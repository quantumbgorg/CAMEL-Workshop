\title{COMPOSITE ADIABATIC TECHNIQUES}

\underline{B. T. Torosov}
\index{Torosov B}

{\normalsize{

\vspace{-4mm} Dipartimento di Fisica, Politecnico di Milano, Piazza Leonardo da Vinci 32, I-20133 Milano, Italy

%\vspace{-4mm} $^2$\unisofia

%\vspace{-4mm} $^3$\dijon

\email torosov@phys.uni-sofia.bg}}

We present a method for optimization of the technique of adiabatic passage between two quantum states by composite sequences of chirped pulses: composite adiabatic passage (CAP). The nonadiabatic losses can be canceled to any desired order with sufficiently long sequences, regardless of the nonadiabatic coupling, by choosing the relative phases between the constituent pulses appropriately. The values of the composite phases are universal for they do not depend on the pulse shapes and the chirp. The accuracy of the CAP technique and its robustness against parameter variations make CAP suitable for ultrahigh-fidelity quantum information processing. We also introduce a high-fidelity technique for coherent control of three-state quantum systems, which combines two popular control tools --- stimulated Raman adiabatic passage (STIRAP) and composite pulses. By using composite sequences of pairs of partly delayed pulses with appropriate phases the nonadiabatic transitions, which prevent STIRAP from reaching unit fidelity, can be canceled to an arbitrary order by destructive interference, and therefore the technique can be made arbitrarily accurate. The composite phases are given by simple analytic formulas, and they are universal for they do not depend on the specific pulse shapes, the pulse delay and the pulse areas.

{\normalsize
[1] B. T. Torosov, S. Gu\'{e}rin and N. V. Vitanov, Phys. Rev. Lett. \textbf{106}, 233001 (2011).
\vsp

[2] B. T. Torosov and N. V. Vitanov, Phys. Rev. A \textbf{87}, 043418 (2013).
}

\vspace{\baselineskip}
