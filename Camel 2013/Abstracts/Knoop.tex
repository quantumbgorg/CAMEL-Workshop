\title{LARGE ION CLOUDS IN RADIOFREQUENCY TRAPS}

\underline{M. Knoop}\index{Knoop M}

{\normalsize{\vspace{-4mm}
Universit\'e d'Aix-Marseille, CNRS, PIIM, UMR7345, Centre de St J\'er\^ome, Case C21, 13397 Marseille Cedex 20, France

\email martina.knoop@univ-amu.fr}}

The trapping of large ion clouds or crystals is gaining interest for various applications. Quantum information processing and microwave metrology are only two possible topics. Our group experimentally studies the dynamics and thermodynamics of trapped ions. In particular the use of very large ion clouds is a challenge but may allow to reach interesting regimes for the study of phase transition and crystallization behavior, or long-range interactions.

Our trapping device is composed of zones of different geometry aligned along a common z-axis. A quadrupole and an octupole linear trap are mounted in line, the quadrupole part being separated in two zones by a center electrode. The traps have been dimensioned to allow for the confinement of an ion cloud filling half the trap and reaching crystallization. Variation of the geometry of the trapping potential has an influence on the ion density distribution in the trap. One of the experimental challenges is the shuttling of a large cloud from the quadrupole part to the multipole part without heating or loss of the ions. In order to maximize ion clouds and to optimize shuttling probabilities, protocols have been optimized numerically. We have developed a FORTRAN90 code which allows to simulate ion cloud transport using the ``real'' trap potential generated by SIMION8.1. Major differences exist compared to the transport in microtraps, mainly the distances of the transport and the
number of control electrodes that can be incorporated in the design, but also the size and shape of the ion cloud. The equations governing the dynamics have been obtained and they allow to achieve 100$\%$ transport of clouds containing more than 10 thousand ions. Results of experiments and simulations will be reported.

\vspace{\baselineskip}
