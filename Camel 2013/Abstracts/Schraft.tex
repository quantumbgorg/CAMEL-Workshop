\title{REPHASING OF ATOMIC COHERENCES BY COMPOSITE SEQUENCES IN A DOPED SOLID}

\underline{D. Schraft} \index{Schraft D}

{\normalsize{\vspace{-4mm}}
Institute of Applied Physics, Technical University of Darmstadt, Hochschulstrasse 6, 64289 Darmstadt, Germany

\email daniel.schraft@physik.tu-darmstadt.de}

The phenomenon of electromagnetically induced transparency (EIT) permits the storage of light pulses, and hence plays a central role in optical and/or quantum information storage. In EIT-based light storage a probe pulse is encoded in (and retrieved from) atomic coherences of a three-level quantum system.
Rare earth ion-doped solids are an attractive medium for implementation of EIT-based memories. The media exhibit long decoherence times and small homogenous optical line width, while maintaining the advantages of solids, i.e. large density and scalability. In doped solids the atomic coherences are created between inhomogeneous broadened hyperfine levels. Hence, robust rephasing protocols are required to cope with dephasing and reach long storage times. However, standard rephasing (e.g. by $\pi$ pulses) only work properly under ideal conditions.
We investigated rephasing of atomic coherences in a doped solid (Pr:YSO) by composite sequences. The sequences consist of radio frequency-$\pi$-pulses with appropriately chosen phases. We experimentally confirmed their robustness with regard to errors in pulse area and transition frequency. Moreover, the efficiency of light storage supported by composite rephasing improved by 20$\%$ compared to $\pi$ pulse rephasing. As another alternative, we applied a recently developed composite version of rapid adiabatic passage (RAP), i.e. composite adiabatic passage (CAP) [1].  Our data clearly show, that CAP is more robust compared both to RAP as well as $\pi$ pulses.

{\normalsize
[1] B. T. Torosov, S. Gu\'erin and N. V. Vitanov, Phys. Rev. Lett. \textbf{106}, 233001 (2011).
}

\vspace{\baselineskip}
