\title{MICROWAVE ENTANGLEMENT GENERATION, ION CHIPS AND CRYOGENIC ION TRAPPING}

\underline{W. Hensinger} \index{Hensinger W}

{\normalsize{\vspace{-4mm}
Ion Quantum Technology Group, Department of Physics and Astronomy,
University of Sussex, Falmer, Brighton, East Sussex, BN1 9QH, United Kingdom

\email w.k.hensinger@sussex.ac.uk}}

The combination of global microwave fields and static magnetic field gradients are a promising route
to scalability by providing ion selectivity as well as the simultaneous entanglement of a large
number of ions. Such gate operations are vulnerable to decoherence due to fluctuating magnetic
fields, however the use of microwave-dressed states protects against this source of noise; with
radio-frequency fields being used for qubit manipulation. I will discuss the preparation of such
states in an ytterbium ion, and the increase in coherence time it produces. The 2nd order Zeeman
effect removes the degeneracy of the radio-frequency transitions within the F=1 manifold of the
ion's ground state, and we use this to demonstrate a simple yet powerful method for manipulating the
dressed state qubit, which I will contrast with the method originally proposed.

We have successfully trapped ions in a specialized setup incorporating strong magnetic field
gradients. I will report on our progress of using dressed states towards achieving high-fidelity
quantum gates with microwaves.

I will present several new trap ion chips developed towards scaling ion trap architectures. I will
report trapping and shuttling within a 2-dimensional microfabricated ion trap array which produces a
two-dimensional ion trap lattice. Each lattice site has 4 neighbouring lattice sites where
interactions can occur. Shuttling of single ions between lattice sites allows for additional
flexibility increasing the overall functionality of the ion trap array. Our ion chip represents a
versatile architecture for 2-dimensional quantum simulations with trapped ions. We are also able to
deterministically introduce defects into the lattice adding further classes of quantum simulations
that can be performed. I will also report progress on a number of other ion chips including a
circular ion trap storage ring on a chip.  I will also present a new study on how to more than
double the breakdown voltages in ion chips and other microfabricated device. I will also show how to
design optimal two-dimensional ion trap arrays for quantum simulation.

Ion trapping in a cryogenic environment has a multitude of application.  We have successfully
constructed a cryogenic ion trap experiment and I will discuss details of the experiment.

\vspace{\baselineskip}
