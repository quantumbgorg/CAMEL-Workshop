\title{QUANTUM NETWORKING WITH QUDITS STORED IN SINGLE PHOTONS}

\underline{A. Kuhn}\index{Kuhn A}

{\normalsize{\vspace{-4mm}
University of Oxford, Clarendon Laboratory, Parks Road, OX1 3PU Oxford, United Kingdom

\email axel.kuhn@physics.ox.ac.uk}}

Photons acting as flying information carriers are key to many modern applications of quantum technologies like, e.g., linear optics
quantum computing (LOQC). The arbitrary control of their quantum states in space and time is crucial to the success of these
schemes. To this end, we looked into the deterministic single-photon emission from a single atom into an optical cavity and
found that the properties of these photons are controllable to an unprecedented degree. We verified the perfect singleness and
indistinguishability of the deterministically emitted photons, which are produced with a probability of 80$\%$ upon each trigger and
at a repetition rate of 1 MHz. Over and above that, we successfully encode arbitrary qubits, qutrits and even ququads within the
spatio-temporal mode profile of individual photons. The fidelity of this groundbreaking state preparation technique has been
verified in time resolved quantum-homodyne measurements to be better than 96$\%$. Such a close-to-perfect control of photonic
wavefunctions paves the way towards novel applications in quantum computing and communication. For instance, when using
qutrits or ququads, powerful ternary or quaternary quantum logic concepts could be implemented without the need for any
additional resources.

\vspace{\baselineskip}
