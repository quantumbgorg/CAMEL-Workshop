\title{MULTI-MODE QUANTUM PROCESSES WITH PHOTONS}

\underline{B. Garraway} \index{Garraway B}

{\normalsize{\vspace{-4mm}
Department of Physics and Astronomy, University of Sussex, Falmer,
Brighton, BN1 9QH, United Kingdom

\email b.m.garraway@sussex.ac.uk}}

We have developed theory for multi-photon atom-cavity interactions based on off-resonant interactions with multiple single photons. The locations of sharp resonances are predicted from the use of effective two-level, three-level, and even more complex Hamiltonians and an evaluation is made for configurations which are least sensitive to external parameters and decoherence. Possible applications to photonic QIP are presented.

The theory starts from a multi-mode, multi-level Jaynes-Cummings model and uses the effective Hamiltonian procedure of Shore [1] to eliminate the off-resonant levels. This gives good estimates for the detunings and couplings required for multi-photon resonances.

Quantum information is processed using a dual-rail encoded qubit where cavity pairs are represented by logical qubits. For example, the state $|a1010\rangle$ can represent the qubit state $|10\rangle$. In this case the population swapping seen is part of the logic table of an iSWAP gate. Using a six level system we can extend the approach to a Fredkin gate where there are rich possibilities for the choice of different combinations of resonant levels. As quantum gates these systems are extremely sensitive and we have made studies of the effects of decoherence and accuracy of the effective models.

{\normalsize
[1] B. W. Shore, Phys. Rev. A \textbf{24}, 1413 (1981).
}

\vspace{\baselineskip}
