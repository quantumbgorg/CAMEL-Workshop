\title{CONTROL OF QUBITS BY SHAPED PULSES OF FINITE DURATION}

\underline{I. I. Boradjiev} \index{Boradjiev I}

{\normalsize{\vspace{-4mm}

\unisofia

\email boradjiev@phys.uni-sofia.bg}}

We consider the interaction of a two-state quantum system with a class of pulses of finite temporal duration.
The pulse shape function $f(t)$ of such a pulse is necessarily a nonanalytic function of time, with discontinuous derivatives at the turn-on and turn-off times.
The excitation line width --- the excited-state population versus the detuning --- is determined primarily by the magnitude of the jumps of the derivative $f^{(n)}(t)$ at the points of nonanalyticity, where $n$ is the order of the first discontinuous derivative; this nonanalyticity shows up in the $n$-th superadiabatic basis.
The excitation line width for such pulses exhibits weak power broadening --- it scales up as $\Omega_0^{1/(n+1)}$, where $\Omega_0$ is the peak Rabi frequency of the transition: $\Omega(t)=\Omega_0 f(t)$.
As a specific example, we consider the power-of-sine class $f(t) = \sin^n(\pi t/T)$ ($0\leqslant t \leqslant T$) and a truncated Gaussian pulse, and we compare their excitation line widths with the well-known excitation profile of the rectangular pulse (the Rabi formula).
We find that, because of the reduced power broadening, the $\sin^n$ and truncated Gaussian pulses may accelerate manipulation of qubits compared to rectangular pulses.
The reason is that the lower power broadening allows one to use higher Rabi frequency, and hence shorter pulse duration, without affecting significantly other closely lying states.

\vspace{\baselineskip}
