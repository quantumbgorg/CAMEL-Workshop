\title{LASING WITHOUT INVERSION IN MERCURY}

\underline{T. Walther} \index{Walther T}

{\normalsize{\vspace{-4mm}
Institute for Applied Physics, TU Darmstadt, Schlossgartenstr. 7, D-64289 Darmstadt, Germany

\email thomas.walther@physik.tu-darmstadt.de}}

In this talk we present recent progress towards the implementation of a four-level Lasing without Inversion (LWI) scheme in mercury.
It is based on theoretical calculations by Fry \textit{et al.} [1] and aims at demonstrating LWI based UV cw-lasers at 253.7
nm or 185 nm, respectively. Laser radiation of two lasers, both of which are at a longer wavelength than that of the LWI laser output,
is required to set up the necessary coherence in a mercury filled gas cell. The particular alignment of the lasers enables
compensation for the Doppler effect leading to high efficiency and a relatively straightforward setup. During the talk we will present
the details of the experimental idea as well as the present status of the experiment.

{\normalsize
[1] E. S. Fry, M. D. Lukin, Th. Walther, and G. R. Welch, Optics Communications, 179:0 499-504 (2000).
}

\vspace{\baselineskip}
