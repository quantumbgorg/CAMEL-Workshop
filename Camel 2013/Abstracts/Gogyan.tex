\title{NON-LINEAR FARADAY EFFECT FOR THE PULSED PROBE FIELD}

\underline{A. Gogyan} \index{Gogyan A}

{\normalsize{\vspace{-4mm}
Institute for Physical Research,
National Academy of Sciences of Armenia,
Gitavan-2, 0203 Ashtarak, Armenia

\email agogyan@gmail.com}}

We have studied non-linear magneto-optical effects in media of lambda atoms, which are confined in a
hollow-core crystal fiber (HCF). For the first time we have investigated the non-linear Faraday
effect for pulsed linearly polarized weak probe field. We have shown, that the pulsed input field
that propagates in HCF under the electromagnetically induced transparency conditions, induces steep
dispersion of the medium at probe frequency leading to large Faraday rotation (FR) in very weak
magnetic fields. We have shown that the front edge of the temporal distribution of FR of probe pulse
polarization plane has similar dependence on the magnetic field, i.e. for small values of the latter
the polarization rotation is increasing and then decreasing. FR for the pulsed probe field is
steeper than is in the case of CW probe field for the same feasible parameters, which can lead to
experimental higher sensitivities. Another interesting observation is that this polarization plane
rotation not only depends on the applied magnetic field strength, but also has different values at
different parts of the probe pulse's temporal distribution. This property may have range of
applications in quantum information science, e. g. the phase dependence of the probe pulse can carry
information, which is then  possible to map on a medium that is sensitive to the direction of light
polarization.

\vspace{\baselineskip}
