\title{LASER CONTROL OF 9BE+ ION QUBITS USING AN OPTICAL FREQUENCY COMB}

\underline{A.-G. Paschke} \index{Paschke A-G}

{\normalsize{\vspace{-4mm}
Institut f\"ur Quantenoptik, Universit\"at Hannover,
Welfengarten 1, 30167 Hannover

\email paschke@iqo.uni-hannover.de}}

A CPT test experiment based on a g-factor comparison between single trapped (anti-)protons is currently being designed in our group. The challenging spin-flip frequency measurement will be realized by transferring the (anti-)proton's spin state to a co-trapped ``logic'' 9Be+ ion for readout using quantum logic operations according to the proposal by Heinzen and Wineland [1].
The essential manipulation of 9Be+ is typically carried out using two-photon stimulated Raman transitions. Because of the high required magnetic field and the resulting large qubit splitting of 9Be+, the widely used CW laser approach is rather unattractive. Instead, we intend to explore pulsed lasers to implement the necessary operations (following [2]).
A considerable experimental challenge arises from the atomic structure of 9Be+, which requires a careful design of the comb's spectral properties. To obtain the optimized spectrum near 313 nm for maximum two-photon stimulated Rabi Frequency and minimum single-photon spontaneous scattering rate the spectral bandwidth as well as the spectral profile need to be precisely controlled in order to keep the necessary UV power acceptably low. For efficient generation of the optimal narrow bandwidth UV pulses we discuss using spectral compression of broadband pulses at 626 nm in a second-order nonlinear crystal, followed by spectral pulse shaping to control the UV comb's spectrum.

{\normalsize
[1] Heinzen and Wineland, Phys. Rev. A \textbf{42}, 2977 (1990).
\vsp

[2] Hayes \textit{et al.}, Phys. Rev. Lett. \textbf{104}, 140501 (2010).
}

\vspace{\baselineskip}
