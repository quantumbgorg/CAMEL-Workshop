\title{NONLINEAR OPTICAL PUMPING OF A SLOW AND COLD CS BEAM}

\underline{T. Kirova}\index{Kirova T}

{\normalsize{\vspace{-4mm}
Laser Centre,
University of Latvia,
Riga, LV-1002,
Latvia

\email teo@lu.lv}}

A cesium beam is produced out of a modified pyramidal Magneto-Optical Trap used as an atom
funnel. Briefly, trapping and repumping radiation are sent onto an arrangement of prisms and mirrors
shaped in the form of a hollow pyramid realizing the beam configuration of a standard MOT; a
hole (area $\sim 2$ mm$^2$) is drilled at the pyramid apex, hence no laser radiation is
retroreflected along the pyramid axis. A continuous beam of atoms is then leaving the hole with
longitudinal velocity around 12 m/s.Right after the hole, the beam is collimated by a 2-D transverse
optical molasses.  We investigated the dynamical aspects of the excitation process in the atoms
belonging to the beam, slowly moving through a weak resonant excitation radiation with a Gaussian
profile. The populations of the $6^{2}\mathrm{P}_{3/2}$ HF levels $F_{e}=3, 4, 5$ have been probed
in a two-photon photoionization scheme with  a low ionization rate, leading to negligible
perturbation. The comparison of the experimental data with results of accurate numerical
simulations highlights the effects of optical pumping phenomena under a weak excitation limit,
involving both hyperfine and Zeeman structure of the energy levels. Thanks to the long transit time
($\approx 100$ $\mu$s) enabled by the sub-thermal velocity of the beam, even a tiny mixture
within the HF levels due to the excitation laser coupling results in essential modifications of the
optical pumping effects.

\vspace{\baselineskip}
