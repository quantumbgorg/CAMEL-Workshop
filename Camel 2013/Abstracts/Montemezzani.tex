\title{DEMONSTRATION OF QUANTUM-LIKE EFFECTS USING PHOTOINDUCED WAVEGUIDES}

\underline{G. Montemezzani} \index{Montemezzani G}

{\normalsize{\vspace{-4mm}
Laboratoire Mat\'eriaux Optiques, Photonique et Syst\`emes (LMOPS)
Universit\'e de Lorraine et Sup\'elec 2, rue E. Belin
57070 METZ (France)

\email germano.montemezzani@metz.supelec.fr}}

An important formal analogy exists between the physics governing light propagation in an array of coupled optical waveguides and the physics governing the quantum population dynamics of discrete levels coupled by a coherent field. This is inspiring increasing interest in fundamental studies as well as for potential applications in integrated optics.
We use an experimental platform for photo-inscription of waveguides, which is based on photorefractive crystals, where the refractive index is locally modified in response to light illumination. The wished waveguide structure is impinged on a control light wave that is imaged laterally on the crystal surface. A second light wave, which propagates longitudinally in the recorded waveguides, is used to characterize the propagation effects and test the quantum analogies. The waveguides are reconfigurable and their properties can be tuned locally in terms of the coupling strength (equivalent to the Rabi frequencies in a coupled two-level system) and of the longitudinal propagation constants (equivalent to the energy detuning).
Several quantum-optical analogies are experimentally demonstrated, including multiple level Stimulated-Raman Adiabatic Passage (STIRAP) and related systems that can be useful for novel broadband integrated optical elements, as well as Electromagnetically-Induced-Transparency (EIT) and the Autler-Townes effect. The theoretical aspects as well as the experimental implementation and results are discussed in each case.

\vspace{\baselineskip}
