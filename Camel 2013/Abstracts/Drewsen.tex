\title{MANIPULATING ION COULOMB CRYSTALS BY OPTICALLY INDUCED POTENTIALS}

\underline{M. Drewsen} \index{Drewsen M}

{\normalsize{\vspace{-4mm}
Department of Physics and Astronomy,
Aarhus University,
Ny Munkegade 120,
DK-8000 Aarhus C,
Denmark

\email drewsen@phys.au.dk}}

In the recent past we have carried out a series of experiments where non-neutral plasmas in the form of ion Coulomb crystals have been placed inside an optical cavity to explore fundamental physics in relation to Cavity Quantum Electrodynamics (CQED) [1,2] as well as mode dynamics of such crystals [3]. These experiments all exploited changes in the cavity reflection/transmission of an optical probe field, which is resonantly coupled to an electronic transition of the atomic ions. So far, the strengths of the interactions of the probe fields with the ion coulomb crystal have purposely been so weak that the mechanical back-action of the light field on the ions’ motion could be neglected. It is, however, as well interesting to explore situations where a cavity light field might be strong enough to change the structural properties of the ion Coulomb crystals due to significant optomechanical forces. While optically induced dipole potential for more than 20 years have been widely used to control the motion of cold neutral atoms, only a few experiments have been reported with ions recently [4-6], due to the challenge of competing with typically much stronger electrostatic forces.
In the presentation, I will give a status on our current activities towards manipulating the spatial ordering of both smaller and larger ion Coulomb crystals using dipole potentials including the perspectives of investigating of structural phase transitions [7] and establish improved quantum memory based on ions.

{\normalsize
[1]	P. F. Herskind, A. Dantan, J. P. Marler, M. Albert, and M. Drewsen, Nat. Phys. \textbf{5}, 494 (2009).
\vsp

[2]	M. Albert, A. Dantan, and M. Drewsen, Nat. Phot. \textbf{5}, 633 (2011).
\vsp

[3]	A. Dantan, J. P. Marler, M. Albert, D. Guenot, and M. Drewsen, Phys. Rev. Lett. \textbf{105}, 103001 (2010).
\vsp

[4]  Ch. Schneider, M. Enderlein, T. Huber and T. Schaetz,  Nat. Phot. \textbf{4}, 772 (2010).
\vsp

[5]   M. Enderlein, T. Huber, C. Schneider, and T. Schaetz, Phys. Rev. Lett. \textbf{109}, 233004 (2012).
\vsp

[6]   R. B. Linnet, I. D. Leroux, M. Marciante, A. Dantan, and M. Drewsen, Phys. Rev. Lett. \textbf{109}, 233005 (2012).
\vsp

[7]   P. Horak, A. Dantan, and M. Drewsen, Phys. Rev. A \textbf{86}, 043435 (2012).
}


\vspace{\baselineskip}
