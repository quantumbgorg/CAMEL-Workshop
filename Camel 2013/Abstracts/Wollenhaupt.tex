\title{PHOTOELECTRON DISTRIBUTIONS FROM REMPI OF ATOMS AND CHIRAL MOLECULES}

\underline{M. Wollenhaupt} \index{Wollenhaupt M}

{\normalsize{\vspace{-4mm}
University of Kassel, Institute of Physics und CINSaT, Heinrich-Plett-Str. 40, 34132 Kassel, Germany

\email wollenha@physik.uni-kassel.de}}

Photoelectron Circular Dichroism (PECD) [1] was previously investigated on small chiral molecules using synchrotron radiation.  The observed asymmetries in the PECD arise in forward/backward direction with respect to the light propagation. Recently, we have demonstrated a circular dichroism effect in the plus/minus ten percent regime derived from images of photoelectron angular distributions resulting from 2+1 REMPI of randomly oriented chiral molecules in the gas phase, where Camphor and Fenchone were chosen as prototypes [2]. The PECD was also observed in the Above Threshold Ionization (ATI) photoelectrons. Currently, we study nuclear and electron dynamics on the intermediate resonance with the help coherent control techniques [3]. In addition, we have developed a tomographic reconstruction method to directly measure three-dimensional photoelectron angular distributions resulting from REMPI in a Velocity Map Imaging (VMI) set-up [4,5]. On atoms we have demonstrate
 d the creation of designer electron wave packets using polarization shaped laser pulses.


{\normalsize
[1] I. Powis, Adv. Chem. Phys. \textbf{138}, 267 (2008).
\vsp

[2] C. Lux \textit{et al.}, Angew. Chem. Int. Ed. \textbf{51}, 5001 (2012).
\vsp

[3] M. Wollenhaupt and T. Baumert, Faraday Discuss. \textbf{153}, 9, (2011).
\vsp

[4] M. Wollenhaupt \textit{et al.}, Appl. Phys. B \textbf{95}, 647, (2009).
\vsp

[5] M. Wollenhaupt \textit{et al.}, Phys. Chem. Chem. Phys. \textbf{14}, 1341 (2013).
}

\vspace{\baselineskip}
