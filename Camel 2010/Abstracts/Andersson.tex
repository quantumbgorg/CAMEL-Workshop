\title{PHOTONS WALKING THE LINE}

A. Schreiber, K. N. Cassemiro, V. Potocek, A. Gabris, P. Mosley, \underline{E. Andersson}, I. Jex and C. Silberhorn
\index{Schreiber A}\index{Cassemiro K}\index{Potocek V}\index{Gabris A}\index{Mosley P}\index{Andersson E}\index{Jex I}\index{Silberhorn C}

{\normalsize{\vspace{-4mm}
EPS/Physics, Heriot-Watt University, Edinburgh EH14 4AS, United Kingdom

\email E.Andersson@hw.ac.uk}}

The classical random walk is a fundamental model, describing phenomena ranging from material transport in media to the
evolution of stock market shares. For a quantum walk, a superposition of different paths or positions leads to quantum
interference. Quantum walks have been shown to speed up search problems and to play a role as a computational primitive. We
show how to explore the physics of quantum walks using a robust, flexible and simple optical implementation. A straightforward
setup would use a cascade of beam-splitters. Unfortunately such a complex network suffers from mechanical instabilities and
requires a large number of optical components to realize more steps of the walk. We avoid this with a realization using optical
fibres, employing a compact network loop. The same paths are re-used again and again, and the walk takes place in the time
domain. Since one can easily operate with different coins, and since one may individually address time pulses by using
commercially available fast electro-optical modulators (EOM), our work opens exciting new possibilities for the realization of
quantum information protocols.

\vspace{\baselineskip} 