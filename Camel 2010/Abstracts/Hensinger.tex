\title{SCALABLE QUANTUM TECHNOLOGY WITH TRAPPED YTTERBIUM IONS}

\underline{W. Hensinger} \index{Hensinger W}

{\normalsize{\vspace{-4mm}
Ion Quantum Technology Group, Department of Physics and Astronomy,
University of Sussex, Falmer, Brighton, East Sussex, BN1 9QH, United
Kingdom

\email w.k.hensinger@suss ex.ac.uk}}

We have trapped single ytterbium ions in an experimental setup particularly designed for the
development of advanced ion trap chips. This setup allows for rapid turn-around time, optical access
for all type of ion trap chips and up to 100 electric interconnects. The particular ion trap used
features an ion -- electrode distance of 310 microns and secular frequencies on the order of 1 MHz
and we have observed ion life times in excess of 15 hours. We measured the motional heating rate and
obtained a value for the electric field noise of $S(1\textrm{MHz}) = 3.6\times 10^{11}(\textrm{V/m})^2\textrm{Hz}^{-1}$. We also
accomplished isotope selective loading. Incorporating MEMS fabrication technologies we report on
design and fabrication of a multi-zone y-junction surface-electrode ion trap in which all
dielectrics are optically and electrically shielded. We also present progress on a microfabricated
2-D ion trap array that allows for the creation of ion lattices for the implementation of a quantum
simulator. The separation of ions within an ion trap array is important and we present optimal
electrode configurations in surface ion traps. We also report on our studies for the development of
optimal junctions.

\vspace{\baselineskip} 