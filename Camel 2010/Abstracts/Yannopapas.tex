\title{COHERENT PHENOMENA IN THE VICINITY OF METALLIC NANOSTRUCTURES}

\underline{V. Yannopapas} \index{Yannopapas V}

{\normalsize{\vspace{-4mm}
Department of Materials Science, University of Patras, Panepistimioupolis, GR-26504, Rio, Greece

\email vyannop@upatras.gr}}

The interaction of light with atoms which are located closed to metallic nanostructures enables the observation of phenomena in quantum optics which cannot be unreeled when atoms are in vacuum. Such a case is the observation of quantum interference of the spontaneous emission paths of a V-type three-level atom. When such an atom is placed in the vicinity of properly designed metallic nanostructure, the quantum interference can be greatly enhanced around the surface-plasmon bands of the metallic nanostructure leading to much slower decay of the excited atomic state [1]. Besides this, we present two characteristic examples of analogues of well established quantum coherent phenomena in classical electromagnetism. The first is a classical analogue of electromagnetically-induced transparency. Namely, we will examine the case of a two-dimensional lattice of metallic nanoparticles on top of a guiding substrate. If the substrate supports guided modes which fall within the surface-plasmon absorption band of the lattice of nanoparticles then one can quench light absorption via a mechanism which resembles electromagnetically-induced transparency in atomic gases [2]. Alternatively, the substrate can make metal absorb light in spectral regions well above the surface-plasmon band where metal is essentially transparent. The second analogue is about the coherent control of the optical near-field in nanostructures. Namely, we will show that one can achieve spatio-temporal control over the excitation of surface exciton-polaritons in semiconductor nanoparticles by a proper choice of a chirped light pulse [3]. This is achieved in nanoparticles made from semiconductors with strong excitonic oscillator strength and small absorption linewidth such as copper chloride or oxide.

\begin{enumerate}
\item V. Yannopapas, E. Paspalakis, and N. V. Vitanov, Phys. Rev. Lett. \textbf{103}, 063602 (2009).
\item V. Yannopapas, E. Paspalakis, and N. V. Vitanov, Phys. Rev. B \textbf{80}, 035104 (2009).
\item V. Yannopapas and N. V. Vitanov, Phot. Nanostr., submitted.
\end{enumerate}


\vspace{\baselineskip} 