\title{MEASURE FOR THE NON-MARKOVIANITY OF QUANTUM PROCESSES}

\underline{E.-M. Laine} \index{Laine E-M}

{\normalsize{\vspace{-4mm}
Kaivokatu 16 bA 4, 20520 Turku, Finland

\email emelai@utu.fi}}

A Markov process in the evolution of an open quantum system gives rise to a quantum
dynamical semigroup, for which the most general representation of the dynamics can
be written in the Lindblad form. There exists complex systems for which the relatively
simple description of the open system dynamics given by a Markovian master equation
fails to give a comprehensive picture of the dynamics. Thus in many realistic physical
systems the Markovian approximation of the dynamics gives an overly simplified picture
of the open system evolution and a more rigorous description of the dynamics is required.

To give insight into the nature of non-Markovian effects many analytical methods and
numerical simulation techniques have been developed in recent years. Non-Markovianity
manifests itself in the different approaches in a variety of ways and there exists no recipe
for comparing the degree of non-Markovianity between the different approaches. In order
to give a general quantity determining the degree of non-Markovian behavior in the
open system dynamics, one has to rigorously define what makes a dynamical map non-
Markovian.

We introduce a general measure for the degree of non-Markovian behavior in an open
quantum system by quantifying the information flow from the environment to the open
system [1, 2]. The change in the distinguishability of states of the open system can be interpreted
as information flow between the system and the environment. Thus the measure
for non-Markovianity can be constructed from the concept of trace distance, which quantifies
the distinguishability of quantum states. If the distinguishability is always decreasing,
then the system is Markovian. Increase in the distinguishability at certain times indicates
information flow from the environment to the system and therefore non-Markovian behavior.

This criterion for non-Markovianity does not require knowledge about the details of the
environment or the system-environment interaction. Instead, tomographic measurements
of a system can quantify the extent to which a system exhibits non-Markovian behavior.

\begin{enumerate}
\item H.-P. Breuer, E.-M. Laine, J. Piilo, Phys. Rev. Lett. \textbf{103}, 210401 (2009).
\item E.-M. Laine, J. Piilo, H.-P. Breuer, arXiv: 1002.2583 [quant-ph].
\end{enumerate}

\vspace{\baselineskip} 