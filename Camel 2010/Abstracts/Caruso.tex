\title{NOISE AND ENTANGLEMENT: FROM NATURAL PHOTOSYNTHESIS TO QUANTUM COMMUNICATION}

\underline{F. Caruso} \index{Caruso F}

{\normalsize{\vspace{-4mm}
Ulm University, Institute of Theoretical Physics, Albert-Einstein-Allee 11, D - 89069 Ulm, Germany

\email filippo.caruso@uni-ulm.de}}

Recently, we investigated the intricate interplay of noise and quantum coherence to explain the
remarkable efficiency well above 90\% for excitation energy transfer (EET) in light harvesting
complexes during photosynthesis. Here, we study the role of entanglement in these biological
networks, even in the case of spatially and temporally correlated noise and for different injection
mechanisms, like thermal and coherent laser excitation. While quantum information processing tends
to favour maximal entanglement, the optimal EET efficiency is achieved when the initial part of the
evolution displays intermediate values of entanglement. Finally, we investigate this transport
dynamics in the elegant and powerful framework of quantum communication, and we find that the
dephasing may enhance, in a very remarkable way, the capability of transmitting not only classical
but also, more counterintuitively, quantum information over biologically inspired quantum
communication networks.

\vspace{\baselineskip} 