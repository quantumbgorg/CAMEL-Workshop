\title{CONTROL OF THE SIGN OF SUBNATURAL RESONANCES IN COUNTERPROPAGATING LASER BEAMS}

\underline{V. M. Entin}$^1$, D. V. Brazhnikov$^{2,3}$, A. V. Taichenachev$^{2,3}$, V. I. Yudin$^{2,3}$, I. I. Ryabtsev$^1$
\index{Entin V} \index{Brazhnikov D} \index{Taichenachev A} \index{Yudin V} \index{Ryabtsev I}

{\normalsize{
\vspace{-4mm} $^1$Institute of Semiconductor Physics SB RAS, Pr. Acad. Lavrenteva 13, Novosibirsk, Russia

\vspace{-4mm} $^2$Novosibirsk State University, ul. Pirogova 2 Novosibirsk, Russia

\vspace{-4mm} $^3$Institute of Laser Physics Novosibirsk, Russia

\email ventin@isp.nsc.ru}}

The resonances of electromagnetically-induced transparency (EIT, [1]) and absorption (EIA, [2]) are the bright examples of nonlinear effects based on atomic coherences. Such  EIT and EIA ``subnatural'' resonances have found various applications in nonlinear optics and optical communications (``slow'' and ``fast'' light, four-wave mixing etc.) and metrology (atomic clocks and magnetometers). At the first time EIT as well as EIA were studied in the bichromatic unidirectional laser waves. However, from experimental viewpoint to observe these resonances an easier way may be exploited using so-called Hanle configuration [3]. It requires only one traveling light wave and the spectroscopic signal is absorption versus static magnetic field applied along the vapour cell. At that the subnatural resonance is connected with a crossing of magnetic sublevels in ground state (level-crossing resonance).
Besides studying the parameters of subnatural resonances (width, amplitude, contrast, shift), the problem of its sign has a special interest. It is known that a sign of the subnatural resonance depends on such factors as structure of energy levels [2,4], collisions of atoms with buffer gas or walls of cell [5], microwave [6] or static magnetic fields [7] and some others. In this paper we propose new easy and effective method [8] for controlling of the sign of subnatural resonance in Hanle configuration. We have discussed the scheme being composed of two counterpropagating overlapped light waves with same frequency and static magnetic field applied along the cell. The sign of the resonance can be governed by varying either the angle between the polarizations or the ellipticities. Numerical calculations show that the sign-reversal effect exists for dark-type atomic transitions (e.g. $F=2$ to $F=1$) as well as for bright ones (in particular, $F=2$ to $F=3$). Explicit analytical results and clear physical interpretation are gained using Lambda-atom model. Main theoretical conclusions have been proved by our experiments made at the $F=2$ to $F=1$ transition of D1-line in 87Rb. It should be noted that the method proposed, in principle, may provide high-contrast EIA resonances even in presence of buffer gas, when collisional depolarization of exited state occurs.

\begin{enumerate}
\item G. Alzetta, A. Gozzini, L. Moi, G. Orriols, Nuovo Cimento B 36(1), 5-20 (1976).
\item A.M. Akulshin, S. Barreiro, A. Lezama, Phys. Rev. A \textbf{57}, 2996-3002 (1998).
\item Y. Dancheva, G. Alzetta, S. Cartaleva et al., Opt.Commun. \textbf{178}, 103 (2000).
\item A.S. Zibrov, A.B.Macko, JETP Lett. \textbf{82}, 472 (2005).
\item C. Andreeva, A. Atvars, M. Auzinsh et al., Phys. Rev. A \textbf{76}, 063804 (2007).
\item M.D. Lukin, S.F. Yelin, M. Fleischhauer et al., Phys. Rev. A \textbf{60}, 3225 (1999);
G.S. Agarwal, T.N. Dey, and S. Menon, Phys. Rev. A \textbf{64}, 053809 (2001).
\item K. Nasyrov, S. Cartaleva, N. Petrov et al., Phys. Rev. A \textbf{74}, 013811 (2006).
\item D.V. Brazhnikov, A.V. Taichenachev, V. I. Yudin, I.I. Ryabtsev, V.M. Entin (Submitted to JETP Lett, 2010).
\end{enumerate}

\vspace{\baselineskip} 