\title{ON THE SPECTRAL PROPERTIES OF LAX OPERATORS}

\underline{V. Gerdjikov} \index{Gerdjikov V}

{\normalsize{\vspace{-4mm}
Institute for Nuclear Research and Nuclear Energy, Bulgarian Academy
of Sciences, 72 Tsarigradsko chaussee, 1784 Sofia, Bulgaria

\email gerjikov@inrne.bas.bg}}

I  will consider a class of Lax operators generalizing the
Zakharov-Shabat system:
\begin{equation}\label{eq:L}
L\psi \equiv i\frac{d\psi}{dx} + (Q(x) - \lambda J)\psi(x,\lambda)=0,
\end{equation}
where $Q(x)$ and $J$ take values in a simple Lie algebra $\mathfrak{g}$
and vanishes fast enough for $x\to\pm\infty$.
Applying the inverse scattering method to $L$ one is able to solve a
number of important nonlinear evolution equations (NLEE) such as multicomponent
nonlinear Schrodinger equations, $N$-wave equations, etc.

We will construct the fundamental analytic solutions of $L$ and will show
the important they play in constructing the spectral decompositions of $L$.
The FAS satisfy a Riemann-Hilbert problem, which allows one
 to use the dressing Zakharov-Shabat method for constructing the
reflectionless potentials of $L$ and the soliton solutions of the NLEE.

Another important field of applications of these results is in the quantum mechanics
of multi-level atomic systems.

\vspace{\baselineskip} 