\title{MODE ENTANGLEMENT IN QUANTUM INFORMATION PROCESSING}

\underline{V. Vedral} \index{Vedral V}

{\normalsize{\vspace{-4mm}
Clarendon Laboratory, University of Oxford, Parks Road, Oxford OX1 3PU, UK

\email vlatko.vedral@gmail.com}}

Natural particle-number entanglement resides between spatial modes in coherent ultra-cold atomic
gases. However, operations on the modes are restricted by a superselection rule that forbids coherent superpositions of different particle numbers. This seemingly prevents mode entanglement being used as a resource for quantum communication. In my talk I will argue that mode entanglement
of a single (massive or massless) particle can be used for dense coding and quantum teleportation despite the superselection rules. I will present schemes where the dense coding linear photonic
channel capacity is reached without a shared reservoir and where the full quantum channel capacity
is achieved if both parties share a coherent particle reservoir. If time permits, I will discuss the type of non-locality arising from mode entanglement. For instance, one photon superposed symmetrically over many distant sites (i.e. in a W state) can give an ``all or nothing'' violation of local realism (in a similar manner to the GHZ state). This shows that mode entanglement is as viable and useful as any other form of entanglement exploited in experimental physics.

\vspace{\baselineskip} 