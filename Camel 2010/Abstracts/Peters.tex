\title{ALL-OPTICAL CAVITIES FOR LIGHT: FROM STORED TO STATIONARY LIGHT}

\underline{T. Peters} \index{Peters T}

{\normalsize{\vspace{-4mm}
Institute of Applied Physics, Technical University of Darmstadt,
64289 Darmstadt, Germany

\email thorsten.peters@physik.tu-darmstadt.de}}

Creating light pulses with ultralow group velocities and ``storage and retrieval'' of light pulses based on electromagnetically induced transparency (EIT) have become standard techniques in recent years. As the interaction time of light pulses with a medium can be dramatically increased without inducing additional losses, future applications in the field of quantum information processing (QIP) are under development. However, non-linear optical interactions, which are required for QIP, are so far rather inefficient as their efficiency is proportional to the intensity and the interaction time of the light pulses.
To overcome this problem, the formation of light pulses with stationary spatio-temporal envelopes has been proposed and demonstrated a few years ago in a medium comprised of hot atoms. Since then there had been a theoretical discussion whether or not stationary light pulses (SLPs) could also be formed in cold media.
In this talk, I will discuss the underlying physics of SLP formation, their relation to slow and stored light and why the medium temperature plays an important role in their formation. Following this discussion I will present experimental results on the first successful creation of SLPs in a medium of laser-cooled atoms and how their formation can be actively controlled. All experimental data will be compared to results of numerical simulations.

\vspace{\baselineskip} 