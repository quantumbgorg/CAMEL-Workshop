\title{ADIABATIC MARKOVIAN DYNAMICS}

\underline{O. Oreshkov} \index{Oreshkov O}

{\normalsize{\vspace{-4mm}
Quantum Optics, Quantum Nanophysics, Quantum Information, University of Vienna, Boltzmanngasse 5, Room 3315, 1090 Vienna

\email ooreshkov@gmail.com}}

We propose a theory of adiabaticity in quantum Markovian dynamics based on a structural decomposition of the Hilbert space induced by the asymptotic behavior of the Lindblad semigroup. A central idea of our approach is that the natural generalization of the concept of eigenspace of the Hamiltonian in the case of Markovian dynamics is a noiseless subsystem with a minimal noisy cofactor. Unlike previous attempts to define adiabaticity for open systems, our approach deals exclusively with physical entities and provides a simple, intuitive picture at the underlying Hilbert-space level, linking the notion of adiabaticity to the theory of noiseless subsystems. As an application of our theory, we propose a framework for decoherence-assisted computation in noiseless codes under general Markovian noise. We also formulate a dissipation-driven approach to holonomic computation based on adiabatic dragging of subsystems that is generally not achievable by non-dissipative means.

\vspace{\baselineskip} 