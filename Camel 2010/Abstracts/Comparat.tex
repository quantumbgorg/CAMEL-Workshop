\title{IS THE ADIABATIC THEOREM VALID?}

\underline{D. Comparat} \index{Comparat D}

{\normalsize{\vspace{-4mm}
Laboratoire Aim\'e Cotton, CNRS, Universit\'e Paris-Sud, B\^atiment 505, 91405 Orsay, France

\email daniel.comparat@u-psud.fre}}

Adiabaticity occurs when, during its evolution, a physical system remains in the instantaneous
eigenstate of the hamiltonian. Following Landau-Zener or STIRAP processes, It is widely believed that the condition of a slow hamiltonian variation rate, compared to the spectral gap, is sufficient to ensure adiabaticity. We shall discuss this theorem and show that the theorem is true but only when the hamiltonian is real and non oscillating (for instance containing exponential or polynomial but no sinusoidal functions). Furthermore, we shall improve the quantum adiabatic theorem based on a slow down evolution ($H(\epsilon t)$, $\epsilon \rightarrow 0$), which is insufficient to describe an evolution driven by the hamiltonian $H(t)$ itself. Indeed, we shall derive general criteria and exact bounds, for the state and its phase, ensuring an adiabatic evolution for any hamiltonian $H(t)$.

\vspace{\baselineskip} 