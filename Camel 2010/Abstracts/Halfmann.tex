\title{COHERENT OPTICAL DATA STORAGE AND PROCESSING IN A DOPED SOLID}

\underline{T. Halfmann} \index{Halfmann T}

{\normalsize{\vspace{-4mm}
Institute of Applied Physics, Technical University of Darmstadt,
64289 Darmstadt, Germany

\email thomas.halfmann@physik.tu-darmstadt.de}}

Coherent interactions between light and matter provide powerful tools to control optical properties and processes in any type of quantized medium. Among others, contemporary research efforts aim at efficient storage and processing of optically encoded data, e.g. as required in quantum information processing. The talk presents experimental implementations of coherent laser-driven interactions in particular solid media, i.e. rare-earth doped solids. The latter media combine the advantages of atoms in the gas phase (i.e. spectrally narrow transitions) with the advantages of solids (i.e. large density and scalability). The talk reports on experimental implementations of electro-magnetically-induced transparency (EIT) and dark-state polaritons (DSPs), different versions of stimulated Raman adiabatic passage (STIRAP), dynamic decoherence control, and feedback-controlled pulse shaping in a rare-earth doped solid, to store and process optically-encoded information in a doped solid.

\vspace{\baselineskip} 