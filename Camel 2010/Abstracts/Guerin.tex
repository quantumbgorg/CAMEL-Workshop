\title{CONTROL BY FAST PARALLEL ADIABATIC PASSAGE. APPLICATION FOR QUANTUM MACHINES}

\underline{S. Gu\'erin} \index{Gu\'erin S}

{\normalsize{\vspace{-4mm}
Institut Carnot de Bourgogne, Universite de Bourgogne, 9, Av A.
Savary, 21078 Dijon, France

\email sguerin@u-bourgogne.fr}}

We present the technique of fast parallel adiabatic passage and its application to the control of quantum processes such as state-selectivity, quantum information processing, and to the manipulation of the rotation of molecules to implement quantum machines.
This technique corresponds to an adiabatic passage for which the instantaneous eigenvalues are parallel at each time. It induces a very efficient population transfer in terms of pulse energy and duration, while preserving the standard robustness of adiabatic techniques.
It can be implemented with the modern technology of liquid crystal spatial light modulators.
This opens the possibility to implement a STIRAP-like technique in a sub-picosecond regime.
It can be applied to the control of the alignment and the rotation of molecules in order to produce quantum machines such as a quantum gear.

\begin{enumerate}
\item G. Dridi S. Guerin, V. Hakobyan, H.R. Jauslin, and H. Eleuch, Phys. Rev. A \textbf{80}, 043408 (2009).
\end{enumerate}

\vspace{\baselineskip} 