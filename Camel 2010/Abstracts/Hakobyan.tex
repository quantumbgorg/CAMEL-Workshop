\title{MULTIPHOTON PARALLEL ADIABATIC PASSAGE BY SHAPED PULSES}

\underline{V. Hakobyan} \index{Hakobyan V}

{\normalsize{\vspace{-4mm}
Institut Carnot de Bourgogne, Universite de Bourgogne, 9, Av A.
Savary, 21078 Dijon, France

\email vahe.hakobyan@u-bourgogne.fr}}

The strategy to optimally populate the excited state from the initial ground state by adiabatic
passage is to follow a level line in the diagram of the difference of the instantaneous eigenenergies.
This corresponds to a dynamics featuring parallel instantaneous eigenenergies at all times. This
strategy is referred to as an optimal adiabatic passage as it allows the nonadiabatic correction to be
suppressed in the adiabatic limit, as shown with the Davis-Dykhne-Pechukas formalism.
We study a multiphoton process between an initially populated ground state  and an excited states for
the atom of Cs  driven by a two-photon process such that the Stark shift can be fully compensated to
recover the efficiency of the population transfer by a one-photon resonant process. This can be
produced using the technique of pulse shaping in the frequency domain.

\vspace{\baselineskip} 