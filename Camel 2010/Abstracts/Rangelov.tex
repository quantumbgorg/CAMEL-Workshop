\title{STIMULATED RAMAN ADIABATIC PASSAGE IN POLARIZATION PHYSICS; BROADBAND ADIABATIC CONVERSION OF LIGHT POLARIZATION}

\underline{A. A. Rangelov}$^{1}$, U. Gaubatz$^{2}$, N. V.
Vitanov$^{1}$  \index{Rangelov A}\index{Gaubatz U}\index{Vitanov N}

{\normalsize{

\vspace{-4mm} $^1$\unisofia

\vspace{-4mm} $^2$Nokia Siemens Networks GmbH \& Co. KG,
St.-Martin-Strasse 76, 81541 Munich, Germany

\email rangelov@phys.uni-sofia.bg}}

A broadband technique for robust adiabatic rotation and conversion
of light polarization is proposed. It uses the analogy between the
equation describing the polarization state of light propagating
through an optically anisotropic medium and the Schr\"{o}dinger
equation describing coherent laser excitation of a three-state atom.
The proposed technique is analogous to the stimulated Raman
adiabatic passage (STIRAP) technique in quantum optics;
 it is applicable to a wide range of frequencies and it is robust to variations in the propagation length and the rotary power.


\begin{enumerate}
\item U. Gaubatz, P. Rudecki, S. Schiemann, and K. Bergmann, J. Chem. Phys. \textbf{92}, 5363 (1990).
\item A. A. Rangelov, N. V. Vitanov, and B. W. Shore, J. Phys. B \textbf{42}, 055504 (2009).
\item A. A. Rangelov, U. Gaubatz, and N. V. Vitanov, submitted to Opt. Comm.
\end{enumerate}

\vspace{\baselineskip}
