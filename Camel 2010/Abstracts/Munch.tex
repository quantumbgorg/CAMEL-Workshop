\title{COHERENT CONTROL OF FREQUENCY CONVERSION TOWARDS SHORT (PICOSECOND) VACUUM-ULTRAVIOLET RADIATION PULSES}

\underline{H. M\"{u}nch}, S. Chakrabarti, T. Halfmann \index{M\"{u}nch H} \index{Chakrabarti S} \index{Halfmann T}

{\normalsize{\vspace{-4mm}
Institute of Applied Physics, Technical University of Darmstadt,
64289 Darmstadt, Germany

\email holger.muench@physik.tu-darmstadt.de}}

We present experimental data on coherent control of frequency conversion towards the vacuum-ultraviolet (VUV) regime in xenon atoms, applying intense ultra-short picoseconds (ps) radiation pulses. The coupling scheme involves resonantly-enhanced fifth harmonic generation (FHG) via the 5p $^1S_0 � 8d^2[1/2]_1$ transition in xenon, driven by a short (ps) radiation pulse at 530 nm, yielding VUV radiation at 106 nm. A simultaneous resonantly-enhanced four-wave mixing (FWM) process, driven by the fundamental (ps) laser pulse at 530 nm and it�s phase-locked second harmonic at 265 nm, also yields radiation at 106 nm. The two frequency conversion pathways interfere with each other. The interference is either constructive or destructive, depending on the relative phase of the radiation pulses. We adjust the phase of the fundamental (ps) laser pulse, subsequent to the frequency doubling, by passing it through a closed cell filled with argon. Therefore, by variation of the argon pressure and thus the phase in the fundamental (ps) laser pulse, we observe pronounced enhancement and suppression of the conversion efficiency towards VUV radiation.

\vspace{\baselineskip} 