\title{VIBRATIONAL QUANTUM DEFECT TO ANALYZE WEAKLY-BOUND MOLECULE SPECTRA.
APPLICATION TO CS$_2$ TO DETECT COUPLED STATES}

\underline{L. Pruvost}, H. Jelassi \index{Pruvost L} \index{Jelassi H}

{\normalsize{
\vspace{-4mm}$^1$Laboratoire Aim\'e Cotton, CNRS, Universit\'e Paris-Sud, B\^atiment 505, 91405 Orsay, France

\vspace{-4mm}$^2$Centre National des Sciences et Technologies Nucl\'eaires 2020 Sidi Thabet, Tunisie

\email laurence.pruvost@lac.u-psud.fr}}

To form cold molecules starting from cold atoms, many processes have been
demonstrated: for example photoassociation (PA) or magneto-association. In the case
of the PA use, the scheme which ends to a molecule in the ground state is generally composed
by many steps. The simplest scheme uses two steps: (1) the PA of a pair of cold atoms to an
excited molecular level, and (2) a natural or induced decay of the excited molecule to a
ground state.
The processes (1) and (2) are optimized when the intermediate state (the excited molecule)
has a wavefunction with two regions of large probability: one at long range distance which
allows the PA process and another one at short range distance which enhances the decay into
a vibrational ground state. Such wavefunctions occasionally exist when the intermediate state
results of the coupling between two molecular curves, in the vicinity of quasi resonant levels.
To detect such intermediate states, a careful analysis of the spectroscopic data is required.
In this context, we have applied the method of the vibrational quantum defect [1] to analyse
spectroscopic data of weakly-bound molecules. The method joins the LeRoy-Bernstein
formula and the quantum defect theory. A plot of the quantum defect versus the energy allows
to exhibit the coupled states, and its fit by a model of coupled series allows to determine the
amplitude of the coupling, the location of the levels and the mixing of the wavefunctions.
We illustrate the method with the analysis of $\textrm{Cs}_2$ data [2] of weakly bound levels of the
$6\textrm{s}_{1/2}-6\textrm{p}_{1/2} 0_\textrm{u}^{~+}$ and $0_\textrm{g}^{~-}$ series [3,4]. We show and characterize many coupling regions which
are interested for cold molecule formation.

\begin{enumerate}
\item H. Jelassi, B. Viaris de Lesegno, and L. Pruvost, Phys. Rev. A \textbf{73}, 032501 (2006).
\item M. Pichler, H. Chen and W. C. Stwalley, J. Chem. Phys. \textbf{121}, 1796 (2004).
\item H. Jelassi, B. Viaris de Lesegno, and L. Pruvost, M. Pichler, W. C. Stwalley, Phys. Rev. A \textbf{78}, 022503(2008).
\item L. Pruvost and H. Jelassi, J. Phys. B: At. Mol. Opt. Phys. \textbf{43}, 125301 (2010).
\end{enumerate}

\vspace{\baselineskip} 