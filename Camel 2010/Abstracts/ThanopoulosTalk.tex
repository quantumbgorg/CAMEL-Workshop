\title{INTRAMOLECULAR ENERGY TRANSFER IN PYRAZINE BY PARTITIONING TECHNIQUE: A TIME-DEPENDENT PERSPECTIVE}

\underline{I. Thanopoulos}, P. Brumer$^2$ , and M. Shapiro$^3$ \index{Thanopoulos I}\index{Brumer P}\index{Shapiro M}

{\normalsize{
\vspace{-4mm}$^1$Theoretical and Physical Chemistry Institute, National Hellenic Research Foundation, Athens 11635, Greece

\vspace{-4mm}$^2$Chemical Physics Theory Group, Department of Chemistry, and Center of Quantum Information and Quantum Control,
University of Toronto, Toronto, Ontario M5S 3H6, Canada

\vspace{-4mm}$^3$Departments of Chemistry and Physics, The University of British Columbia, Vancouver V6T1Z1, Canada and Department of Chemical Physics, The Weizmann Institute, Rehovot 76100, Israel

\email ithano@eie.gr}}

We investigate the intramolecular energy transfer dynamics of the S$_2$ excited electronic state of
pyrazine due to radiationless transitions to energetically lower-lying singlet electronic states. The
femtosecond decay of S$_2$ to the S$_1$ excited state and the picosecond decay of S$_2$ to the ground
electronic state S$_0$ are studied within an efficient methodology for computing the intramolecular
energy transfer dynamics in multi-dimensional configurational spaces. Our method is based on
partitioning the full configuration space into the (small) subspace of interest $\mathcal{Q}$, and the rest, the
subspace $\mathcal{P}$. The exact equations of motion for the states in $\mathcal{Q}$, under the influence of $\mathcal{P}$, are
derived in the time-domain in form of a system of integro-differential equations. Their numerical
solution is readily obtained when the $\mathcal{Q}$ space is consisting of just a few states. Otherwise, the
integro-differential equations for the states in $\mathcal{Q}$ are transformed into a (larger) system of ordinary
differential equations, which can be solved by a single diagonalization of a general complex matrix.
The former approach is applied to the investigation of the picosecond S$_2$ $\longrightarrow$ S$_0$ energy transfer
dynamics and the latter is applied to the study of the ultrafast S$_2$ $\longrightarrow$ S$_1$ decay dynamics.


\vspace{\baselineskip} 