\title{ATOM-PHOTON INTERFACES FOR QUANTUM NETWORKING}

\underline{M. Himsworth} \index{Himsworth M}

{\normalsize{\vspace{-4mm}
Atomic and Laser Physics, Clarendon Laboratory, University of Oxford, OX1 3PU Oxford, United Kingdom

\email m.himsworth1@physics.ox.ac.uk}}

Atoms coupled to high finesse cavities are seen as a promising candidate for quantum network nodes. A stream of identical photons can be emitted from the cavity, using vacuum-STIRAP, with a high probability which are necessary for several quantum networking protocols. The photons can be controlled in both polarization and spatio-temporal profile and may be entangled with the cavity-atom or in photon pairs.
An efficient scalable quantum network would require additional elements such as memory and routing, and these may be realized with light storage via electromagnetically induced transparency and controllable dipole traps, respectively. This talk with cover the theoretical and experimental progress in addressing each of the above elements at the University of Oxford.

\vspace{\baselineskip} 