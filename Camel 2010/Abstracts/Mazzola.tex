\title{FROZEN DISCORD AND DYNAMICS OF CORRELATIONS IN NON-MARKOVIAN CHANNELS}

\underline{L. Mazzola} \index{Mazzola L}

{\normalsize{\vspace{-4mm}
Linnankatu 37 b 35, 20100, Department of Physics and Astronomy, University of Turku, Finland

\email laumaz@utu.fi}}

In the last couple of years a big deal of attention has been devoted to the study of quantum and classical correlations in quantum mechanical systems. After the realization that there can be nonclassical correlations which are not necessarily captured by entanglement, we have witnessed a flourishing of works aiming at defining, quantifying and studying the dynamics of these properties in different kinds of systems.
Recently we have discovered that there exist physical systems in which quantum correlations described by quantum discord can be unaffected by environmental noise for a long period of time. In particular in Ref. [1] we have shown that two qubits under local depolarizing channels can exhibit what we have called a transition between classical and quantum decoherence.
Here we study the dynamics of quantum and classical correlations and spot the transition between quantum and classical decoherence for a more general type of noise. We use a well established model describing the dynamics of a two level system under depolarizing channel in which white noise is replaced by colored noise. The memory effects and the backflow of information from the environment to the system blur the simple picture of the white noise case. However it is still possible to distinguish different dynamical regimes for quantum and classical correlations. We also present a geometrical interpretation of invariant-under decoherence discord in terms of the distance to the closest classical state.

\begin{enumerate}
\item L. Mazzola, J. Piilo, and S. Maniscalco, arXiv:1001.5441v3, accepted by Phys. Rev. Lett.
\end{enumerate}

\vspace{\baselineskip} 