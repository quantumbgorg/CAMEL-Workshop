\title{SMOOTH COMPOSITE PULSES IN COHERENT ATOMIC EXCITATION}

\underline{B. T. Torosov}$^{1,2}$, N. V. Vitanov$^{1}$  \index{Torosov B}\index{Vitanov N}

{\normalsize{

\vspace{-4mm} $^1$\unisofia

\vspace{-4mm} $^2$ICB, Universit\'{e} de Bourgogne, UMR CNRS 5209, Dijon, France

\email torosov@phys.uni-sofia.bg}}

Excitation by composite pulses is routinely used in nuclear-magnetic-resonance experiments, in order to achieve excitation profiles with prescribed shapes [1,2]. The method consists of applying several consecutive pulses with appropriate relative phases. In such way one can achieve a robust analog of the traditional resonance pulses. This technique is, however, mainly developed for pulses with rectangular temporal shape.

In this work we show that composite pulses with smooth temporal shape can be used to obtain analogues of the $\pi$-pulses with arbitrarily flat excitation profile. The transition probability can be made robust against variations in the pulse area and/or the detuning. As the number of pulses increases, the excitation profile becomes increasingly insensitive (flat) to small deviations. In order to achieve this, we use the well-known analytic solution of the Rosen-Zener model [3], which assumes a hyperbolic-secant pulse shape and a constant detuning:
\begin{equation}
\Omega(t)=\Omega_0 \textrm{sech} (t/T),\quad \Delta(t)=\Delta_0,
\end{equation}
where $\Omega(t)$ is the Rabi frequency, and $\Omega_0$, $\Delta_0$ and $T$ are constant parameters. We calculate the total propagator by multiplying the phased propagators for each pulse. Then we take the Taylor expansion of the full propagator and nullify the first few terms in the respective series (vs. the pulse area deviation or vs. the detuning). The more composite pulses we use, the more terms we are able to nullify, and the more flat the profile will be.

\begin{enumerate}
\item M.H. Levitt, Prog. NMR Spectrosc. \textbf{18}, 61 (1986).
\item S. Wimperis, J. Magn. Reson. \textbf{109}, 221 (1994).
\item N. Rosen and C. Zener, Phys. Rev. \textbf{40}, 502 (1932).
\end{enumerate}

\vspace{\baselineskip}
