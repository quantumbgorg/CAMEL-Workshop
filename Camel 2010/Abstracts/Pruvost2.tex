\title{COLD ATOM GUIDED IN A SLM-GENERATED LAGUERRE-GAUSSIAN LASER MODE}

B. Viaris de Lesegno, F. Diry, M. Mestre, \underline{L. Pruvost}
\index{Viaris de Lesegno B}\index{Diry F}\index{Mestre M}\index{Pruvost L}

{\normalsize{
\vspace{-4mm}Laboratoire Aim\'e Cotton, CNRS, Universit\'e Paris-Sud, B\^atiment 505, 91405 Orsay, France

\email laurence.pruvost@lac.u-psud.fr}}

In the context of cold atom manipulation or trapping, optical potentials realized with a laser
far-detuned from the atomic transition, constitute efficient tools to perform experiments
without destroying the initial properties of the cold sample. For example, the use of very far
detuned optical potentials avoids sample heating and preserves the coherence properties in the
case of Bose-Einstein condensate use.
The shape of the optical potential is determined by the intensity profile of the laser beam.
Shaping the laser thus would allow to design optical potentials with particular geometries or
with shapes. To shape the laser, we apply a holographic method in which the hologram is a
phase-only spatial light modulator (SLM). We first used such a device to transform a TEM$_{00}$
laser mode into a Laguerre-Gaussian (LG) mode. Then, we generated LG modes deformed
into polygons or opened rings.
The LG modes were applied to rubidium cold atoms to guide them in the dark center of the
mode. We have performed a quantitative study [1] of the guiding efficiency versus the order
of the LG mode and versus the laser detuning. We explained the increase of the efficiency
versus the mode order using a two-dimensional trap model giving the capture efficiency.
Other applications of LG modes and applications of deformed LG modes are conceivable. For
example, crossed LG modes could provide 3-dimensional configurations allowing power-law
potentials for Bose-Einstein condensation [2]. Opened, and deformed LG modes could design
trap geometry for study atom dynamics -regular or chaotic.

\begin{enumerate}
\item M. Mestre, F. Diry, B. Viaris de Lesegno and L. Pruvost, Eur. Phys. J. D \textbf{57}, p 87, 2010.
\item A. Jaouadi, N. Gaaloul, B. Viaris de Lesegno, M. Telmini, L. Pruvost, and E. Charron, submitted to PRA.
\end{enumerate}

\vspace{\baselineskip} 