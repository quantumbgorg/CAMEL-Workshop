\title{TRANSITION TIME IN STIMULATED RAMAN ADIABATIC PASSAGE}

\underline{I. I. Boradjiev} and N. V. Vitanov
\index{Boradjiev I} \index{Vitanov N}

{\normalsize{\vspace{-4mm}
\unisofia

\email boradjiev@phys.uni-sofia.bg}}

The technique of stimulated Raman adiabatic passage (STIRAP) is an efficient and robust technique for population transfer in three-state chainwise connected quantum systems, 1-2-3.
In this work we analyze the transition time in STIRAP, that is the time it takes for the population to be transferred from state 1 to state 3.
We derive a very simple formula for this transition time, which shows that it is inversely proportional to
the nonadiabatic coupling, $\dot\theta(t_0)$, estimated at the time $t_0$ when the pump and Stokes couplings have the same value.
Explicit formulas for several pairs of pump and Stokes pulses are derived.
Contrary to the naive expectation that the transition time might be proportional to the delay between the pump and Stokes pulses,
or to the width of their overlap region, we find that the transition time is inversely proportional to the delay,
i.e. larger delays produce faster transitions.

\vspace{\baselineskip} 