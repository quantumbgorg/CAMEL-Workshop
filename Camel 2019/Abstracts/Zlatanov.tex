\title{GENERATION OF ARBITRARY QUANTUM SUPERPOSITION STATES BY ADIABATIC EVOLUTION SPLIT BY A PHASE JUMP}
% Generation of arbitrary quantum superposition states by adiabatic evolution split by a phase jump

\underline{K. Zlatanov} \index{Zlatanov K.} 
%Kaloyan Zlatanov

{\normalsize{\vspace{-4mm}
Department of Physics, Sofia University, 5 James Bourchier Blvd, Sofia, Bulgaria



\email k.zlatanoff@gmail.com}}

Coherent superposition states have a key position in contemporary quantum physics. Due to their various 
application and their intriguing nature a wide range of techniques has been developed for their generation.
Adiabatic excitation has proven quite robust to pulse shape errors, with a clear trade-off for large pulse areas. However adiabatic evolution is a reliable choice in a 
variety of protocols aiming in the generation of coherent superposition states. 

We report on a new technique for generation of coherent superposition states based on sequence of two
adiabatic pulses divided by a phase jump. Phase jumps in the field of the pulse can have a dramatic
influence over the evolution of the system and have proven to fit well in robust coherent control techniques such
as composite pulses. This gives them a prominent role,
as control parameter in superposition generation schemes. We based our study on repeated pulses, i.e pulses which differ up to a change in the signs of the Rabi frequency and/or the detuning. We show that excitation with repeated
pulses can generate two different robust pulse sequences, that shift the control of the target superposition state
entirely to the phase jump. In the first case we consider pulse sequence in which the phase jump dictates the
design of the second pulse and also plays the role of control parameter that regulates the target superposition
state. In the second case the design is dictated by the chirping of the pulse, while the final position of the Bloch
vector is controlled by the phase jump. In both cases adiabatic excitation is a crucial requirement for the error
minimization in the transition probability of the pulse sequences.

\vspace{\baselineskip}