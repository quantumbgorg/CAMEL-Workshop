\title{ION-NODES FOR QUANTUM NETWORKS}
% Ion-Nodes for Quantum Networks

\underline{M. Keller} \index{Keller M.} 
%Matthias Keller

{\normalsize{\vspace{-4mm}
Pevensey 2,
University of Sussex,
Brighton BN1 9QH,
UK



\email m.k.keller@sussex.ac.uk}}

The complementary benefits of trapped ions and photons as carriers of quantum information make it appealing to combine them in a joint system. Ions provide low decoherence rates, long storage times and high readout efficiency, while photons are ideal candidates for the transmission of quantum states over long distances. To interface the quantum states of ions and photons efficiently, we use calcium ions coupled to an optical high-finesse cavity via a Raman transition.
To achieve strong ion-cavity coupling we employ fibre tip cavities integrated into the electrodes of an endcap style ion trap. With a cavity length of 380 mm the resulting ion-cavity coupling strength is 17 MHz with a cavity line width of 8 MHz. We trap single calcium ions with a life time of several hours and have optimised the ion-cavity overlap to observe the interaction of the cavity with the ion.
While fibre cavities are ideal tools for ion-photon interfaces the limited coupling between the cavity mode and the fibre mode poses severe limitations on their usability in efficient quantum interfaces. We have developed a system to integrate mode matching optics into a fibre system and have demonstrated a mode matching between cavity and fibre on the order of 90\%.
Crucial for quantum networks is the indistinguishability of the photons that are generated by the ion-cavity system. We have investigated two scheme to generate single photons and compared their single photon efficiency and the indistinguishability of the photons using HOM interference.


\vspace{\baselineskip}
