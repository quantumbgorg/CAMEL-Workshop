\title{STRONG INTERACTION BETWEEN RYDBERG IONS}
% Strong interaction between Rydberg ions

\underline{M. Hennrich} \index{Hennrich M.} 
%Markus Hennrich

{\normalsize{\vspace{-4mm}
Stockholm University,
Albanova University Center,
Roslagstullsbacken 21,
10691 Stockholm,
Sweden



\email markus.hennrich@fysik.su.se}}

Trapped Rydberg ions are a novel quantum system, which combine the strong dipolar interactions of Rydberg atoms and the precise quantum control of trapped ions [1-2]. For trapped ions, this technology promises to speed up entanglement operations and make them available in larger ion crystals.
In this presentation, we will report on our recent experimental results using trapped 88Sr+ Rydberg ions.  This includes modifications in the trapping potential due to the large polarizability of trapped Rydberg ions [3], and the observation of a strong Rydberg interaction between two microwave-dressed Rydberg ions. These are important steps towards realizing entanglement operations and quantum gates with trapped Rydberg ions.
The achieved quantum control and strong interactions between trapped Rydberg ions open new pathways for future application in quantum computation, simulation and sensing.      

{\normalsize
[1] M. M\"uller, et al., New J. Phys. 10, 093009 (2008).
\vsp

[2] F. Schmidt-Kaler, et al., New J. Phys. 13, 075014 (2011).
\vsp

[3] G. Higgins, et al., arXiv:1904.08099.
}

\vspace{\baselineskip}
