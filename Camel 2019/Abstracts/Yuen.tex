\title{QUANTUM MECHANICS IN TECHNICOLOR}
% Quantum Mechanics in Technicolor

\underline{B. Yuen} \index{Yuen B.} 
%Ben Yuen

{\normalsize{\vspace{-4mm}
Clarendon Laboratory,
Parks rd,
Oxford,
OX1 3PU,
United Kingdom



\email benjamin.yuen@physics.ox.ac.uk}}

A dipole oscillating in an electromagnetic field is ubiquitous in modern physics. Physically and theoretically this system is beautifully simple. Despite this simplicity, it is striking that we are yet to discover an exact analytic solution to Rabi’s model which describes it. Only its approximated form - the Jaynes-Cummings model - has been solved exactly. Furthermore, we must ask what happens beyond the very simplest of situations to understand the real world.  The world is full of colour. What happens when the light with which atoms interact is polychromatic?

I will present a promising new method for tackling the polychromatic Jaynes-Cummings and Rabi models — where a dipole is driven by a quantised field with multiple frequency modes [1]. This method can in principle be used to find solutions to an arbitrary degree of accuracy. I will indicate how considering the single frequency Rabi model within a multi-frequency formalism allows progress to be made on finding an exact closed form analytic solution to this 80-year old problem. This work was motivated by recent experiments to trap ultra-cold atoms with multi-radio frequency fields [2, 3], but its applicability extends far beyond this to a wide range of topics in quantum and nonlinear optics. In this context I explore some simple approximate expression for the time evolution of the atom and the polychromatic field -- valid even when the interaction is strong.

{\normalsize
[1] Yuen, B., arXiv preprint arXiv:1805.05922 (2018).
\vsp

[2] Harte, T. L., Bentine, E., Luksch, K., Barker, A. J., Trypogeorgos, D., Yuen, B., Foot, C. J. Physical Review A, 97(1), 013616 (2018).
\vsp

[3] Luksch, K., Bentine, E., Barker, A. J., Sunami, S., Harte, T. L., Yuen, B., Foot, C. J. arXiv preprint arXiv:1812.05545 (2018).
}

\vspace{\baselineskip}