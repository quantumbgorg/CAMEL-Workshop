\title{PROGRESS IN DEVELOPING A MODULAR MICROWAVE TRAPPED ION QUANTUM COMPUTER}
% Progress in developing a modular microwave trapped ion quantum computer

\underline{W. Hensinger} \index{Hensinger W.} 
%Winfried Hensinger

{\normalsize{\vspace{-4mm}
Sussex Centre for Quantum Technologies, Department of Physics and Astronomy, University of Sussex, Brighton BN1 9QH, United Kingdom



\email w.k.hensinger@sussex.ac.uk}}

Trapped ions are arguably the most mature technology capable of constructing practical large scale quantum computers. We are now moving away from fundamental physics studies towards tackling the required engineering tasks in order build such machines. 

By inventing a new method where voltages applied to a quantum computer microchip are used to implement entanglement operations, we have managed to remove one of the biggest barriers traditionally faced to build a large-scale quantum computer using trapped ions, namely having to precisely align billions of lasers to execute quantum gate operations. This new approach, quantum computing with global radiation fields, is based on the use of well-developed microwave technology [1].

In order to be able to build large scale device, a quantum computer needs to be modular. One approach features modules that are connected via photonic interconnect, however, only very small connection speeds between modules demonstrated have been demonstrated so far.  We have invented an alternative method where modules are connected via electric fields, allowing ions to be transported from one module to another giving rise to much faster connection speeds [2].

Incorporating these two inventions, we recently unveiled the first industrial blueprint on how to build a large-scale quantum computer which I will discuss in this talk [2]. I will show progress in constructing a quantum computer prototype at the University of Sussex featuring this technology and I will discuss a new method we have demonstrated recently in order to make quantum gates with trapped ions more resilient to sources of decoherence such as motional heating, stray magnetic fields and noise in electrical components [3].

{\normalsize
[1] S. Weidt, J. Randall, S. C. Webster, K. Lake, A. E. Webb, I. Cohen, T. Navickas, B. Lekitsch, A. Retzker, and W. K. Hensinger, Phys. Rev. Lett. 117, 220501 (2016).
\vsp
 
[2] B. Lekitsch, S. Weidt, A.G. Fowler, K. Mølmer, S.J. Devitt, Ch. Wunderlich, and W.K. Hensinger, Science Advances 3, e1601540 (2017).
\vsp

[3] A. E. Webb, S. C. Webster, S. Collingbourne, D. Bretaud, A. M. Lawrence, S. Weidt, F. Mintert and W. K. Hensinger, Phys. Rev. Lett. 121, 180501 (2018).
}

\vspace{\baselineskip}