\title{TOWARDS OPTICAL TRAPPING OF YTTERBIUM RYDBERG ATOMS}
% Towards optical trapping of Ytterbium Rydberg atoms

\underline{S. Lepoutre} \index{Lepoutre S.} 
%Steven Lepoutre

{\normalsize{\vspace{-4mm}
Laboratoire Aim\'e Cotton, CNRS, Universit\'e Paris-Saclay, B\^at. 505, 91405 Orsay, France



\email steven.lepoutre@u-psud.fr}}

Thanks to their strong dipolar interactions, Rydberg atoms emerge as a powerful tool for the development of quantum simulation, as already demonstrated by pioneering experiments using alkaline atoms trapped in their ground state in dipolar traps [1] or optical lattices [2]. Several teams have recently turned to alkaline-earth species in order to investigate many-body physics with two-electron atoms [3]. Indeed, the presence of a second electron enables optical manipulation of the atom while the other electron is in a Rydberg state.

Here, we study experimentally the light shift of a 6snl Rydberg state of ytterbium coupled to 6pn’l states using 369 nm laser light close to the ionic core resonance. Strong auto-ionization is generally observed because 6pn’l states are above the ionization limit. Nevertheless, the auto-ionization spectra presents modulations due to interferences between the different n’ levels addressed, resulting in the existence of frequencies for which auto-ionization cancels [4]. We have shown that a light shift of the Rydberg states exists in the absence of auto-ionization. These results pave the way towards quantum simulation with alkaline-earth atoms trapped in their Rydberg state.

{\normalsize
[1] V. Lienhard, et al., Phys. Rev. X 8, 021070 (2018).
\vsp

[2] J. Zeiher, et al., Nat. Phys. 12, 1095 (2016).
\vsp

[3] R. Mukherjee, et al., J. Phys. B 44, 184010 (2011).
\vsp

[4] N. H. Tran, et al., Phys. Rev. A 26, 3016(R) (1982).
}


\vspace{\baselineskip}