\title{NON RECIPROCAL WAVE PLATE}
% Non reciprocal wave plate

\underline{M. Al-Mahmoud} \index{Al-Mahmoud M.} 
%Mouhamad AL-MAHMOUD

{\normalsize{\vspace{-4mm}
5 James Bourchier Blvd, 
1164 Sofia, 
Bulgaria



\email mouhamadmahmoud1@gmail.com}}

In polarization optics there are different polarization instruments such as polarizers, rotators and wave plates [1-4]. The rotator can be reciprocal (a sequence of half wave plates), or non-reciprocal (Faraday rotator). Then by using a reciprocal rotator and a Faraday rotator, it produces a non-reciprocal rotator, so the rotation angles are added in forward direction and they are subtracted in the backward direction.
Finally, by putting them between two quarter wave plates, the sequence acts like a non-reciprocal retarder. By choosing the rotation angles of the rotators, the retardations will be tunable. %A particular couple of angles give  retardation in forward direction and /2 retardation in the backward direction. 
In this talk, we will present the proof of principal of this non-reciprocal wave plate.

{\normalsize
[1] E. Hecht, Optics, 4th ed. (Addison Wesley, 2002).
\vsp

[2] M. Born and E. Wolf, Principles of Optics, (Pergamon, 1975).
\vsp

[3] M. A. Azzam and N. M. Bashara, Ellipsometry and Polarized Light, (North Holland, 1977).
\vsp

[4] D. Goldstein and E. Collett, Polarized Light, (CRC, 2003).
}

\vspace{\baselineskip}