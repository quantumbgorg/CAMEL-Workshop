\title{NEW CAVITY-BASED PHOTON GENERATION SCHEMES}
% New Cavity-Based Photon Generation Schemes

\underline{J. R. Alvarez Velasquez} \index{Alvarez Velasquez J. R.} 
%Juan Rafael Alvarez Velasquez

{\normalsize{\vspace{-4mm}
Clarendon Laboratory, Parks Road, Oxford, OX1 3PU, United Kingdom



\email juan.alvarezvelasquez@physics.ox.ac.uk}}

The strong coupling of an atom to a cavity offers unparalleled control on the generation of single photons with tunable properties such as their temporal distribution. Such atom-cavity systems, based on vacuum stimulated Raman processes (V-STIRAP) can provide a priori deterministic photon sources with arbitrary control of the photon wave-packet, being able to produce long photons which are orders of magnitude larger than the single photon detector resolutions.

Attempts at producing polarized single photons with these schemes have alternated the driving of magnetic sub-levels of atoms. However, due to nonlinear Zeeman effects, these schemes show low efficiencies when attempting to produce more than two photons at a time. We propose a method to generate polarized photons by re-preparing the original magnetic state of a Rb 87 atom using two alternating STIRAP beams.

\vspace{\baselineskip}
