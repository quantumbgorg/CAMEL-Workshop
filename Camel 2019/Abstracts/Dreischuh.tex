\title{FAR-FIELD BEAM SHAPING BY SINGULAR OPTICAL LATTICES}
% Far-Field Beam Shaping By Singular Optical Lattices

\underline{A. Dreischuh} \index{Dreischuh A.} 
%Alexander Dreischuh

{\normalsize{\vspace{-4mm}
Department of Quantum Electronics,
Faculty of Physics,
Sofia University,
5 J Bourchier Blvd, Sofia, Bulgaria



\email ald@phys.uni-sofia.bg}}

Due to the point phase dislocations in their helical wavefronts, optical vortices (OVs) are the only known truly two-dimensional singular beams. The angular momentum they carry is proportional to their topological charge -- a positive or a negative integer number, corresponding to the total phase change over the azimuthal coordinate.
In this talk, we will present evidences for the creation of controllable multi-spot focal arrays composed of bright beams with flat phase profiles. The input phase structures sent to spatial light modulators are square-shaped and hexagonal OV lattices of different periods, containing hundreds of OVs. In order to stabilize each of these lattices in space, all used OVs are singly charged and their signs vary alternatively across the structures. It is proven that the node spacings of the lattices can be used as control parameters for reshaping the multi-spot focal arrays generated. Each peak of these arrays is shown to be able to additionally host a singular beam. Method for generating long range nondiffracting Gauss-Bessel beams by annihilating multiple-charged OVs will be discussed.

\vspace{\baselineskip}