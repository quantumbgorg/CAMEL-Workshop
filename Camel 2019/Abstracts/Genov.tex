\title{MIXED DYNAMICAL DECOUPLING}
% Mixed Dynamical Decoupling

\underline{G. Genov} \index{Genov G.} 
%Genko Genov

{\normalsize{\vspace{-4mm}
Institute for Quantum Optics, Ulm University, Albert-Einstein-Allee 11, Ulm 89081, Germany



\email genko.genov@uni-ulm.de}}

Developments in quantum technologies are increasingly important nowadays for various applications in sensing, transmission and processing of information. However, protection of quantum systems from unwanted interactions with the environment remains a challenge. Continuous dynamical decoupling (DD), where the system is driven with a protecting dressing field for the entire duration of the experiment has already been demonstrated to compensate for noise sources in various media, e.g., in color centers in diamond and trapped ions. Dynamical decoupling by sequences of time-separated pulses (pulsed DD) is another widely used approach for compensation of environmental noise.

In this work we propose a scheme for mixed dynamical decoupling, where we combine continuous and pulsed DD [1]. Specifically, we use two fields for decoupling, where the first continuous driving field creates dressed states that are robust to environmental noise. Then, a second field implements a robust sequence of phased pulses to perform inversions of the dressed qubits, thus achieving robustness to amplitude fluctuations of both fields. We show that mixed DD outperforms standard concatenated continuous dynamical decoupling in realistic numerical simulations for dynamical decoupling in NV centers in diamond. Finally, we also demonstrate how our technique can be utilized for improved sensing.

{\normalsize
[1] Genko T. Genov, Nati Aharon, Fedor Jelezko, and Alex Retzker, arXiv:1903.0074.
}

\vspace{\baselineskip}