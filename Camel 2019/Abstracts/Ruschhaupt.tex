\title{NOISE-STABILISED PARTICLE TRANSPORT VIA SHORTCUTS TO ADIABATICITY AND ASYMMETRIC SCATTERING BY NON-HERMITIAN POTENTIALS}
% Noise-stabilised particle transport via shortcuts to adiabaticity and asymmetric scattering by non-Hermitian potentials

\underline{A. Ruschhaupt} \index{Ruschhaupt A.} 
%Andreas Ruschhaupt

{\normalsize{\vspace{-4mm}
Department of Physics,
University College Cork,
Cork,
Ireland



\email aruschhaupt@ucc.ie}}

In the first part of the talk, we present new applications of quantum control via shortcuts to adiabaticity. Following [1,2], we  investigate the transport of a quantum particle in a harmonic trap with the spring constant perturbed by weak coloured noise. We find expressions for the final excitation energy in terms of static (independent of trap motion) and dynamical sensitivities and demonstrate that the excitation can be reduced by using the techniques of shortcuts to adiabaticity.
In the second part, we consider asymmetric scattering by non-Hermitian (generally non-local) potentials in one dimension [3]. We show that these non-Hermitian, non-local potentials allow the design of basic devices which cannot be designed with ``standard'' local, Hermitian potentials. These basic devices can be variously described pictorically as a one-way mirror, a one-way barrier (a Maxwell pressure demon), a one-way transmission filter, a one-way reflection filter, a mirror with unidirectional transmission, and a transparent, one-way reflector. We finish with a discussion of possible physical realisations of these non-Hermitian, non-local potentials.

{\normalsize
[1] X.-J. Lu, A. Ruschhaupt and J. G. Muga, Phys. Rev. A 97, 053402 (2018).
\vsp

[2] X.-J. Lu, J. G. Muga, X. Chen, U. G. Poschinger, F. Schmidt-Kaler and A. Ruschhaupt, Phys. Rev. A 89, 063414 (2014).
\vsp

[3] A. Ruschhaupt, T. Dowdall, M.A. Simon and J. G. Muga, EPL 120, 20001 (2017).
}

\vspace{\baselineskip}