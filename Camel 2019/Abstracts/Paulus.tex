\title{THE HEH ION IN STRONG LASER FIELDS}
% The HeH ion in strong laser fields

\underline{G. Paulus} \index{Paulus G.} 
%Gerhard Paulus

{\normalsize{\vspace{-4mm}
Friedrich Schiller University,
Institute of Optics and Quantum Electronics,
Max-Wien-Platz 1,
07743 Jena,
Germany



\email gerhard.paulus@uni-jena.de}}

Since its first experimental observation in 1925, the HeH$^+$ molecular ion, the simplest polar heteronuclear molecule, has served as a fundamental benchmark system for understanding principles of molecular formation and electron correlation. Moreover, HeH$^+$ continues to intrigue researchers as the first molecular species to arise in the universe, but has been discovered in astronomical spectra only this year. Despite of its fundamental nature and broad-ranging importance, the behavior of HeH$^+$ in strong fields is largely unexplored, until now.

So far, the molecular hydrogen ion, the simplest molecule, has typically been used as the prototype for the dynamics of molecules in strong laser fields. This understanding is then extended to interpret and predict the behavior of more and more complex molecules. However, the dynamics of this and other homonuclear molecules lack fundamental properties as they are symmetric and do not have a permanent dipole moment. In stark contrast, HeH$^+$ is a two-electron system with a large mass asymmetry, a strong electronic asymmetry, and a permanent dipole moment.

Since its first experimental observation in 1925, the HeH$^+$ molecular ion, the simplest polar heteronuclear molecule, has served as a fundamental benchmark system for understanding principles of molecular formation and electron correlation. Moreover, HeH$^+$ continues to intrigue researchers as the first molecular species to arise in the universe.
So far, the molecular hydrogen ion, has typically been used as the prototype for the dynamics of molecules in strong laser fields. This understanding is then extended to interpret and predict the behavior of more complex molecules. However, the dynamics of this and other homonuclear molecules lack fundamental properties as they are symmetric and do not have a permanent dipole moment. In stark contrast, HeH$^+$ is a two-electron system with a large mass asymmetry, a strong electronic asymmetry, and a permanent dipole moment.

A sophisticated setup for measuring laser-induced fragmentation of an ion beam of helium hydride and an isotopologue at various wavelengths and intensities enables us to study the dynamics of this most fundamental polar molecule. In contrast to the prevailing interpretation of strong-field fragmentation, in which stretching of the molecule results primarily from laser-induced electronic excitation, experiment and theory for non-ionizing dissociation, single ionization and double ionization both show that the direct vibrational excitation plays the decisive role here. We are able to reconstruct fragmentation pathways and determine the times at which each ionization step occurs as well as the bond length evolution before the electron removal. The dynamics of this extremely asymmetric molecule contrast the well-known symmetric systems yielding a much clearer picture of strong-field molecular dynamics in general and facilitating interpolation to other systems. 

\vspace{\baselineskip}