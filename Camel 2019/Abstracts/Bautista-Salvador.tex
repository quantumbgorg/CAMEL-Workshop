\title{TOWARDS FUTURE LARGE-SCALE QUANTUM DEVICES WITH TRAPPED IONS}
% Towards future large-scale quantum devices with trapped ions

\underline{A. Bautista-Salvador} \index{Bautista-Salvador A.} 
%Amado Bautista-Salvador

{\normalsize{\vspace{-4mm}
Bundesallee 100, 38116 Braunschweig



\email bautista@iqo.uni-hannover.de}}

Quantum computing with trapped ions is currently reaching an unprecedented maturity towards a practical realization of a scalable platform. Besides developing the tools for performing quantum operations, the field requires a significant effort in developing scalable hardware. This progressive technical quest requires, for instance, advances in material science, the development of new designs or fabrication methods. Here we present a novel method for the realization of large-scale quantum devices [1]. The method allows the integration of complex 3D structures into the trap itself. In a first part of the talk, I will discuss the trap fabrication and show preliminary results on the characterization of a multilayer ion trap with integrated 3D microwave circuitry for the implementation of high-fidelity quantum logic control with 9Be+ ions. We demonstrate ion trapping and single-qubit microwave control of a laser cooled 9Be+ ion held at a distance of 35 $\mu$m from the surface. Towards the implementation of multi-qubit operations, we fit the measured magnetic field emanating from the structures to the 2D near-field model of [2] and find excellent agreement with numerical simulations. Finally, I will discuss new routes and potential new integrated devices in which the multilayer method can be exploited.

{\normalsize
[1] A. Bautista-Salvador et al., New J. Phys. 21, 043011 (2019).
\vsp

[2] M. Wahnschaffe et al., Appl. Phys. Lett. 110, 034103 (2017).
}

\vspace{\baselineskip}