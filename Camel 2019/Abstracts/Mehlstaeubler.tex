\title{NON-EQUILIBRIUM DYNAMICS IN ION COULOMB SYSTEMS}
% Non-Equilibrium Dynamics in Ion Coulomb Systems

\underline{T. Mehlstaeubler} \index{Mehlstaeubler T.} 
%Tanja Mehlstaeubler

{\normalsize{\vspace{-4mm}
Bundesallee 100,
38116 Braunschweig,
Germany



\email tanja.mehlstaeubler@ptb.de}}

Single trapped and laser-cooled ions in Paul traps allow for a high degree of control of atomic quantum systems. They are the basis for modern atomic clocks, quantum computers and quantum simulators. Our research aims to use ion Coulomb crystals, i.e. many-body systems with complex dynamics, for precision spectroscopy. This paves the way to novel optical frequency standards for applications such as relativistic geodesy and quantum simulators in which complex dynamics becomes accessible with atomic resolution.
The high-level of control of self-organized Coulomb crystals open up a fascinating insight into the non-equilibrium dynamics of coupled many-body systems, displaying atomic friction and symmetry-breaking phase transitions. We discuss the creation of topological defects and Kibble-Zurek tests in 2D crystals and present recent results on the study of tribology and transport mediated by the topological defect.


\vspace{\baselineskip}