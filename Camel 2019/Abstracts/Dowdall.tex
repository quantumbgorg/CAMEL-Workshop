\title{COOLING OF QUANTUM PARTICLES AND TRANSPORT OF A BOSE-EINSTEIN CONDENSATE ACROSS AN OPTICAL LATTICE}
% Cooling of quantum particles and transport of a Bose-Einstein Condensate across an optical lattice

\underline{T. Dowdall} \index{Dowdall T.} 
%Tom Dowdall

{\normalsize{\vspace{-4mm}
Department of Physics,
University College Cork,
Cork,
Ireland



\email t.dowdall@umail.ucc.ie}}

In the first part,  we present a theoretical scheme for cooling i.e. the
compression of both velocity and position distribution of particles in
motion [1]. This is achieved by collisions of the particles in
combination of a moving atomic mirror and a moving atom diode. An atom
diode is a unidirectional barrier, i.e., an optical device through which
an atom can pass in one direction only. We examine both the classical
and quantum mechanical descriptions of the scheme, along with the
numerical simulations to show the efficiency in each case. In the second
part, we use the techniques of shortcuts to adiabaticity to design a
scheme for a fast and robust transport of a Bose-Einstein condensates
trapped in an harmonic trap across an optical lattice [2]. We study the
fidelities of this scheme for different physical settings.

{\normalsize
[1] T. Dowdall and A. Ruschhaupt, Phys. Rev. A 97, 013412 (2018).
\vsp

[2] T. Dowdall and A. Ruschhaupt, in preparation
}

\vspace{\baselineskip}