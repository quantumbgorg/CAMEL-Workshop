\title{RYDBERG-RYDBERG INTERACTION STRENGTH AND DIPOLE BLOCKADE RADIUS IN 87RB ATOMS IN THE PRESENCE OF FORSTER RESONANCES}
% Rydberg-Rydberg Interaction Strength and Dipole Blockade Radius in 87Rb Atoms in the Presence of Forster Resonances

\underline{T. Kirova} \index{Kirova T.} 
%Teodora Kirova

{\normalsize{\vspace{-4mm}
Institute of Atomic Physics and Spectroscopy,
University of Latvia,
Jelgavas iela 3,
Riga, LV-1004,
Latvia



\email teo@lu.lv}}

Dipole blockade [1] is a phenomenon, where due to dipole-dipole interaction between Rydberg atoms the simultaneous excitation of two/multiple Rydberg atoms in a \"blockade sphere\" [2] is suppressed. Our aim is to find the best experimental parameters necessary to achieve dipole blockade radius of  50 micrometers which will be later measured experimentally. We are especially interested in the resonant type dipole-dipole interaction, which happens in the presence of Forster resonances.
 
With our purpose in mind, we calculate the magnitude of C6  coefficients for specific Forster transitions in 87Rb of the form  na,la,ja + nb,lb,jb-->nc,lc,jc+ nd,ld,jd. A large C6 coefficient is associated with a minimum in the absolute value of the Forster defect dk, which we plot as a function of the principle quantum number na for the transitions described above. We found that in some cases under study, the \"dk vs na\" curves show diverging behavior and no minimum of the absolute value of dk is observed. In other cases, however, such as n,p1/2 + (n + 20),p1/2-->(n-1)d3/2 + (n + 18)d3/2, the minimum occurs at n = 65, corresponding to dk= 3.47MHz, and giving a blockade radius of 18.2micrometers. For other cases, e.g. n,p1/2 + (n + 30)p1/2-->(n-1)d3/2 + (n + 28)d3/2, the minimum of dk=0.68 MHz is at n=83 and we can achieve blockade radius of 41.1 micrometers, which is close enough to the experimentally desired one.

{\normalsize
[1] A. Gaetan et al., Nature 5, 115 (2009).
\vsp

[2] D. Tong, et al., Phys. Rev. Lett 93, 063001 (2004).
}


\vspace{\baselineskip}