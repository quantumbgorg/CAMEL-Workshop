\title{RYDBERG-RYDBERG INTERACTION STRENGTH AND DIPOLE BLOCKADE RADIUS IN $^{87}$RB ATOMS IN THE PRESENCE OF FORSTER RESONANCES}
% Rydberg-Rydberg Interaction Strength and Dipole Blockade Radius in 87Rb Atoms in the Presence of Forster Resonances

\underline{T. Kirova} \index{Kirova T.} 
%Teodora Kirova

{\normalsize{\vspace{-4mm}
Institute of Atomic Physics and Spectroscopy,
University of Latvia,
Jelgavas iela 3,
Riga, LV-1004,
Latvia



\email teo@lu.lv}}

Dipole blockade [1] is a phenomenon, where due to dipole-dipole interaction between Rydberg atoms the simultaneous excitation of two/multiple Rydberg atoms in a "blockade sphere" [2] is suppressed. Our aim is to find the best experimental parameters necessary to achieve dipole blockade radius of $R_b\approx50\mu$m, which will be later measured experimentally. We are especially interested in the resonant type dipole-dipole interaction, which happens in the presence of  F\"orster resonances.
With our purpose in mind, we calculate the magnitude of $C_6$ coefficients for specific F\"orster transitions in $^{87}$Rb of the form $n_{a}l_{a}j_{a}+n_{b}l_{b}j_{b}\rightarrow n_{c}l_{c}j_{c}+n_{d}l_{d}j_{d}$.
A large $C_6$ coefficient is associated with a minimum in the absolute value of the F\"orster defect $\delta_k$, which we plot as a function of the principle quantum number $n_{a}$ for the transitions described above.
We found that in all cases under study, the  ``$\delta_k$ vs $n_{a}$'' curves show diverging behavior and no minimum of the absolute value of $\delta_k$ is observed.
In other cases, however, such as $np_{1/2}+(n+20)p_{1/2}\rightarrow(n-1)d_{3/2}+(n+18)d_{3/2}$, the minimum occurs at $n_{a}=65$, corresponding to $\delta_{k}=3.47$ MHz, $C_6=-2.19.10^{5}$
GHz $\mu$m$^6$, and giving a blockade radius of  $R_b=18.21\mu$m.
For some transitions, e.g. $np_{1/2}+(n+30)p_{1/2}\rightarrow(n-1)d_{3/2}+(n+28)d_{3/2}$, the minimum of $\delta_{k}=0.69$ MHz is at $n_{a}=97$, $C_6=-3.04.10^{7}$ GHz$\mu$m$^6$, and we can achieve a  blockade radius of  $41.41\mu$m, which is close enough to the experimentally desired one.

{\normalsize
[1] A. Gaetan et al., Nature 5, 115 (2009).
\vsp

[2] D. Tong, et al., Phys. Rev. Lett 93, 063001 (2004).
}


\vspace{\baselineskip}