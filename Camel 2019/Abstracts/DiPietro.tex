\title{THERMALLY-INDUCED NONLINEAR SPATIAL SHAPING OF INFRARED FEMTOSECOND PULSES IN NEMATIC LIQUID CRYSTALS}
% Thermally-induced nonlinear spatial shaping of infrared femtosecond pulses in nematic liquid crystals

\underline{V. di Pietro} \index{di Pietro V.} 
%Vittorio Maria di Pietro

{\normalsize{\vspace{-4mm}
Avenue de la republique 34



\email vittorio.dipietro@fastilte.com}}

Liquid crystals (LC) offer numerous prospects in photonics, for instance, nematics LC-based electro-optical devices enables tunable phase-shifting of optical and THz radiations, or phase-sensing. Indium-Tin-Oxide (ITO) is one of the most employed electrode coating for this applications, thanks to its good transmission in the visible
part of the optical spectrum. However, extension towards infrared wavelengths finds some difficulties, such as ITO strong absorption and nonlinear behavior around 1.55um. Although some optimized versions of ITO are available in this spectral range, with a remaining weak absorption might have a significant impact on the LC optical answer. We demonstrate that an infrared femtosecond oscillator focalized up to 7.6kW/cm$^2$, undergoes strong spatial selfphase modulation in a 180um thick E7 nematic mixture cell, due to partial laser light absorption (20\%) in the yet
optimized ITO coating. The heat absorption in the ITO layer is transferred to the nematic layer and a well-confined thermal gradient is established due to the different heat transfer coefficients. Large and nonlinear sensitivity of
the thermotropic LC with the temperature generates a typical spatial self-phase modulation pattern in thin media (i.e. multiple-ring pattern) when the average power density is high enough and dephasing is larger than 2$\pi$. The number of rings  is a function of the index gradient and, therefore, increases with the power density, meanwhile the polarization sets the shape of the rings and the sign of the non-linearity. From the number of rings we can estimate the introduced dephasing , so the refractive index variation. Close to the focus, the phase transition between nematic and isotropic phases (Tc = 331K for E7) is reached. We can also
recover Temperature excursion inside the LC across the laser spot from and verify that the anisotropy of the heat transfer coefficients (two times lower in the ordinary axe) makes that $T\text{o} >T\text{e}$. The stability of the non-linear pattern is better than $\pi =5$ for an overall of phase shift of $30\pi$. The refractive index variration is temporally stable and the spectral distribution in the rings is found unchanged, therefore the thermal gradient is stable and well-confined with a spatial resolution measured below 140 mm. The usual spatial nonlinear effects achieved with fs pulses are often
generated with bulk media, so self-focusing or defocusing effects limit spatial phase shift. Here, no modulation of the temporal phase occurs, and thus the spectro-temporal characteristics of the pulse are not affected. The fidelity and stability of the process open new prospects for polarization-dependent spatial shaping devices and delimits the operating wavelength range for ultrafast liquid-crystal based electro-optic application.

\vspace{\baselineskip}
