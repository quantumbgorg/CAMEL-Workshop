\title{LIGHT SOURCES FOR EXPERIMENTS IN LASER COOLING, LASING WITHOUT INVERSION AND QUANTUM KEY DISTRIBUTION}
% Light Sources for Experiments in Laser Cooling, Lasing without Inversion and Quantum Key Distribution

\underline{T. Walther} \index{Walther T.} 
%Thomas Walther

{\normalsize{\vspace{-4mm}
Inst. for Applied Physics,
TU Darmstadt,
Schlossgartenstr. 7,
D-64289 Darmstadt



\email thomas.walther@physik.tu-darmstadt.de}}

In recent years, we have pursued experiments where narrowband cw radiation or Fourier-transform (FT) limited pulses are a requirement. In addition, the experiments regularly require the use of non-linear processes to either reach the UV range or generate entangled photons by spontaneous parametric down conversion (SPDC).

During the first part of the talk, we will detail some of the physics and experimental features of our cw UV laser systems to be used in our experiments regarding laser cooling relativistic ion beams and lasing without inversion. In both cases, we need UV radiation in the region of 254 nm. A challenge in the generation of long-term stable operation of these systems is the degradation of the employed non-linear crystals. Recently, we have found a way around this problem by a special  built-up cavity with elliptical focusing designed with the help of evolutionary algorithms [1], [2].

The second part of the talk is dedicated to the discussion of an QKD experiment geared towards the implementation of a quantum hub. We have setup a photon source based on SPDC and the phase-time bin entanglement. Currently parts of this system are being tested in collaboration with the Deutsche Telekom. In the talk, I will discuss some of the features as well as the current status of the experiment.

{\normalsize
[1] D. Kiefer, D. Prei\ss ler, T. F{\"u}hrer, and T. Walther, Laser Physics Letters, in print, (2019).
\vsp

[2] D. Prei\ss ler, D. Kiefer,  T. F{\"u}hrer, and T. Walther,
Appl. Phys. B, to be submitted (2019).
}

\vspace{\baselineskip}