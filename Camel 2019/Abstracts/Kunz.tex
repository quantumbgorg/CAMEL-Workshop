\title{SUB-SHOT-NOISE RECEIVER FOR BINARY COHERENT ALPHABETS IN THE PRESENCE OF PHASE NOISE}
% Sub-shot-noise receiver for binary coherent alphabets in the presence of phase noise

\underline{L. Kunz} \index{Kunz L.} 
%Ludwig Kunz

{\normalsize{\vspace{-4mm}
University of Warsaw,
Krakowskie Przedmiescie 26/28,
00-927 Warszawa, Poland,
VAT Reg. no. PL5250011266



\email l.kunz@cent.uw.edu.pl}}

For reliable optical communication state discrimination is a critical task. Quantum measurements can provide significant enhancement in information transfer compared to classical techniques and enable detection below the shot-noise limit. Noise in the communication channel or the measurement limits the benefits of quantum-enhanced techniques. While linear losses result in simple rescaling of the complex field amplitude, the effects of phase diffusion are less trivial. Phase noise can arise as linear noise or as nonlinear phase noise caused by nonlinear interactions. The effect of phase noise on the information transfer is particularly severe if the information is encoded in coherent states. To mitigate the detrimental effects new strategies for information retrieval are required. We investigate a single-shot measurement which shows robustness when communicating over a channel with linear or nonlinear phase noise. In these communication scenarios the information is encoded in a binary alphabet of coherent states where the average energy is limited. We consider a measurement based on a displacement operation followed by photon counting with finite photon number resolution. By optimizing the displacement operation, the information transfer can be maximized while at the same time the effect of phase noise is minimized. This communication strategy provides enhancement compared to conventional detection when the amplitudes of the alphabet have been optimized for both techniques.

\vspace{\baselineskip}