\title{THREE-PHOTON PRECISION SPECTROSCOPY OF A LARGE ION CLOUD}
% Three-photon precision spectroscopy of a large ion cloud

\underline{M. Knoop} \index{Knoop M.}
%Martina Knoop

{\normalsize{\vspace{-4mm}
Aix-Marseille Universite, CNRS, PIIM, UMR 7345, 13397 Marseille, France

\email martina.knoop@univ-amu.fr}}

Coherent population trapping allows to exploit multi-photon spectroscopy with a very high precision [1]. The right combination of three wavevectors gives rise to a scheme which is inherently Doppler-free. A cloud of trapped ions is then an excellent sample to measure, for example, transition dipoles with an uncertainty inferior to $10^{-3}$, necessary to push forward todays best calculations [2].
The experimental observation of a three-photon dark line in an ion cloud is explored and its building blocks will be presented. The set-up relies on a linear radio-frequency trap where up to $10^6$ laser-cooled Ca$^+$ ions can be trapped. We have developed and realized a narrow linewidth laser at 729 nm that reaches a relative frequency stability below $10^{-14}$ per second for periods inferior to 10 seconds to probe the electric quadrupole transition. The required fixed phase-relation of this source and the two cooling lasers is reached by simultaneous lock on a frequency comb. Measurements are ongoing and results will be reported at the conference [3].

{\normalsize
[1] C. Champenois, et al., Phys. Rev. Lett. \textbf{99}, 013001 (2007).
\vsp

[2] M. S. Safronova et al., Phys. Rev. A \textbf{83}, 012503 (2011).
\vsp

[3] R. Khayatzadeh, et al., IEEE Photonics Technology Letters \textbf{29}, 322-325 (2017).
}

\vspace{\baselineskip} 