\title{RECENT PROGRESS TOWARDS QUANTUM LOGIC INSPIRED COOLING AND READOUT TECHNIQUES FOR SINGLE (ANTI-)PROTONS}
% Recent progress towards quantum logic inspired cooling and readout techniques for single (anti-)protons

\underline{A. Paschke} \index{Paschke A.}
%Anna-Greta Paschke

{\normalsize{\vspace{-4mm}
Institut fuer Quantenoptik,
Leibniz Universitaet Hannover,
Welfengarten 1,
30167 Hannover,
Germany

\email paschke@iqo.uni-hannover.de}}

Our research group aims to develop and apply novel laser-based, quantum logic inspired cooling and manipulation techniques to single (anti-)protons in a cryogenic Penning trap system. Within the BASE collaboration [1] this will support ongoing tests of CPT symmetry based on a g-factor comparison between the proton and antiproton by significantly improving the particle localization and detection times. For implementation we have designed a multi-zone Penning trap apparatus, in which a single (anti-)proton will be coupled to a co-trapped ``logic'' 9Be$^+$ ion and be sympathetically cooled, controlled and read out indirectly using quantum logic operations [2,3].
In this contribution we report on the status of the project and present recent experimental results on quantum control of 9Be$^+$ ions based on direct frequency comb control [4] using a spectrally tailored UV frequency comb.

{\normalsize
[1] Smorra et al., Eur. Phys. J. Special Topics \textbf{224}, 3055-3108 (2015).
\vsp

[2] Heinzen and Wineland, PRA \textbf{42}, 2977 (1990).
\vsp

[3] Wineland et al., J. Res. NIST \textbf{103}, 259 (1998).
\vsp

[4] Hayes et al., PRL \textbf{104}, 140501 (2010).
}

\vspace{\baselineskip} 