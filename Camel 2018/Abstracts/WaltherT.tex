\title{FIBER AMPLIFIERS AND THEIR APPLICATIONS IN DOPPLER COOLING AND QUANTUM KEY DISTRIBUTION}
% Fiber Amplifiers and their Applications in Doppler Cooling and Quantum Key Distribution

\underline{T. Walther} \index{Walther T.}
%Thomas Walther

{\normalsize{\vspace{-4mm}
Institute for Applied Physics,
TU Darmstadt,
Schlossgartenstr. 7,
D-64289 Darmstadt,
Germany

\email thomas.walther@physik.tu-darmstadt.de}}

In recent years, we have pursued experiments where narrowband cw radiation or Fourier-transform (FT) limited pulses are a requirement. We therefore ventured into exploring the possibilities of fiber amplifiers as they enable fulfilling these specifications in a relatively straight forward manner. In some cases this was combined with the need of non-linear frequency conversion into the UV spectral range.

During the first part of the talk, we will detail some of the physics and experimental features of our cw and pulsed fiber amplifiers. In particular, we describe a cw system frequency quadrupled into the UV [1], a fiber amplifier with FT-limited ns-pulses [2] as well as an advanced version producing ps-pulses with flexible pulse duration at high repetition rates [3].

The second part is dedicated to the discussion of two applications of these fiber amplifiers in our laboratory. The first is the transfer of the ideas of Doppler cooling to cooling highly relativistic ion beams in accelerators. Instead of stopping the ion beam, the goal of this experiment is to considerably narrow the momentum distribution. We will introduce the motivation for such experiments and the basic steps in order to achieve such cooling as well as first results.
Lastly, we present an experiment in quantum key distribution (QKD) geared towards the setup of a quantum hub intended as a basic building block of a small QKD network.

{\normalsize
[1] T. Beck, B. Rein, F. S\"orensen, and T. Walther, Opt. Lett. \textbf{41}, 4186--4189 (2016).
\vsp

[2] Kai Schorstein and Thomas Walther, Appl. Phys. B \textbf{97}, 591--597 (2009).
\vsp

[3] Daniel Kiefer and Thomas Walther, submitted to CLEO (2018).
}

\vspace{\baselineskip} 