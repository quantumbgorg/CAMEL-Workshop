\title{OPTICALLY STIMULATED THIRD HARMONIC GENERATION}
% Optically stimulated third harmonic generation

\underline{K. Zlatanov} \index{Zlatanov K.}
%Kaloyan Zlatanov

{\normalsize{\vspace{-4mm}
Hochschulstr. 6D-64289 Darmstadt, Germany



\email k.zlatanoff@gmail.com}}


Coherent nonlinear microscopy (CNM) is widely used in three-dimensional imaging of transparent samples, particularly for biological tissues [1]. Examples for parametric and label free CNM processes are second harmonic generation (SHG) and third harmonic generation (THG). However, working under far off-resonant excitation conditions usually leads to low optical conversion efficiencies. Hence, the typically low signals may lead to a bad signal-to-noise ratio and low image contrast. There are several approaches to enhance the signal yield in CNM, e.g. using dispersion to optimize the phase-matching integral [2]. One promising concept is the ``Enhancement of Second-Order Nonlinear-Optical Signals by Optical Stimulation'' [3]. The basic idea is to stimulate a nonlinear signal by already ``seeding'' with some radiation at the harmonic frequency. The previous and first demonstration of the concept for SHG yielded signal enhancements more than 10 in a biologically relevant medium. We present strong enhancements of third harmonic generation by optical stimulation in a microscopy setup. The effect is most pronounced at low laser intensity and weak nonlinear susceptibilities, making it suitable in harmonic microscopy.

{\normalsize
[1] S. Yue, M.N. Slipchenko, J.-X. Cheng, Laser Photon. Rev. \textbf{5}, 496–512 (2011).
\vsp

[2] Ch. Stock, K. Zlatanov, and Th. Halfmann, Optics Communications Vol. \textbf{393}, 289-293 (2017).
\vsp

[3] A. J. Goodman and W. A. Tisdale, Phys. Rev. Lett. \textbf{114}, 183902 (2015).
}

\vspace{\baselineskip}
