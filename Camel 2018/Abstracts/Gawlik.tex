\title{COHERENCE AND NONLINEAR SPECTROSCOPY IN NV DIAMOND}
% Coherence and nonlinear spectroscopy in NV diamond

\underline{W. Gawlik} \index{Gawlik W.}
%Wojciech GAWLIK

{\normalsize{\vspace{-4mm}
Instytut Fizyki Uniwersytet Jagiellonski,
Profesora Lojasiewicza 11,
30-348 Krakow, Poland

\email gawlik@uj.edu.pl}}

Nitrogen-Vacancy (NV-) color centers in diamond are characterized by a nonzero electron spin ($S$=1) which allows them to be optically pumped and probed via microwave (MW) spectroscopy. In our study we focus on the case of two MW fields tuned to the transitions between the $m$=0 $\leftrightarrow$ $m$ = $\pm$ 1 spin sublevels of the NV-ensemble in the 3A ground state. One field saturates the transition (burns a hole) while the second is acting as the probe field. The fluorescence intensity reflects the spin state populations and enables monitoring of the optically detected magnetic resonance (ODMR). The recorded spectra are more complex spectra than the standard ODMR ones and exhibit hole burning effect positioned at the pump-field frequency. We found different behavior of the spectra when the pump and probe microwave fields are tuned to either the same or distinct transitions and interpret the difference as a result of coherent population oscillations (CPO). The hole-burning method enables isolation of the homogeneous contribution to the magnetic-resonance width on the background of inhomogeneous broadening and may become a powerful toll for studies of the line broadening issues in NV- diamonds. In particular, I will demonstrate its possible applications to study powders and suspensions of nanodiamonds.

\vspace{\baselineskip}
