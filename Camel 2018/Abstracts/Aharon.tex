\title{HIGH RESOLUTION SENSING OF HIGH-FREQUENCY FIELDS WITH CONTINUOUS DYNAMICAL DECOUPLING}
% High resolution sensing of high-frequency fields with continuous dynamical decoupling

\underline{N. Aharon} \index{Aharon N.}
%Nati Aharon

{\normalsize{\vspace{-4mm}
Racah Institute of Physics, The Hebrew University of Jerusalem, Jerusalem 91904, Israel

\email nati.aharon@gmail.com}}

State-of-the-art methods for sensing weak AC fields are only efficient in the low frequency domain ($<$10 MHz). The inefficiency of sensing high-frequency signals is due to the lack of ability to use dynamical decoupling. In this work we show that dynamical decoupling can be incorporated into high-frequency sensing schemes and by this we demonstrate that the high sensitivity achieved for low frequency can be extended to the whole spectrum. While our scheme is general and suitable to a variety of atomic and solid-state systems, we experimentally demonstrate it with the nitrogen-vacancy center in diamond. We achieve coherence times up to 1.43 ms resulting in a smallest detectable magnetic field strength of 4 nT at 1.6 GHz. Attributed to the inherent nature of our scheme, we observe an additional increase in coherence time due to the signal itself. In this talk I will also present a few other dynamical decoupling schemes, that could be utilized to further improve the resolution of sensing oscillating signals, and in particular, high frequency fields.

\vspace{\baselineskip} 