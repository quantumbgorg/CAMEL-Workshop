\title{RELATIONS BETWEEN THE SINGLE-PASS AND DOUBLE-PASS TRANSITION PROBABILITIES IN QUANTUM SYSTEMS WITH TWO AND THREE STATES}
% Relations between the single-pass and double-pass transition probabilities in quantum systems with two and three states

\underline{N. Vitanov} \index{Vitanov N.}
%Nikolay Vitanov

{\normalsize{\vspace{-4mm}
Department of Physics, St Kliment Ohridski University of Sofia, 5 James Bourchier Blvd, 1164 Sofia, Bulgaria



\email vitanov@phys.uni-sofia.bg}}

In the experimental determination of the population transfer efficiency between discrete states of a coherently driven quantum system it is often inconvenient to measure the population of the target state. Instead, after the interaction that transfers the population from the initial state to the target state, a second interaction is applied which brings the system back to the initial state, the population of which is easy to measure and normalize. If the transition probability is $p$ in the forward process, then classical intuition suggests that the probability to return to the initial state after the backward process should be $p^2$. However, this classical expectation is generally misleading because it neglects interference effects.This paper presents a rigorous theoretical analysis based on the SU(2) and SU(3) symmetries of the propagators describing the evolution of quantum systems with two and three states, resulting in explicit analytic formulas that link the two-step probabilities to the single-step ones. Explicit examples are given with the popular techniques of rapid adiabatic passage and stimulated Raman adiabatic passage. The present results suggest that quantum-mechanical probabilities degrade faster in repeated processes than classical probabilities. Therefore, the actual single-pass efficiencies in various experiments, calculated from double-pass probabilities, might have been greater than the reported values.

\vspace{\baselineskip} 