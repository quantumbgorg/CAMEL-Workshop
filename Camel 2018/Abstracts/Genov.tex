\title{EXPERIMENTAL DEMONSTRATION OF COMPOSITE STIMULATED RAMAN ADIABATIC PASSAGE}
% Experimental Demonstration of Composite Stimulated Raman Adiabatic Passage

\underline{G. Genov} \index{Genov G.}
%Genko Genov

{\normalsize{\vspace{-4mm}
Institut f\"ur Angewandte Physik, Technische Universität Darmstadt, Hochschulstrasse 6, 64289 Darmstadt, Germany

\email genko.genov@physik.tu-darmstadt.de}}

Efficient and robust population transfer by external fields remains an important challenge in many areas of physics. Resonant pulses can deliver high efficiency but are not robust to even the slightest experimental imperfections. Composite pulses are sequences of pulses with suitable relative phases and improved performance with applications in nuclear magnetic resonance and quantum optics. Adiabatic techniques are another useful alternative, e.g., stimulated Raman adiabatic passage (STIRAP) is widely used in physics and chemistry, but its fidelity is limited by imperfect adiabaticity. A combination of STIRAP with composite pulses, composite STIRAP (CSTIRAP), has been proposed recently to address these limitations [1].

We present here the first proof-of-principle experimental implementation of CSTIRAP by applying it for population transfer between the hyperfine ground states of Pr atoms in a rare-earth doped solid. We compare the performance with traditional STIRAP and show that CSTIRAP improves both the robustness and peak transfer efficiency substantially in comparison to repeated STIRAP in the regime of large single photon detuning. In the latter, STIRAP is insensitive to the initial state of the system, which makes it suitable for repeated inversion processes, e.g. for rephasing of atomic coherences for quantum memories. Our experimental findings match very well the corresponding numerical simulations.

{\normalsize
[1] B. T. Torosov and N. V. Vitanov, Phys. Rev. A \textbf{87}, 043418 (2013).
}

\vspace{\baselineskip}