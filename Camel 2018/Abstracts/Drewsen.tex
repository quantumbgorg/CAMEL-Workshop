\title{COHERENT INTERNAL STATE MANIPULATION OF SINGLE ATOMIC AND MOLECULAR IONS BY OPTICAL FREQUENCY COMBS}
% Coherent internal state manipulation of single atomic and molecular ions by optical frequency combs

\underline{M. Drewsen} \index{Drewsen M.}
%Michael Drewsen

{\normalsize{\vspace{-4mm}
Department of Physics and Astronomy, Aarhus University, Denmark

\email drewsen@phys.au.dk}}

Coherent manipulation of state-prepared molecular ions are of interest for a wide range of research fields, including ultra-cold chemistry, ultra-high resolution spectroscopy for test of fundamental physics, and quantum information science. Recently, there has been significant advances with respect to both preparing trapped molecular ions in their quantized motional ground state [1-3], and controlling their rovibrational degrees of freedom through a series of different techniques [4-11]. This talk will focus on coherent manipulation of the internal states of single atomic and molecular ions through the exposure of such ions to optical frequency comb (OFC) [12, 13]. Based on recent coherent manipulations of the population between the metastable 3d $^2\text{D}_{3/2}$ and 3d $^2\text{D}_{5/2}$ levels in the Ca$^+$ ion separated by 1.8 THz, we will discuss next experiments on coherent rotational state manipulation of the MgH$^+$ ion, as well as comment on the perspective of using this manipulation technique in connection coherently controlled reaction experiments.

{\normalsize
[1] G. Poulsen, PhD thesis, Aarhus University, 2011.
\vsp

[2] Y. Wan et al., Phys. Rev. A 043425, 91 (2015).
\vsp

[3] R. Rugango et al., New J. Phys. 03009, 17 (2015).
\vsp

[4] P. F. Staanum et al., Nat. Phys. 271, 6 (2010).
\vsp

[5] T. Schneider et al., Nat. Phys. 275, 6 (2010).
\vsp

[6] X. Tong et al., Phys. Rev. Lett. 143001, 105 (2010).
\vsp

[7] W. G.Rellergert et al., Nature 490, 495 (2013).
\vsp

[8] A. K. Hansen et al., Nature 76, 508 (2014).
\vsp

[9] C.-Y. Lien et al., Nat. Comm. 4783, 5 (2014).
\vsp

[10] F. Wolf et al., Nature 457, 530 (2016).
\vsp

[11] C.-W. Chou et al., Nature 203, 545 (2017).
\vsp

[12] D. Hayes et al., Phys. Rev. Lett. 140501, 104 (2010).
\vsp

[13] S. Ding and D. N. Matsukevich, New J. Phys. 023028, 12 (2012); D. Leibfried, ibid., 023029, 12 (2012).
}

\vspace{\baselineskip} 