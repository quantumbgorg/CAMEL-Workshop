\title{CONSTRUCTING A MULTI-MODULE TRAPPED-ION QUANTUM COMPUTER PROTOTYPE}
% Constructing a multi-module trapped-ion quantum computer prototype

\underline{W. Hensinger} \index{Hensinger W.}
%Winfried Hensinger

{\normalsize{\vspace{-4mm}
Sussex Centre for Quantum Technologies,
Department of Physics and Astronomy,
University of Sussex,
Falmer, Brighton, East Sussex, BN1 9QH,
United Kingdom




\email w.k.hensinger@sussex.ac.uk}}

I will discuss progress in constructing a multi-module trapped-ion quantum computer prototype at the University of Sussex. I will provide a short overview of the overall architecture. Previously, it had been proposed to use photonic interconnects to connect individual computer modules. Our new invention introduces connections created by electric fields that allow ions to be transported from one module to another. This architecture also features a method where quantum gates with trapped ions are executed by applying voltages in the presence of a few global rf radiation fields similar in nature to the operation of transistors in a classical computer. I will also describe a coherent control method to make quantum gates more resilient to parameter fluctuations and show experimental results. Finally I will discuss a technique to transform existing two-level quantum control methods to new multi-level control methods and provide an application of this method where we map two different qubit types coherently with a fidelity well above the relevant fault-tolerant threshold.

\vspace{\baselineskip}
