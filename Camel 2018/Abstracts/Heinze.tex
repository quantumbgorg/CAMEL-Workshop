\title{TOWARDS ENTANGLEMENT BETWEEN DISPARATE MATTER QUANTUM NODES}
% Towards Entanglement between Disparate Matter Quantum Nodes

\underline{G. Heinze} \index{Heinze G.}
%Georg Heinze

{\normalsize{\vspace{-4mm}
ICFO -- The Institute of Photonic Sciences, Av. Carl Friedrich Gauss 3, 08860 Castelldefels (Barcelona), Spain



\email georg.heinze@icfo.eu}}

Building quantum networks –- consisting of matter quantum nodes and photonic communication channels –- is a key to enable future distributed quantum information applications.
In this talk, we first briefly report on our recent efforts to build the first hybrid quantum interface connecting a cold ensemble of Rubidium atoms and a Praseodymium-ion doped crystal [1]. We show that a paired single photon from the cold atomic ensemble can be frequency converted to the telecom C-band and back to the visible range to be stored and retrieved in the solid-state memory. Using time-bin encoding, we afterwards demonstrate that a qubit can be faithfully transferred between the disparate quantum systems.
To further expand the capabilities of that approach, one would like to achieve entanglement between both matter systems. In the second part of the talk, we discuss how this goal can be reached. To that end, we demonstrate the first direct generation of entanglement between a photonic time-bin qubit and a single collective atomic spin excitation (spin wave) in the cold atomic ensemble [2]. A magnetic field that induces a periodic dephasing and rephasing of the atomic excitation ensures the temporal distinguishability of the two time bins and plays a central role in the entanglement generation. We analyze the generated quantum state by performing projective measurements in different qubit bases and verify the entanglement by violating a CHSH Bell inequality.

{\normalsize
[1] N. Maring, P. Farrera, K. Kutluer, M. Mazzera, G. Heinze, H. de Riedmatten, Nature 551, 485 (2017).
\vsp

[2] P. Farrera, G. Heinze, H. de Riedmatten, arXiv:1801.05723 (2018).
}

\vspace{\baselineskip}
