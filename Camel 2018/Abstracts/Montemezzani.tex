\title{GENERATION OF RADIALLY AND AZIMUTHALLY STRUCTURED VECTORIAL BEAMS BY POLARIZATION-SCRAMBLED CASCADED CONICAL DIFFRACTION}
% Generation of radially and azimuthally structured vectorial beams by polarization-scrambled cascaded conical diffraction

\underline{G. Montemezzani} \index{Montemezzani G.}
%Germano Montemezzani

{\normalsize{\vspace{-4mm}
Laboratory LMOPS,
University of Lorraine and CentraleSupelec,
2 rue E. Belin,
F - 57070 Metz

\email germano.montemezzani@univ-lorraine.fr}}

When a light beam travels along one of the two optical axes of an anisotropic biaxial crystal it experiences conical diffraction (or conical refraction), a phenomenon known since nearly two centuries. The unique propagation vector is then associated to a multitude of different directions for the Poynting vector, all propagating on the surface of a cone and each one associated to a different linear polarization. The last 15 years or so have witnessed a strong renewed interest in this peculiar and fascinating effect, principally due to several promising applications in modern photonics. One of the major recent advances consists in the development of cascaded configurations, where two or more biaxial crystals with their optical axes aligned are put in series. It was shown that a cascade of $N$ crystals leads to the formation of 2 to the power $N$-1 concentric cones. In this case the intensity distribution is modulated radially but remains homogeneous in the azimuthal direction, at least in the case of an input light being unpolarized or circularly polarized. Here it will be shown theoretically and experimentally that by introducing some polarization transforming elements (waveplates, polarizers) in between the cascaded crystals the azimuthal light distribution becomes highly structured. Potentially the involved elements can be activated extremely fast by using electro-optical devices, leading to the possibility of a fast reshaping of the complex structure of the vectorial beam.

\vspace{\baselineskip}