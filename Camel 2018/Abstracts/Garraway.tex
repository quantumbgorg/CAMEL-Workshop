\title{DECAY OF QUANTUM SYSTEMS ANALYSED WITH PSEUDOMODES OF RESERVOIR STRUCTURES}
% Decay of quantum systems analysed with pseudomodes of reservoir structures

\underline{B. Garraway} \index{Garraway B.}
%Barry Garraway

{\normalsize{\vspace{-4mm}
Department of Physics and Astronomy,
University of Sussex,
UK



\email b.m.garraway@sussex.ac.uk}}

Reservoir structures result from certain types of non-uniform bath
spectral density with canonical examples such as that of an atom in a
cavity. When these structures are coupled to simple quantum systems
the resulting decay can be analysed by the method of ``pseudomodes'',
where the reservoir structure is replaced by an effective
mode [1]. The approach is useful for strongly coupled,
i.e. non-Markovian problems, since exact master equations can be
derived. In this talk, an introduction to the basics of pseudomode
theory will be given, together with developments on reservoir memory
[2, 3] and entanglement in such reservoir structures [4]. Our latest
results involving an analysis of multiple observers will be
presented [5].

{\normalsize
[1] B.M. Garraway, Phys. Rev. A \textbf{55}, 4636 (1997).
\vsp

[2] L. Mazzola, S. Maniscalco, J. Piilo, K.-A. Suominen, and
    B.M. Garraway, Phys. Rev. A. \textbf{80}, 012104 (2009).
\vsp

[3] G. Pleasance and B.M. Garraway, in preparation (2017).
\vsp

[4] B.M. Garraway, J. Piilo, and S. Maniscalco, J. Phys. B \textbf{44}, 065505
    (2011).
\vsp

[5] G. Pleasance and B.M. Garraway, Phys Rev A \textbf{96}, 062105 (2017).
}

\vspace{\baselineskip}
