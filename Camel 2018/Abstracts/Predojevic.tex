\title{SOLID STATE SOURCE OF PHOTON TRIPLETS}
% Solid State Source of Photon Triplets

\underline{A. Predojevic} \index{Predojevic A.}
%Ana Predojevic

{\normalsize{\vspace{-4mm}
Stockholm University,
Department of Physics,
Albanova University Center,
Roslagstullsbacken 21,
SE-106 91 Stockholm,
Sweden



\email ana.predojevic@fysik.su.se}}

While remarkable progress has been made on single photons and photon pairs, multipartite correlated photon states are usually produced in purely optical systems by post-selection or cascading. On the other hand, multipartite states enable improved tests of the foundations of quantum mechanics as well as implementations of complex quantum optical networks and protocols. Therefore, it would be favorable to directly generate these states using solid state systems, for better scaling, simpler handling, and the promise of reversible transfer of quantum information between stationary and flying qubits.
Here, we use the ground states of two optically active coupled quantum dots to directly produce photon triplets. The formation of a triexciton leads to a triple cascade recombination and sequential emission of three photons with strong correlations. The quantum dot molecule is embedded in an epitaxially grown nanowire engineered for single-mode wave-guiding and improved extraction efficiency at the emission wavelength. We record 65.62 photon triplets per minute. Our structure and data represent a breakthrough towards implementing multipartite photon entanglement and multi-qubit readout schemes in solid-state devices, suitable for integrated quantum information processing.


\vspace{\baselineskip}