\title{PECULIARITIES OF BRIGHT AND DARK STATES FORMATION IN 3D3/2 AND 5D5/2  NA STATES}

\underline{T. Kirova}\index{Kirova T}

{\normalsize{\vspace{-4mm}
Laser Centre,
University of Latvia,
Riga, LV-1002,
Latvia

\email teo@lu.lv}}

We discuss the formation of bright and dark states in a generalized ladder scheme of three hyperfine
manifolds upon coupling by a strong laser field.  We investigate theoretically two excitation
schemes, where a weak probe field is used in the first excitation step between the 3S1/2 and 3P3/2
states, while the strong field further couples the 3P3/2 to the 3D3/2 (or 3D5/2) states in Na. The
population of the final state in the ladder scheme is investigated numerically as a function of the
weak probe field detuning under different coupling field parameters, e.g. detunings and intensities.
Depending on the number of the involved hyperfine levels, the coupling by a sufficiently strong
laser field leads to the formation of only bright or bright and dark states. In the 3D3/2 case the
intensities of the outermost bright peaks diminish until their full disappearance with sufficiently
strong coupling field, while in the 3D5/2 case the intermediate bright states disappear and the
outermost ones survive. This observation is confirmed by calculations of the dressed-states
eigenvalues which match the positions of the bright and dark peaks. 

\vspace{\baselineskip}
