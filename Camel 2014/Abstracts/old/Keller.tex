\title{TOWARDS HIGH RESOLUTION SPECTROSCOPY OF SINGLE MOLECULAR IONS}

\underline{M. Keller}\index{Keller M}

{\normalsize{\vspace{-4mm}
Department of Physics and Astronomy,
University of Sussex,
Brighton, BN1 9QH, UK

\email m.k.keller@sussex.ac.uk}}

High resolution spectroscopy of molecules has many applications such as tests of fundamental theories, detecting changes in fundamental constants and, potentially, quantum information processing. Prerequisite for these applications is the cooling of the molecules' motion and its non-invasive identification. Furthermore, the internal state of the molecules needs to be prepared and non-destructively detected.
The cooling of the motion and trapping of molecular ions can be accomplished by trapping them in an rf-trap alongside laser cooled atomic ions.
While blackbody assisted laser cooling was recently demonstrated, the non-destructive state detection is still beyond current experiments. Employing state selective laser induced dipole forces we aim to detect the internal state of molecular ions by mapping the state information onto the ions' motion. The scheme promises mitigation of the effect of laser polarisation and the distribution of population across Zeeman sublevels and it may be applicable for a larger number of simultaneously trapped molecules.

\vspace{\baselineskip}
