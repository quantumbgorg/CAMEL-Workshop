\title{DICKE-MODEL PHASE TRANSITION WITH ULTRACOLD ATOMS IN AN OPTICAL CAVITY}

\underline{D. Nagy} \index{Nagy D}

{\normalsize{\vspace{-4mm}
Wigner Research Centre for Physics, Hungarian Academy of Sciences,
H-1121, Budapest, Konkoly-Thege M. 29-33, Hungary

\email nagy.david@wigner.mta.hu}}

We discuss the dispersive coupling of a Bose-Einstein condensate to
the field of a high-Q optical cavity. The optical field mediates
an infinite-range atom-atom interaction which can induce the
self-organization of a homogeneous BEC into a periodically patterned
distribution above a critical driving strength [1]. This
self-organization effect can be identified with the superradiant
quantum phase transition of the Dicke model, however in our system,
the role of the internal atomic states are played by the motional
states of the condensate [2]. The cavity photon loss limits
the observation of the quantum phase transition in the ground state
and for long times one observes a nonequilibrium phase transition in
the steady state of the system. We show that the critical fluctuations
survive in the steady state, however the critical exponents [3] and the
finite-size scaling exponents [4] are different from those in the ground
state. Furthermore, the atom-field entanglement is peaked but not
divergent in the steady state.

{\normalsize
[1] D. Nagy, G. Szirmai and P. Domokos, Eur. Phys. J D \textbf{48}, 127 (2008).
\vsp

[2] D. Nagy, G. Konya, G. Szirmai, and P. Domokos, Phys. Rev. Lett. \textbf{104}, 130401 (2010).
\vsp

[3]  D. Nagy, G. Szirmai and P. Domokos, Phys. Rev. A \textbf{84}, 043637 (2011).
\vsp

[4] G. Konya, D. Nagy, G. Szirmai, and P. Domokos, Phys. Rev. A \textbf{86}, 013641 (2012).
}

\vspace{\baselineskip}
