\title{STRONG QUANTUM INTERFERENCES IN FREQUENCY CONVERSION}

\underline{P. Ackermann} \index{Ackermann P}

{\normalsize{\vspace{-4mm}
Institute of Applied Physics, Technical University of Darmstadt, Hochschulstrasse 6, 64289 Darmstadt, Germany

\email patric.ackermann@physik.tu-darmstadt.de}}

We present experimental data on quantum interference in resonantly enhanced frequency upconversion towards the vacuum-ultraviolet spectral regime. The process is driven in Xenon atoms by ultra-short (picosecond) laser pulses. We use two simultaneous frequency conversion pathways via a highly excited state: (A) Fifth harmonic generation of the fundamental wavelength and (B) sum frequency mixing of one fundamental photon and two photons of the second harmonic frequency. Both conversion pathways yield radiation at 102 nm. The two pathways interfere, depending on the relative phase of the fundamental and second harmonic. By appropriate choice of the phase we get constructive interference (i.e. increased efficiency) or destructive interference (i.e. reduced efficiency). Previous experiments used gas cells to change the relative phase and yielded only very moderate contrast in the quantum interference. We apply a Mach-Zehnder type interferometer to align and measure the relative phase (rp. the delay) between the laser pulses. Our setup permits observation of very pronounced quantum interferences with a large modulation depth and a large number of oscillation cycles. By measurements of fluorescence decay we also monitor the atomic population, driven by the ultra-short laser pulses to the excited state in Xenon. The atomic population shows the same quantum interferences dependent upon the phase of the driving laser pulses as the resonantly-enhanced frequency conversion processes.

\vspace{\baselineskip}
