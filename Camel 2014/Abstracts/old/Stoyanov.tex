\title{BRIGHT BEAM SELF-FOCUSING CONTROLLED BY SINGULAR DARK BEAMS}

\underline{L. Stoyanov} \index{Stoyanov L}

{\normalsize{\vspace{-4mm}

\unisofia

\email l\_stoja@yahoo.com}}

Propagation of optical beams in nonlinear media (NLM) has been a subject of continuing interest for more than four decades, partially due to the possibility for creation of reconfigurable waveguides through the intensity-dependent refractive index change [1,2]. Particular interest in singular dark beams (optical vortices, one-dimensional dark beams and ring dark waves) is motivated by their ability to propagate as dark spatial solitons or dark solitary waves and to induce gradient optical waveguides in bulk self-defocusing NLM [1-4]. Necessary (but not sufficient) condition for this is to propagate them in a self-defocusing NLM, in which the dark beam diffraction is compensated for by the medium's nonlinearity. In contrast, in self-focusing NLM  the positive nonlinearity leads to an accelerated dark beam broadening and energy density redistribution on the host background beam. As a result, controllable self-focusing of the bright structures on the background could be initiated [5-7].

In this work we show experimental results and computer simulations confirming this hypothesis. Nonlocal and anisotropic self-focusing photorefractive nonlinear medium (crystal SBN) is considered. Geometry-controlled condition used is the dark ring radius and the presence/absence of an on-axis optical vortex. In this way we succeeded to control the self-focusing longitudinal speed and the type of self-focusing structure (single peak or bright ring of variable radius). The experimental data are in fairly good agreement with the numerical simulations.

{\normalsize
[1] G. I. Stegeman and M. Segev, Science \textbf{286}, 1518-1523 (1999).
\vsp

[2] Yu. S. Kivshar and B. Luther-Davies, Phys. Rep. \textbf{298}, 81-197 (1998).
\vsp

[3] G. Swartzlander, Jr. and C. Law, Phys. Rev. Lett. \textbf{69}, 2503-2506 (1992).
\vsp

[4] C. T. Law, X. Zhang, and G. A. Swartzlander, Jr., Opt. Lett. \textbf{25}, 55-57 (2000).
\vsp

[5] P. Hansinger, A. Dreischuh, and G. G. Paulus, Opt. Commun. \textbf{282}, 3349-3355 (2009).
\vsp

[6] P. Hansinger, A. Dreischuh, and G. G. Paulus, Appl. Physics B \textbf{104}, 561-567 (2011).
\vsp

[7] G. Maleshkov, P. Hansinger, A. Dreischuh, and G. G. Paulus, Proc. SPIE 7747, art. N 77471P (2011).
}

\vspace{\baselineskip}
