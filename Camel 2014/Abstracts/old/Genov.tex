\title{UNIVERSAL BROADBAND COMPOSITE SEQUENCES OF ARBITRARY PULSES}

\underline{G. T. Genov} \index{Genov G}

{\normalsize{

\vspace{-4mm} \unisofia

\email genko.genov@phys.uni-sofia.bg}}

Composite pulse sequences have been used for several decades in nuclear magnetic resonance , and
since recently, in quantum information processing as a versatile control tool for quantum systems,
e.g. robust population transfer, suppression of excitation channels in multistate quantum systems,
etc. The main goal of composite pulses is to improve the performance of a single pulse by applying a
sequence of pulses with suitably chosen phases. In this way, errors from various sources are
canceled by the destructive interference of their effects in each of the constituent pulses. A
common drawback of the existing composite pulses is that they usually compensate deviations in a
single parameter only (e.g. pulse duration, pulse amplitude, detuning) or simultaneous deviations in
at most two-three parameters. Furthermore, the optimal phases of the constituent pulses often depend
on their pulse shape.

We propose broadband composite pulses for robust population transfer, which compensate deviations in
any experimental parameter and are applicable with any constituent pulse shape, i.e. universal
broadband composite sequences of arbitrary pulses. We describe a general procedure for the
derivation of such universal broadband sequences and illustrate their application for rectangular,
truncated Gaussian and pulses of an arbitrary random shape in the presence of Stark shift and/or a
chirp. Finally, we discuss their possible application for dynamical decoupling for coherent data
storage. 

\vspace{\baselineskip}
