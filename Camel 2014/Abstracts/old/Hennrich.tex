\title{EXPERIMENTAL QUANTUM INFORMATION PROCESSING WITH TRAPPED (RYDBERG) IONS}

\underline{M. Hennrich} \index{Hennrich M}

{\normalsize{\vspace{-4mm}
University of Innsbruck,
Experimental Physics,
Technikerstr. 25,
6020 Innsbruck,
Austria

\email markus.hennrich@uibk.ac.at}}

Strings of trapped ions form a useful system to realize quantum algorithms and quantum simulations. Using a combination of high-fidelity gate operations with optical pumping we can currently implement unitary and non-unitary operations with up to 14 qubits and up to 100 gate operations.

In the first part of this talk, I will present some of our recently realised algorithms using this technology, in particular the realization of open-system and digital quantum simulations, and the quantum error correction of a measurement projection.

In the second part of my talk I will describe a novel approach that will bring together two technologies for quantum systems: trapped ions and Rydberg atoms. This idea promises to combine the advanced quantum control of trapped ions with the strong dipolar interaction between Rydberg atoms. Joining them will form a novel quantum system with advantages from both sides. Here, I will describe our progress towards realising such an experimental system.

\vspace{\baselineskip}
