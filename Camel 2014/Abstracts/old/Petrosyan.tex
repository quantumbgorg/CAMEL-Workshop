\title{RETRIEVAL OF A STORED SINGLE PHOTON IN MULTIPLE TEMPORAL MODES WITHIN HOLLOW FIBER}

\underline{S. Petrosyan} \index{Petrosyan S}

{\normalsize{\vspace{-4mm}
Institute for Physical Research, Armenia, Ashtarak 2

\email shushanpet@gmail.com}}

Here we propose a method of preparing a single photon (SP) in temporally delocalized entangled
modes which can be transferred over significantly large distances without appreciable losses [1].
Such states, which allow sharing quantum information among many users, are highly demanded
for long-distance quantum communication. At first, we consider the storage of a short narrow-
band SP pulse in a cold O-type atoms confined inside a hollow core of a single-mode photonic-
crystal fiber, while the incoming photon is in the fiber mode. We use the robust far off-resonant
Raman interaction with SP and control laser fields. This architecture offers a drastic enhancement
of the atom-photon interaction due to increasing the electric field amplitude of SP, while the off-
resonant nature of the scheme makes the system immune to spontaneous losses. The duration of the
SP pulse is also smaller than the inverse relaxation rate of ground state atomic coherence. Therefore, even for sub-microsecond pulses, all relaxations can be neglected thus ensuring the robust and efficient storage of the SP pulse. We derive the SP propagation law in general case, when the control field is an arbitrary function of time. Upon the SP pulse entering the medium, the control field is adiabatically switched off and after a programmable delay a train of readout control pulses is applied, which coherently recalls the stored photon in many well-separated temporal modes, thus producing a single-photon state entangled in multi-time-bins with the amplitudes easily controlled by the intensities and relative phases of readout pulses. We show also the results for cw control field, when the outgoing photon displays temporal oscillations, which are the quantum counterpart of classical field ringing [2].

{\normalsize
[1] H. de Riedmatten \textit{et al.}, Phys. Rev. Lett. \textbf{92}, 047904 (2004).
\vsp

[2] J. Rothenberg \textit{et al.}, Phys. Rev. Lett. \textbf{53}, 552 (1984).
}

\vspace{\baselineskip}
