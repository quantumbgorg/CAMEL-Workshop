\title{ENTANGLEMENT PROPERTIES OF A SPIN-1/2 DIAMOND CHAIN}

\underline{L. Chakhmakhchyan} \index{Chakhmakhchyan L}

{\normalsize{
\vspace{-4mm}
\dijon

\vspace{-4mm}
Institute for Physical Research, 0203 Ashtarak-2, Armenia

\email levon.chakhmakhchyan@u-bourgogne.fr}}

We consider the entanglement properties of a generalized symmetrical spin-1/2 diamond chain with various competing interactions. The system can be used for description of the natural mineral azurite (Cu$_3$(CO$_3$)$_2$(OH)$_2$). Since the rigorous theoretical treatment of geometrically frustrated quantum Heisenberg model is unattainable, we adopt the approximation of the Ising-Heisenberg diamond chain model. The latter introduces Ising spins at the nodal sites and the Heisenberg dimers on the interstitial decorating sites of the chain.  Because of the separable nature of the Ising-type exchange interactions between neighboring Heisenberg dimers, calculation of the entanglement can be performed separately for each of them. For quantifying the thermal entanglement of the system we use the concurrence. Note that despite the long-standing interest towards the properties of the mineral azurite, the question of the type and strength of interactions between the Cu$^{2+}$
ions is still open. Thus, a detailed analysis of the phase structure of the system in a wide range of exchange interaction strengths is strongly motivated. Particularly, we reveal various regimes, depending on the relation between competing interaction strengths, with different values of ground state entanglement. Finally, some novel effects, such as the two-peak behavior of concurrence versus temperature and coexistence of phases with different values of magnetic entanglement, are pointed out.

\vspace{\baselineskip}
