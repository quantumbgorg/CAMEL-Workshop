\title{MICROWAVE-CONTROLLED MULTI-QUBIT INTERACTIONS WITH TRAPPED IONS}

\underline{M. Kohnen} \index{Kohnen M}

{\normalsize{\vspace{-4mm}}
Physikalisch-Technische Bundesanstalt,
QUEST Institut,
Bundesallee 100,
D-38116 Braunschweig,
Germany

\email kohnen@iqo.uni-hannover.de}

Spin-dependent interactions as used extensively in trapped-ion Quantum Information Processing (QIP) can be mediated by shared motional states and sideband transitions. The latter couple the motional states to the internal states of each ion. Experimentally, they require a steep gradient of the driving field over the ion's motional wave packet, which is typically realized using laser light.
We present a design for a planar chip geometry capable of creating such a field configuration with a single microwave electrode. Key advantages of using microwave near-fields instead of laser fields are the lack of spontaneous-emission decoherence, low motional-state dependence, potentially superior classical control and integration for scalability [1].
We have performed efficient and accurate numerical simulations (see also [2]) of microwave guiding structures, taking into account the current distribution in the electrode itself as well as the currents induced in nearby conductors. We use our simulations to optimise the electrode dimensions with respect to the ratio of sideband to carrier excitations addressed by the microwave fields.
Furthermore, we point out applications of these techniques in the context of a CPT test with single (anti-) protons [3].

{\normalsize
[1] C. Ospelkaus et al., Nature \textbf{476}, 181-184 (2011).
\vsp

[2] D. T. C. Allcock et al., Appl. Phys. Lett. \textbf{102}, 044103 (2013).
\vsp

[3] D. J. Heinzen and D. J. Wineland, PRA \textbf{42}, 2977 (1990); D. J. Wineland et al., J. Res. NIST.
}

\vspace{\baselineskip}
