\title{SPIN SPLITTING AND ZITTERBEWEGUNG OF RELATIVISTIC PARTICLES}

\underline{T. G. Tenev}
\index{Tenev T}

{\normalsize{
\vspace{-4mm}\unisofia

\email tenev@phys.uni-sofia.bg}}

The behavior of relativistic particles in an electric and/or magnetic field is considered in the
general case when the direction of propagation may differ from the direction of the field. For both
neutral and charged particles, the Larmor frequency shows a longitudinal motional red shift. For a
neutral particle, there is a dynamical upper bound; moreover, the transverse motion leads to a blue
shift of the Larmor frequency. For a charged particle, the relativistic spin splitting depends on
the Landau levels and decreases for higher Landau levels, thereby signalling the presence of a
Landau ladder red shift effect.

Furthermore Zitterbewegung of neutral relativistic particles propagating along a constant magnetic
and/or electric field is studied. It is shown that spin Zitterbewegung, when superimposed on the
Larmor precession frequency, leads to a beating pattern. The existence of a forbidden frequency of
spin precession is predicted. Modifications of position and velocity Zitterbewegung due to lifted
spin degeneracy manifested in the appearance of longitudinal and transversal Zitterbewegung, each
with two Zitterbewegung frequencies and resulting beating patterns, are reported.

\vspace{\baselineskip}
