\title{THE QUANTUM DYNAMICS EXPERIENCED BY A SINGLE MOLECULAR EIGENSTATE}

\underline{M. Shapiro} \index{Shapiro M}

{\normalsize{\vspace{-4mm}}
Dept. of Chemistry,
University of British Columbia,
2036 Main Mall,
Vancouver BC V6T1Z1,
Canada

\email mshapiro@chem.ubc.ca}

Contrary to conventional wisdom that all dynamics is a result of interference (or ``dephasing'')
between many (at least 2) energy eigenstates, we show that when a continuum of states
is present, even a single molecular eigenstate undergoes ``steady-state'' quantum dynamics.
Moreover, this type of dynamics can be initiated by incoherent (e.g., solar) light sources.
Continua are invariably involved in molecular systems due to a variety of sources such as the
ever present bath modes; spontaneously emitted photons; the detachment of electrons; or
the dissociation of chemical bonds. Contrary to a single bound energy-eigenfunction which
is a real (``standing-waves'') function that carries no flux, hence has no dynamics, a single
(complex) continuum energy-eigenfunction carries ``steady-state'' flux given by the group velocity
of the energetically narrow wave packet it represents. When this energy eigenfunction
is a multi-mode resonance embedded in a continuum via a chain of intramolecular couplings,
this dynamics may be initiated by any (light) source, and is controlled, contrary to coherent
wave packet dynamics, by the position of the resonance rather than its width.

\vspace{\baselineskip}
