\title{MOVING IONS}

\underline{K. Singer} \index{Singer K}

{\normalsize{\vspace{-4mm}}
Institut f\"{u}r Physik, Staudingerweg 7, 55128 Mainz, Germany

\email kilian.singer@uni-mainz.de}

Novel ion trap geometries for deterministic high resolution ion implantation are presented which are
obtained by highly efficient field calculation methods [1]. I will present our recent progress with
a segmented ion trap with mK laser cooled ions which serves as a high resolution deterministic
single ion source. It can operate with a huge range of sympathetically cooled ion species, isotopes
or ionic molecules. We have deterministically extracted a predetermined number of ions on demand
[2], performed transport operations [3] without exciting any motional quanta. These results a first
step in the realization of an atomic nano assembler, a novel device capable of placing an exactly
defined number of atoms or molecules into solid state substrates with sub nano meter precision in
depth and lateral position. Current state of the art production techniques do not offer these
possibilities and pose a major production problem for the realization of scaled solid state quantum
devices. The project is motivated by the quest for novel tailored solid state quantum materials
generated by deterministic high resolution ion implantation. The major goals are the deterministic
generation of colour centers or quantum dots, placing them in special geometries in order to exploit
the mutual coupling for the realization of macroscopic functional systems and interfacing them to
the macroscopic world with the help of electrode structures, single electron transistors and optical
micro cavities. Targeted applications range from quantum repeater, correlated triggered multi photon
sources, calibrated single photon sources, quantum computation circuits and sensors with
unprecedented sensitivity. In addition, I will discuss a spin-off of our efforts in ion trap design,
a custom-designed trap intended for the first realization of a single-ion heat engine capable of
working in the quantum regime [4]. Observing the Kibble-Zurek scaling law for ion crystal defect formation [5].

{\normalsize
[1] K. Singer, U. G. Poschinger, M. Murphy, P. A. Ivanov, F. Ziesel, T. Calarco, F. Schmidt- Kaler,
Rev. Mod. Phys. \textbf{82}, 2609 (2010).
\vsp

[2] W. Schnitzler, N. M. Linke, R. Fickler, J. Meijer, F. Schmidt-Kaler, and K. Singer, Phys. Rev.
Lett. \textbf{102}, 070501 (2009).
\vsp

[3] A. Walther, F. Ziesel, T. Ruster, S. T. Dawkins, K. Ott, M. Hettrich, K. Singer, F. Schmidt-
Kaler, U. G. Poschinger, Phys. Rev. Lett. \textbf{109}, 080501 (2012).
\vsp

[4] O. Abah, J. Rossnagel, G. Jacob, S. Deffner, F. Schmidt-Kaler, K. Singer, E. Lutz, Phys. Rev.
Lett. accepted, arXiv:1205.1362 (2012).
\vsp

[5] S. Ulm, J. Rossnagel, G. Jacob, C. Deg\"unther, S. T. Dawkins, U. G. Poschinger, R. Nigmatullin and
A. Retzker amd M. B. Plenio, F. Schmidt-Kaler, arXiv:1302.5343 (2013).
}

\vspace{\baselineskip}
