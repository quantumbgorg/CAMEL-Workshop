\title{CONTROL OF TRAPPED IONS AT ETH ZURICH}

\underline{J. Home} \index{Home J}

{\normalsize{\vspace{-4mm}
Schafmattstrasse 16, HPF E8,
8093 Zurich,
Switzerland

\email jhome@phys.ethz.ch}}

I will describe theoretical and experimental results from three channels of research at ETH Zurich.
In the first, a microstructured multi-zone trap has been built for controlling two-species ion
chains, with a view to exploring open-system quantum dynamics and scalable quantum computation. The
second setup involves a cryogenic surface-electrode ion trap with nanosecond switches placed close
to the trap in vacuum, allowing changes to be made to the trapping potentials much faster than the
frequencies of the ions. I will describe theoretical investigations of transport and squeezing
protocols [1]. Finally, I will describe new fabrication methods for ion traps based on photonic-
crystal-fibre technologies, which has promise for reaching a range of length scales, and provides a
unique geometry with high optical access.

{\normalsize
[1] J. Alonso \textit{et al.}, New J. Phys. \textbf{15}, 023001 (2013).
}

\vspace{\baselineskip}
