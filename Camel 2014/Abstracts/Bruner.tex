\title{ENHANCEMENT OF HIGH HARMONIC GENERATION USING TWO-COLOUR LASER FIELDS}

\underline{B. Bruner} \index{Bruner B}

{\normalsize{\vspace{-4mm}
Department of Physics of Complex Systems, Rm. 306 Weizmann Institute
of Science, Rehovot 76100, Israel

\email barry.bruner@weizmann.ac.il}}

The interaction between a strong laser field and an atomic or molecular gas can lead to the emission of coherent radiation in the XUV or x-ray region.  This process, known as high harmonic generation (HHG), is leading to the development of table top, high flux XUV or x-ray sources that are invaluable tools in many areas of ultrafast physics. However, the very low conversion efficiency from the near or mid-IR laser fields to the short wavelength HHG light poses a significant limitation for the development of these XUV sources.  We show that the use of a two-colour driving field can lead to a considerable enhancement of the HHG efficiency.  Using a tunable mid-IR (1300-1600 nm) source as a driving field and an 800 nm source as an assisting field, the enhancement occurs only when the two fields have mutually orthogonal polarizations.  The effect appears to be general and has been observed in a number of atomic and molecular systems.

\vspace{\baselineskip}
