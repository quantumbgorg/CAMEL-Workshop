\title{QUANTUM INFORMATION WITH RYDBERG BLOCKED ATOMIC ENSEMBLES}

\underline{E. Brion} \index{Brion E}

{\normalsize{\vspace{-4mm}
Laboratoire Aim\'e Cotton, Universit\'e Paris-Sud, B\^atiment 505, 91405 Orsay, France

\email etienne.brion@u-psud.fr}}

Neutral atoms are one of the candidates for the physical implementation
of a quantum processor. Most of the proposals in atomic quantum information
are based on the very strong dipole-dipole interaction which exists
between highly excited atoms (Rydberg atoms) [1]. In an
atomic ensemble, this interaction \textendash{} and the energy shifts
it results in, may be strong enough to forbid the resonant excitation
of the sample into multiply Rydberg excited states, leading to the
so-called Rydberg blockade [2]. We proposed to use this
phenomenon to encode quantum information in the highly entangled collective
symmetric states of an atomic ensemble [3,4]. This storage
procedure is robust [5] and the information thus encoded
can be processed in a very practical way, only through collective
manipulations of the ensembles [4]. In this talk, I shall
briefly summarize the basic features of our collective encoding scheme
and I will evoke some of its possible uses in quantum computation
[6], quantum simulation [7] and quantum communication [8].

{\normalsize
[1] M. Saffman, T. G. Walker, and K. Molmer, Rev. Mod. Phys. \textbf{82}, 2313 (2010).
\vsp

[2] M. D. Lukin et al., Phys. Rev. Lett. \textbf{87}, 037901 (2001).
\vsp

[3] E. Brion, A. S. Mouritzen, and K. Molmer, Phys. Rev. A \textbf{76}, 022334 (2007).
\vsp

[4] E. Brion, K. Molmer, and M. Saffman, Phys. Rev. Lett. \textbf{99}, 260501 (2007).
\vsp

[5] E. Brion et al., Phys. Rev. Lett. \textbf{100}, 110506 (2008).
\vsp

[6] E. Brion et al., Quantum Computers and Computing \textbf{10}, 26 (2010).
\vsp

[7] C. Guerlin, E. Brion, T. Esslinger, and K. Molmer, Phys. Rev. A \textbf{82}, 053832 (2010).
\vsp

[8] E. Brion, F. Carlier, V. M. Akulin, and K. Molmer, Phys. Rev. A \textbf{85}, 042324 (2012).
}

\vspace{\baselineskip}
