\title{HOW TO CONTROL ATOM-CHIP COHERENT DYNAMICS}

\underline{F. Caruso} \index{Caruso F}

{\normalsize{
\vspace{-4mm}
Physics Dept.-LENS-QSTAR, Florence University,
Via Carrara 1,
Sesto Fiorentino, 50019 Florence,
Italy

\email filippo.caruso@lens.unifi.it}}

Controlling the coherent dynamics of a quantum system is an extremely useful but very challenging task, especially in presence of
decoherence processes. Here we theoretically and experimentally apply different techniques to successfully perform it on a Rubidium
atom-chip Bose-Einstein condensate. First of all, we show how to exploit the back action of quantum measurements and strong
couplings to tailor and protect the coherent evolution of a quantum system inside a two-level subspace (quantum Zeno dynamics).
Secondly, optimal control strategies are experimentally applied to realize high-fidelity state preparation and to drive forth and back
the system through several paths in its five-level Hilbert space. These results are important steps forward in protecting and
controlling quantum dynamics, e.g. enhancing atom interferometry sensitivity, and, broadly speaking, in testing new tools for
quantum information processing.

{\normalsize
[1] F. Schafer et al., Nature Communications \textbf{5}, 3194 (2014).
\vsp

[2] C. Lovecchio et al., paper in preparation (2014).
\vsp

[3] F. Caruso et al., paper in preparation (2014).
}

\vspace{\baselineskip}
