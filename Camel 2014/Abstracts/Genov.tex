\title{EFFICIENT AND BROADBAND FREQUENCY GENERATION BY COMPOSITE CRYSTALS}

\underline{G. T. Genov} \index{Genov G T}

{\normalsize{\vspace{-4mm}
Institute of Applied Physics, Technical University of Darmstadt,
Hochschulstrasse 6, 64289 Darmstadt, Germany

\email genko.genov@physik.tu-darmstadt.de}}

Composite pulse sequences have been used for several decades in nuclear magnetic resonance and, since recently, in quantum information processing as a versatile control tool. In addition, novel universal broadband composite pulses have been introduced recently that perform robust population transfer and compensate errors in any experimental parameter for any pulse shape [1]. An important application of these sequences has been efficient and robust rephasing of atomic coherences in doped solids, which has also been verified experimentally [1].

We introduce another interesting application that uses an analogy with the universal composite pulses: composite crystals for efficient broadband sum and difference frequency generation [2]. This technique delivers high efficiency and robustness to parameter variations, e.g. when the phase matching condition is not fulfilled. It is a viable alternative to the adiabatic approaches because it requires much lower input intensity and shorter nonlinear crystals. It also works both with continuous-wave and pulsed lasers, as well as in the linear and nonlinear regimes of depleted and undepleted pumps, respectively.

{\normalsize
[1] G. T. Genov, D. Schraft, T. Halfmann, N. V. Vitanov, arXiv:1403.1201.
\vsp

[2] G. T. Genov, A. A. Rangelov, N. V. Vitanov, J. Opt. \textbf{16} 062001 (2014).
}

\vspace{\baselineskip}
