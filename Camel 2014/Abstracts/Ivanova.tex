\title{FEEDBACK COOLING OF THERMAL ATOMS USING BRAGG SCATTERING}

\underline{T. Ivanova}
\index{Ivanova T}

{\normalsize{
\vspace{-4mm} Faculty of physics, St-Petersburg State University, Ulianovskaya 3,
Petrodvorets, St. Petersburg, 198504, Russia

%\vspace{-4mm} $^{2}$ \unisofia

\email tanya00@yandex.ru}}

The new method of cooling and spatial organization of polarizable  particles in a periodic light
potential is proposed and theoretically justified. It is based on the detection of Bragg reflected
light as an error signal for a feedback loop. It has been inspired by the cavity self-organization
of atoms. The principal difference from the cavity self-organization is that instead of optical
feedback mediated by the cavity we propose to use an electronic feedback that is more flexible and
controllable by currently available facilities. Unlike in the systems such as collective atomic
recoil lasers, we assume that the probe light is extremely weakly coupled to the atoms and there is
no kick of atoms due to the probe light scattering. Nevertheless we show that the scattered light
can be used in an electronic feedback loop to control additional potential and finally to cool the
atoms. The role of the additional potential can be played by strong lasers forming an optical
lattice.
The simulation model is simplified to omit all quantum effects for both external and
internal degrees of freedom of the atoms. The numerical analysis with realistic parameters has shown
the possibility to reach the regime where the atoms get localized in the wells of the periodic
potential simultaneously losing their kinetic energy. This is the threshold process with respect to
the feedback gain. The timescale of the cooling process was shown to be comparable to the trapping
time in conventional dipole traps.

\vspace{\baselineskip}
