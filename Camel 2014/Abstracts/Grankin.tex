\title{QUANTUM STATISTICS OF LIGHT TRANSMITTED THROUGH AN INTRACAVITY RYDBERG MEDIUM}

\underline{A. Grankin} \index{Grankin A}

{\normalsize{\vspace{-4mm}
Laboratoire Charles Fabry, Institut d'Optique, 2 Avenue Fresnel, 91127 Palaiseau, France

\email andrey.grankin@institutoptique.fr}}

We theoretically investigate the quantum statistical properties of light transmitted through an
atomic medium with strong optical non-linearity induced by Rydberg-Rydberg van der Waals
interactions. In our setup, atoms are located in a cavity and non-resonantly driven on a two-photon
transition from their ground state to a Rydberg level via an intermediate state by the combination of
the weak signal field and a strong control beam. To characterize the transmitted light we
compute the second-order correlation function $g^{\left(2\right)}\left(\tau\right)$. The simulations
we obtained on the specific case of rubidium atoms suggest that the bunched or antibunched nature of
the outgoing beam can be chosen at will by appropriately tuning the physical parameters.

\vspace{\baselineskip}
