\title{IMPROVEMENT OF ADIABATICITY: COMPOSITE PULSES AND NON-HERMITIAN SHORTCUTS}

\underline{B. Torosov} \index{Torosov B}

{\normalsize{\vspace{-4mm}
Institute of Solid State Physics, Bulgarian Academy of Sciences, Sofia, Bulgaria

\email torosov@phys.uni-sofia.bg}}

We study two different methods for the optimization of adiabatic passage in two and three-level systems. The first method is based on composite pulses and the second on non-Hermitian shortcuts to adiabaticity. The technique of composite pulses uses a sequence of pulses with relative phases, used as control parameters. This allows the nonadiabatic losses to be canceled to any desired order with sufficiently long sequences, regardless of the nonadiabatic coupling, by choosing the phases between the constituent pulses appropriately. This technique is applied for the improvement of the fidelity of rapid adiabatic passage (RAP) and stimulated Raman adiabatic passage (STIRAP). On the other hand, the method of shortcuts to adiabaticity allows the adiabatic processes, which are usually considered slow, to increase their speed. We achieve this by adding a non-Hermitian term in the Hamiltonian, which cancels exactly the nonadiabatic coupling. We show how this is applied in RAP and STIRAP.

{\normalsize
[1] B. T. Torosov, S. Gu\'{e}rin and N. V. Vitanov, Phys. Rev. Lett. \textbf{106}, 233001 (2011).
\vsp

[2] B. T. Torosov and N. V. Vitanov, Phys. Rev. A \textbf{87}, 043418 (2013).
\vsp

[3] B. T. Torosov, G. Della Valle and S. Longhi, Phys. Rev. A \textbf{87}, 052502 (2013).
}

\vspace{\baselineskip}
