\title{PREPARATION OF AN ULTRACOLD MEDIUM AT EXTREME OPTICAL DEPTH}

\underline{T. Peters} \index{Peters T}

{\normalsize{\vspace{-4mm}
Institute of Applied Physics,
Hochschulstrasse 6,
64289 Darmstadt

\email thorsten.peters@physik.tu-darmstadt.de}}

Engineering strong coupling between individual photons at the few-photon level has been a long-standing goal in quantum optics. Strong coupling between photons enables single-photon switches and phase gates for quantum information processing and, e.g., also allows the generation of strongly correlated many-body photonic systems.
Interactions between photons are always mediated by a medium. Strong coupling is achieved, e.g., by placing atoms into high-Q micro-cavities, harnessing dipole-dipole interactions between Rydberg atoms, or coupling the light fields to atomic ensembles. In the latter approach, the coupling strength is determined by the opaqueness of the ensemble, i.e., the optical depth (OD).
In recent years, several experiments were proposed aiming at the all-optical generation of strongly correlated photonic systems, based on narrow-band electromagnetically induced transparency (EIT) in media of extreme OD beyond 1000. Such extreme ODs are typically not in reach with, e.g., laser-cooled atoms in magneto-optical traps. A promising approach towards extreme ODs is based on loading hollow-core fibers with atoms [1-5]. Here, light and atoms can be tightly confined over a macroscopic distance.
In this talk, we will present our recent experimental results on loading laser-cooled rubidium atoms from a magneto-optical trap into the few micron-sized core of a hollow-core photonic crystal fiber [5]. By applying an optical dipole trap inside the fiber, we prevent collisions of the cold atoms with the room-temperature fiber wall. Via comparison of spectroscopic measurements and a calculation we determine the OD of our fiber-based medium as 1000. This represents the highest OD observed so far with cold atoms on a transition relevant to EIT and sets the basis for future experiments on strongly-correlated photonic systems.

{\normalsize
[1] C. Christensen, S. Will, M. Saba, G.-B. Jo, Y.-I. Shin, W. Ketterle, and David Pritchard, Phys. Rev. A \textbf{78}, 033429 (2008).
\vsp

[2] M. Bajcsy, S. Hofferberth, V. Balic, T. Peyronel, M. Hafezi, A. S. Zibrov, V. Vuletic, and M. D. Lukin, Phys. Rev. Lett. \textbf{102}, 203902 (2009).
\vsp

[3] M., S. Hofferberth, T. Peyronel, V. Balic, Q. Liang, A. S. Zibrov, V. Vuletic, and M. D. Lukin, Phys. Rev. A \textbf{83}, 063830 (2011).
\vsp

[4] S. Vorrath, S. A. Möller, P. Windpassinger, K. Bongs, and K. Sengstock, New J. Phys. \textbf{12}, 123015 (2010).
\vsp

[5] F. Blatt, T. Halfmann, and T. Peters, Opt. Lett. \textbf{39}, 446 (2014).
}

\vspace{\baselineskip}
