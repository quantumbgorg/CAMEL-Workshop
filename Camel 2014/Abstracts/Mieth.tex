\title{SOLID-STATE, HIGH POWER, TUNABLE CW LASER SYSTEM FOR QUANTUM OPTICS APPLICATIONS}

\underline{S. Mieth} \index{Mieth S}

{\normalsize{\vspace{-4mm}
TU Darmstadt, Institut f\"{u}r Angewandte Physik, AG Halfmann,
Hochschulsstrasse 6, 64289 Darmstadt

\email Simon.Mieth@physik.tu-darmstadt.de}}

We present a compact all-solid-state laser system based on an optical parametric oscillator (OPO), pumped by a fiber laser, extended by intra-cavity sum frequency generation (SFG) and frequency stabilization, to provide intense radiation in the visible spectral regime [1]. The setup is based on a commercially available cw OPO system for mid-infrared output. The SFG and OPO processes are driven on a single periodically-poled lithium niobate crystal. Variation of the poling periods on the crystal allows for coarse wavelength tuning in a range between 605 and 616 nm. Pump wavelength tuning achieves a single-longitudinal mode tuning range of around 20 GHz. The robust, combined SFG-OPO approach is also applicable to other wavelength regimes. The system provides more than 1W output power over the full spectral range. A Pound-Drever-Hall frequency stabilization reduces the laser linewidth to the regime of 100 kHz (FWHM).

We apply the system for coherent, optical data storage in rare-earth doped solids, driven at a wavelength of 606 nm -- which otherwise is only accessible by dye lasers or larger setups for frequency mixing of two phase-locked lasers. As a demonstration, the talk presents data on storage of light pulses in atomic frequency combs, driven by the SFG-OPO system in a Pr:YSO crystal.

{\normalsize
[1] S. Mieth, A. Henderson, and T. Halfmann, Opt. Expr. \textbf{22}, 11182 (2014).
}

\vspace{\baselineskip}
