\title{ROBUST CONTROL OF LINEAR AND NON-LINEAR QUANTUM SYSTEMS BY SHAPED PULSES}

\underline{S. Gu\'erin} \index{Gu\'erin S}

{\normalsize{\vspace{-4mm}
\dijon

\email sguerin@u-bourgogne.fr}}

The control of quantum systems by external fields (such as laser and cavity) is at the heart of modern applications ranging from
quantum information processing to the control of chemical reactions.  It can be in general formulated as the transfer from an initial
to a target state that can feature photonic [1], atomic [2], molecular [3], and combined (dressed) states [4].

One of the most challenging issues is the ability to achieve a high-fidelity transfer to a given target state in a robust way with respect
to fluctuations, the partial knowledge of the system, and decoherence. We present techniques that allow the design of fields for such
a robust control, ranging from ultrafast adiabatic methods [5] to single-shot shaped pulses [6] and composite pulses [7].

Its extension to non-linear quantum systems, such as the one featuring the creation of molecular Bose-Einstein condensates from
atomic ones, is shown [8].

{\normalsize
[1] A. Gogyan, S. Gu\'erin , C. Leroy, and Yu. Malakyan, Phys. Rev. A \textbf{86}, 063801 (2012).
\vsp

[2] B. Rousseaux, S. Gu\'erin, and N.V. Vitanov, Phys. Rev. A \textbf{87}, 032328 (2013).
\vsp

[3] M. Sala,  M. Saab, B. Lasorne, F. Gatti, and S. Gu\'erin, J. Chem. Phys. \textbf{140}, 194309 (2014).
\vsp

[4] S. Gu\'erin and H.R. Jauslin, Adv. Chem. Phys. \textbf{125}, 147 (2003).
\vsp

[5] G. Dridi, S. Gu\'erin, V. Hakobyan, H.R. Jauslin, and H. Eleuch, Phys. Rev. A \textbf{80}, 043408 (2009);
S. Gu\'erin, V. Hakobyan, and H.R. Jauslin, ibid \textbf{84}, 013423 (2011).
\vsp

[6] D. Daems, A. Ruschhaupt, D. Sugny, and S. Gu\'erin, Phys. Rev. Lett. \textbf{111}, 050404 (2013).
\vsp

[7] B. T. Torosov, S. Gu\'erin, N. V. Vitanov, Phys. Rev. Lett. \textbf{106}, 233001 (2011).
\vsp

[8] S. Gu\'erin, M. Gevorgyan, C. Leroy, H.R. Jauslin, and A. Ishkhanyan, Phys. Rev. A \textbf{88}, 063622 (2013).
}

\vspace{\baselineskip}
