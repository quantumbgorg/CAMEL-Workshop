\title{COHERENT ELECTRONIC MOTION IN FEMTOSCALE: GENERATING GIANT MOLECULAR ANTENNAS}

\underline{I. Sola} \index{Sola I}

{\normalsize{
\vspace{-4mm}
Departamento de Quimica Fisica I, Universidad Complutense, 28040 Madrid, Spain

\email isola@quim.ucm.es}}

Attophysics deals with electron dynamics. In this talk we propose different schemes that employ
ultrafast (attosecond) pulses and/or static fields or low frequency pulses in order to create and
manipulate oscillating electric dipoles in homonuclear diatomic cations. We argue that any working
control mechanism needs i) to break the symmetry of the system and ii) to sustain highly correlated
electronic and nuclear motion, thus pushing the period of oscillation of the dipoles from the
electronic (attosecond) timescale to the nuclear (femtosecond) timescales.

Using reduced dimensionality models with regularized soft-core Coulomb potentials for the Hydrogen
molecular cation we show that one can create dipoles as large as 40 Debyes oscillating in the far-
infrared regime, that act as molecular antennas.

\vspace{\baselineskip}