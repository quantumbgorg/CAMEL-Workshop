\title{DISPERSION-ENHANCED THIRD-HARMONIC MICROSCOPY}
% Dispersion-enhanced third-harmonic microscopy

\underline{C. Stock} \index{Stock C}
%Christian Stock

{\normalsize{\vspace{-4mm}
\darmstadt

\email christian.stock@physik.tu-darmstadt.de}}

In the past two decades coherent nonlinear microscopy (CNM) developed into a powerful and broadly applied tool for three-dimensional imaging of transparent sample. CNM utilizes frequency conversion processes in strongly focused, ultrashort laser pulses. It even delivers images of otherwise fully transparent samples, which are not accessible by conventional linear microscopy without marking or staining. One example for CNM processes is third harmonic generation (THG). In the case of tight focussing, the Gouy phase shift in the focus causes destructive interference for third harmonic generation in a bulk sample. Thus, net THG emission occurs only at interfaces. Three-dimensional imaging is possible by scanning the laser focus across the sample. This permits high contrast 3D imaging of boundaries in heterogeneous samples.
In this talk we demonstrate strong enhancements of signal yield and image contrast in third-harmonic microscopy by appropriate choice of driving laser wavelength to modulate the phase-matching conditions of the conversion process by dispersion control [1]. Tuning the laser wavelength in the range of 1010-1350 nm at samples containing interfaces with water and glass, we obtained large signal enhancements up to a factor of 19, and improvements in the image contrast by an order of magnitude. A simple theoretical calculation, using a constant ratio of susceptibilities in the media, matches very well with the experimental data.

{\normalsize
[1] C. Stock, K. Zlatanov, T. Halfmann, Opt. Commun. \textbf{393}, 289–293 (2017).
}

\vspace{\baselineskip} 