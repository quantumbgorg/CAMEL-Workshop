\title{SINGLE ATOM EDGE-LIKE STATES VIA QUANTUM INTERFERENCE}
% Single atom edge-like states via quantum interference

\underline{V. Ahufinger} \index{Ahufinger V}
%Veronica Ahufinger

{\normalsize{\vspace{-4mm}
Grup d'Optica. Departament de Fisica. Universitat Autonoma de Barcelona. E-08193 Bellaterra, Spain

\email veronica.ahufinger@uab.cat}}

We demonstrate that quantum interference may lead to the generation of robust edge-like states (ELS) of a single ultracold atom in two-dimensional optical ribbons [1]. These states can be engineered either within the manifold of local ground states of the sites forming the ribbon, or of states carrying one unit of orbital angular momentum (OAM). First, we consider a system of three in-line sites and we show that in this system quantum interference effects give rise to spatial dark states (SDS). Then, by using the SDS as basic building blocks, global ELS can be created in arbitrarily large ribbons. These ELS are very robust against defects of the ribbon and perturbations in the phase differences between the local eigenstates of the sites. For the manifold of local ground states, quantum interference effects are solely due to phase differences in the local states of the sites, allowing to create ELS in a large variety of geometrical configurations. In addition, the different ELS could be coupled with laser pulses inducing  oscillations  between global eigenstates of the ribbon. In the case of states carrying one unit of OAM, quantum interference is due to complex tunneling amplitudes, whose phases are modulated by the relative orientation between sites [2] and we suggest to use the winding number associated to the angular momentum as a synthetic dimension.

{\normalsize
[1] G. Pelegrí et al., Phys. Rev. A \textbf{95}, 013614 (2017);
\vsp

[2] J. Polo et al., Phys. Rev. A \textbf{93}, 033613 (2016).
}

\vspace{\baselineskip} 