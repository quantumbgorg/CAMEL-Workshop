\title{QUANTUM DYNAMICS OF TWO COUPLED SPINS UNDER CONTROLLABLE AND FLUCTUATING MAGNETIC FIELDS}
% Quantum dynamics of two coupled spins under controllable and fluctuating magnetic fields

\underline{R. Grimaudo} \index{Grimaudo R}
%Roberto Grimaudo

{\normalsize{\vspace{-4mm}
Dipartimento di Fisica e Chimica, Universita degli Studi di Palermo, Palermo, Italy.

\email bobogrima@gmail.com}}

The quantum dynamics of two spins $\hat{\mathbf{j}}_1$ and $\hat{\mathbf{j}}_2$ (with values $j_1\geq j_2$), subjected to external and controllable time-dependent magnetic fields and under a $\hat{\mathbf{J}}^2=(\hat{\mathbf{j}}_1+\hat{\mathbf{j}}_2)^2$-conserving bilinear coupling is investigated.
Each eigenspace of $\hat{\mathbf{J}}^2$ is dynamically invariant and the Hamiltonian of the total system restricted to any one of such ($2j_2+1$) eigenspaces, possesses the SU(2) structure of the Hamiltonian of a single fictitious spin acted upon by the given controllable magnetic field.
We show that such a reducibility holds regardless of the time-dependence of the externally applied field as well as of the statistical properties of the Overhauser noise, here represented as a classical fluctuating magnetic field.
Exploiting such a remarkable result, the time evolution of the joint transition probabilities of the two spins $\hat{\mathbf{j}}_1$ and $\hat{\mathbf{j}}_2$ between two prefixed factorized states is examined, bringing to light peculiar dynamical properties of the system under scrutiny.
In particular, when the noise-induced non unitary dynamics of the two coupled spins is properly taken into account, the paper reports explicit analytical expressions for the joint Landau-Zener transition probabilities.
The possibility of taking advantage of these exact results to envision a feedback-test on the reliability of the modelling adopted for unavoidable environmental effects in a given set-up, is finally briefly discussed.

\vspace{\baselineskip} 