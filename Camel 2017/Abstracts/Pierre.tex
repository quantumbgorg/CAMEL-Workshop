\title{FAST THERMALIZATION OF A FROZEN RYDBERG GAS IN LONG-RANGE INTERATOMIC DIPOLE-DIPOLE COUPLING}
% Fast thermalization of a frozen Rydberg gas in long-range interatomic dipole-dipole coupling

\underline{P. Pillet} \index{Pillet P}
%Pierre Pillet

{\normalsize{\vspace{-4mm}
Laboratoire Aime Cotton, CNRS, Univ Paris-Sud, ENS Paris-Saclay
Bat. 505, Campus d'Orsay, 91405 Orsay cedex, France

\email pierre.pillet@u-psud.fr}}

The long-range dipole-dipole interactions in a gas of cold atoms lead to numerous possible applications for instance the formation of ultra-cold molecules via the photoassociation of cold atoms, or the realization of scalable quantum gates via the dipole blockade of the Rydberg excitation, and the realization of Rydberg ``crystals'' again using the dipole blockade. The origin of such an interest is contained in the exaggerated properties of the Rydberg atoms, which can acquire huge electric momentum scaling up to $n^2$ in atomic units, meaning a few thousand times those of very polar molecules. Today the accurate control of such interatomic interactions in ultracold gases or quantum gases opens the way to a very active research field in quantum simulation, where the frozen Rydberg atoms permit one, the realization of many physical configurations, mimicking problems in condensed matter physics.

We consider here the case of a cold, disordered and dense cesium Rydberg gas in a configuration of resonance of Foerster [1,2], where two atoms or a group of atoms exchange internal energy by a resonant way. We will discuss the case of the two-body Foerster resonance, and we introduce the one of three-body Foerster resonance, also qualified as Borromean [3] because occurring in the absence of any effect due two-body Foerster resonances. In the case of the two-body Foerster resonance, we can reach the saturation regime, characterized by a quite amazing behavior corresponding to the ``thermalization'' of the atomic sample, meaning an equal- distribution of the populations of the relevant level of the resonance. The dynamics of the thermalization seems to be the result of few-body effects. The interplay between two-, few- [3,4] and many-body regime in a dipole coupling dense frozen Rydberg gas will be discussed.

{\normalsize
[1] B. Pelle, R.Faoro, J. Billy, E. Arimondo, P. Pillet, P. Cheinet, Physical Review A \textbf{93}, 3417 (2016);
\vsp

[2] W. Maineult, B. Pelle, R. Faoro, E. Arimondo, P. Pillet, P. Cheinet, J. Phys. B: At. Mol. Opt. Phys. \textbf{49}, 214001 (2016);
\vsp

[3] R. Faoro, B. Pelle, A. Zuliani, P. Cheinet, E. Arimondo, P. Pillet, Nature Communications \textbf{6}, 8173 (2015);
\vsp

[4] J.H. Gurian, P. Cheinet, P. Huillery, A. Fioretti, J. Zhao, P.L. Gould, D. Comparat, P. Pillet, Phys Rev. Lett. \textbf{108}, 023005 (2012).
}

\vspace{\baselineskip} 