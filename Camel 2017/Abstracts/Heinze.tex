\title{INTERFACING DISPARATE QUANTUM MEMORY SYSTEMS FOR HYBRID QUANTUM OPTICS EXPERIMENTS}
% Interfacing disparate quantum memory systems for hybrid quantum optics experiments

\underline{G. Heinze} \index{Heinze G}
%Georg Heinze

{\normalsize{\vspace{-4mm}
ICFO -- The Institute of Photonic Sciences,
Av. Carl Friedrich Gauss 3,
08860 Castelldefels (Barcelona), Spain

\email georg.heinze@icfo.eu}}

The interconnection of fundamentally different quantum platforms via photons is a key requirement to build future hybrid quantum networks. Such heterogeneous architectures hold promise to offer more powerful capabilities than their homogeneous counterparts, as they would benefit from the individual strengths of different quantum matter systems.

We report on our recent experiments interfacing a cold atomic ensemble of Rubidium atoms –operated as read-only quantum memory (QM)– with two other very distinct quantum platforms. First, we demonstrate storage and retrieval of a paired single photon, emitted from the cold Rb QM, on a highly excited Rydberg state of a second, separate Rb ensemble via electromagnetically induced transparency. We show that nonclassical correlations between both photons persist after the storage and retrieval process. Second, we interface the paired single photon from the cold Rb QM with a rare-earth ion-doped solid state QM. As both systems exhibit very different optical transitions, we apply cascaded frequency conversion techniques to bridge the wavelength gap and moreover transmit the single photon at telecom wavelength which is favorable for long distance communication. We demonstrate that the coherence of the single photon is preserved after frequency conversion and storage and retrieval from the solid state QM. Finally, we show qubit transfer between the two fundamentally different QM systems with fidelities surpassing the classical threshold.

\vspace{\baselineskip} 