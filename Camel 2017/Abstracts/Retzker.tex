\title{SUB-MILLIHERZ MAGNETIC SPECTROSCOPY WITH A NANOSCALE QUANTUM SENSOR}
% Sub-milliherz magnetic spectroscopy with a nanoscale quantum sensor

\underline{A. Retzker} \index{Retzker A}
%Alex Retzker

{\normalsize{\vspace{-4mm}
Edmond J. Safra campus.,
Danciger B Building, room 203,
Hebrew University of Jerusalem,
Israel 91904

\email retzker@phys.huji.ac.il}}

Precise timekeeping is critical to metrology, forming the basis by which standards of time, length and fundamental constants are determined. Stable clocks are particularly valuable in spectroscopy as they define the ultimate frequency precision that can be reached. In quantum metrology, where the phase of a qubit is used to detect external fields, the clock stability is defined by the qubit coherence time, and therefore determines the spectral linewidth and frequency precision. I will present a demonstration of a quantum sensing protocol for oscillating fields where the spectral precision goes beyond the sensor coherence time and is limited by the stability of a classical clock. Using this technique, we observe a precision in frequency estimation scaling as $1/T^{3/2}$ for classical fields. The narrow linewidth magnetometer based on single quantum coherent spins in diamond is used to sense magnetic fields with an intrinsic frequency resolution of 607 µHz, 8 orders of magnitude narrower than the qubit coherence time.

\vspace{\baselineskip} 