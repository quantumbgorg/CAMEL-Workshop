\title{ARBITRARILY ACCURATE PULSE SEQUENCES FOR ROBUST DYNAMICAL DECOUPLING}
% Arbitrarily Accurate Pulse Sequences for Robust Dynamical Decoupling

\underline{G. Genov} \index{Genov G}
%Genko Genov

{\normalsize{\vspace{-4mm}
\darmstadt

\email genko.genov@physik.tu-darmstadt.de}}

Quantum technologies are increasingly important nowadays for sensing, processing, and communication of information. However, protection of quantum systems from the environment remains a major challenge. Dynamical decoupling (DD) is a practical and widely used approach that aims to achieve this goal by applying sequences of pulses. Most DD schemes focus on dephasing processes due to their high contribution to data loss. Then, pulse imperfections remain the main challenge. Robust DD sequences have also been designed but most compensate errors in one or two parameters only (e.g., flip angle) or for a specific initial state. Ultra-high fidelity can potentially by achieved by nesting of sequences, but this requires an exponential growth in the number of pulses.

We introduce universally robust sequences for dynamical decoupling, which simultaneously compensate pulse imperfections and the detrimental effect of a dephasing environment to an \emph{arbitrary} order, work with any pulse shape, and improve performance for any initial condition [1]. Moreover, the number of pulses in a sequence grows only linearly with the order of error compensation. Our sequences outperform the state-of-the-art robust DD sequences. Beyond the theoretical proposal, we also present convincing experimental data for dynamical decoupling of atomic coherences in a solid-state optical memory.

{\normalsize
[1] G. T. Genov, D. Schraft, N. V. Vitanov, and T. Halfmann, Phys. Rev. Lett. \textbf{118}, 133202 (2017).
}

%{\normalsize
%}

\vspace{\baselineskip} 