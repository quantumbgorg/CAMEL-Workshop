\title{TOWARDS A HIGH RESOLUTION RB$^+$ FIB}
% Towards a high resolution Rb$^+$ FIB

\underline{E. Vredenbregt} \index{Vredenbregt E}
%Edgar Vredenbregt

{\normalsize{\vspace{-4mm}
Coherence and Quantum Technology group,
Department of Applied Physics,
Eindhoven University of Technology,
PO Box 513, 5600 MB Eindhoven,
Netherlands

\email e.j.d.vredenbregt@tue.nl}}

The atomic beam laser-cooled ion source (ABLIS) project [1] is aimed at creating an ultracold rubidium ion beam by the photo-ionisation of a laser-cooled and compressed thermal atomic beam. This ion beam will be used in a new focused ion beam (FIB) apparatus. Such systems are widely used in science and industry for sample inspection and manipulation at the nano-scale. Current gallium liquid metal ion source (LMIS) based FIBs can image and alter structures with a resolution down to 5 nm at a beam current of 1 pA and a beam energy of 30 keV. However, with the decrease in feature size achievable with photo-lithographic techniques, the FIB spot size needs to shrink as well in order to keep circuit edit and repair viable. A smaller spot size in a FIB can be realised by increasing the ion beam brightness and/or by decreasing its longitudinal energy spread. The ABLIS project aims at creating an ion source that achieves both. In short, laser cooling and compression is applied to a thermal atomic beam of rubidium atoms to achieve the desired brightness. Then, two-step photo-ionisation in an electric field is applied to turn all atoms into ions while minimising heating due to Coulomb effects, thus preserving the brightness [2]. Keeping the ionisation volume small ensures that the longitudinal energy spread is minimised. Note that other research groups are working on ion sources based on similar ideas [3]. In this contribution an overview is given of the experimental realisation of the atomic beam laser-cooled ion source [4]. A high degree of ionisation is achieved in order to generate an ion beam with similar brightness as the atomic beam. Furthermore, to minimise the longitudinal energy spread that is introduced due to the accelerating electric field, ionisation should take place in an as small as possible length. Experiments were carried out to find the highest ionisation degree and to verify the modelling. Apart from the beam brightness, the longitudinal energy spread of the ion beam plays an important role in the formation of a small spot due to chromatic abberations in the electrostatic lens column. Here, a retarding field analyser was used to measure the energy spread. Currently, steps are taken to mount the ion source on a commercial FIB system.

{\normalsize
[1] S. H. W. Wouters et al., Phys. Rev. A \textbf{90}, 063817 (2014);
\vsp

[2] G. ten Haaf et al., J. Appl. Phys. \textbf{116}, 244301 (2014);
\vsp

[3] J. J. McClelland et al., Appl. Phys. Rev. \textbf{3}, 011302 (2016);
\vsp

[4] G. ten Haaf et al., Phys. Rev. Appl. \textbf{7}, 054013 (2017).
}

\vspace{\baselineskip} 