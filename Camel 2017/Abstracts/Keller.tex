\title{INTERFACING IONS AND PHOTONS}
% Interfacing ions and photons

\underline{M. Keller} \index{Keller M}
%Matthias Keller

{\normalsize{\vspace{-4mm}
University of Sussex,
Department of Physics and Astronomy,
Pevensey 2,
Brighton BN1 9QH,
UK

\email m.k.keller@sussex.ac.uk}}

The complementary benefits of trapped ions and photons as carriers of quantum information make it appealing to combine them in a joint system. Ions provide low decoherence rates, long storage times and high readout efficiency, while photons travel over long distances. To interface the quantum states of ions and photons efficiently, we use calcium ions coupled to an optical high-finesse cavity via a Raman transition.
To achieve strong ion-cavity coupling we employ fibre tip cavities integrated into the electrodes of an endcap style ion trap. With a cavity length of 380 mm the resulting ion-cavity coupling strength is 18 MHz with a cavity line width of 10 MHz. We trap single calcium ions with a life time of several hours and have optimised the ion-cavity overlap to observe the interaction of the cavity with the ion.
In another experiment, we combine a conventional cavity with a linear ion trap to facilitate the investigation of the interaction of multiple ions with a single cavity mode. We have demonstrated the localisation of several ions in a collinear cavity-trap system and have demonstrated the emission of polarised single photons from this system.
To enable the use of fibre cavities in applications such as single photon sources and nodes in quantum networks the coupling between the cavity and the fibre must be significantly improved. We have developed a system to integrate mode matching optics into a fibre system and have demonstrated a mode matching between cavity and fibre on the order of 90%.


\vspace{\baselineskip} 