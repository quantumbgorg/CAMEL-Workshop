\title{GENERATION OF NON-ABELIAN GEOMETRIC PHASES IN DEGENERATE ATOMIC TRANSITIONS}
% Generation of non-Abelian geometric phases in degenerate atomic transitions

\underline{L. Simeonov} \index{Simeonov L}
%Lachezar Simeonov

{\normalsize{\vspace{-4mm}
\unisofia

\email lsimeonov@phys.uni-sofia.bg}}

A popular quantum system, in which the Pancharatnam-Berry non-Abelian geometric phase has
been generated and exploited, is the atomic tripod system. It is conveniently created by linking a
single atomic state with three other states by three electromagnetic fields. Such a linkage pattern
naturally emerges between the magnetic sublevels of two atomic levels with angular momenta $J = 0$
and $J = 1$, although tripod implementations between other suitable sublevels are also used. Here
we go beyond the limitation of a tripod system and show that it is possible to generate the non-
Abelian geometric phase in a quantum system composed of $N$ lower and $N-2$ upper sublevels.
The theoretical instrument is the Morris-Shore transformation which reveals the existence of two
uncoupled (dark) states composed of the lower sublevels only. A possible physical implementation
is the atomic transition $J$ to $J-1$, with $J$ arbitrary, which is driven, as in the case of tripod system,
by three electromagnetic fields of different polarizations. This generalization considerably broadens the range of systems that can be used to generate a geometric phase, with the the same experimental complexity as in the tripod system. Specific calculations of the non-Abelian geometric phase are presented for $J = 3/2$ to $J=1/2$ and $J=2$ to $J =1$ systems.  A method for measuring the geometric
phase is proposed.

\vspace{\baselineskip} 