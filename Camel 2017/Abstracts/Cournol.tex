\title{RO-VIBRATIONAL COOLING OF MOLECULES}
% Ro-vibrational cooling of molecules

\underline{A. Cournol} \index{Cournol A}
%Anne Cournol

{\normalsize{\vspace{-4mm}
Laboratoire Aime Cotton,
Bat. 505 Campus d'Orsay,
Orsay Cedex,
France

\email anne.cournol@u-psud.fr}}

Laser techniques applied to precision spectroscopy or to the control of internal and external degrees of freedom have considerably improved our knowledge on molecular physics. One of the greatest challenges of modern physical chemistry is to push forward the limits of electromagnetic and laser techniques to manipulate and probe molecules at low temperatures where molecular interactions are dominated by pure quantum phenomena.

In this context we have developed an original technique that enables us to manipulate the internal degrees of freedom of diatomic molecules. The principle consists in using broadband lasers to pump population of all the internal levels towards a target level. We have performed such technique on cesium dimers [1]. We apply this method to ro-vibrationally cool barium monofluoride (BaF) molecules produced in a supersonic beam. BaF is a good candidate as it has a suitable vibrational structure [2].

{\normalsize
[1] I. Manai, R. Horchani, H. Lignier, P. Pillet, D. Comparat, A. Fioretti and M. Allegrini, Phys. Rev. Lett. \textbf{109}, 183001 (2012);
\vsp

[2] C. Effantin, A. Bernard, J. d'Incan, G. Wannous, J. Verges and R. F. Barrow, Mol. Phys. \textbf{70}, 735-745 (1990).
}

\vspace{\baselineskip} 