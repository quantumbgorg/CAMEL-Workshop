\title{CONSTRUCTING A MICROWAVE TRAPPED ION QUANTUM COMPUTER}
% Constructing a microwave trapped ion quantum computer

\underline{W. Hensinger} \index{Hensinger W}
%Winfried Hensinger

{\normalsize{\vspace{-4mm}
Sussex Centre for Quantum Technologies,
Department of Physics and Astronomy, University of Sussex,
Brighton, East Sussex, BN1 9QH, United Kingdom

\email w.k.hensinger@sussex.ac.uk}}

Trapped ions are a promising tool for building a large-scale quantum computer. The number of radiation fields (such as lasers) required for the realisation of quantum gates in any proposed ion-based architecture scales with the number of ions inside the quantum computer, posing a major challenge when imagining a device with millions of qubits. Here I present a fundamentally different approach [1], where this scaling entirely vanishes. The method is based on individually controlled voltages applied to each logic gate location to facilitate the actual gate operation analogous to a traditional transistor architecture within a classical computer processor. Instead of aligning numerous laser beams into designated entanglement zones, the use of a single microwave source outside the vacuum system is sufficient. We have demonstrated the key principle of this approach by implementing a two-qubit quantum gate based on long-wavelength radiation where we generate a maximally entangled two-qubit state with fidelity 0.985(12). I will also discuss the engineering blueprint for a large-scale microwave trapped-ion quantum computer we have recently released [2]. The work features a new invention permitting actual quantum bits to be transmitted between individual quantum computing modules using electric fields in order to obtain a fully modular large-scale machine. Finally I will introduce a powerful technique to transform existing two-level quantum control methods to new multi-level quantum control operations and illustrate the process using experiments with trapped ions.

{\normalsize
[1] S. Weidt, J. Randall, S. C. Webster, K. Lake, A. E. Webb, I. Cohen, T. Navickas, B. Lekitsch, A. Retzker, and W. K. Hensinger, Phys. Rev. Lett. \textbf{117}, 220501 (2016);
\vsp

[2] B. Lekitsch, S. Weidt, A.G. Fowler, K. Molmer, S.J. Devitt, Ch. Wunderlich, and W.K. Hensinger, Science Advances \textbf{3}, e1601540 (2017).
}

\vspace{\baselineskip} 