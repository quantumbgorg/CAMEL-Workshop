\title{HARMONIC GENERATION IN GAS-FILLED WAVEGUIDES IN THE VICINITY OF MULTI-PHOTON RESONANCES}
% Harmonic generation in gas-filled waveguides in the vicinity of multi-photon resonances

\underline{X. Laforgue} \index{Laforgue X.}
%Xavier Laforgue

{\normalsize{\vspace{-4mm}
\darmstadt

\dijon

\email xavier.laforgue@physik.tu-darmstadt.de}}

Frequency conversion techniques extend the spectral region accessible by lasers from the mid-infrared towards soft x-rays. However, only gaseous media are applicable in the VUV and XUV, which reduces the conversion efficiency due to the low density in such media. Hollow-core optical waveguides permit spatial confinement of both the driving radiation as well as the medium. This provides long interaction length, while still maintaining phase-matching conditions. Tuning to multi-photon resonances offers another option to increase the nonlinear susceptibilities of the medium. We investigate phase-matched harmonic generation of ultrashort (ps) laser pulses, tuned in the vicinity of a five-photon resonance in Argon, confined in a capillary with a diameter of 100 $\mu$m. We study the dependence of the conversion efficiency, resonance shifts, and phase-matching conditions with gas pressure, driving fundamental wavelength, intensity, and admixture of buffer gas. We compare the experimental data with extended numerical simulations, taking higher spatial modes and cascade frequency conversion into account, identifying also the significant contribution of quasi-phase-matching by polarization beating. Our investigations show, that proper choice of experimental parameters enables significant resonance enhancements in the conversion efficiencies.

\vspace{\baselineskip} 