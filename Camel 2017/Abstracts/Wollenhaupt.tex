\title{BICHROMATIC CONTROL OF MULTI-PHOTON IONIZATION}
% Bichromatic control of multi-photon ionization

\underline{M. Wollenhaupt} \index{Wollenhaupt M}
%Matthias Wollenhaupt

{\normalsize{\vspace{-4mm}
Carl von Ossietzky Universitat Oldenburg,
Institut für Physik,
Carl-von-Ossietzky-Strasse 9-11,
D-26129 Oldenburg,
Germany

\email matthias.wollenhaupt@uni-oldenburg.de}}

Polarization-tailored bichromatic laser fields have emerged as new twist to steer ultrafast electron dynamics on their intrinsic timescales. Recently, we introduced a novel approach to the generation of polarization-tailored bichromatic fields, based on ultrafast pulse shaping applied to an octave-spanning CEP-stable white light supercontinuum (WLS) [1-3]. The setup provides full control over all bichromatic pulse parameters such as the frequency and amplitude ratio, the spectral phase profile (including CEP and relative phase) and the polarization state of both colors. Bichromatic pulse shaping opens up a new class of polarization-tailored waveforms with application to multi-path coherent control of ultrafast dynamics, generation and control of high harmonics and the design of polarization-sensitive two-color pump-probe experiments with phase-locked CEP-stable laser pulses at a broad range of excitation wavelengths.
In our experiments we employ polarization-shaped bichromatic fields to study resonance-enhanced multi-photon ionization (REMPI) of atoms as a prototype scenario for multi-path coherent control. Three-dimensional detection of the photoelectron momentum distribution by photoelectron imaging tomography provides detailed insights into the excitation and ionization dynamics. In addition, the generation of vortex-shaped photoelectron wave packets from single color REMPI of potassium atoms with sequences of two time-delayed CRCP few-cycle pulses is demonstrated [3].

{\normalsize
[1] S. Kerbstadt, L. Englert, T. Bayer and M. Wollenhaupt, Journal Modern Optics, \textbf{64}, 1010-1025;
\vsp

[2] S. Kerbstadt, L. Englert, T. Bayer, M. Wollenhaupt, Optics Express, (in print) (2017);
\vsp

[3] D. Pengel, S. Kerbstadt, D. Johannmeyer, L. Englert, T. Bayer and M. Wollenhaupt, Phys. Rev. Lett. \textbf{118}, 053003 (2017).
}

\vspace{\baselineskip} 