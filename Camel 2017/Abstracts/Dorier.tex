\title{REVERSE ENGINEERING CONTROL FOR MOLECULAR BOSE-EINSTEIN CONDENSATION}
% Reverse engineering control for molecular Bose-Einstein condensation

\underline{V. Dorier} \index{Dorier V}
% Vincent Dorier

{\normalsize{\vspace{-4mm}
\dijon

\email vincent.dorier@u-bourgogne.fr}}

In the second quantization formalism, the dynamics between condensed atoms and molecules is described by a Gross-Pitaevskii equation in the mean-field approximation. In this talk, we describe how to adapt the tools of control initially developed for linear quantum systems to nonlinear systems. We show in particular a method to shape the control fields of a \emph{one-step nonlinear adiabatic passage} from free atoms to ground-state molecules (as early proposed in Ref [1] without specific shaping), instead of the usual (limited) two-step passage through the production of Feshbach molecules. This new method is efficient, robust and features a transient controlled (low) population in the lossy state. It is based on an exact solution of the nonlinear Schr\"odinger equation which is tracked by the actual dynamics via the control fields. The atom-molecule 2nd order nonlinearities are thus taken into account by the control.

In such molecular systems, 3rd order nonlinearities arise from the elastic collisions between species [2]. We describe how these effects, known as \emph{mean-field shifts}, can be exactly dynamically compensated using time-dependent detunings. This method generalizes nonlinear adiabatic passage techniques [3,4] to exact passage.

{\normalsize
[1] Mackie, M., Kowalski, R., Javanainen, J., Physical Review Letters, 84, 3803 (2000);
\vsp

[2] Timmermans, E., Tommasini, P., Hussein, M., Kerman, A., Physics Reports, 315, 199-230 (1999);
\vsp

[3] Gu\'erin, S., Gevorgyan, M., Leroy, C., Jauslin, H. R., Ishkhanyan, A., Physical Review A, 88, 063622 (2013);
\vsp

[4] Gevorgyan, M., Gu\'erin, S., Leroy, C., Ishkhanyan, A., Jauslin, H. R., The European Physical Journal D, 70, 253 (2016).
}

\vspace{\baselineskip}
