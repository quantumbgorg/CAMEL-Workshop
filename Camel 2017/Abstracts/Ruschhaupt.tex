\title{NEWS ABOUT SHORTCUTS TO ADIABATICITY: SPATIAL NON-ADIABATIC PASSAGE}
% News about Shortcuts to Adiabaticity: Spatial Non-adiabatic Passage

\underline{A. Ruschhaupt} \index{Ruschhaupt A}
%Andreas Ruschhaupt

{\normalsize{\vspace{-4mm}
Department of Physics,
University College Cork,
Cork,
Ireland,

\email aruschhaupt@ucc.ie}}

Quantum technologies based on adiabatic techniques can be highly effective, but often at the cost of being very slow. In the first part, we will introduce a set of shortcut protocols for spatial state preparation, which yield the same fidelity as their adiabatic counterparts, but on fast timescales [1]. In particular, we consider a charged particle in a system of three tunnel-coupled quantum wells, where the presence of a magnetic field can induce a geometric phase during the tunnelling processes. We show that this leads to the appearance of complex tunnelling amplitudes and allows for the implementation of spatial non-adiabatic passage. We demonstrate the ability of such a system to transport a particle between two different wells and to generate a delocalised superposition between the three traps with high fidelity in short times.

In the second part, we switch back to adiabatic processes and we present a detailed derivation of the effect of Poisson Noise on adiabatic quantum control [2]. We discuss the limiting cases of Poisson white noise and provide approximations for the different noise strength regimes. We show that using the eigenstates of the noise superoperator as a basis can be a useful way of expressing the master equation. Using this, we simulate various settings to illustrate different effects of Poisson noise. In particular, we show a dip in the fidelity as a function of noise strength where high fidelity can occur in the strong-noise regime for some cases.

{\normalsize
[1] A. Benseny, A. Kiely, Y. Zhang, T. Busch and A. Ruschhaupt, EPJ Quantum Technology \textbf{4}, 3 (2017);
\vsp

[2] A. Kiely, J.G. Muga and A. Ruschhaupt, Phys. Rev. A \textbf{95} 012115 (2017).
}

\vspace{\baselineskip} 