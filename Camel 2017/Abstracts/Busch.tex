\title{CONTROLLING SUPERFLUID FLOW IN BOSE-EINSTEIN CONDENSATES}
% Controlling superfluid flow in Bose-Einstein condensates

\underline{T. Busch} \index{Busch T}
%Thomas Busch

{\normalsize{\vspace{-4mm}
OIST Graduate University,
1919-1 Tancha, Onna-son,
904-0495 Okinawa, Japan

\email thomas.busch@oist.jp}}

Superfluidity in Bose-Einstein condensates is a manifestation of long range correlations. Its hallmark is the existence of quantised vortices with their characteristic core and 1/r azimuthal velocity profile. However, other flow patterns in finite sized systems are possible, and in this presentation I will discuss two experimentally realistic examples.

In the first one I will show that the introduction of short range correlations into a superfluid system can lead to a dynamics that is reminiscent of classical rotation, where the velocity profile is directly  proportional to the radius r. The required short-range correlations exist in two-component condensates and in the phase-separation regime they lead to radial flow in azimuthally symmetric potentials.

In the second example I will show how artificial gauge fields for neutral atoms can be used to create single vortex-rings and vortex-ring-lattices in a controlled manner. The basis for this is the presence of artificial gauge fields created by the optical near field potentials around optical nano fibers, which are highly tunable and allow to study the dynamics of vortex rings in many different situations.


\vspace{\baselineskip} 