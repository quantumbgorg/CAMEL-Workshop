\title{COHERENT POPULATION OSCILLATIONS IN NV– COLOUR CENTRES IN DIAMOND}
% Coherent Population Oscillations in NV– colour centres in diamond

\underline{K. Sycz} \index{Sycz K}
%Krystian Sycz

{\normalsize{\vspace{-4mm}
Institute of Physics,
Jagiellonian University,
Lojasiewicza 11,
30-348, Poland

\email krystian.sycz@gmail.com}}

Nitrogen-vacancy (NV-) colour centres are point defects in the diamond lattice, consisting of a substitutional nitrogen atom adjacent to a lattice vacancy. A nonzero electron spin ($S=1$) of those centres allows them to be optically pumped, and then probed via microwave spectroscopy. This property makes them useful in numerous applications including electric-field, magnetic-field, pressure, and temperature sensing, as well as nanoscale NMR. Nanodiamonds can also be used as fluorescent markers or sensors in biological materials.
We present the results of our research on microwave spectroscopy in nitrogen-vacancy colour centres in a diamond. In particular, we focus on the case of two microwave fields and apply the optically-detected magnetic resonance technique to study the microwave hole burning. When both microwaves are tuned to transitions between $\text{ms}=0\leftrightarrow\text{ms}=+1$ spin sublevels of the NV-ensemble in the 3A2 ground state, the observed spectra exhibit a complex narrow structure composed of three Lorentzian resonances positioned at the pump-field frequency. The resonance widths and amplitudes depend on the lifetimes of the levels involved in the transition. We attribute the spectra to coherent population oscillations induced by the two nearly degenerate microwave fields, which we have also observed in real time. The observations agree well with a theoretical model and can be useful for investigation of the NV- relaxation mechanisms.

\vspace{\baselineskip} 