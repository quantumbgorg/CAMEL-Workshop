\title{COHERENT MANIPULATION OF ULTRACOLD RUBIDIUM ATOMS AND MOLECULES}

\underline{J. Petrovic}

\index{Petrovic J}

Clarendon Laboratory, Parks Road, OX1 3PU, Oxford, United Kingdom\\
email: j.petrovic1@physics.ox.ac.uk

Optical control of ultracold atoms and molecules has set the ground for ultrahigh precision spectroscopy and study of chemical
reactions at low temperatures. As opposed to atoms, formation of ultracold molecules has been a challenge  due to the absence
of closed loop transitions. Recently, indirect cooling techniques, which aim to associate translationally cold atoms into ro-
vibrationally stable molecules, have been successfully used to form Cs2 in the lowest ground state as well as a range of
heteronuclear dimers. While the Stimulated Raman Adiabaic Passage (STIRAP) approach demonstrated by Danzl et al [1],
preserves coherence of the initial state, broadband optical cycling realizes cooling by spontaneous decay [2]. Here we report on
the progress of an alternative photoassociation technique - the broadband pump-dump scheme proposed by Koch et al. [3].
In this scheme, a broadband pulse pumps atoms into a range of vibrational levels in the excited state, the superposition of
which results in a wavepacket. The wavepacket moves in the attractive potential and when it reaches a point of good Franck-
Condon overlap with the ground state it can be dumped by another laser pulse [3]. The success of the scheme relies on the
capability to photoassociate ultracold atoms into molecules occupying a range of vibrational levels in the excited state and to
steer their coherent superposition into a wavepacket. We have recently reported the fulfillment of the first requirement [4], and
here we concentrate on coherent manipulation of rubidium atoms and molecules by shaped femtosecond pulses. In particular, we
investigate two sources of excited state molecules and discuss their role in dynamics of the excited state. To this end, several
coherent control schemes have been performed and their results are presented. Finally, we give an outlook of the merits of
broadband coherent control in the context of recent achievements and the major goals in the field.

[1] Danzl et al, Faraday Discussion 142, 2009; J. G. Danzl, E. Haller, M. Gustavsson, M. J. Mark, R. Hart, N. Bouloufa, O. Dulieu, H.
Ritsch, H-C. N\"{a}gerl, Science 321, 1062, 2008.\newline
[2] M. Viteau, A. Chotia, M. Allegrini, N. Bouloufa, O. Dulieu, D. Comparat and P. Pillet, Science 321, 232, 2008.\newline
[3] C. P. Koch, J. P. Palao, R. Kosloff, R. and F. Masnou-Seeuws, Phys. Rev. A \textbf{70}, 013402, 2004.\newline
[4] D. J. McCabe, D. England, H. E. L. Martay, M. Friedman, J. Petrovic, E. Dimova, B. Chatel and I. A. Walmsley, arXiv:0904.0244.

\vspace{\baselineskip}