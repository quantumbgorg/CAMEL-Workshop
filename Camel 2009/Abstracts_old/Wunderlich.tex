\title{ELEMENTS OF QUANTUM INFORMATION PROCESSING WITH TRAPPED YB+ IONS}

\underline{C. Wunderlich}

\index{Wunderlich C}

Fachbereich Physik,
Universitaet Siegen,
57068 Siegen,
Germany\\
email: wunderlich@physik.uni-siegen.de

For quantum information processing it is vital to characterize and
improve the experimental performance of quantum logic gates. Various
types of single-qubit channels have been experimentally implemented
with an individual trapped ion and complete tomography of these
quantum channels has been performed employing (and comparing) different
reconstruction methods taking into account imperfect state preparation and measurement. 
That is, complete channel tomography is done with mixed test states and
biased measurements.

Also, robust coherent operations developed using optimal control theory have been
demonstrated for the first time with trapped ions. Their performance as a function
of error parameters is systematically investigated and compared to composite pulses.
Such pulses are basic building blocks for single and multi-qubit quantum
gates.

Two essential experimental steps towards implementing ion spin
molecules will be reported: i) individual addressing of trapped ions
in frequency space using rf-radiation, and ii) the demonstration of
an interaction between motional and spin states induced by an rf
field. The development of novel micro-structured traps will allow
for using large spin-spin coupling constants, sculpting the strength
and range of these coupling constants, and for transport of ions.
This will make ion spin molecules a versatile system for
investigating many facets of quantum information science.

\vspace{\baselineskip}