\title{PRINCIPLE OF MINIMUM ENTROPY PRODUCTION}

\underline{S. Stenholm}

\index{Stenholm S}

Ekv\"{a}gen 9,
02270 Esbo,
Finland\\
email: stenholm@atom.kth.se

The entropy is known to ever increase. The time dependence of thermal quantities was introduced by Onsager, who linearized the deviations from equilibrium. Within this formalism, Prigogine formulated his principle of minimum entropy production, when the system is constrained to stay away from equilibrium. The validity of this principle has been controversial and its very meaning has remained obscure. Here I consider this principle within the formalism earlier developed to describe irreversible evolution in a dynamical system. The goal is to define an expression for the entropy production as a functional of the state of the system. It is shown that the steady state solution gives a minimum with respect to virtual variations around this state. The principle can be formulated only near the thermally defined equilibrium state.

\vspace{\baselineskip}