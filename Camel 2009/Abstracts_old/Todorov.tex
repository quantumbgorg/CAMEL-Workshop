\title{COHERENT LASER SPECTROSCOPY IN EXTREMELY THIN CELLS}

\underline{P. Todorov}

\index{Todorov P}

Institute of Electronics,
Bulgarian Academy of Sciences,
72 Tsarigradsko chausse,
1784 Sofia, Bulgaria\\
email: petkoatodorov@yahoo.com

Extremely thin cell (ETC) filled with thermal alkali vapour is a tool for creating an atomic layer with one-dimension thickness $L<10$ micrometers (the two other dimensions are in the cm range). The cell thickness has a wedge shape, allowing study of different $L$ at similar conditions.
We will present the main properties of laser spectroscopy in ETCs.
In case of dilute vapour, the space dimension anisotropy leads to atom light interaction time anisotropy when the light beam diameter exceeds $L$. If the mean flight time of an atom between two successive collisions with the cell walls is shorter than the life time of the exited state, changing $L$ one can study absorption and fluorescence processes in evolution.
Spectroscopy in ETC reveals many interesting phenomena with different dependences on $L$. Coherent Dicke narrowing in optical domain is observable, allowing separation of hyperfine optical transitions. Sub-Doppler resonances in absorption and fluorescence are obtained due to velocity selective saturation and optical pumping.

\vspace{\baselineskip}