\title{CONTROLLED DYNAMICS AND QUANTUM INTERFERENCE IN NANOSYSTEMS-LIGHT INTERACTION}

\underline{E. Paspalakis}

\index{Paspalakis E}

Materials Science Department,
University of Patras, Patras 265 04, Greece\\
email: paspalak@upatras.gr

In this talk we will present our recent results on the interaction
of several types of nanosystems with light. Three different
systems will be studied. Initially, we will consider the
interaction of modulation-doped semiconductor quantum wells with
electromagnetic fields. Emphasis will be given to the controlled
behavior of the quantum well system under the interaction with
continuous wave and pulsed fields. The case of chirped pulses
will also be analyzed. Then, we will present the interaction of a
hybrid structure consisting of a metal nanoparticle and a
semiconductor quantum dot with an external electromagnetic field.
In this case, too, the system will interact with both continuous
wave and pulsed fields. Finally, we will study the quantum
interference effects in spontaneous emission of a three-level
V-type atom placed near metalic nanostructures, such as thin
metallic films, metal nanospheres and periodic arrays of
metal-coated spheres. An enhancement of quantum interference in
spontaneous emission will be shown. This is attributed to the
strong dependence of the spontaneous emission rate on the
orientation of an atomic dipole relative to surface of the
nanostructure at the excitation frequencies of surface plasmons.

\vspace{\baselineskip}