\title{INVESTIGATION OF COLD RB RYDBERG ATOMS IN A MAGNETO-OPTICAL TRAP}

\underline{V. M. Entin}, D.B. Tretyakov, I.I. Beterov, and I.I. Ryabtsev

\index{Entin V}
\index{Tretyakov D}
\index{Beterov I}
\index{Ryabtsev I}

Institute of Semiconductor Physics SB RAS,
Pr. Acad. Lavrenteva 13,
630090, Novosibirsk, Russia\\
email: ventin@isp.nsc.ru

Rydberg atoms excited from cold atomic clouds attracts
large interest due to the fact that the gas consisting of almost frozen
Rydberg atoms behaves in a way similar to amorphous solids, in which
the strong collective inter-atomic interactions can lead to
broadening and shift of the spectral lines, as well as to ionization of
atoms and formation of ultra cold plasma. The high atomic density and
long time of interaction of atoms in a cold media provide the best
conditions for long range interactions experiments. The flexibility of
experimental techniques with Rydberg atoms and availability of
single-atom sensitive detection methods (e.g., selective field
ionization technique (SFI) [1]) make them a good candidate for quantum
logic gates on neutral atoms.

In the current paper we describe our progress in the experiments with
cold Rb Rydberg atoms. The aim of the work was realization of
excitation in a small volume, development of diagnostic methods for
cold Rydberg atoms, spectroscopy of the microwave transitions between
Rydberg states in the presence of the magneto-optical trap (MOT) quadrupole magnetic field, as
well as observation of dipole-dipole interaction of a small number of
Rydberg atoms.

The Rb atoms were captured with a conventional MOT using three pairs of counter-propagating
beams from diode lasers in a quadrupole magnetic field. The initial $37P$ Rydberg state was excited by two pulsed lasers in a
crossed-beam geometry. Rydberg states were detected with the SFI
technique using a channeltron detector to collect electrons, produced
via field ionization. In order to probe the magnetic/electrical
fields inside of our cold atom cloud, we have performed additional
experiments on microwave spectroscopy of Rydberg states. The
experimental data on the width and positions of the microwave
resonances were analyzed and compared with theory. These data were used
to reduce the stray static magnetic and electric fields. The
appropriate steps were undertaken for stronger dipole-dipole
interaction between Rydberg atoms. In the comparison to our previous
results [2] in the current paper we have reduced interaction volume for
the laser radiation (Rydberg state excitation) from $\sim 100
\mu m\textsuperscript{3}$ to a few tens of
$\mu m\textsuperscript{3}$.

Stark tuning of Rydberg levels was used to put the dipole-dipole
interaction: $Rb(37P)+Rb(37P)\rightarrow Rb(37S)+ Rb(38S)$ on the
resonance. Prior to the ionization pulse, a small dc electric field
(tuned by computer) was applied. The detected channeltron pulses were
sorted by the amplitudes corresponding to 1-5 detected electrons.
Obtained data were averaged over a large number of laser pulses and
stored simultaneously for different number of detected atoms. The
resonances obtained for 1 to 5 of detected Rydberg atoms have been
analyzed and compared with the theory.

The dependences of the measured
signals on the number of atoms, excitation volume, energy and
interaction time were investigated. It has been shown that due to the
cooling and trapping of atoms, the effective lifetime of the cold
Rydberg atoms in a MOT is close to the natural one. This makes possible
to perform measurements at a large timescale, taking all advantages of
Rydberg atoms. The localization in a small excitation volume allows to
select the position of atomic interaction within the inhomogeneous
external fields, and to map the spatial distribution of these fields
using microwave spectroscopy. In particular, localizing the excitation
volume close to the zero magnetic-field point improves the spectral
resolution and yields narrow microwave resonances even without turning
off the MOT quadrupole magnetic field.

This work was supported by the Russian Academy of Science and
Russian Foundation for Basic Research.

[1] T.F.Gallagher, Rydberg Atoms , (Cambridge
University Press, Cambridge,1994).\newline
[2] D. B. Tretyakov, I. I. Beterov, V. M. Entin, I. I. Ryabtsev, P. L.
Chapovsky, arXiv:0810.5427 [physics.atom-ph] (2008) (to
appear in JETP, 2009).

\vspace{\baselineskip}