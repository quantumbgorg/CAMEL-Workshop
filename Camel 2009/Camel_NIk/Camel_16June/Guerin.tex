\title{CONTROL OF QUANTUM PROCESSES BY PARALLEL \mbox{ADIABATIC} \mbox{PASSAGE}}

\underline{S. Gu\'erin} \index{Gu\'erin S}

{\normalsize{\vspace{-4mm}
Institut Carnot de Bourgogne, Universite de Bourgogne, 9, Av A.
Savary, 21078 Dijon, France

\email sguerin@u-bourgogne.fr}}

We present the technique of parallel adiabatic passage and its
application to the control of quantum processes such as
state-selectivity for atoms and molecules, and quantum information
processing. Parallel adiabatic passage corresponds to an adiabatic
passage for which the instantaneous eigenvalues are parallel at each
time from a specific design of the pulses. This allows one to
optimize the adiabaticity in the sense that the DDP non-adiabatic
correction is zero. In practice the technique induces a very
efficient population transfer in terms of pulse energy and duration,
while preserving the standard robustness of adiabatic techniques. As
an application, we have developed a STIRAP-like technique with
appropriate chirped fields. It leads to an adiabatic population
transfer with a pulse area as low as $3.5\pi$. This technique opens
the possibility to implement STIRAP in picosecond regimes with
interesting applications in quantum information processing.

\vspace{\baselineskip}