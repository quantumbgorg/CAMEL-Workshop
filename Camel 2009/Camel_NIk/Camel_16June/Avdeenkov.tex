\title{COLLISIONS OF ULTRACOLD POLAR MOLECULES IN \mbox{MICROWAVE} TRAPS}

\underline{A. Avdeenkov}$^{12}$ \index{Avdeenkov A}

{\normalsize{\vspace{-4mm} $^1$ National Institute for Theoretical Physics, Stellenbosch
Institute of Advanced Study, Private BagX1, Matieland, 7602, South
Africa

\vspace{-4mm} $^2$ Institute of Nuclear Physics, Moscow State
University, Moscow, 119992, Russia

\email avdeenkov@sun.ac.za}}

The collisions between linear polar molecules, trapped in a
microwave field with a circular polarization, are theoretically
analyzed. The microwave trap suggested by DeMille [1] seems to be
rather advantageous in comparison with other traps. Here we have
demonstrated that the microwave trap can provide the successful
evaporative cooling for polar molecules not only in their absolute
ground state but also in almost any strong-field seeking states and
the collision losses should not be of much concern to them [2]. But
the state in which molecules can be safely loaded and trapped
depends on the frequency of the AC-field. The AC-Stark splitting is
characterized by $\left|J,M,n\right>$ states, where $M$ is the
projection of rotation spin $J$ and n is the deviation of the photon
number. We are mostly interested in the lowest energy
strong-field-seeking state of the ground vibrational state,
$\left|JM\right>= \left|00\right>$ and found rather large inelastic
cross sections for molecules even in this state. But we found that
the nature of this inelasticity is simply the ``undressing'' of
molecules by one photon each and should not cause the loss of
molecules from the trap.

\vspace{-4mm} {\normalsize
\begin{enumerate}
\item D. DeMille, D. R. Glenn, and J. Petricka, %Microwave traps for cold polar molecules
Euro. Phys. J. D \textbf{31}, 375 (2004).  %
\item A.V.Avdeenkov, %Collisions of bosonic ultracold polar molecules in microwave traps,
New J. Phys. \textbf{11}, (2009) in print . %
\item A.V.Avdeenkov and J.L.Bohn, %Collisional dynamics of ultracold OH molecules in an electrostatic field,
Phys. Rev. A \textbf{66}, 052718 (2002). %
\end{enumerate}
}

\vspace{\baselineskip}