\title{COHERENT TRANSIENTS IN THE FORMATION OF \mbox{ULTRACOLD} MOLECULES WITH SHAPED FEMTOSECOND PULSES}

\underline{T. G. Mullins}$^1$, W. Salzmann$^1$, S. G\"{o}tz$^{12}$,
M. Albert$^1$, J. Eng$^1$, R. Wester$^1$, M. Weidem\"{u}ller$^{12}$,
A. Merli$^3$, S. Weber$^3$, M. Plewicki$^3$, F. Sauer$^3$, F.
Weise$^3$, L. W\"{o}ste$^3$, A. Lindinger$^3$ \index{Mullins T,
Salzmann W} \index{Mullins T} \index{Mullins T} \index{Mullins T}
\index{Mullins T}

{\normalsize{\vspace{-4mm} $^1$Physikalisches Institut, Universit\"{a}t Freiburg,
Hermann-Herder-Str. 3, 79104 Freiburg, Germany

\vspace{-4mm} $^2$Ruprecht Karls Universit\"{a}t Heidelberg, D-69120 Heidelberg, Germany

\vspace{-4mm} $^3$Fachbereich Physik der Freien Universit\"{a}t
Berlin, D-14195 Berlin, Germany

\email terry.mullins@physik.uni-freiburg.de}}

We present experiments on the formation of ultracold molecules by
femtosecond laser pulses in a pump-probe scheme [1]. Previous
experiments [2,3] have only managed to dissociate molecules by
femtosecond pulses, whereas now active photoassociation is observed.
A shaped pump pulse excites a collision pair of laser cooled
rubidium atoms to a bound molecular state below the $5s5p_{1/2}$
asymptote, from where the molecule is transferred to the molecular
ionic state by a probe pulse a few picoseconds later. A femtosecond
pulse shaper is used to apply a sharp low-pass optical filter to the
pump pulse spectrum with a cut-off frequency a few wavenumbers below
the atomic D1 resonance. The photoassociation signal shows
oscillatory dynamics between the electronic molecular states coupled
by the pump pulse, resulting from the pump pulse's high electric
field strength and its specific spectral shape close to the
molecular dissociation limit. Simulations of the pump pulse
excitation have been performed by numerically solving the
time-dependent Schr\"{o}dinger equation using a
mapped-fourier-grid-hamiltonian algorithm and are in excellent
agreement with the experimental results.

\vspace{-4mm} {\normalsize \begin{enumerate}
\item W. Salzmann et al., Phys. Rev. Lett. \textbf{100}, (2008) 233003.
\item W. Salzmann et al., Phys. Rev. A \textbf{73}, (2006) 023414.
\item B. L. Brown et al., Phys. Rev. Lett. \textbf{96}, (2006) 173002.
\item M. J. Wright et al., Phys. Rev. A \textbf{75}, (2007) 051401.
\end{enumerate}
}

\vspace{\baselineskip}