\title{QUANTUM STATE MANIPULATION USING STRUCTURED PULSES}

\underline{B. W. Shore}$^1$, A. A. Rangelov$^2$, and N. V.
Vitanov$^{2}$ \index{Shore B} \index{Rangelov A} \index{Vitanov N}

{\normalsize{\vspace{-4mm} $^1$618 Escondido Cir. Livermore CA 94550 (925) 455
0627

\vspace{-4mm}  $^2$Department of Physics, Sofia University, 5 James
Bourchier Blvd., 1164 Sofia, Bulgaria

\email bwshore@alum.mit.edu}}

Changes to quantum states induced by laser pulses have long provided
opportunities to produce complete transfer of probability between two
states or to create coherent superpositions. Techniques that rely on
adiabatic time evolution, such as rapid adiabatic passage (RAP) in two-state
systems or stimulated Raman adiabatic passage (STIRAP)  amongst
three states, have been widely studied, both theoretically and
experimentally, as means of producing complete population transfer.
Utilizing procedures that can craft temporally shaped laser pulses,
including carrier-frequency variation,  experimenters can produce a
variety of coherent quantum-state changes via adiabatic passage.  I
will discuss several idealized excitation scenarios, based on
adiabatic evolution of the statevector,  that are possible with the
use of suitably structured pulses, with examples drawn from RAP and
STIRAP.  In all cases a simple torque equation, involving two real-valued
coordinates,  provides a simple picture of the desired
adiabatic following as motion along a circle in a fixed plane. For two
states the torque equation is the familiar (Bloch) equation of
Feynman, Vernon and Hellwarth; for three states it is the time-dependent
Schroedinger equation for components of the ``dark'' adiabatic
eigenvector.  The examples will illustrate how, for the two-state
system the use of temporally asymmetric pulses of detuning and Rabi
frequency can produce RAP without the customary crossing of diabatic
energy curves. For the three-state system such ``zero-area'' pulses can
produce fractional- or multiple-STIRAP, allowing control of the phase
of the final state relative to the initial state.

\vspace{\baselineskip}