\title{MULTI-COMPONENT NLS MODELS AND SPINOR BOSE-EINSTEIN \mbox{CONDENSATES}}

\underline{V. Gerdjikov} \index{Gerdjikov V}

{\normalsize{\vspace{-4mm}
Institute for Nuclear Research and Nuclear Energy, Bulgarian Academy
of Sciences, 72 Tsarigradsko chaussee, 1784 Sofia, Bulgaria

\email gerjikov@inrne.bas.bg}}

Consider BEC's of alcali atoms in the $F=1$ hyperfine state,
elongated in $x$ direction and confined in the transverse directions
$y,z$ by purely optical means. The dynamics of this assembly of
atoms is described by a  3-component normalized spinor wave vector
${\bf\Phi}(x,t)=(\Phi_1,\Phi_0 ,\Phi_{-1})^T(x,t)$ satisfying the
nonlinear Schr\"{o}dinger (MNLS) equation [1,2]:
\begin{eqnarray}
&& i\partial_{t}\Phi_{1}+\partial^{2}_{x}\Phi_{1}+2\left(|\Phi_{1}|^2
+2|\Phi_{0}|^2\right)\Phi_{1} +2\Phi_{-1}^{*}\Phi_{0}^2=0, \nonumber\\
&& i\partial_{t}\Phi_{0}+\partial^{2}_{x}
\Phi_{0}+2\left(|\Phi_{-1}|^2
+ |\Phi_{0}|^2+|\Phi_{1}|^2\right)\Phi_{0} +2\Phi_{0}^{*}\Phi_{1}\Phi_{-1}=0,\nonumber \\
&& i\partial_{t}\Phi_{-1}+\partial^{2}_{x}
\Phi_{-1}+2\left(|\Phi_{-1}|^2+ 2|\Phi_{0}|^2\right)\Phi_{-1}
+2\Phi_{1}^{*}\Phi_{0}^2=0. \nonumber
\end{eqnarray}
This  model has natural Lie algebraic interpretation and is related
to the symmetric spaces ${\bf BD.I}\simeq {\rm SO(5)}/{\rm
SO(3)\times SO(2)}$. It is integrable by means of inverse scattering
transform method [2,3]. Using a modification of the Zakharov-Shabat
``dressing method'' we describe the different classes of soliton
solutions of the above equation, see [2] and the references therein.
BEC with $F=2$ for rather specific choices of the scattering lengths
in dimensionless coordinates takes the form:
\begin{eqnarray}
&&i\partial_t\Phi_{\pm 2}+\partial_{xx}\Phi_{\pm 2}= -2\epsilon
\left(\vec{{\bf \Phi}},\vec{{\bf \Phi^{*}}}\right) \Phi_{\pm 2} +\epsilon
\left(2\Phi_{2}\Phi_{-2}-2 \Phi_{1}\Phi_{-1}+\Phi_{0}^2\right) \Phi_{\mp
2}^*,\nonumber \\
&&i\partial_t\Phi_{\pm 1}+\partial_{xx}\Phi_{\pm 1}= -2\epsilon
\left(\vec{{\bf \Phi}},\vec{{\bf \Phi^{*}}}\right) \Phi_{\pm 1} -\epsilon
\left(2\Phi_{2}\Phi_{-2}-2 \Phi_{1}\Phi_{-1}+\Phi_{0}^2\right) \Phi_{\mp
1}^*,\nonumber \\
&&i\partial_t\Phi_{0}+\partial_{xx}\Phi_{0}=  -2\epsilon
\left(\vec{{\bf \Phi}},\vec{{\bf \Phi^{*}}}\right) \Phi_{0} +\epsilon
\left(2\Phi_{2}\Phi_{-2}-2 \Phi_{1}\Phi_{-1}+\Phi_{0}^2\right)
\Phi_{0}^*.\nonumber
\end{eqnarray}
where $\epsilon =\pm 1$. It can also be solved by the inverse
scattering method with Lax operator related to the symmetric space
of ${\bf BD.I}$ type $ {\rm SO(7)}/{\rm SO(5)\times SO(2)}$.


\vspace{-4mm}

{\normalsize \begin{enumerate}
\item Ieda J., Miyakawa T. and Wadati M., Phys. Rev. Lett. \textbf{93} 194102, (2004).

\item V. S. Gerdjikov, N. A. Kostov, T. I. Valchev,  ArXiv:0802.4398 [nlin.SI];\newline Physica D,
doi:10.1016/j.physd.2008.06.007 (in press).

\item Fordy A. P. and Kulish P. P., Commun. Math. Phys. \textbf{89}, 427 (1983).

\item Salerno M., Phys. Rev. A \textbf{72} (2005), 063602, cond-mat/0503097.
\end{enumerate}
}

\vspace{\baselineskip}