\title{NARROW DARK AND BRIGHT RESONANCES IN ALKALI ATOMS}

\underline{S. Cartaleva} \index{Cartaleva S}

{\normalsize{\vspace{-4mm}
Bulgarian Academy of Sciences, 72 Tsarigradsko chausse, 1784 Sofia,
Bulgaria

\email stefka-c@ie.bas.bg}}

Applications of Electromagnetically Induced Transparency (EIT, dark) and Electromagnetically Induced Absorption (EIA, bright) resonances are expanding rapidly.
Here we present a comparison of dark and bright resonances in different alkali atoms (Cs, Rb, Na, K) with the aim of clarification of their potential for various applications.
EIT in Cs and Rb thermal vapour has been prepared using frequency modulated (in GHz and kHz regions) laser light. The kHz-modulation approach is more effective and well applicable for magnetic field measurement. The disadvantage is the hyperfine optical pumping, which can be overcome using K vapour.
Unlike the cm-sized cells, the nano-metric thin cells allow the observation of well resolved hyperfine transitions and the dark/bright resonance study in better-defined conditions. In Cs vapour confined in nano-cell, the bright resonance sign reversal is demonstrated.
The richness of alkali atom properties makes possible tailoring of particular methodology depending on the aimed application.

\vspace{\baselineskip}