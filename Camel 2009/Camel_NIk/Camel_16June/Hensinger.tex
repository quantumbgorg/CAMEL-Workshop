\title{ARCHITECTURES FOR ION QUANTUM TECHNOLOGY: TRAP ARRAYS ON A CHIP, AND BEYOND}

\underline{W. Hensinger} \index{Hensinger W}

{\normalsize{\vspace{-4mm}
Ion Quantum Technology Group, Department of Physics and Astronomy,
University of Sussex, Falmer, Brighton, East Sussex, BN1 9QH, United
Kingdom

\email w.k.hensinger@suss ex.ac.uk}}

Quantum technology with trapped ions has already been successfully realized in experiments with a small number of quantum bits for example to realize quantum algorithms such as search, the generation of particular entangled states of up to 8 ions, teleportation, ion-photon entanglement, and others. In order to build useful devices, the next step must include the systematic development of suitable architectures for large scale ion quantum technology applications. The scalable fabrication of ion trap arrays involves the production of chip structures using advanced nanofabrication techniques. I will discuss fabrication and successful operation of such chips for example an integrated ion chip we etched in a multi-layer Gallium-Arsenide substrate. I will also report progress on a chip we created from polysilicon and discuss general considerations in the implementation of such chips. Furthermore, I will discuss transport of ions within ion trap arrays and illustrate a general theoretical framework for shuttling atomic ions along any desired trajectory. I will discuss issues particular to the shuttling through multidimensional junctions. In order to produce ion trap arrays one may want to keep the size of a single trapping zone inside the array as small as possible. However, motional heating of trapped ions can become significant with very small trap dimensions. I will also discuss the scaling of motional heating from the quantum ground state in an ion trap. At Sussex, we have developed an experimental setup for the implementation of large scale ion trap arrays on a chip using ytterbium ions and I will report progress on the experiment as well as the development of ion chips and advanced ion trap arrays at Sussex. Finally I will present our progress in analyzing general ion trap geometries.

\vspace{\baselineskip}