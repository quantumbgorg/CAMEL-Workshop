\title{QUANTUM INFORMATION AT THE ION-PHOTON \mbox{INTERFACE}}

\underline{W. Lange} \index{Lange W}

{\normalsize{\vspace{-4mm}
University of Sussex, Department of Physics and Astronomy, Falmer,
Brighton, East Sussex, BN1 9QH, United Kingdom

\email W.Lange@sussex.ac.uk}}

The complementary benefits of trapped ions and photons as carriers of
quantum information make it appealing to join them. Cavity QED provides
a setting in which the quantum states of ions and photons can be
interfaced efficiently. The actual implementation depends on the
strength of the coherent ion-field coupling and the cavity damping rate.
For moderate coupling, quantum entanglement may be generated
probabilistically. Detecting a photon emitted from a cavity with
multiple ions projects the ions to an entangled state, as there is no
way to determine the source ion. For stronger coupling, deterministic
transfer of quantum states between ions and photons is possible by
linking different electronic states of an ion with different
polarization modes of the cavity. Entanglement of ions in a cavity may
even be generated through the exchange of virtual photons. This requires
very strong coupling. Progress in the realization of these schemes at
the University of Sussex will be reported.

\vspace{\baselineskip}