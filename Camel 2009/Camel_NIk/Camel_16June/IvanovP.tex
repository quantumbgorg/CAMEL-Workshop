\title{SIMULATION OF A QUANTUM PHASE TRANSITION OF \mbox{POLARITONS} WITH TRAPPED IONS}

\underline{P. A. Ivanov}$^{12}$, S. S. Ivanov$^{2}$, N. V.
Vitanov$^{2}$, A. Mering$^{3}$, M. Fleischhauer$^{3}$, and K.
Singer$^{1}$ \index{Ivanov P} \index{Ivanov S} \index{Vitanov N}
\index{Mering A} \index{Fleischhauer M} \index{Singer K}

{\normalsize{\vspace{-4mm} $^1$University Ulm, Albert-Einstein-Allee 11, 89081
Ulm, Germany

\vspace{-4mm} $^2$Department of Physics, Sofia University, 5 James
Bourchier Blvd., 1164 Sofia, Bulgaria

\vspace{-4mm} $^3$Technical University of Kaiserslautern, D-67653
Kaiserslautern, Germany

\email pivanov@phys.uni-sofia.bg}}

We present a novel system for the simulation of quantum phase transitions of collective internal qubit and phononic states with a linear crystal of trapped ions.
The laser-ion interaction creates an energy gap in the excitation spectrum, which induces an effective phonon-phonon repulsion and a Jaynes-Cummings-Hubbard interaction.
This system shows features equivalent to phase transitions of polaritons in coupled cavity arrays.
Trapped ions allows for easy tunabilty of the hopping frequency by adjusting the axial trapping frequency and the phonon-phonon repulsion by the laser detuning and intensity.
We propose an experimental protocol to access all observables of the system, which allows one to obtain signatures of the quantum phase transitions even with a small number of ions.

%[1] J. I. Cirac and P. Zoller, Phys. Rev. Lett. \textbf{74}, 4091 (1995).\\
%
%[2] R. P. Feynman, Int. J. Theoret. Phys. \textbf{21}, 467--468 (1982).\\
%
%[3] C. Monroe \emph{et al}., Phys. Rev. Lett. \textbf{75}, 4714
%(1995); D. Leibfried, \textit{et al.}, Nature 422, 412 (2003); F.
%Schmidt-Kaler, \textit{et al.} Nature 422, 408 (2003); S. Gulde, \textit{et
%al.}, Nature 421, 48--50 (2003); M. D. Barrett \textit{et al.}, Nature 429,
%737--739 (2004); M. Riebe, \textit{et al.} Nature 429, 734--737 (2004); J.
%Chiaverini \textit{et al}, Nature 432, 602--605 (2004); J. Chiaverini,
%\textit{et al.}, Science 308, 997--1000 (2005); K.-A. Brickman, \textit{et
%al.}, Phys. Rev. A 72, 050306(R) (2005).\\
%
%[4] D. Kielpinski \emph{et al}., Nature 417, 709-711 (2002).\\
%
%[5] D. Porras and J. I. Cirac, Phys. Rev. Lett. \textbf{93},
%263602 (2004); X. L. Deng \emph{et al}., Phys. Rev. A \textbf{77}, 033403
%(2008).\\
%
%[6] D. Porras \emph{et al}., Phys. Rev. A \textbf{87}, 010101 (R) (2008).\\
%
%[7] R. Schutzhold \emph{et al}., Phys. Rev. Lett. \textbf{99},
%201301 (2007); L. Lamata \emph{et al}., Phys. Rev. Lett.\textbf{\ 98},
%253005 (2007); D. Porras and J. I. Cirac, Phys. Rev. Lett. \textbf{92},
%207901 (2004).\\
%
%[8] A. Friedenauer \emph{et al}., Nature Phys. \textbf{4}, 757 (2008).\\
%
%[9] A. D. Greentree \emph{et al}., Nat. Phys. \textbf{2}, 856
%(2006); M. J. Hartmann \emph{et al}., Nat. Phys. \textbf{2}, 849 (2006). M.
%I. Makin \emph{et al}., Phys. Rev. A \textbf{77}, 053819 (2008); D. Rossini
%and R. Fazio, Phys. Rev. Lett. \textbf{99}, 186401 (2007).\\
%
%[10] J. Hubbard, Proc. Roy. Soc. A \textbf{276}, 238 (1963).\\
%
%[11] M. P. A. Fisher \emph{et al}., Phys. Rev. B \textbf{40}, 546 (1989).\\
%
%[12] B. W. Shore and P. L. Knight, J. Mod. Opt. \textbf{40}, 1195 (1993).\\
%
%[13] A. Imamoglu \emph{et al}., Phys. Rev. Lett. \textbf{79}, 1467
%(1997). K. M. Birnbaum \emph{et al}., Nature \textbf{436}, 87 (2005).\\
%
%[14] E. K. Irish \emph{et al}., Phys. Rev. A \textbf{77}, 033801
%(2008); D. G. Angelakis \emph{et al}., Phys. Rev. A \textbf{76}, 031805(R)
%(2007).\\
%
%[15] D. F. V. James, Appl. Phys. B: Lasers Opt. \textbf{66}, 181
%(1998).
%
%[16] D. J. Wineland \emph{et al}., {J. Res. Natl. Inst.
%Stand. Technol.} \textbf{103}, 259 (1998).\\

\vspace{\baselineskip}