\title{CAVITY QUANTUM ELECTRODYNAMICS WITH ION COULOMB \mbox{CRYSTALS}}

\underline{M. Drewsen} \index{Drewsen M}

\vspace{-4mm}
{\normalsize{Institute of Physics and Astronomy, University of Aarhus, Ny
Munkegade 120, DK-8000 Aarhus C, Denmark

\email drewsen@phys.au.dk}}

Cavity Quantum ElectroDynamics (CQED) is a rich research field with multiple potential applications within quantum information science. Due to the prospect of creating efficient light-matter quantum interfaces at the single photon level, CQED is e.g. an ideal tool for swapping information between flying and stationary qubits. Ion Coulomb crystals have a series of specific solid state properties of relevance for CQED studies and for the development of quantum information devices. One is their uniform density with the ions positioned at distances of ~10 microns, which leads to neither temporal nor spatial inhomogeneous spectral broadening, due to the lack of close ion-ion interactions. Another is their stable trapping for hours, which allows for repeatable experiments with a coupling strength being constant to within a few percent.
In the talk, I will present the first results on collective strong coupling between atomic ions in a solid in the form of an ion Coulomb crystal and an optical field, and discuss the potential of exploiting Coulomb crystal based CQED for creating single- and multimode quantum memories for light, for cavity-mediated cooling, for the studies of optomechanical effects of solid-like objects at the quantum level, as well as for gaining information on structural and thermodynamics properties of ion Coulomb crystals.

\vspace{\baselineskip}