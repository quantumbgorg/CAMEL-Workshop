\title{OPTICAL VORTICES IN SELF-FOCUSING KERR \mbox{NONLINEAR} MEDIA}

\underline{A. Dreischuh} \index{Dreischuh A}

{\normalsize{\vspace{-4mm}
Department of Quantum Electronics, Faculty of Physics, Sofia
University ``St. Kliment Ohridski'', 5 James Bourchier Blvd.,
BG-1164 Sofia, Bulgaria

\email ald@phys.uni-sofia.bg}}

An isolated point singularity with a screw-type phase distribution is associated with an optical vortex (OV). We will report the first experimental generation of femtosecond supercontinuum in a solid state medium by a singly-charged optical vortex beam. We will show that, despite the strong self-focusing resulting in multiple filaments ordered along the vortex ring, the optical vortex remains well preserved in both the near- and far-field. Quasi-(3+1)-dimensional numerical simulations confirming qualitatively the observations will be presented. Next we will numerically compare the interaction of optical vortices in self-defocusing and self-focusing Kerr nonlinear media. We will show that the basic scenarios in the interaction of two and three vortices with equal and alternative topological charges are the same in both media. However, the vortex dynamics under self-focusing conditions is influenced by the reshaping of the surrounding part of the background.

\vspace{\baselineskip}