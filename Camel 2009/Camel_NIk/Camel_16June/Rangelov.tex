\title{STIMULATED RAMAN ADIABATIC PASSAGE ANALOGS IN CLASSICAL PHYSICS}

\underline{A. A. Rangelov}, N. V. Vitanov, and B. W. Shore
\index{Rangelov A} \index{Vitanov N} \index{Shore B}

{\normalsize{\vspace{-4mm}
Department of Physics, Sofia University, 5 James Bourchier Blvd.,
1164 Sofia, Bulgaria

\email rangelov@phys.uni-sofia.bg}}

Stimulated Raman adiabatic passage (STIRAP) is a well established technique
for producing coherent population transfer in a three-state quantum system
[1-3]. We here exploit the resemblance between the Schr\"{o}dinger equation for
such a quantum system and the Newton equation of motion for a classical
system undergoing torque to discuss several classical analogs of STIRAP,
notably the motion of a moving charged particle subject to the Lorentz force
of a quasistatic magnetic field, the orientation of a magnetic moment in a
slowly varying magnetic field and the Coriolis effect. Like STIRAP, these
phenomena occur for \emph{counterintuitive} motion of the torque and are
robustly insensitive to small changes in the interaction properties.

\vspace{-4mm} {\normalsize \begin{enumerate} \item U. Gaubatz, P.
Rudecki, S. Schiemann, and K. Bergmann, J. Chem. Phys. \textbf{92},
5363 (1990).\item B. W. Shore, Acta Physica Slovaca \textbf{58}, 243
(2008).\item N. V. Vitanov and B. W. Shore, Phys. Rev. A
\textbf{73}, 053402 (2006).\item A. A. Rangelov, N. V. Vitanov and
B. W. Shore, J. Phys. B: At. Mol. Opt. Phys. \textbf{42}, 055504
(2009).
\end{enumerate}
}

\vspace{\baselineskip}