\title{OPTIMIZING GAUSSIAN COMMUNICATION FOR HYBRID QUANTUM REPEATER}

\underline{L. Praxmeyer}$^{12}$, P. van Loock$^1$ \index{Praxmeyer
L} \index{van Loock P}

{\normalsize{\vspace{-4mm} $^1$Max Planck Institute for the Science of Light and
Institute of Theoretical Physics I, University Erlangen-Nuremberg,
Staudstr. 7/B2, 91058 Erlangen, Germany

\vspace{-4mm} $^2$Department of Physics, Sofia University, 5 James
Bourchier blvd., 1164 Sofia, Bulgaria

\email lpraxm@gmail.com}}

For long distance quantum communication, distribution of
entanglement over long distances, overcoming effects of noise and
decoherence, is needed. Various quantum repeater schemes [1] were
proposed to provide a theoretical solution of this problem. We shall
present an optimized version of the hybrid quantum repeater [2,3],
based on Gaussian optical-state communication, and Gaussian,
homodyne-based, conditional entangled state preparation. We show
that use od squeezed light significantly increases the fidelity of
states obtained while probability of success is maintained.

\vspace{-4mm} {\normalsize \begin{enumerate} \item H.-J. Briegel {\sl et
al.}, Phys. Rev. Lett. \textbf{81}, 5932 (1998). \item P. van Loock
{\sl et al.}, Phys. Rev. Lett. \textbf{96}, 240501 (2006).
\item T. D. Ladd {\sl et al.}, New J. Phys. \textbf{8},
164, (2006).
\end{enumerate}
}

\vspace{\baselineskip}