\title{NON-MARKOVIAN DYNAMICS OF BIPARTITE ENTANGLEMENT}

\underline{S. Maniscalco} \index{Maniscalco S}

{\normalsize{\vspace{-4mm}
Department of Physics and Astronomy, University of Turku, 20014
Turku, Finland

\email smanis@utu.fi}}

The non-Markovian dynamics of entanglement between two-qubits and between two continuous variable quantum channels is
investigated.
For the case of two qubits immersed in a common environment, we demonstrate the optimal conditions for reservoir-mediated
entanglement generation and we study the effects of the reservoir memory on entanglement deterioration [1,2].
We then consider two possible strategies for protecting two-qubit entanglement for both the common reservoir and the
independent reservoir cases [3,4].
Finally we tackle the problem of the non-Markovian description of the time evolution of entanglement in continuous variable
quantum channels [5] and we compare the dynamics of entanglement loss for different types of structured reservoirs.

\vspace{-4mm} {\normalsize
\begin{enumerate}
\item F. Francica, S. Maniscalco, J. Piilo, F. Plastina, and
K.-A. Suominen, Phys. Rev. A \textbf{79}, 032310 (2009).
\item L. Mazzola, S. Maniscalco, J. Piilo, K.-A.
Suominen, and B. M. Garraway, Phys. Rev. A \textbf{79}, 042302
(2009). \item S. Maniscalco, F. Francica, R. L. Zaffino, N. Lo
Gullo, and F. Plastina, Phys. Rev. Lett. \textbf{100}, 090503
(2008). \item B. Bellomo, R. Lo Franco, S. Maniscalco, and G.
Compagno Phys. Rev. A \textbf{78}, 060302(R) (2008).
\item S. Maniscalco, S. Olivares, and M. G. A. Paris, Phys. Rev. A
\textbf{75}, 062119 (2007).
\end{enumerate}
}

\vspace{\baselineskip}