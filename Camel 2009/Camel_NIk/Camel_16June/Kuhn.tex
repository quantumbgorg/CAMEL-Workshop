\title{TOWARDS SCALABLE QUANTUM NETWORKS OF \mbox{NEUTRAL} ATOMS}

\underline{A. Kuhn} \index{Kuhn A}

\vspace{-4mm}
{\normalsize{University of Oxford,
Clarendon Laboratory
Parks Road
Oxford, OX1 3PU,
United Kingdom

\email axel.kuhn@physics.ox.ac.uk}}

Mastering the interaction of matter and light at the single-atom and single-photon level is one of the most challenging tasks in
modern quantum technology. In particular, the entanglement of atoms and photons as well as the entanglement of photon pairs
play a key role in quantum information physics and quantum communication, with atoms and photons acting as information
carriers in a scalable quantum network. Such a network would be ideal for quantum computing and the quantum simulation of
complex systems.
I will shortly review our previous work which has been addressing some of the fundamental issues, such as reliable atom-photon
interfaces in high-finesse optical cavities, atom-photon entanglement and quantum state mapping between atoms and photons. Then I will present our current projects which are aiming at the implementation of a new scheme for increasing the number of
atoms and cavities in a scalable fashion. The scheme relies on an array of optical tweezers, formed by a spatial light modulator,
to arbitrarily manipulate and control the positions of individual dipole-trapped atoms. Eventually this will allow for coupling of random atoms to fibre-tip cavities, for controlling atomic collisions at the single-atom level, and for atom-interferometric
measurements under full quantum control.

\vspace{\baselineskip}