\title{ATOMIC COHERENCE PREPARATION BY MODULATED LIGHT FOR MAGNETIC FIELD MEASUREMENT}

\underline{D. Slavov} \index{Slavov D}

{\normalsize{\vspace{-4mm}
Bulgarian Academy of Sciences, 72 Tsarigradsko chausse, 1784 Sofia,
Bulgaria

\email slavov\_d\_g@yahoo.com}}

In this communication, we present a comparative study of different approaches for weak magnetic field measurement where Coherent Population Trapping (CPT) resonances are obtained using frequency or intensity modulated diode laser light. Thermal alkali vapour is irradiated by the modulated light and CPT resonances are registered in absorption and fluorescence. The CPT resonance parameters are compared in different alkali metals as well as depending on the atomic level configuration.
Based on the simplicity of laser diode modulation, several methodologies are developed for all-optical magnetic field measurement. The principal differences will be illustrated discussing two compact laboratory Rb magnetometers.
The influence of different mechanisms for reducing of ground-level coherence relaxation (like cell wall coating or buffer gas implementation) will be discussed. A strong improvement of CPT-resonance parameter (amplitude, contrast and width) will be reported based on resent results obtained in Potassium.

\vspace{\baselineskip}