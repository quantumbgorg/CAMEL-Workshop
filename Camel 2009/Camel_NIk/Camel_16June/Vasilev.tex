\title{OPTIMUM PULSE SHAPES FOR STIMULATED RAMAN \mbox{ADIABATIC} PASSAGE}

\underline{G. Vasilev}$^{12}$, A. Kuhn$^1$, and N. V. Vitanov$^{2}$ \index{Vasilev G} \index{Kuhn A} \index{Vitanov N}

{\normalsize{\vspace{-4mm}
$^1$Department of Physics, University of Oxford, Parks Road, OX1 3PU Oxford, UK%

\vspace{-4mm} $^2$Department of Physics, Sofia University, 5 James Bourchier 5 Blvd., 1164 Sofia, Bulgaria%

\email gvasilev@phys.uni-sofia.bg}}

Stimulated Raman adiabatic passage (STIRAP) is a well established
and widely used technique for coherent population transfer in atoms
and molecules [1]. Because STIRAP is an adiabatic technique it is
insensitive to small to moderate variations in most of the
experimental parameters. A particularly remarkable and very useful
feature of STIRAP is its insensitivity to the properties of the
intermediate state $\psi_2$. For these reasons STIRAP is a very
attractive technique for quantum information processing (QIP) [2].
However, it is widely recognized that QIP requires very high
fidelities, with the admissible error of gate operations being below
$10^{-4}$ for a reliable quantum processor [3]. Utilizing a recent
idea of Guerin \emph{et al.} [4] who applied the well-known
Dykhne-Davis-Pechukas (DDP) method [5] to optimize adiabatic passage
in a two-state system, we propose how to achieve an ultrahigh
fidelity in STIRAP, and thus make it fully suitable for QIP. In
order to adapt this approach to STIRAP, we reduce the three-level
Raman system to effective two-state systems in two limits: on exact
resonance and for large single-photon detuning. The optimization,
which minimizes the nonadiabatic transitions and maximizes the
fidelity, leads to a particular relation between the pulse shapes of
the driving pump and Stokes fields. The DDP-optimized version of
STIRAP maintains its robustness against variations in the pulse
intensities and durations, the single-photon detuning and possible
losses from the intermediate state.

{\normalsize

\begin{enumerate}
\item U. Gaubatz, P. Rudecki, S. Schiemann, K. Bergmann, J. Chem.
Phys. \textbf{92}, 5363 (1990); S. Schiemann, A. Kuhn, S.
Steuerwald, K. Bergmann, Phys. Rev. Lett. \textbf{71}, 3637 (1993);
N. V. Vitanov, M. Fleischhauer, B. W. Shore and K. Bergmann, Adv.
At. Mol. Opt. Phys. \textbf{46}, 55 (2001).

\item M. Hennrich, T. Legero, A. Kuhn and G. Rempe, Phys. Rev. Lett.
\textbf{85}, 4872 (2000); A. Kuhn, M. Hennrich, and G. Rempe, Phys.
Rev. Lett. \textbf{89}, 067901 (2002); C. Wunderlich, T. Hannemann,
T. K\"{o}rber, H. H\"{a}ffner, C. Roos, W. H\"{a}nsel, R. Blatt and
F. Schmidt- Kaler, J. Mod. Opt. \textbf{54}, 1541 (2007).

\item M. A. Nielsen and I. L. Chuang, \emph{Quantum Computation and
Quantum Information} (Cambridge University Press, 1990); A. Steane,
Rep. Prog. Phys. \textbf{61}, 117 (1998).

\item S. Gu\'{e}rin, S. Thomas, and H. R. Jauslin, Phys. Rev. A
\textbf{65}, 023409 (2002);
 X. Lacour, S. Gu\'{e}rin and H. R. Jauslin, Phys. Rev. A \textbf{78}, 033417 (2008).

\item J. P. Davis and P. Pechukas, J. Chem. Phys. \textbf{64}, 3129
(1976);
 A. M. Dykhne, Sov. Phys. JETP \textbf{11}, 411 (1960).\newline

\end{enumerate}
}

\vspace{\baselineskip}