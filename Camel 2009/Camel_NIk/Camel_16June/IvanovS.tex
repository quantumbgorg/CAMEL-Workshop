\title{SELECTIVE EXCITATION OF MULTIPLE STATES IN ATOMIC SODIUM BY A SINGLE INTENSE CHIRPED FEMTOSECOND LASER PULSE}

M. Krug$^1$, T. Bayer$^1$, M. Wollenhaupt$^1$, C. Sarpe-Tudoran$^1$, T. Baumert$^1$,
\underline{S. S. Ivanov}$^2$, N. V. Vitanov$^{2}$ \index{Krug M}
\index{Bayer T} \index{Wollenhaupt M} \index{Sarpe-Tudoran C}
\index{Baumert T} \index{Vitanov N}

{\normalsize{\vspace{-4mm}
$^1$Universit\"at Kassel, Institut f\"ur Physik und CINSaT,
Heinrich-Plett-Str. 40, D-34132 Kassel, Germany

\vspace{-4mm} $^2$Department of Physics, Sofia University, 5 James
Bourchier Blvd, 1164 Sofia, Bulgaria

\email svetivanov@phys.uni-sofia.bg}}

We present a joined experimental and theoretical study on the
strong-field photo-ionization of sodium atoms using intense chirped
femtosecond laser pulses. By tuning the chirp parameter, a high
degree of selectivity among the population in the highly excited
states ($5p$, $6p$, $7p$ and $4f$, $5f$, $6f$) is observed. The
different excitation and ionization pathways enabling control are
identified by measuring Photoelectron Angular Distributions (PADs)
employing the Velocity Map Imaging (VMI) technique.  Free electron
wave packets at an energy of around 1~eV with $d$ ($l=2$) and $g$
($l=4$) symmetry are observed. These photoelectrons originate from
two channels. The first 2+1+1 Resonance Enhanced Multi-Photon
Ionization (REMPI) processes via the strongly driven two-photon
transition $4s \leftarrow\leftarrow3s$, and subsequent
ionization from the $5p$, $6p$ and $7p$ states whereas the second pathway
involves 3+1 REMPI via the $4f$, $5f$ and $6f$ states. In addition,
electron wave packets with $f$ ($l=3$) symmetry from two-photon
ionization of the non-resonant transiently populated $3p$ state are
observed at 0.2~eV. A five state model is studied theoretically to
provide insights into the physical mechanisms at play. Our analysis
shows that by tuning the chirp parameter the dynamics is effectively
controlled by dynamic Stark shifts and level crossings.  In
particular, we show that under the experimental conditions the
passage through an uncommon three-state ``bow-tie'' level
crossing allows the preparation of a coherent superposition state.

\vspace{\baselineskip}