\title{PHOTON DIODE: PERFORMING NONUNITARY \mbox{OPERATIONS} ON QUANTUM LIGHT}

\underline{G. Nikoghosyan} \index{Nikoghosyan G}

\vspace{-4mm}
{\normalsize{Fachbereich Physik,
University of Kaiserslautern,
67663 Kaiserslautern,
Germany

\email gor@physik.uni-kl.de}}

We discuss the interaction of two quantized modes of light with a
spectrally broadened atomic ensemble. We show that the system is
analogous to a two level system interacting with a bosonic
reservoir, where the photonic modes correspond to the atomic states
and the atomic ensemble corresponds to the modes of the reservoir.
In contrast to the photonic reservoirs, the atomic ensembles can be
easily controlled which can be used to simulate the dynamics of an
open two level system in a reservoir with tunable spectrum. Due to
the coupling with the atoms the analog of spontaneous decay for
photons is obtained. This process leads to an irreversible transfer
of photons from one mode to the other. The effect can be used for
large variety of applications; e.g. the creation of new quantum
states, the transfer of photons of optical frequency to microwave
domain and vice versa, or the construction of a diode for photons,
i.e. a device where single photon pulses injected in any of the two
input ports will be directed to the same output port.

\vspace{\baselineskip}