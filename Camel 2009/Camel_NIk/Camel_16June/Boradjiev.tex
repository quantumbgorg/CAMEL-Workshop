\title{IS TWO-PHOTON RESONANCE OPTIMAL FOR STIRAP?}

\underline{I. I. Boradjiev}, A. R. Rangelov, N. V. Vitanov
\index{Boradjiev I} \index{Rangelov A} \index{Vitanov N}

{\normalsize{\vspace{-4mm}
Department of Physics, Sofia University, 5 James Bourchier Blvd.,
1164 Sofia, Bulgaria

\email boradjiev@phys.uni-sofia.bg}}

The technique of stimulated Raman adiabatic passage is a highly efficient tool for coherent population transfer in a chainwise-connected three-state quantum system, 1-2-3.
It is widely accepted that the two-photon resonance between the two end states of the chain, 1 and 3, is crucial for the success of the process.
This assumption, which is almost never questioned, is perfectly justified when the two driving pulses --- pump and Stokes --- produce nearly equal couplings for the two transitions, 1-2 and 2-3.
We show, numerically and analytically, that when the pump and Stokes couplings differ significantly, the two-photon excitation profile is distorted and asymmetric with respect to two-photon resonance.
We present examples when the line profile is shifted from the two-photon resonance so much that the optimal operation of STIRAP demands significant two-photon detuning.
The results are of potential importance for a number of applications of STIRAP, particularly in situations when the pump and Stokes fields are of different nature.

\vspace{\baselineskip}