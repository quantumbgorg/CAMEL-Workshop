\title{MANY-BODY REGIME IN A RYDBERG ATOMIC SAMPLE: DIPOLE-BLOCKADE AND FEW-BODY PROCESSES}

\underline{P. Pillet} \index{Pillet P}

{\normalsize{\vspace{-4mm}
Laboratoire Aim\'e Cotton, Universit\'e Paris-Sud, B\^atiment 505, 91405 Orsay, France

\email pierre.pillet@lac.u-psud.fr}}

Due to their large polarizability, cold Rydberg atoms constitute a fantastic laboratory for studying an
ensemble of particles strongly interacting at huge atomic distances. They open many research routes
for quantum information, quantum simulation, or for the characterization of few-body effects. Two
different results will be presented and discussed.
The first one will discuss recent experimental results demonstrating for the first time a four-body
Stark-tuned F\"{o}rster resonance. We take advantage of a nearly coincidence between with two two-body
F\"{o}rster resonances, where two Rydberg atoms exchange resonantly internal energy
\[23\text{p}_{3/2}(\text{m} = 1/2) + 23\text{p}_{3/2}(\text{m} = 1/2) \rightarrow 23\text{s} + 24\text{s}\]
and
\[24\text{s} + 23\text{s} \rightarrow 23\text{p}_{1/2}(\text{m} = 1/2) + 23\text{d}_{5/2}(\text{m} = 1/2)\]
in the presence of a static electric field tuned at 80.2 V/cm and 80.8 V/cm respectively. In both cases,
the signature of the F\"{o}rster resonance is observed by detecting the higher. We observe at the
expected value of the electric field 80.3 V/cm a transfer in the 23d$_{5/2}$ level corresponding to the
signature of the four-body resonance
\[
4\times 23\text{p}_{3/2}(\text{m} = 1/2) \rightarrow 2\times 23\text{s} + 23\text{p}_{1/2} (\text{m}=1/2) + 23\text{d}_{5/2}(\text{m} = 1/2)\]
The second result will concern the treatment of the interplay between the cooperativity of the laser
excitation of an ensemble of atoms and the dipole-dipole interactions in the blockade regime. We
will show that several unexpected features of the resonance lines are the result of many-body
effects.

\vspace{\baselineskip} 