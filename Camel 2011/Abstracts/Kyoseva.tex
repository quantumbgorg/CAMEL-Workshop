\title{LUTTINGER LIQUIDS AND SPIN-CHARGE SEPARATION IN A QUANTUM OPTICAL SYSTEM}

\underline{E. Kyoseva}\index{Kyoseva E}

{\normalsize{\vspace{-4mm}
Centre for Quantum Technologies, 3 Science Drive 2, 117543 Singapore

\email cqtesk@nus.edu.sg}}

In this work we show that light-matter excitations (polaritons) generated inside a hollow-core one-dimensional fiber filled with two types of atoms, can exhibit Luttinger liquid behavior. We first explain how to prepare and drive this quantum-optical system to a strongly interacting regime, described by a bosonic two-component Lieb-Liniger model. Utilizing the connection between strongly interacting bosonic and fermionic systems, we then show how spin-charge separation could be observed by probing the correlations in the polaritons. This is performed by first mapping the polaritons to propagating photon pulses and then measuring the effective photonic spin and charge densities and velocities by analyzing the correlations in the emitted photon spectrum. The necessary regime of interactions is achievable with current quantum-optical technology.

\vspace{\baselineskip} 