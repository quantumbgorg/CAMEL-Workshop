\title{ULTRAFAST CONTROL OF ELECTRON DYNAMICS}

\underline{M. Wollenhaupt} \index{Wollenhaupt M}

{\normalsize{\vspace{-4mm}
University of Kassel, Institute of Physics und CINSaT, Heinrich-Plett-Str. 40, 34132 Kassel, Germany

\email wollenha@physik.uni-kassel.de}}

The ability to generate ultrashort laser pulses of controllable shape, instantaneous frequency and polarization with sub-attosecond precision [1] has made it possible to precisely manipulate the dynamics of quantum systems. Effective control of electron dynamics implies large population transfer rates, i.e. non-perturbative interactions, and control of the three dimensional light-matter interactions. In this contribution, both aspects of controlling electron dynamics are discussed.

In the first experiment, control of three-dimensional (3D) free electron wave packets from Resonance Enhanced Multiphoton Ionization (REMPI) of potassium atoms is achieved by polarization-shaped laser pulses [2] combined with direct 3D tomographic reconstruction based on measurements of Photoelectron Angular Distributions (PADs) [3].

In the second experiment we demonstrate strong-field control of the bound electron dynamics on potassium dimers. In our two-color energy resolved photoelectron spectroscopy experiment, the first pulse serves to prepare an electronic wave packet. An intense second pulse, the phase of which is tailored to the electronic wave packet oscillation [4], guides the system towards a preselected target channel.

Measurements of PADs are carried out in order to investigate the Circular Dichroism in the photo-ionization of chiral molecules. Asymmetric PADs resulting from REMPI of R- and S-camphor and fenchone are observed using left- and right-handed circularly polarized femtosecond laser pulses.

{\normalsize
[1] J. K\"{o}hler et al., Optics Express \textbf{19}, accepted, (2011).
\vsp

[2] M. Wollenhaupt et al., Appl. Phys. B \textbf{95}, 245 (2009).
\vsp

[3] M. Wollenhaupt et al., Appl. Phys. B \textbf{95}, 647 (2009).
\vsp

[4] T. Bayer et al., Phys. Rev. Lett. \textbf{102}, 023004-1 (2009).
}

\vspace{\baselineskip} 