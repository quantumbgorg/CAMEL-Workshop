\title{COOLING AND DETECTION OF MOLECULAR IONS}

\underline{M. Keller}\index{Keller M}

{\normalsize{\vspace{-4mm}
Department of Physics and Astronomy, University of Sussex, Pevensey 2, BN1 9QH Brighton

\email m.k.keller@sussex.ac.uk}}

Cold molecules have a multitude of applications ranging from high resolution spectroscopy and tests of fundamental theories to cold chemistry and, potentially, quantum information processing. Prerequisite for these applications is the cooling of the molecules' motion and its localisation. Furthermore, the internal state of the molecules needs to be prepared and non-destructively detected. The cooling of the motion and trapping of molecular ions can be accomplished by trapping them in an rf-trap alongside laser cooled atomic ions. The trapped ions form crystal like structures (Coulomb crystals) in which the ions are well localised within a volume of a few $\mu$�m$^3$. While blackbody assisted laser cooling was recently demonstrated, the non-destructive state detection is still beyond current experiments. Our aim is to employ tools from trapped ion QIP and quantum optics to develop novel techniques for fast molecular cooling, non-destructive state detection and molecular species detection. Mapping the internal state of molecular ions onto the ions' motion the state can be detected via atomic ions trapped alongside the molecules. Fast and effective cooling of complex molecules may be achieved by coupling the molecules to the field of an optical cavity.  Cavity assisted Raman transition can be applied to map the internal state onto the cavity field.

\vspace{\baselineskip} 