\title{ARCHITECTURES FOR ION QUANTUM TECHNOLOGY}

\underline{W. Hensinger} \index{Hensinger W}

{\normalsize{\vspace{-4mm}
Ion Quantum Technology Group, Department of Physics and Astronomy,
University of Sussex, Falmer, Brighton, East Sussex, BN1 9QH, United
Kingdom

\email w.k.hensinger@sussex.ac.uk}}

I will discuss pathways how architectures for quantum technology may be realized and recent progress
in our group. At the University of Sussex we are developing a new generation of chip scale ion trap
arrays that may accommodate large numbers of single atomic ions. The scalable fabrication of ion
trap arrays involves advanced nanofabrication techniques including photolithography. Incorporating
MEMS fabrication technologies we report on design and fabrication of a multi-zone y-junction
surface-electrode ion trap. We have also trapped single ytterbium ions in an experimental setup
particularly designed for the development of advanced ion trap chips. This setup allows for rapid
turn-around time, optical access for all type of ion trap chips and up to 100 electric
interconnects. I will also present progress on a microfabricated 2-D ion trap array that allows for
the creation of ion lattices for the implementation of a quantum simulator.  Shuttling ions in
multidimensional structures will likely form an important tool for the interchange of quantum
information and I will provide in an introduction into the development of optimal electrode
geometries including junctions.  I will also show a perspective of the work that still needs to be
carried out in order to produce practical devices including the realisation of advanced on-chip
features and highlight the importance of the condensed matter in atomic physics interface.

\vspace{\baselineskip} 