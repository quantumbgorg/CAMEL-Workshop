\title{SLOW-LIGHT AND RYDBERG POLARITONS}

\underline{M. Fleischhauer}
\index{Fleischhauer M}

{\normalsize{

\vspace{-4mm} Department of Physics \& Research Center OPTIMAS, University of Kaiserslautern
Germany

\email mfleisch@physik.uni-kl.de}}

Dark-state or slow-light polaritons are quasi-particles generated
in the interaction of light with multi-level atoms driven
by an external laser close to a Raman resonance. Their
dispersion relation can be controlled to a large extend,
representing massive Schr\"{o}dinger particles
on the one hand or multi-component objects with
a Dirac-like spectrum on the other. In the latter case
the ``relativistic'' length and energy scales can be widely tuned,
making effects accessible in the lab. Also more exotic systems can
be realized and experimentally studied such as
the random-mass Dirac model, which shows anomalous
localization phenomena. In the second part of the talk the prospects to
create strong interactions between dark-state polaritons using Rydberg atoms
will be discussed. The dipole-dipole coupling between atoms in a Rydberg state
leads to a strong and long-range interaction between polaritons, as well
as to a blockade phenomenon. This interaction can give rise to
interesting many-body phenomena, such as hard-sphere two-particle correlations,
crystallization or quantum Hall states.

\vspace{\baselineskip}
