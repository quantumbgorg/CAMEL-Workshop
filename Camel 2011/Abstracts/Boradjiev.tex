\title{PIECEWISE ADIABATIC PASSAGE DYNAMICS REVEALED BY ADIABATIC FLOQUET THEORY}

\underline{I. I. Boradjiev}
\index{Boradjiev I}

{\normalsize{\vspace{-4mm}
\dijon

\email boradjiev@phys.uni-sofia.bg}}

Coherent excitation of quantum systems with ultrashort laser pulses is essential for a variety of
fields such as spectroscopy, quantum information and control of molecular dynamics. Recently, the
proposed method of piecewise adiabatic passage (PAP) appears to be a promising technique for executing
complete and robust population transfer between two quantum states using a series of femtosecond laser
pulses with constant (but varying from pulse to pulse) phases. Exploiting the periodicity of the
femtosecond pulses combined with the pulse to pulse slowly varying peak amplitudes and phases, we
analyze this piecewise problem with the adiabatic Floquet theory. Then, with the use of Kolmogorov-
Arnold-Moser type perturbation theory we obtain an effective smooth (nonpiecewise) Hamiltonian, and
analyze the conditions for adiabatic transfer. Due to its effectively adiabatic nature the method of
PAP is robust to a variation of intensities, durations and shapes of the pulses.

\vspace{\baselineskip} 