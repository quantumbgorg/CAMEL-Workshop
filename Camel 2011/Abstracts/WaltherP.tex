\title{PHOTONIC ENTANGLEMENT FOR QUANTUM COMPUTATION AND QUANTUM SIMULATION}

\underline{P. Walther}
\index{Walther P}

{\normalsize{

\vspace{-4mm} Faculty of Physics, University of Vienna, Boltzmanngasse 5, A-1090 Vienna, Austria

\email philip.walther@univie.ac.at}}

Single photons are ideal carriers of quantum information due to their low-noise properties and
ease of manipulation at the single-qubit level. Therefore, the applications are manifold and reach
from quantum communication and quantum metrology to optical quantum computing and
lately quantum simulations.
Quantum simulators, as suggested originally by Richard Feynman, are controllable quantum systems that can be used to mimic other quantum systems, thus being able to tackle problems that are intractable on classical computers. The usage of entangled photon states for the simulation of other quantum systems is an entirely new research direction and has just become possible by the latest theoretical and experimental developments. Since quantum simulations are conjectured to be less demanding than quantum computations implies that this might be the most feasible application for small- and mediumscale quantum systems, such as in quantum chemistry.
The development and combination of new quantum technologies for the generation and
manipulation of multi-photon entanglement will enable quantum simulations that provide
insight into other quantum systems, but also quantum computation experiments in quantum
networks.

\vspace{\baselineskip}
