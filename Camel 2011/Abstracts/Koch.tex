\title{CONTROLLING A SHAPE RESONANCE WITH NON-RESONANT LIGHT}

\underline{C. Koch} \index{Koch C}

{\normalsize{\vspace{-4mm}}
Universit\"{a}t Kassel,
Fachbereich 10 -- Mathematik und Naturwissenschaften,
Institut f\"{u}r Physik,
Heinrich-Plett-Str. 40,
D-34132 Kassel

\email christiane.koch@uni-kassel.de}

A (diatomic) shape resonance is a metastable state of a pair of colliding atoms quasi-bound by the centrifugal barrier imposed by the angular momentum involved in the collision. The temporary trapping of the atoms' scattering wavefunction corresponds to an enhanced atom pair density at low interatomic separations which can be useful for making molecules from atom pairs. However, for an ensemble of atoms, the atom pair density will only be enhanced if the energy of the resonance comes close to the temperature of the atomic ensemble. Here, we seek to control the energy of a shape resonance shifting it toward the temperature of atoms confined in a trap. The shifts are imparted by the interaction of non-resonant light with the anisotropic polarizability of the atom  pair, which affects both the centrifugal barrier and the pair's rotational and vibrational levels. We study the effect on the pair density of rubidium and strontium atoms as a function of the nonresonant light intensity.

\vspace{\baselineskip}
