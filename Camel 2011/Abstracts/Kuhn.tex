\title{A CAVITY-BASED INTERFACE FOR MATTER AND LIGHT AT THE SINGLE-PARTICLE LEVEL}

\underline{A. Kuhn}\index{Kuhn A}

{\normalsize{\vspace{-4mm}
University of Oxford, Clarendon Laboratory, Parks Road, OX1 3PU Oxford, United Kingdom

\email axel.kuhn@physics.ox.ac.uk}}

A novel scheme for generating arbitrarily shaped single photons from coupled atom-cavity systems is introduced, and its
experimental implementation is shown. Nearly indistinguishable photons get deterministically produced at high repetition rate and
efficiency. Based on this approach, I also show how to capture photons of arbitrary temporal shape with one atom coupled to an
optical cavity. A control pulse of suitable shape ensures impedance matching throughout the pulse, resulting in complete state
mapping from photon to atom. For most possible photon shapes,  an unambiguous analytic expression for the temporal shape of
the control pulse is derived.

\vspace{\baselineskip} 