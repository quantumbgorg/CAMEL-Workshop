\title{CAVITY QUANTUM ELECTRODYNAMICS WITH A RYDBERG-BLOCKED ATOMIC ENSEMBLE}

\underline{E. Brion} \index{Brion E}

{\normalsize{\vspace{-4mm}
Laboratoire Aim\'e Cotton, Universit\'e Paris-Sud, B\^atiment 505, 91405 Orsay, France

\email etienne.brion@u-psud.fr}}

The realization of a Jaynes-Cummings model in the optical domain is proposed for an atomic ensemble. The scheme exploits the collective coupling of the atoms to a quantized cavity mode and the nonlinearity introduced by coupling to high-lying Rydberg states. A two-photon transition resonantly couples the single-atom ground state $\ket{g}$ to a Rydberg state $\ket{e}$ via a nonresonant intermediate state $\ket{i}$, but due to the interaction between Rydberg atoms only a single atom can be resonantly excited in the ensemble. This restricts the state space of the ensemble to the collective ground state $\ket{G}$ and the collectively excited state $\ket{E}$ with a single Rydberg excitation distributed evenly on all atoms. The collectively enhanced coupling of all atoms to the cavity field with coherent coupling strengths which are much larger than the decay rates in the system leads to the strong coupling regime of the resulting effective Jaynes-Cummings model. We use numerical simulations to show that the cavity transmission can be used to reveal detailed properties of the Jaynes-Cummings ladder of excited states and that the atomic nonlinearity gives rise to highly nontrivial photon emission from the cavity. Finally, we suggest that the absence of interactions between remote Rydberg atoms may, due to a combinatorial effect, induce a cavity-assisted excitation blockade whose range is larger than the typical Rydberg dipole-dipole interaction length.

\vspace{\baselineskip} 