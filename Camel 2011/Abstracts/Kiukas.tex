\title{FULL COUNTING STATISTICS FOR PARTICLE BEAMS}

\underline{J. Kiukas} \index{Kiukas J}

{\normalsize{\vspace{-4mm}
Leibniz Universit\"{a}t Hannover, Institute for Theoretical Physics,
Appelstrasse 2, 30167 Hannover, Germany

\email jukka.kiukas@itp.uni-hannover.de}}

We present a general, theoretical framework for treating particle beams as time-stationary limits of many particle systems. Due to
stationarity, the total particle number diverges, and a description in Fock space is no longer possible. Nevertheless, we show that
when describing the particle detection via the 2nd quantized arrival time observables, such beams exhibit a well-defined ``local''
counting statistics, that is, full counting statistics of all clicks falling into any given finite time interval. We also treat in detail a
realization of such a beam via the long time limit of a source creating particles in a fixed initial state from which they then evolve
freely. The scheme applies to massive as well as massless particles. In particular, it provides an approach to photon counting based
on the axiomatic definition of the detection time.

\vspace{\baselineskip} 