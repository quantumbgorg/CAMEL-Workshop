\title{OSCILLATIONS IN A THREE-LEVEL NEUTRINO SYSTEM}

\underline{H. Hristova}, S. S. Ivanov, and N. V. Vitanov
\index{Hristova H}\index{Ivanov S}\index{Vitanov N}

{\normalsize{

\vspace{-4mm} \unisofia

\email tinardie@abv.bg}}

We consider the level-crossing problem in a system with two active neutrino flavors and one sterile neutrino. The energies of the three states depend linearly on matter density, whereas the interactions between them are constant. We suppose a connection between mass differences and mixing angles to obtain a simple form for the Hamiltonian. The propagator and the transition probabilities are calculated by assuming independent pairwise Landau-Zener transitions occurring instantly at the avoided crossings, and adiabatic evolution is assumed elsewere. Different evolution paths are possible in Hilbert space between an initial and a final flavor state and quantum interferences are identified. These results are of potential interest in neutrino physics.

%(We use the results to obtain some limits  for sterile neutrino mixing angles and mass differences.)

\vspace{\baselineskip} 