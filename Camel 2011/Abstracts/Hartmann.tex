\title{NON-CLASSICAL STATES OF NANO-MECHANICAL OSCILLATORS}

\underline{M. Hartmann} \index{Hartmann M}

{\normalsize{\vspace{-4mm}
Technische Universit\"{a}t M\"{u}nchen, Physik Department, T34 James Franck Strasse, 85748 Garching, Germany

\email michael.hartmann@ph.tum.de}}

As versatile qubit transducers, for high precision displacement measurements and to explore possible fundamental
limits of quantum theory, nano-mechanical oscillators are currently emerging as promising devices since cooling
them to their quantum ground states has recently been achieved. A next step for studying quantum dynamics in
these structures will now be to prepare them in states with distinguished features of non-classicality. In this talk, I
will discuss optomechanical schemes to prepare such mechanical oscillators in states that are entangled or have
negative Wigner functions.

First, I will consider two dielectric membranes suspended inside a Fabry-P\'{e}rot-cavity, which are cooled via a suitable
lasers drive. In this scenario, the mechanical vibrations of the membranes can become entangled in the steady state.
They thus form two mechanical degrees of freedom that share steady state entanglement.

In the second part of the talk, I will discuss mechanical oscillators that couple to the evanescent field of high finesse
optical cavities and show how the intrinsic geometric nonlinearity of these oscillators can be enhanced to such an
extent that it becomes comparable line-width of the cavity. This regime opens up many new possibilities to
manipulate the mechanical oscillator, including the generation of non-classical states. I will focus here on the
preparation of phonon Fock states which feature negative Wigner functions.

\vspace{\baselineskip} 