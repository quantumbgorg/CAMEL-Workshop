\title{COLD ATOMS IN LAGUERRE-GAUSSIAN LASER BEAMS}

\underline{L. Pruvost}
\index{Pruvost L}

{\normalsize{
\vspace{-4mm}
Laboratoire Aim\'e Cotton, CNRS, Universit\'e Paris-Sud, B\^atiment 505, 91405 Orsay, France

\email laurence.pruvost@lac.u-psud.fr}}

A Laguerre-Gaussian (LG) laser mode, blue-detuned from the atomic resonance, constitutes a power-law trap for cold atoms because the laser intensity profile varies as $r^{2x}$, where $x$ is the order of the LG mode.
We have shown that very accurate LG modes can be generated from a usual Gaussian mode, and applying a phase-hologram with a spatial light modulator and that they can be used to guide cold atoms. The effect which has been experimentally studied versus the order of the mode and the laser detuning, has been quantitatively interpreted by a 2D-capture model.
LG modes or deformed LG modes could be used for other applications, where specific traps are required. For example, a Bose-Einstein condensate realized in a power-law trap is expected to have different properties than the condensate in a harmonic trap. Condition for condensation is less severe and the condensate would be homogeneous.

\vspace{\baselineskip} 