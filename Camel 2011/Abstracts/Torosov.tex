\title{COMPOSITE ADIABATIC PASSAGE}

\underline{B. T. Torosov}$^{1,2}$, S. Gu\'{e}rin$^{2}$ and N. V. Vitanov$^{1}$
\index{Torosov B}\index{Gu\'{e}rin S}\index{Vitanov N}

{\normalsize{

\vspace{-4mm} $^1$\unisofia

\vspace{-4mm} $^2$\dijon

\email torosov@phys.uni-sofia.bg}}

We present a method for optimization of the technique of adiabatic passage between two quantum states by composite sequences of chirped pulses: composite adiabatic passage (CAP). The nonadiabatic losses can be canceled to any desired order with sufficiently long sequences, regardless of the nonadiabatic coupling, by choosing the relative phases between the constituent pulses appropriately. The values of the composite phases are universal for they do not depend on the pulse shapes and the chirp. The accuracy of the CAP technique and its robustness against parameter variations make CAP suitable for ultrahigh-fidelity quantum information processing.

{\normalsize
[1] B. T. Torosov, S. Gu\'{e}rin and N. V. Vitanov, accepted in Phys. Rev. Lett.
}

\vspace{\baselineskip} 