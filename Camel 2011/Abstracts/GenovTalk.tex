\title{COMPOSITE PULSES FOR MULTISTATE QUANTUM SYSTEMS}

\underline{G. T. Genov}$^{1}$, B. T. Torosov$^{1,2}$, N. V. Vitanov$^{1}$
\index{Genov G} \index{Torosov B} \index{Vitanov N}

{\normalsize{

\vspace{-4mm} $^1$\unisofia

\vspace{-4mm} $^2$Institute of Solid State Physics, Bulgarian Academy of Sciences, Tsarigradsko Chaussee 72, 1784 Sofia, Bulgaria

\email genko.genov@phys.uni-sofia.bg}}

Composite pulses are used in nuclear magnetic resonance [1-2], and most recently, in trapped-ion quantum information processing [3-4] for various applications, including selective excitation and qubit rotations that are more robust against variations in the pulse area and the detuning than single resonant pulses of precise area.

We describe the design and applications of composite pulses for two types of multistate quantum systems: (1) systems with SU(2) dynamic symmetry and (2) systems, which are suitable for Morris-Shore decomposition [5]. For multistate systems with SU(2) symmetry, we use the Majorana decomposition [6] to reduce their dynamics to the ones of effective two-state systems. These transformations allow us to find the propagators for the multistate systems analytically and use the pool of available composite pulses. For the second type of multistate systems, we use the Morris-Shore decomposition, which again reduces their dynamics to the ones of effective two-state systems and allows us to find their propagators analytically. Additionally, we show some applications of composite pulses to achieve pulse area compensation, detuning compensation, and selective excitation in such multistate systems. Finally, we give explicit examples for the application of composite pulses in three-state quantum systems with these characteristics.

{\normalsize
[1] M. H. Levitt, Prog. NMR Spectrosc. \textbf{18}, 61 (1986).
\vsp

[2] S. Wimperis, J. Magn. Reson. \textbf{109}, 221 (1994).
\vsp

[3] H. Haffner, C.F. Roos, R. Blatt, Phys. Rep. \textbf{469}, 155 (2008).
\vsp

[4] B. T. Torosov and N. V. Vitanov, Phys. Rev. A \textbf{83}, 053420(7) (2011).
\vsp

[5] J. R. Morris and B. W. Shore, Phys. Rev. A \textbf{27}, 906 (1983).
\vsp

[6] E. Majorana, Nuovo Cimento \textbf{9}, 43 (1932).
}

\vspace{\baselineskip} 