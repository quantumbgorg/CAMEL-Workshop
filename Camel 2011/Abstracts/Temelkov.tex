\title{TEMPERATURE OF COLD RB CLOUD BY SHAKING THE MOT}

\underline{I. Temelkov}, G. Dobrev, E. Dimova, A. Pashov, K. Blagoev, N. V. Vitanov
\index{Temelkov I}\index{Dobrev G}\index{Dimova E}\index{Pashov A}\index{Blagoev K}\index{Vitanov N}

{\normalsize{\vspace{-4mm}
\unisofia

\email i\_temelkov@phys.uni-sofia.bg}}

We present a characterization of the first MOT in Bulgaria realized with
Rb atoms. For measuring the temperature a non-standard technique was used, namely
by observing the oscillations of the cold ensemble when shaking the zero point of the
trapping magnetic field (as proposed in M. I. den Hertogin's thesis, Utrecht University, 2004).
The main theoretical background of this method
is presented and compared with experimental results. Advantages and
drawbacks of the method are discussed.

\vspace{\baselineskip} 