\title{IONS (AND ATOMS) FOR QUANTUM SIMULATIONS}

\underline{T. Sch\"{a}tz} \index{Sch\"{a}tz T}

{\normalsize{\vspace{-4mm}
Hans Kopfermann Str. 1, 85748 Garching, Germany

\email tschaetz@mpq.mpg.de}}

Direct experimental access to some of the most intriguing and puzzling quantum phenomena is difficult due to their fragility to noise. Their simulation on conventional computers is impossible, since quantum behaviour is not efficiently translatable in classical language. However, one could gain deeper insight into complex quantum dynamics via experimentally simulating the quantum behaviour of interest in another quantum system, where not all but the relevant parameters and interactions can be controlled and robust effects detected sufficiently well. One example is simulating quantum-spin systems with trapped ions.

I will summarize our status and our view of the perspectives of the following topics:

(1) Analogue Quantum simulations in surface radiofrequency traps:
We recently trapped Mg ions in a linear surface trap (SANDIA-Eurotrap). We are close to enthrone our 2D surface trap (collaboration with NIST, SANDIA and R. Schmied) and are eager to explore its capabilities for simulations (collaboration with D. Porras and A. Bermudez).

(2) Optical trapping of ions towards a new class of Analogue Quantum Simulations:
We recently trapped an Mg ion in an optical dipole trap. We very recently trapped in a 1D optical lattice. We currently try to cool the optically trapped ion (collaboration with A. Retzker and A. Albrecht), investigate altered trapping conditions (collaboration with C. Cormick and G. Morigi) and propose first potential simulations (collaboration with P. Hauke).

\vspace{\baselineskip} 