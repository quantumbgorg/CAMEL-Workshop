\title{MAGNETO-OPTICAL TRAPPING OF NEUTRAL MERCURY}

\underline{T. Walther} \index{Walther T}

{\normalsize{\vspace{-4mm}
Institute for Applied Physics, TU Darmstadt, Schlossgartenstr. 7, D-64289 Darmstadt, Germany

\email thomas.walther@physik.tu-darmstadt.de}}

We report on our work of cooling neutral mercury in a magneto-optical trap. Cooling transition is the spin forbidden $^1\text{S}_0\rightarrow\,^3\text{P}_1$ transition at 253.7 nm.  We trapped the bosonic $^{202}$Hg isotope and the fermionic $^{199}$Hg isotope.

Based on these results many exciting exciting experiments in cold atomic and molecular physics will be possible including a time standard based on trapping the $^{199}$Hg isotope in an optical lattice at the magic wavelength.

However, our main goal are experiments towards the formation of ultracold diatomic Hg-molecules by photo-association. Of particular interest is that the molecular structure of the Hg dimers allows for the additional vibrational cooling of the dimers in a quasi-closed cycling transition after they have been formed. We will discuss the details and opportunities of this approach.

\vspace{\baselineskip} 