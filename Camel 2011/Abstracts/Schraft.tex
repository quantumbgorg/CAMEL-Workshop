\title{FEEDBACK-CONTROLLED OPTIMIZATION OF LIGHT STORAGE BY EIT IN A DOPED CRYSTAL}

\underline{D. Schraft} \index{Schraft D}

{\normalsize{\vspace{-4mm}
Institute of Applied Physics, Technical University of Darmstadt, Hochschulstrasse 6, 64289 Darmstadt

\email daniel.schraft@physik.tu-darmstadt.de}}

In recent years, coherent-adiabatic interactions opened a wide field of applications. The phenomenon of electromagnetically induced transparency (EIT) leads to the concept of light storage, which may play an essential role in optical (or quantum) information storage. In light storage by EIT a probe (data) pulse is encoded in atomic coherences of a quantum system. Though the effect is in principle well understood, future real-life applications will require efficient coherent storage of large amounts of data.
The efficiency of light storage by EIT can be controlled by the temporal intensity and frequency profiles of the driving laser pulses. However, analytic solutions for optimal temporal pulse shapes are limited to simple systems without unknown perturbations. Feedback-controlled pulse shaping involving evolutionary algorithms (EA) offers a powerful approach also in rather complex systems or with perturbations. EAs enable us to search for optimal solutions over a complex parameter space.
We experimentally implemented a feedback-controlled pulse shaping involving an EA for light storage by EIT in Pr:YSO. The EA finds optimal solutions for an efficient preparation pulse, which optically prepares the crystal prior to light storage. The required preparation time could be reduced by a factor of 15. Furthermore, the EA enhances the optical depth in the crystal by a factor of 6. Moreover, the EA determines an optimal intensity profile of the coupling pulse for a given probe pulse.

\vspace{\baselineskip} 