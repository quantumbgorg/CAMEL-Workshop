\title{TOPOLOGICAL PHOTONIC MODES IN METAMATERIALS}

\underline{V. Yannopapas} \index{Yannopapas V}

{\normalsize{\vspace{-4mm}
Department of Materials Science, University of Patras, Panepistimioupolis, GR-26504, Rio, Greece

\email vyannop@upatras.gr}}

In topological phases of matter, properties like conductivity and optical reflectivity are not determined by the crystalline order but by the topological order of the electronic states. Such states of matter are the integer quantum Hall state [1], and the recently discovered topological insulators [2]. A typical example of an integer quantum Hall state is graphene under the influence of a periodic magnetic field while topological insulators are binary alloys based on Bi, e.g., Bi1-xSbx. Here we propose that we can have topological states of the electromagnetic field in certain metamaterial structures operating as photonic analogues of the above electronic states of matter.
We show in particular, that a gyrotropic (chiral) medium supporting a longitudinal-wave excitation exhibits a Dirac point in the corresponding photon dispersion lines. By breaking the time-reversal symmetry in such a medium, the dispersion relation resembles the energy dispersion of a spin-polarized two-dimensional electron gas with Rashba spin-orbit coupling. The resulting split bands of the dispersion relation correspond to nonzero Chern numbers implying the existence of nontrivial topological states of the electromagnetic field [3].
We also show that a tetragonal lattice of weakly interacting particles with uniaxial electromagnetic response is the photonic counterpart of topological crystalline insulators, a new topological phase of atomic band insulators [4]. Namely, the frequency band structure stemming from the interaction of resonant modes of the individual cavities exhibits an omnidirectional band gap within which gapless surface states emerge for finite slabs of the lattice [5]. Due to the equivalence of a topological crystalline insulator with its photonic analog, the frequency band structure of the latter can be characterized by a Z2 topological invariant.

{\normalsize
[1] D. Xiao, M. C. Chang, and Q. Niu, Rev. Mod. Phys \textbf{82}, 1959 (2010).
\vsp

[2] M. Z. Hasan and C. L. Kane, Rev. Mod. Phys. \textbf{82}, 3045 (2010).
\vsp

[3] V. Yannopapas, Phys. Rev. B \textbf{83}, 113101 (2011).
\vsp

[4] L. Fu, Phys. Rev. Lett. \textbf{106}, 106802 (2011).
\vsp

[5] V. Yannopapas, submitted.
}

\vspace{\baselineskip} 