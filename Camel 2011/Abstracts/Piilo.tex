\title{CONTROLLING THE TRANSITION FROM MARKOVIAN TO NON-MARKOVIAN QUANTUM DYNAMICS}

B.-H. Liu, L. Li, Y.-F. Huang, C.-F. Li,  G.-C. Guo, E.-M. Laine, H.-P. Breuer, and \underline{J. Piilo}
\index{Piilo J}

{\normalsize{\vspace{-4mm}
Department of Physics and Astronomy, University of Turku, Finland

\email jyrki.piilo@utu.fi}}

Realistic quantum mechanical systems are always exposed to an external environment. The presence of the environment is often
considered to give rise to a Markovian process in which the quantum system irretrievably loses information to its surroundings.
However, many quantum systems exhibit a pronounced non-Markovian behavior in which there is a flow of information from the
environment back to the open system, signifying the presence of quantum memory effects. The environment is usually composed
of a large number of degrees of freedom which are difficult to control, but some sophisticated schemes for modifying the
environment have been developed. We report on an experiment which allows through selective preparation of the initial
environmental states to drive the open system dynamics from the Markovian to the non-Markovian regime, to control the
information flow between the system and the environment, and to determine the degree of non-Markovianity by direct
measurements on the open system.

\vspace{\baselineskip} 