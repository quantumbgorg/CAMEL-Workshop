\title{ION-PHOTON CAVITY QED}

\underline{W. Lange} \index{Lange W}

{\normalsize{\vspace{-4mm}
University of Sussex, Department of Physics and Astronomy, Falmer, Brighton, East Sussex, BN1 9QH, United Kingdom

\email w.lange@sussex.ac.uk}}

Interfacing the quantum states of ions and photons is a task of fundamental importance in quantum information processing. It is the basis for quantum networking and distributed quantum computation which extend dramatically the capabilities of individual quantum processors. Local interactions between qubits could also benefit from replacing phonon-mediated gates by their photonic counterparts, as they are less sensitive to thermal fluctuations. Coupling the quantum states of ions and photons with high fidelity uses cavity-QED processes and hence requires an optical cavity with small mode volume to enhance the coupling strength. It is a technological challenge to integrate miniature cavities in an ion trap.

At Sussex, three different approaches are currently investigated, accessing different regimes of atom-photon interactions. In a cavity collinear with the axis of a linear trap, moderate ion-photon coupling strength can be achieved. This can be exploited for probabilistic processes based on the entanglement of ions and single photons emitted from the cavity. Projective measurement of two photons with orthogonal polarization can be used to entangle selected pairs of ions in a string. Stronger coupling is obtained in a cavity oriented transverse to the axis of the trap. In this case, deterministic transfer of quantum states between ions and photons is possible. By analogy with ions coupled to a collective phonon mode, ions simultaneously interacting with the cavity mode may be entangled through the exchange of actual or virtual photons. For the strongest interaction, conventional mirrors have to be replaced by the end-facets of optical fibres. We have machined them to a radius of curvature of 300$\mu$m, leading to a cavity with ultra-small mode volume. In our setup, a pair of optical fibers is tightly integrated in a miniature endcap trap. With this system, we have efficiently captured the fluorescence of a single calcium ion, measured its $g^{(2)}$-function and produced single photons on demand, even without coating the optical fiber facets. With no adjustments required for the injection and extraction of radiation by optical fibers, this is a very robust system for interfacing ions and photons.

\vspace{\baselineskip} 