\title{TEMPORAL EVOLUTION OF ULTRACOLD POLAR MOLECULES IN CIRCULARLY POLARIZED MW FIELD}

\underline{T. Kirova}\index{Kirova T}

{\normalsize{\vspace{-4mm}
National Institute for Theoretical Physics, 10 Marais Street, Stellenbosch, South Africa

\email kirova@sun.ac.za}}

The cooling and trapping of atoms,easily achieved in numerous labs, has moved in the direction of cooling and trapping of polar molecules. Possessing a permanent dipole moment and interacting via long range dipole-dipole (DD) forces, ultracold polar molecules offer new prospects of exciting physics. We analyze the temporal evolution of ultracold molecular population in a microwave (mw) field with a circular polarization. The polar molecules are treated as rigid rotors with a permanent dipole moment. The Hamiltonian of two ground state polar molecules is constructed in the basis set of two-molecule bare states and the equation of motion is solved.

We study the effect of different mw field parameters and molecular gas characteristics on the dynamics of all states. Several ranges of the intermolecular distance $R$ are tested, corresponding to various densities of a molecular gas. The behavior of populations dynamics is defined by the ratio of the mw field Rabi frequency $\Omega$ and the magnitude of the DD interaction $V_{\text{DD}}$, with beating and oscillations occurring in the populations time-development as the $V_{\text{DD}}$ becomes comparable to $\Omega$.

The behavior of bare and dressed states populations as a function of the mw field detuning shows a three-peak structure for certain intermolecular distances. We associate the origin of these peaks with the existence of three avoided crossings in the eigenvalue spectrum. The latter occurs at certain value of the DD interaction when it becomes equal to the absolute value of the mw field detuning.

We found that only the ``exchange'' channel, which is caused by the exchange of ground and first excited rotational states is responsible for the appearance of these peaks, while at small enough distances inelastic processes play a larger role and cause shifts of peaks into the region of positive detunings.

\vspace{\baselineskip} 