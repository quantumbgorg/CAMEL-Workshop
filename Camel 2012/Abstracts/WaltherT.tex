\title{QUANTUM OPTICAL EXPERIMENTS BASED ON NEUTRAL MERCURY}

\underline{T. Walther}, B. Rein and H. John\index{Walther T}

{\normalsize{\vspace{-4mm}
Institute for Applied Physics, TU Darmstadt, Schlossgartenstr. 7, D-64289 Darmstadt, Germany

\email thomas.walther@physik.tu-darmstadt.de}}

We report on our work on quantum optical experiments based on mercury.
%\vspace{-3mm}

The first experiment concerns cooling neutral mercury in a magneto-optical trap [1]. Cooling transition is the spin forbidden $^1S_0\rightarrow \,^3P_1$ transition at 253.7 nm.  We give an update of the experimental progress towards experiments on photo association spectroscopy in order to form ultra-cold Hg dimers.

The second experiment deals with the implementation of a four-level Lasing without Inversion (LWI) scheme in mercury. It is based on theoretical calculations by Fry {\it et al.} [2] and aims at demonstrating LWI based UV cw-lasers at 253.7 nm or 185 nm, respectively. Laser radiation of two lasers, both of which are at a longer wavelength than that of the LWI laser output, is required to set up the necessary coherence in a mercury filled gas cell. The particular alignment of the lasers enables compensation for the Doppler effect leading to high efficiency and a relatively straightforward setup. During the talk we will present the details of the experimental idea as well as the present status of the experiment.

{\normalsize
[1] P. Villwock, S. Siol, and T. Walther, Eur. Phys. J. D, 65:0 251-255 (2011).
\vsp

[2] E.S. Fry, M.D. Lukin, Th. Walther, and G.R. Welch, Optics Communications, 179:0 499-504 (2000).
}

\vspace{\baselineskip} 