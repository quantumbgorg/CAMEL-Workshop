\title{THERMOMETRY OF COLD ATOMIC GASES BY ELECTROMAGNETICALLY INDUCED TRANSPARENCY}

\underline{T. Peters} \index{Peters T}

{\normalsize{\vspace{-4mm}
Institute of Applied Physics, Technical University of Darmstadt,
64289 Darmstadt, Germany

\email thorsten.peters@physik.tu-darmstadt.de}}

Precise thermometry of ultra-cold atomic gases is a crucial requirement for many experiments. Thermometry is typically realized by time-of-flight (TOF) measurements.
This technique is easy to implement and precise -- but also slow and destructive to the atomic cloud. In this talk we present experimental results on temperature measurements of an ultra-cold atomic gas by electromagnetically-induced transparency (EIT). As an important feature in EIT, the
absorption of a probe beam depends on the Doppler broadening, i.e., the temperature of the medium. This enables determination of temperatures from rather simple EIT spectra. The technique is robust, fast and quasi non-destructive. We compare the data to numerical simulations, as well as temperature measurements by TOF.

\vspace{\baselineskip} 