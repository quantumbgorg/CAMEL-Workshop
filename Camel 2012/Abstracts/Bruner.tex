\title{CLOCKING OF ELECTRON TUNNELING AND ITS APPLICATIONS FOR RESOLVING ULTRAFAST ELECTRON DYNAMICS}

\underline{B. Bruner} \index{Bruner B}

{\normalsize{\vspace{-4mm}
Department of Physics of Complex Systems, Rm. 306 Weizmann Institute
of Science, Rehovot 76100, Israel

\email barry.bruner@weizmann.ac.il}}

The tunneling of a particle through a barrier is one of the most fundamental and
ubiquitous quantum processes.  When induced by an intense laser field, electron tunneling from
atoms and molecules initiates a broad range of phenomena that can evolve on an attosecond (1 as =
$10^{-18}$ s) time scale [1].  Here we study laser induced tunneling by using a weak probe field to
manipulate the dynamics of the tunneled electron and monitor the effect on the attosecond light
bursts emitted when the liberated electron re-encounters the parent ion.  This allows for
investigation of a general issue much debated since the early days of quantum mechanics, namely, the
link between the tunneling of an electron through a barrier and its dynamics outside the barrier
[2].  This method can also be used to study the validity of the strong field approximation -- the
physical basis behind the generation of high harmonics and attosecond light pulses -- in
increasingly complex systems.  In addition, using aligned carbon dioxide molecules, we demonstrate
the ability to detect signatures of electron tunneling in systems where ionization from more than
one orbital occurs, paving the way for time resolving multielectron rearrangements in atoms and
molecules -- one of the key challenges in ultrafast science [3].

{\normalsize
[1] M. Hentschel et al., Nature, vol. \textbf{414}, 509-513 (2001).
\vsp

[2] M. Buettiker and R. Landauer, Phys. Rev. Lett., vol. \textbf{49} (23), 1739-1742 (1982).
\vsp

[3] D. Shafir et al., Nature, vol. \textbf{485}, 343-346 (2012).
}

\vspace{\baselineskip} 