\title{CAVITY OPTOMECHANICAL SYSTEMS AND THE QUANTUM REGIME}

\underline{W. Wieczorek} \index{Wieczorek W}

{\normalsize{\vspace{-4mm}
University of Vienna,
Faculty of Physics,
Quantum Optics, Quantum Nanophysics, Quantum Information,
Boltzmanngasse 5,
1090 Vienna,
Austria

\email witlef.wieczorek@univie.ac.at}}

Cavity optomechanical systems are comprised of (at least) two coupled harmonic oscillators: a mechanical oscillator and a single mode of the light field. Their mutual interaction -- e.g. via radiation pressure -- allows to control the mechanical oscillator as well as to engineer the optical field. The optical field usually acts as a zero entropy bath that is able to cool the mechanical oscillator into its quantum ground state, which has recently been achieved. On the other hand, the mechanical oscillator will have an effect on the light field, which has been for example demonstrated via a strongly coupled optomechanical system exhibiting normal mode splitting. It is anticipated that cavity optomechanical systems will enter a regime, where macroscopic quantum physics experiments can be performed. This hinges on entering the quantum regime of a cavity optomechanical system. We will present experiments towards reaching that goal.

\vspace{\baselineskip} 