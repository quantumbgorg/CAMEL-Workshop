\title{DRESSED ATOMS: NEW REGIMES, TRAPS AND LIMITS TO TRAPPING}

\underline{B. Garraway} \index{Garraway B}

{\normalsize{\vspace{-4mm}
Department of Physics and Astronomy, University of Sussex, Falmer,
Brighton, BN1 9QH, United Kingdom

\email b.m.garraway@sussex.ac.uk}}

A wide range of atom traps can be achieved with RF dressed atoms [1]
with potential applications to sensing, metrology and interferometry.
This wide range is due to the flexibility inherent in the vector
coupling of a magnetic dipole moment to static and EM fields which can
be each be varied in time, frequency and space. In addition, multiple
frequencies can be used to control and probe the atoms [2]. In this
talk we will report on some new designs of dressed atom traps for cold
atoms. Demanding applications require design compromise and in this
work we also examine the limits to dressed RF trapping from
non-adiabatic effects. A simple model of decay is developed for this
system which exhibits a decay threshold as a function of trap
parameters [3].

{\normalsize
[1] O. Zobay and B. M. Garraway, Phys. Rev. Lett. \textbf{86} 1195 (2001);
Y. Colombe et al., Europhys. Lett. \textbf{67}, 593 (2004);
T. Schumm et al. Nat. Phys. \textbf{1} 57 (2005).
\vsp

[2] Carlos L. Garrido Alzar et al., Phys. Rev. A \textbf{74}, 053413 (2006);
Ph. W. Courteille et al., J. Phys. B \textbf{39}, 1055 (2006);
B. M. Garraway and H. Perrin, Phys. Scr. T \textbf{140}, 014006 (2010);
R. Kollengode Easwaran et al., J. Phys. B \textbf{43}, 065302 (2010).
\vsp

[3] K. Burrows and B. M. Garraway, in preparation.
}

\vspace{\baselineskip} 