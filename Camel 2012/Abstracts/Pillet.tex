\title{RO-VIBRATIONAL LASER COOLING OF MOLECULES VIA OPTICAL PUMPING}

\underline{P. Pillet} \index{Pillet P}

{\normalsize{\vspace{-4mm}
Laboratoire Aim\'e Cotton, Universit\'e Paris-Sud, B\^atiment 505, 91405 Orsay, France

\email pierre.pillet@lac.u-psud.fr}}

An optical pumping method is used for the transfer of molecules from several rotational and vibrational molecular levels into a given
ro-vibrational state, and the cooling of Cs$_2$ molecules in the ground state ($v = 0$; $J = 0$) is demonstrated. This manipulation is
realized by the use of two lasers, exciting all the populated rovibrational state but the chosen final one. This target state thus
behaves like a dark state where molecules pile up thanks to the repetition of absorption-spontaneous emission cycles. A broadband
laser is used to reduce the extended vibrational spectra to the sole $v = 0$ branch, whereas a narrowband laser is scanned in order to
manipulate the more compressed rotational spectra. The efficiency of the cooling process is mainly limited by the power of the
vibrational cooling laser. The simplicity of the method suggests that it can be extended to other molecules and to molecular beams.

\vspace{\baselineskip} 