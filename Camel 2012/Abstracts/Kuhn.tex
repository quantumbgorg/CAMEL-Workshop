\title{ENCODING QUBITS, QUTRITS AND QUQUADS IN SINGLE PHOTONS FROM ATOM-CAVITY SYSTEMS}

\underline{A. Kuhn}\index{Kuhn A}

{\normalsize{\vspace{-4mm}
University of Oxford, Clarendon Laboratory, Parks Road, OX1 3PU Oxford, United Kingdom

\email axel.kuhn@physics.ox.ac.uk}}

The ability of encoding arbitrary qubits in elementary quantum systems heralds the beginning of a novel approach to computing
and information security. Particular attention has been paid toward the field of linear optics quantum computing (LOQC) as
quantum circuits made from linear optical elements are in principle scalable, although they rely on parametric down-conversion
(PDC) sources which are of an intrinsic spontaneous nature. In this paper, I will show that single photons deterministically emitted
from a single atom into an optical cavity can be used for LOQC to get past these limits. With a coherence time greater than 500
ns, even a subdivision of photons into D timebins of arbitrary amplitudes and phases is possible, which we use for encoding
arbitrary qu-D-its in one single photon. We verify the fidelity of the quantum state preparation using time-resolved quantum-
homodyne measurements. These are performed by sending single ``signal'' and ``reference'' photons into an elementary photonic
circuit. Photon correlations monitored in a time-resolved manner then allow for the quantum state to be partially reconstructed.

\vspace{\baselineskip} 