\title{TOWARDS THE ULTIMATE QUANTUM REGIME OF THE LIGHT-MATTER INTERACTION}

\underline{I. Mekhov}\index{Mekhov I}

{\normalsize{\vspace{-4mm}
University of Oxford, Clarendon Laboratory, Parks Road, OX1 3PU Oxford, United Kingdom

\email Igor.Mekhov@physics.ox.ac.uk}}

Although quantum gases trapped by light have been actively studied, the quantum properties of light are usually neglected in this field. I will show, how the study of phenomena, where the quantizations of both light and atomic motion are crucial, will lead to novel effects, beyond traditional physics of many-body systems trapped in prescribed potentials, e.g., optical lattices. First, light serves as a quantum nondemolition (QND) probe of atomic [1] or molecular [2] states: the particle high-order correlation functions (beyond the density-density ones) and full statistics can be measured. Second, due to the light-matter entanglement, the measurement-based preparation of many-body states is possible (number squeezed, Schr\"odinger cat states, etc.). Light scattering constitutes the quantum measurement with controllable form of measurement back-action, allowing the dissipation tailoring in a strongly correlated system. Third, in cavity QED with quantum gases, the selfconsistent solution for light and atoms is required, enriching phases of atoms trapped in fully quantum potentials. In general, interfacing quantum gases with quantum light enables the quantum control of many-body systems, unachievable using classical optical lattices. For a recent review of this field, cf. Ref. [1].

{\normalsize
[1] I. B. Mekhov and H. Ritsch, Journ. Phys. B. \textbf{45}, 102001 (2012) (REVIEW).
\vsp

[2] B. Wunsch, N. T. Zinner, I. B. Mekhov, S.-J. Huang, D.-W. Wang, and E. Demler, Phys. Rev. Lett. \textbf{107}, 073201 (2011).
}

\vspace{\baselineskip} 