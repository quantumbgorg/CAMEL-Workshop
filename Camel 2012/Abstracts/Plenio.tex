\title{SENSING SINGLE SPINS WITH DIAMONDS}

\underline{M. Plenio}\index{Plenio M}

{\normalsize{\vspace{-4mm}
University of Ulm,
Institute of Theoretical Physics,
Albert-Einstein-Allee 11
D-89069 Ulm, Germany

\email martin.plenio@uni-ulm.de}}

The detection of a nuclear spin in an individual molecule represents a key challenge in physics and biology whose solution has been pursued for many years. The small magnetic moment of a single nucleus and the unavoidable environmental noise present the key obstacles for its realization. Here, we provide a theoretical demonstration that a single nitrogen-vacancy (NV) center in diamond can be used to construct a nano-scale single molecule spectrometer that is capable of detecting the position and spin state of a single nucleus and can determine the distance and alignment of a nuclear or electron spin pair. This device would find applications in single molecule spectroscopy in chemistry and biology, such as in determining protein structure or monitoring macromolecular motions and could thus provide a tool to help unravelling the microscopic mechanisms underlying bio-molecular function. In this context the loss of coherence represents a key obstacle that needs to be ad
dressed. The efficiency of dynamical decoupling schemes, which have been introduced to address this problem, is limited itself by the fluctuations in the driving fields which will themselves introduce noise. We address this challenge by introducing the concept of concatenated continuous dynamical decoupling, which can overcome not only external noise but also fluctuations in driving fields that implement the decoupling sequences and thus holds the potential for achieving relaxation limited coherence times. We implement the scheme experimentally with nitrogen-vacancy (NV) centers in diamond, and demonstrate an improvement of the decoherence time by an order of magnitude. The proposed scheme can be applied to a wide variety of other physical systems including, trapped atoms and ions, quantum dots, and may be combined with other quantum technologies challenges such as quantum sensing and quantum information processing.

\vspace{\baselineskip} 