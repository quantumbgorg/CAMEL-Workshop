\title{COHERENT EXCITATION BY LORENTZIAN PULSED FIELD}

\underline{G. S. Vasilev} and N. V. Vitanov
\index{Vasilev G} \index{Vitanov N}

{\normalsize{

\vspace{-4mm} \unisofia

%\vspace{-4mm} $^2$\ssinstitute

\email vasilev@phys.uni-sofia.bg}}

Two-state quantum
models can be found in a variety of problems across quantum
physics, ranging from nuclear magnetic resonance, coherent atomic
excitation and quantum information to chemical physics,
solid-state physics and neutrino oscillations. Moreover, many
problems involving multiple states and complicated linkage
patterns can very often be understood only by reduction to one or
more two-state problems. It is known, that the two-state
problem with arbitrary time-dependent fields is related to the
Riccati equation and hence a general solution cannot be given.
There are several exactly soluble two-state models, including the
Rabi, Landau-Zener, Rosen-Zener, Allen-Eberly-Hioe, Bambini-Berman,
Demkov-Kunike, Demkov  and Nikitin
models. Because of the importance of the two-state models, the
search for analytical solutions continues. Due to its complexity
most of these models use various special functions to solve the
particular two-state problem. If this is not possible, there exist
also methods for approximate solutions, such as adiabatic
approximations, Magnus approximation, Dykhne-Davis-Pechukas
approximation.
In the present work, we derive analytically the transition
probability for a two-state system driven by a pulsed external
field with Lorentzian temporal envelope and constant carrier
frequency. The solution is expressed in terms of confluent Heun
functions. This field, for which no exact analytic solution was
known yet, is among the most important pulsed fields. Using
Delos-Thorson approach presented solution can be extended to
solution of class of models. Despite the active researh the Heun
equations and Heun function have not been well studied. Because of
this the Lorentzian model will also be investigated with the
Dykhne-Davis-Pechukas (DDP) method, which involves integration
in the complex time plane, to derive a very accurate approximation
to the transition probability and the width of the excitation line
profile.


\vspace{\baselineskip}
