\title{BROADBAND FARADAY ISOLATOR}

\underline{M. Berent}$^{1,2}$, A. A. Rangelov$^{1}$ and N. V. Vitanov$^{1}$ \index{Berent M}\index{Rangelov A} \index{Vitanov N}

{\normalsize{
\vspace{-4mm} $^{1}$\unisofia

\vspace{-4mm} $^{2}$Faculty of Physics, Adam Mickiewicz University, Umultowska 85, 61-614 Pozna\'n, Poland

\email mberent@amu.edu.pl}}

Drawing on an analogy with the powerful technique of composite pulses in quantum optics [1] and polarization optics [2,3] we present a broadband optical diode (optical isolator) made of sequences of ordinary 45$^\circ$ Faraday rotators sandwiched with quarter-wave plates rotated at the specific angles with respect to their fast polarization axes.

{\normalsize
[1] B. T. Torosov and N. V. Vitanov, Phys. Rev. A \textbf{83}, 053420 (2011).
\vsp

[2] A. Ardavan, New J. Phys. \textbf{9}, 24 (2007).
\vsp

[3] S. S. Ivanov, A. A. Rangelov , N. V. Vitanov, T. Peters and T. Halfmann, J. Opt. Soc. Am. A \textbf{29}, 265 (2012).
}

\vspace{\baselineskip} 