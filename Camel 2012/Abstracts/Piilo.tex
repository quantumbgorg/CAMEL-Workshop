\title{QUANTIFYING AND CONTROLLING NON-MARKOVIAN QUANTUM DYNAMICS}

\underline{J. Piilo}\index{Piilo J}

{\normalsize{\vspace{-4mm}
Department of Physics and Astronomy,
University of Turku,
FI-20014 Turun Yliopisto,
Finland

\email jyrki.piilo@utu.fi}}

Realistic quantum mechanical systems are always exposed to an external environment. The presence of the environment often gives
rise to a Markovian process in which the system loses information to its surroundings. However, many quantum systems exhibit a
pronounced non-Markovian behavior signifying the presence of quantum memory effects. In this talk, we show how to quantify
non-Markovianity of a process based on the concept of information flow  and how to control the transition from Markovian to
non-Markovian quantum dynamics in an optical system. Moreover, we show that initial correlations in a composite environment can lead
to a nonlocal open system dynamics which exhibits strong memory effects although the local dynamics is Markovian. The latter
results demonstrate that, contrary to conventional wisdom, enlarging an open system can change the dynamics from Markovian to
non-Markovian, and that in an optical setup, one can use the polarization degrees of freedom of photons to probe their frequency
correlations.

\vspace{\baselineskip} 