\title{NON-MARKOVIANITY, LOSCHMIDT ECHO AND CRITICALITY: A UNIFIED PICTURE}

\underline{S. Maniscalco} \index{Maniscalco S}

{\normalsize{\vspace{-4mm}
School of Engineering \& Physical Sciences, Heriot-Watt University, EH14 4AS Edinburgh, United Kingdom

\email smanis@utu.fi}}

A simple relationship between recently proposed measures of non-Markovianity and the Loschmidt echo is established, holding for
a two-level system (qubit) undergoing pure dephasing due to a coupling with a many-body environment. We show that the
Loschmidt echo is intimately related to the information flowing out from and occasionally back into the system. This, in turn,
determines the non-Markovianity of the reduced dynamics. In particular, we consider a central qubit coupled to a quantum Ising ring
in the transverse field. In this context, the information flux between system and environment is strongly affected by the
environmental criticality; the qubit dynamics is shown to be Markovian exactly and only at the critical point. Therefore non-
Markovianity is an indicator of criticality in the model considered here.

\vspace{\baselineskip} 