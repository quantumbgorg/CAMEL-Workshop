\title{CIRCULAR DICHROISM IN THE PAD OF CAMPHOR AND FENCHONE MOLECULES}

\underline{M. Wollenhaupt} \index{Wollenhaupt M}

{\normalsize{\vspace{-4mm}
University of Kassel, Institute of Physics und CINSaT, Heinrich-Plett-Str. 40, 34132 Kassel, Germany

\email wollenha@physik.uni-kassel.de}}

A femtosecond laser based approach to chiral recognition of randomly oriented molecules in the gas phase enabling implementations in analytics is presented. Photoelectron Angular Distributions (PAD) resulting from Resonance Enhanced Multi-Photon Ionization (REMPI) of camphor and fenchone molecules are studied in order to investigate the Circular Dichroism in the photo-ionization of chiral molecules.  The observed circular dichroism effect using left- and right-handed circularly polarized femtosecond laser pulses is in the $\pm$ ten percent regime [1].

{\normalsize
[1] C. Lux, M. Wollenhaupt, T. Bolze, Q. Liang, J. K\"ohler , C Sarpe and T.Baumert,
Angewandte Chemie int. Ed., Vol. \textbf{51}, 5001-5005 (2012).
}

\vspace{\baselineskip} 