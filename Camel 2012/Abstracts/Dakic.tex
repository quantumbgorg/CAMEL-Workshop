\title{QUANTUM DISCORD AS RESOURCE FOR QUANTUM COMMUNICATION}

\underline{B. Dakic}\index{Dakic B}

{\normalsize{\vspace{-4mm}
Faculty of Physics,
University of Vienna,
Boltzmanngasse 5,
A-1090 Vienna,
Austria

\email borivoje.dakic@univie.ac.at}}

Quantum entanglement is widely recognized as one of the key resources for the advantages of quantum information processing, including universal quantum computation, reduction of communication complexity or secret key distribution. However, computational models have been discovered, which consume very little or no entanglement and still can efficiently solve certain problems thought to be classically intractable. The existence of these models suggests that separable or weakly entangled states could be extremely useful tools for quantum information processing as they are much easier to prepare and control even in dissipative environments. It has been proposed that a requirement for useful quantum states is the generation of so-called quantum discord, a measure of non-classical correlations that includes entanglement as a subset. Although a link between quantum discord and few quantum information tasks has been studied, its role in computation speed-up and quantum comm
unication is still open. Here we show that quantum discord is the optimal resource for the remote quantum state preparation, a variant of the quantum teleportation protocol. Using photonic quantum systems, we explicitly show that the geometric measure of quantum discord is related to the fidelity of this task, which provides an operational meaning. Moreover, we demonstrate that separable states with non-zero quantum discord can outperform entangled states. Therefore, the role of quantum discord might provide fundamental insights for resource-efficient quantum information processing. The experimental implementation was performed on a quantum optical platform using polarization-correlated single photons.

\vspace{\baselineskip} 