\title{RF AND MICROWAVE BASED QUANTUM INFORMATION SCIENCE WITH TRAPPED IONS}

\underline{Ch. Wunderlich} \index{Wunderlich Ch}

{\normalsize{\vspace{-4mm}
Department of Physics,
University Siegen,
57068 Siegen,
Germany

\email wunderlich@physik.uni-siegen.de}}

With trapped ions unsurpassed control of the quantum degrees of freedom of individual particles has been achieved [1].  When contemplating the scalability of trapped ions for quantum information science one notes that the use of laser light for {\em coherent manipulation} gives rise to fundamental and technical issues [2-5].  We discuss the manipulation of quantum information encoded into spin states of Doppler-cooled ions using long wavelength radiation [3,4]. Such radiation with sub-Hertz frequency stability can be easily obtained from commercial sources. Addressing of single ions can be achieved by applying a magnetic gradient along the spin chain which lifts the degeneracy of the transition frequencies and thus allows for addressing in frequency space [6,7]. Furthermore,  Magnetic Gradient Induced Coupling (MAGIC) between ion spins and their motion using radio-frequency radiation has been observed [6].

Here, we report on the first demonstration of essential
features of a trapped ion spin ``molecule'' that exhibits long range spin-spin interactions due to MAGIC, analogous to J-coupling between nuclear spins in molecules [8]. Resonances of individual spins are well separated and are addressed with high fidelity. Quantum CNOT gates are carried out using microwave radiation. Importantly, these measurements are carried out with thermally excited $^{171}$Yb$^+$  ions with a mean quantum number of 23$\pm$7 characterizing the excitation of the COM axial mode. Demonstrating Conditional-NOT gates between non-nearest neighbors serves as a proof-of-principle of a quantum bus employing a spin chain. In addition, we characterize experimentally the spin-spin-coupling in strings of two and three ions and prove the dependence of this coupling on the trap frequency which can be used to create tailored coupling patterns relevant for quantum simulations.


Decoherence due to fluctuating magnetic fields can be strongly suppressed using microwave-dressed states [9] and coherence times up to about 1~s are achieved.  At the same time, using dressed  states  eliminates carrier transitions by interference and retains the magnetic gradient-induced coupling. Thus, fast quantum gates even with a small effective Lamb-Dicke parameter are possible. This approach is generic and applicable also to laser-based gates as well as other types of physical qubits.

{\normalsize
[1] R. Blatt, and D. Wineland, Nature {\bf 453}, 1008-1015 (2008).
\vsp

[2] R. Ozeri, et al., Phys. Rev. A {\bf 75}, 042329 (2007).
\vsp

[3] F. Mintert, Ch. Wunderlich, Phys. Rev. Lett. {\bf 87}, 257904 (2001); ibid. {\bf 91}, 029902 (2003).
\vsp

[4] Ch. Wunderlich in {\em Laser Physics at the Limit}, pp. 261- 271 (Springer, 2002);
Ch. Wunderlich,  Ch. Balzer, Adv. At. Mol. Opt. Phys. {\bf 49}, 293-376 (2003).
\vsp

[5] C. Ospelkaus et al, Phys. Rev. Lett. {\bf 101}, 090502 (2008).
\vsp

[6] M. Johanning et al, Phys. Rev. Lett. {\bf 102}, 073004 (2009).
\vsp

[7] S. X. Wang et al., Appl. Phys. Lett. {\bf 94}, 094103 (2009).
\vsp

[8]  A. Khromova et al., Phys. Rev. Lett. {\bf 108}, 220502 (2012).
\vsp

[9] N. Timoney et al., Nature {\bf 476}, 185 (2011).
}

\vspace{\baselineskip} 