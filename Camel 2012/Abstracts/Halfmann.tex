\title{ENHANCED OPTICAL DATA STORAGE UP TO SEVERAL SECONDS BY EIT IN A DOPED SOLID}

\underline{T. Halfmann} \index{Halfmann T}

{\normalsize{\vspace{-4mm}
Institute of Applied Physics, Technical University of Darmstadt,
64289 Darmstadt, Germany

\email thomas.halfmann@physik.tu-darmstadt.de}}

Coherent light-matter interactions provide powerful tools to control optical properties in quantum systems, e.g. aiming at efficient optical storage (as required in quantum processing). The talk presents implementations of coherent optical interactions in particular solids, i.e. rare-earth doped crystals. The latter media combine the advantages of gases (i.e. spectrally narrow transitions) and solids (i.e. large storage density and scalability). In particular, the talk reports on applications of electromagnetically induced transparency (EIT) to store light pulses and images in atomic coherences in Pr:YSO. We demonstrate efficient operation of the solid memory by combination of EIT and image storage with feedback-controlled pulse shaping/evolutionary algorithms for enhanced optical preparation via spectral hole burning, angular and frequency multiplexing for enlarged storage capacity, as well as dynamic decoupling and zero first order Zeeman shifts to cope with decoherence. Combination of the powerful approaches permits prolongation of storage times in the EIT-driven solid quantum memory up to several seconds.

\vspace{\baselineskip} 