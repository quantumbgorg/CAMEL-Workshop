\title{SCALABLE ION QUANTUM TECHNOLOGY}

\underline{W. Hensinger} \index{Hensinger W}

{\normalsize{\vspace{-4mm}
Ion Quantum Technology Group, Department of Physics and Astronomy,
University of Sussex, Falmer, Brighton, East Sussex, BN1 9QH, United Kingdom

\email w.k.hensinger@sussex.ac.uk}}

Quantum theory can have powerful applications due to the possibility of implementing new quantum
technologies such as the quantum computer. Recent developments in ion trapping experiments show that
it should be possible to implement advanced quantum technology with trapped ions. Quantum
information science and coherent manipulation with trapped ions has already been successfully
applied in experiments with a small number of quantum bits, for example to realize quantum
algorithms such as search, the generation of particular entangled states of up to 14 ions,
teleportation, ion-photon entanglement, error correction and others. In order to build useful
devices, the next step must include the systematic development of suitable architectures for large
scale ion quantum technology applications.

At the University of Sussex we are developing a new generation of on chip ion trap arrays that may
accommodate large numbers of single atomic ions. The scalable fabrication of ion trap arrays
involves advanced nanofabrication techniques including photolithography. I will present progress on
two types of ion trap chips delivering versatile performance for a multitude of applications. I will
also present progress on a microfabricated 2-D ion trap array that allows for the creation of ion
lattices for the implementation of a quantum simulator. I will also present theoretical results
illustrating how to design two dimensional ion trap lattices for quantum simulation.

I will also discuss experiments concerning the coherent manipulation of trapped ions towards the
implementation of large scale entanglement using microwave radiation.

Finally I will present progress on experiments towards demonstrating the inhomogeneous Kibble-Zurek
Mechanism with trapped ions.

More information: http://www.sussex.ac.uk/physics/iqt

\vspace{\baselineskip} 