\title{QUANTUM CONTROL OF SPINS USING NITROGEN VACANCY CENTRES IN DIAMOND}
% Quantum control of spins using Nitrogen Vacancy centres in diamond

\underline{B. Naydenov} \index{Naydenov B.} 
%Boris Naydenov

{\normalsize{\vspace{-4mm}
Helmholtz-Zentrum Berlin für Materialien und Energie (HZB),
Institute for Nanospectroscopy,
Kekuléstraße. 5,
D-12489 Berlin,
Germany



\email boris.naydenov@helmholtz-berlin.de}}

Colour centres in diamond and especially the nitrogen-vacancy centres (NVs), show remarkable physical properties making them good candidates for quantum bits, single photon sources and precise magnetic field sensors with a nanometre spatial resolution. These defects can be measured at the single cite level even at room temperature, allowing to perform a variety of fundamental experiments.
In this talk the physics of NV centres with a focus on their spin properties will be reviewed and various applications will be discussed. It will be shown how a single NV centre can be used for imaging magnetic fields with a nanoscale resolution, thus revealing interesting magnetic phenomena. Then the realisation of nanoscale Nuclear Magnetic Resonance (NMR) will be demonstrated, where a single nuclear spin sensitivity is reached. Furthermore, NV centres can be used not only for the detection of nuclear spins, but also for their control, allowing to implement Dynamic Nuclear Polarisation (DNP) for improving NMR signals by few orders of magnitude. Finally, a solid state quantum simulator based on a 2D layer of nuclear spins in the vicinity of a single NV will be presented. 

\vspace{\baselineskip}