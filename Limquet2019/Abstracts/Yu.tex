\title{PHOTONIC CIRCUITS FOR QUANTUM OPTICS}
% Photonic circuits for quantum optics

\underline{X. Yu} \index{Yu X.} 
%Xiao Yu

{\normalsize{\vspace{-4mm}
Laboratoire Interdisciplinaire Carnot de Bourgogne (ICB) ,
UMR 6303 CNRS - Université Bourgogne-Franche Comté,
Equipe Submicron optics,
Faculté des Sciences Mirande,
B.P. 47 870,
21078 DIJON CEDEX - France



\email xiao.yu@u-bourgogne.fr}}

Enhancing the light-matter interaction at single-atom/photon level is required for quantum technologies. The coupling strength depends on the quality factor Q and effective volume V of the involved mode by the ratio Q/V. High and efficient light matter interaction can be achieved in cavity-based systems. In this project, we aim at transposing quantum optics techniques to nanophotonics using hybrid plasmonic-photonic nanostructures taking benefit of sub-wavelength dimensions. Surface plasmon polaritons (modes sustained by metallic nanostructures) present low mode volumes and can interface light and matter at the nanoscale. We propose to use innovative photonic structures with quantum emitters, coupled to 1D and 2D photonic and/or plasmonics waveguides. 

In the talk, I will report the fabrication progress and share some results about the quantum dots deposition and Yagi-Uda antenna-planar waveguide coupling. The plan and primary design for the first secondement will also be mentioned.

\vspace{\baselineskip}