\title{INTRODUCTION TO DYNAMICAL DECOUPLING FOR QUANTUM TECHNOLOGIES}
% Introduction to Dynamical Decoupling for Quantum Technologies

\underline{G. Genov} \index{Genov G.} 
%Genko Genov

{\normalsize{\vspace{-4mm}
Institute for Quantum Optics, Ulm University, Albert-Einstein-Allee 11, 89081 Ulm, Germany



\email genko.genov@uni-ulm.de}}

Developments in quantum technologies are increasingly important nowadays for various applications in computation, sensing, and communication of data. However, protection of quantum systems from the environment remains a challenge. In this talk, I will present an overview of the technique of dynamical decoupling (DD), which uses pulses or continuous fields to alter the time evolution of a system and suppress the effect of unwanted noise while retaining sensitivity to particular interactions and signals [1].

First, I will discuss pulsed DD, which has been applied extensively for compensation of environmental noise. Important requirements are high pulse powers to limit dephasing during a pulse and high repetition rate, so refocusing is much faster than the correlation time of the environment. A second approach is continuous DD, where the system is driven with a protecting dressing field for the entire duration of the experiment, and has also been demonstrated to compensate for noise sources in various media, e.g., in color centers in diamond and trapped ions. Finally, I will also discuss a mixed scheme where continuous and pulsed DD are combined to use of the advantages of both pulsed and continuous DD [2]. 

{\normalsize
[1] D. Suter and G. A. Alvarez, Rev. Mod. Phys. 88, 041001 (2016); C. L. Degen, F. Reinhard, and P. Cappellaro, Rev. Mod. Phys. 89, 035002 (2017).
\vsp

[2] G. T. Genov, N. Aharon, F. Jelezko, and A. Retzker, Quantum Sci. Technol. 4 035010 (2019).
}

\vspace{\baselineskip}