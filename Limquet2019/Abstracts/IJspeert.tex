\title{SINGLE-ATOM TRAPPING WITH OPTICAL TWEEZERS FOR DETERMINISTIC CAVITY LOADING}
% Single-atom trapping with optical tweezers for deterministic cavity loading

\underline{M. IJspeert} \index{IJspeert M.} 
%Mark IJspeert

{\normalsize{\vspace{-4mm}
Clarendon Laboratory,
Parks Road,
OX1 3PU,
Oxford



\email mark.ijspeert@physics.ox.ac.uk}}

Efficient single-atom single-photon interfaces are essential to many applications in quantum information processing. The potential of such hybrid systems for the delivery of single photons and distributed entanglement has been demonstrated using 87Rb atoms stochastically loaded into a high-finesse optical cavity to achieve strong coupling [1]. Whilst this source constitutes an excellent testbed, probabilistic loading limits the scalability of its design. In addition, the ballistic transit of atoms across the cavity mode gives rise to time-dependent coupling strengths. In this work we present techniques for the cooling, trapping and imaging of single atoms [2]. We demonstrate dipole trapping of individual, laser-cooled 87Rb atoms captured from a magneto-optic trap on time scales exceeding 40 s. We analyse the effect of trap depth and optical molasses density on the lifetime and average occupation of the dipole trap. Such a deterministic means of holding a single atom in place is an important step towards the deterministic cavity loading scheme required for a scalable atom-photon quantum interface.

{\normalsize
[1] T. D. Barrett et. al., Quantum Science and Technology 4(2), 025008, (2019).
\vsp

[2] N. Holland et. al., Journal of Modern Optics 65(18), 2133{2141, (2018).
}

\vspace{\baselineskip}
