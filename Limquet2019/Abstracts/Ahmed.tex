\title{TOWARDS GLASS INTEGRATED PLATFORM DEDICATED TO QUANTUM OPTICS}
% TOWARDS GLASS INTEGRATED PLATFORM DEDICATED TO QUANTUM OPTICS

\underline{M. Ahmed} \index{Ahmed M.} 
%Muhammad Ahmed

{\normalsize{\vspace{-4mm}
4 Quai de France



\email mhamdy@zewailcity.edu.eg}}

In the last few years, the excitation of light sources has been successfully implemented using several integrated optical devices within a few nanometers. For example, the excitation of nano-emitters has been realised by surface plasmon polaritons (SPPs) in nano-antennas optical cavities with nanocrystals, nanowires and metallic waveguides. The used materials (silicon and noble metals) for the excitation process are characterized by high losses in the visible spectral range. Our target is how to enhance the light confinement in design integrated device using low-loss materials. Consequently, the coupling between nanostructures and waveguide will increase. 
Our choice has converged towards ion exchanged optical waveguides (IEW). However, the excitation of the nanostructures with this kind of waveguides is not well-spread because it has low light confinement due to the low index contrast between the core and cladding. In our work one proposal is reported to enhance the field confinement in IEW. Firstly, dielectric thin film of high index material (TiO2, SiC, ZnO, etc.) deposited above IEW is used to increase the light confinement inside the waveguide [11]. The hybrid modes with higher effective indices occur because of the propagation of the guided mode through the structure. Furthermore, two waveguides coupled will appear, due to excited modes of the TiO2 layer resulting from the evanescent wave of the guiding mode in IEW that will enhance the light confinement inside the IEW. The nanosource will be placed on the top of the structure and will be excited by an incident light. 
Simulation results were obtained using the 3-D finite differential time domain (3-D FDTD) method. Our proposed design will enhance light guidance through the IEW due to the presence of an SiC slab. In this work, the role of the dielectric slab is studied at different thicknesses, lengths, materials and widths to get the optimum dimensions of the slab and study their impacts on enhancing the light confinement and the measured transmission. We demonstrated that the proposed structures allow increasing the measured transmission to 68\%. We will also present our recent work on coupling our waveguides with electrically excited plasmon sources.


\vspace{\baselineskip}