\title{DOUBLE PHOTONIC STRUCTURE WITH QUANTUM DOT IN NANOWIRE}
% Double Photonic Structure with Quantum Dot in Nanowire

\underline{A. Balgarkashi} \index{Balgarkashi A.} 
%Akshay Balgarkashi

{\normalsize{\vspace{-4mm}
EPFL STI IMX LMSC,
MXC 317 (Bâtiment MXC),
Station 12,
CH-1015 Lausanne,
Switzerland



\email akshay.balgarkashi@epfl.ch}}

The goal of this project is to achieve quantum dots in nanowires, possibly in the telecom domain. In this presentation, we will report on progress in the growth of InAs segments on GaAs nanowire arrays on Si substrate by the self-catalyzed method. In order to define pristine InAs segments, we consume the Ga droplet on top of GaAs NWs and replace it with an indium droplet or directly deposit InAs segments. EDX results indicate that InAs accumulates preferentially on the facets close to nanowire tip. By following different growth strategies, we aim to maximize accumulation of InAs at the nanowire top facet. Such InAs segments grown on GaAs nanowires are confined using a GaAs shell capping layer. The optical properties of these segments are investigated using cathodoluminescence spectroscopy.

\vspace{\baselineskip}
