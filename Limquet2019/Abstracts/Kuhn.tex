\title{PUSHING PURCELL-ENHANCEMENT BEYOND ITS LIMITS}
% Pushing Purcell-enhancement beyond its limits

\underline{A. Kuhn} \index{Kuhn A.} 
%Axel Kuhn

{\normalsize{\vspace{-4mm}
University of Oxford,
Clarendon Laboratory,
Parks Road,
Oxford,
OX1 3PU, UK



\email axel.kuhn@physics.ox.ac.uk}}

Purcell-enhanced emission from a coupled emitter-cavity system is a fundamental manifestation of cavity quantum electrodynamics. We derive a scheme for photon emission from an emitter coupled to a birefringent cavity that exceeds hitherto anticipated limitations. Based on a recent experimental investigation of the intra-cavity coupling, we decouple the emitter and the photon prior to emission. This is ``hiding'' the emitter from the photon to suppress re-excitation, increasing the overall emission through the mirrors. It is found that birefringence can mitigate the trade-off between stronger emitter-cavity coupling and efficient photon extraction. Furthermore, we generalise our model to consider a variety of equivalent schemes. For instance, detuning a pair of ground states in a three-level emitter coupled to a cavity in a Lambda-system is shown to provide the same enhancement. Additionally, it is found that when connecting multiple ground states of the emitter to form a chain of coupled states, the extraction efficiency approaches its fundamental upper limit.

\vspace{\baselineskip}