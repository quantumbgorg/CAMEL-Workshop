\title{MICROSCOPIC AND PHENOMENOLOGICAL MODELS OF DRIVEN SYSTEMS IN STRUCTURED RESERVOIRS}
% Microscopic and phenomenological models of driven systems in structured reservoirs

\underline{A. Saharyan} \index{Saharyan A.} 
%Astghik Saharyan

{\normalsize{\vspace{-4mm}
6 Rue Maréchal Leclerc,
21000 Dijon,
France 



\email astghik.saharyan@gmail.com}}

We study the paradigmatic model of a qubit interacting with a structured environment and driven by an external field by means of a microscopic and a phenomenological model. The validity of the so-called fixed-dissipator (FD) assumption, where the dissipation is taken as the one of the undriven qubit is discussed. In the limit of a flat spectrum, the FD model and the microscopic one remarkably practically coincide.  For a structured reservoir, we show in the secular limit that steady states can be different from those determined from the FD model, opening the possibility for exploiting reservoir engineering. We explore it as a function of the control field parameters, of the characteristics of the spectral density and of the environment temperature. The observed widening of
the family of target states by reservoir engineering suggests new possibilities in quantum control protocols.

\vspace{\baselineskip}
