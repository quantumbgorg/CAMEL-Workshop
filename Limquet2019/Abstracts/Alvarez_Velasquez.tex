\title{CAVITY-BASED PHOTON-GENERATION SCHEMES USING STIRAP RE-PREPARATION}
% Cavity-based photon-generation schemes using STIRAP re-preparation

\underline{J. R. Alvarez Velasquez} \index{Alvarez Velasquez J. R.} 
%Juan Rafael Alvarez Velasquez

{\normalsize{\vspace{-4mm}
Clarendon Laboratory, Parks Road, Oxford, OX1 3PU, United Kingdom


\email juan.alvarezvelasquez@physics.ox.ac.uk}}

The strong coupling of an atom to a cavity offers unparalleled control over the generation of single photons with tunable properties. Such atom-cavity systems, based on vacuum stimulated Raman processes (V-STIRAP) can provide a priori deterministic photon sources with arbitrary control of the photon amplitude and phase envelope in space, being able to produce photons which are orders of magnitude longer than single photon detector resolutions.
Attempts at producing polarized single photons with these schemes have driven magnetic sub-levels of atoms in an alternating order. However, due to nonlinear Zeeman effects, these schemes show low efficiencies when attempting to produce more than two photons sequentially. In this report, a method to re-prepare the initial magnetic state of a 87Rb atom using two overlapping STIRAP beams is proposed.

\vspace{\baselineskip}
