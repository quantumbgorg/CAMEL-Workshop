\title{CLASSICAL PHYSICS ANALOGUES OF QUANTUM SYSTEMS}
% Classical physics analogues of quantum systems

\underline{A. Rangelov} \index{Rangelov A.} 
%Andon Rangelov

{\normalsize{\vspace{-4mm}
Theoretical Physics Division
Department of Physics, Sofia University
James Bourchier 5 Blvd, 1164 Sofia, Bulgaria



\email rangelov@phys.uni-sofia.bg}}

Analogy is a basic concept for understanding nature, since it analyses and connects different phenomena linked by common properties or similar behavior. Specifically, the analogies between classical physical theories and quantum phenomena reveal the fact that similar mathematical formalisms apply to phenomena that cannot be related in all aspects and are a priori conceptually different. The role of mathematics is crucial, because the essence of the analogy resides in the fact that completely different systems can be modeled by similar mathematical equations.

In particular, analogies between wave optics and quantum mechanics have been highlighted since the early days of quantum mechanics: wave effects like interference and diffraction were borrowed from optics and applied to demonstrate the wavy nature of quantum particles, as electrons, neutrons and atoms. After the full development of quantum theory and the rise of coherent light sources, the exchange of concepts in the opposite direction started to occur.

In this talk I am going to show several formal analogy between the Schrödinger equation of quantum system of two and three states and classical systems.  


\vspace{\baselineskip}
