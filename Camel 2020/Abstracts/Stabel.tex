\title{EIT LIGHT STORAGE TOWARDS THE SINGLE-PHOTON LEVEL IN PR:YSO}
% EIT Light Storage towards the Single-Photon Level in Pr:YSO

\underline{M. Stabel} \index{Stabel M.}
%Markus Stabel

{\normalsize{\vspace{-4mm}
Nonlinear Optics / Quantum Optics,
Institute of Applied Physics,
Technical University of Darmstadt,
Hochschulstraße 6,
D-64289 Darmstadt



\email markus.stabel@physik.tu-darmstadt.de}}

As a key component of quantum repeaters, quantum memories for single photons are a central component in future, large scale quantum communication networks. Electromagnetically induced transparency (EIT) is a prominent protocol to implement a quantum memory by stopping and storing light. In our work we apply EIT in a praseodymium doped yttrium orthosilicate crystal (Pr:YSO) to provide a solid-state memory. The protocol and specific medium allow us to reach both high storage efficiencies and long storage times [1,2]. While we already demonstrated EIT for efficient and long-term storage of classical pulses in Pr:YSO, we aim now at storage of weak coherent pulses in the regime of a single photon. This requires sophisticated techniques for filtering and background suppression. As a basic component towards this goal, in our setup we apply a second, optically prepared Pr:YSO crystal for narrowband spectral filtering. Combined with other measures, this enabled our first successful demonstration of a solid-state EIT-driven quantum memory at the single photon level.

{\normalsize
[1] G. Heinze, C. Hubrich, and T. Halfmann, Phys. Rev. Lett. \textbf{111}, 033601 (2013).
\vsp

[2] D. Schraft, M. Hain, N. Lorenz, and T. Halfmann, Phys. Rev. Lett. \textbf{116}, 073602 (2016).
}

\vspace{\baselineskip}