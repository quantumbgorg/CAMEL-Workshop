\title{DETERMINISTIC FILLING OF A STATIC, SINGLE-ATOM OPTICAL TWEEZER FOR EFFICIENT CAVITY LOADING}
% Deterministic filling of a static, single-atom optical tweezer for efficient cavity loading

\underline{M. IJspeert} \index{IJspeert M.}
%Mark IJspeert

{\normalsize{\vspace{-4mm}
Clarendon Laboratory
University of Oxford
Parks Road, Oxford, OX1 3PU



\email mark.ijspeert@physics.ox.ac.uk}}

Efficient single-atom single-photon interfaces are essential to many applications in quantum information processing. The potential of such hybrid systems for the delivery of single photons and distributed entanglement has been demonstrated using Rb-87 atoms stochastically loaded into a high-finesse optical cavity to achieve strong coupling [1]. Whilst this source constitutes an excellent testbed, probabilistic loading limits the scalability of its design and gives rise to time-dependent coupling strengths. The solution is to trap and hold a single atom in a tightly focused dipole trap [2], a technique that relies on the collisional blockade effect. However, the time-averaged probability of single atom occupation in the collisional blockade regime is limited to 0.5 [3]. In this work, we demonstrate that increasing the depth of a static, optical dipole trap enables the transition from fast loading (at a rate of 0.32 Hz) to long-lifetime trapping (average lifetime of 8.2 s) with a success rate of 98\%. This translates to an achievable filling ratio of 0.72. Such a deterministic means of holding a single atom in place is an important step towards the deterministic cavity loading scheme required for a scalable atom-photon quantum interface.

{\normalsize
[1] T. D. Barrett et. al., Quantum Science and Technology 4(2), 025008, (2019).
\vsp

[2] N. Holland et. al., Journal of Modern Optics 65(18), 2133–2141, (2018).
\vsp

[3] N. Schlosser et. al., Phys. Rev. Lett. 89(2), 023005, (2002).
}

\vspace{\baselineskip}