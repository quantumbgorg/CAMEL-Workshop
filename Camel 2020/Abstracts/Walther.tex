\title{RECENT PROGRESS IN OUR LAB TOWARDS LASING WITHOUT INVERSION AND QKD IN A TELEKOM ENVIRONMENT}
% Recent Progress in our Lab towards Lasing without Inversion and QKD in a Telekom Environment

\underline{T. Walther} \index{Walther T.}
%Thomas Walther

{\normalsize{\vspace{-4mm}
Inst. for Applied Physics,
Schlossgartenstr. 7,
TU Darmstadt,
D-64289 Darmstadt,
Germany



\email thomas.walther@physik.tu-darmstadt.de}}


In recent years, we have pursuing experiments to demonstrate lasing without inversion (LWI) and to investigate quantum key distribution in a real-world telecom environment. In this talk we will report on recent progress in both of these endeavours.

First, we will briefly discuss the basic idea behind our LWI approach as well as its experimental status. Our four-level LWI scheme in neutral mercury is designed to demonstrate LWI in the UV range. Two drive fields are necessary to generate the required coherences. The short term experimental goal is amplification without inversion (AWI); however, for the goal of LWI the gain must be maximized to overcome losses. After a short summary of the experimental limitations of our current setup we will discuss the lessons learned from our recent simulations. In order to identify the most promising parameters for maximizing the gain, we performed extensive simulations based on our earlier theoretical model. Next to the linewidth of the incoherent pump, the power of the strong drive as well as the linewidths of both drive fields are the most crucial parameters.

The second part of the talk is dedicated to the discussion of a quantum key distribution (QKD) experiment geared towards the implementation of a quantum hub which enables simultaneous pairwise QKD between more than two parties. We have setup a photon source based on SPDC and the phase-time bin entanglement. Currently, parts of this system are being tested in collaboration with the Deutsche Telekom in a real-world environment. In the talk, we will discuss some of the features as well as the current status of the experiment as well as the characterization of our photon source. Integral part of the talk will focus on the experimental challenges in such a scenario.

\vspace{\baselineskip}