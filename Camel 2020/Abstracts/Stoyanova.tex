\title{ADIABATIC MOTION OF A CHARGED PARTICLE IN SPATIALLY UNIFORM AND NONUNIFORM STATIC MAGNETIC FIELDS}
% Adiabatic motion of a charged particle in spatially uniform and nonuniform static magnetic fields

\underline{E. Stoyanova} \index{Stoyanova E.}
%Elena Stoyanova

{\normalsize{\vspace{-4mm}
Theoretical Physics Department, Sofia University, James Bourchier 5 Blvd, 1164 Sofia, Bulgaria



\email e.stoyanova@phys.uni-sofia.bg}}

Analogy is a fundamental concept for better understanding nature, since it associates various phenomena related by common properties or comparable behavior. In particular, the analogies between different areas of research in physics show the fact that similar scientific formalisms apply to phenomena that are completely different.

We examine theoretically the general adiabatic solution for the motion of a classical charged particle in spatially uniform and nonuniform static magnetic fields. Our approach benefits from fields like nuclear magnetic resonance (NMR), quantum optics and atomic physics, where a similar mathematical problem to what we come across here describing the dynamics of two-state quantum systems has already been solved. Such mathematical parallels have lead to novel results in various classical systems. This mathematical analogue allows us to use all known solutions for the dynamics of two-state quantum systems to study the motion of a charged particle in a nonuniform magnetic field.

We have derived the evolution matrix for the adiabatic motion of a classical charged particle in a static spatially nonuniform magnetic field. Counter-intuitively, our results suggest that if initially the magnetic field vector is aligned with the particle\'s velocity, being either parallel or anti-parallel to it, as long as the adiabatic conditions are met, the particle follows the field line either in the direction of it or in the opposite direction.

{\normalsize
[1] Elena Stoyanova, Svetoslav S. Ivanov, Andon A. Rangelov and Nikolay V. Vitanov  Phys. Scr. \textbf{94} 055501, (2019).
}

\vspace{\baselineskip}