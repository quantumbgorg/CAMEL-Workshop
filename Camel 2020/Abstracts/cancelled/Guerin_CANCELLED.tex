\title{ROBUST CONTROL OF LINEAR AND NON-LINEAR QUANTUM SYSTEMS}
% Robust control of linear and non-linear quantum systems

\underline{S. Guerin} \index{Guerin S.}
%Stephane Guerin

{\normalsize{\vspace{-4mm}
Laboratoire Interdisciplinaire Carnot de Bourgogne, CNRS UMR 6303, University of Bourgogne, Dijon, France 



\email sguerin@u-bourgogne.fr}}

The control of quantum systems by external fields (laser) possibly combined with cavity QED or (plasmonic) near field is at the heart of modern applications ranging from quantum information processing to the control of chemical reactions.  It can be in general formulated as the transfer from an initial to a target state or as a logic gate for photonic, atomic, molecular, and quantum dot states.

One challenging issue is the ability to achieve a high-fidelity transfer to a given target state in a robust way with respect to fluctuations, the partial knowledge of the system, and leakage to other states. This can be optimally achieved by the single-shot shaped pulse technique [1-2], which extends the composite pulse method. This has been also applied for the stimulated Raman exact passage (STIREP), making the well-known stimulated Raman adiabatic passage (STIRAP) technique exact and robust [3].
Its extension to non-linear quantum systems, such as the one featuring the production of molecular Bose-Einstein condensates from atomic ones, is presented [4].


{\normalsize
[1] D. Daems, A. Ruschhaupt, D. Sugny, and S. Guerin,, Phys. Rev. Lett. \textbf{111}, 050404 (2013).
\vsp

[2] L. Van-Damme, D. Schraft, G.T. Genov, D. Sugny, T. Halfmann, and S. Guerin, Phys. Rev. A \textbf{96}, 022309 (2017).
\vsp

[3] X. Laforgue, Xi Chen, and S. Guerin, Phys. Rev. A. \textbf{100}, 023415 (2019).
\vsp

[4] V. Dorier, M. Gevorgyan, A. Ishkhanyan, C. Leroy, H.R. Jauslin, and S. Guerin, Phys. Rev. Lett. \textbf{119}, 243902 (2017).
}

\vspace{\baselineskip}