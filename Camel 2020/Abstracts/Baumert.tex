\title{PHOTOELECTRON CIRCULAR DICHROISM IN THE LIGHT OF RESONANCE ENHANCED MULTI PHOTON IONIZATION}
% Photoelectron circular dichroism in the light of resonance enhanced multi photon ionization

\underline{T. Baumert} \index{Baumert T.}
%Thomas Baumert

{\normalsize{\vspace{-4mm}
Universitaet Kassel,
Institut fuer Physik,
Heinrich-Plett-Str. 40,
D-34132 Kassel, Germany           



\email baumert@physik.uni-kassel.de}}

Exploiting an electric dipole effect in ionization [1], photoelectron circular dichroism (PECD) is a highly sensitive enantioselective spectroscopy for studying chiral molecules in the gas phase using either single-photon ionization [2] or multiphoton ionization [3].  In the latter case resonance enhanced multiphoton ionization (REMPI) gives access to neutral electronic excited states.  The PECD sensitivity opens the door to study control of the coupled electron and nuclear motion in enantiomers.  A prerequisite is a detailed understanding of PECD in REMPI schemes.  In this contribution I will report on our recent experiments devoted to unravel different aspects of this effect on the fenchone prototype by addressing the range from impulsive excitation on the femtosecond time scale to highly vibrational state selective excitation with the help of high resolution nanosecond laser techniques.  The reflection of the number of absorbed photons in the PECD will be discussed as well as subcycle effects in bichromatic fields.

{\normalsize
[1] B. Ritchie, Phys. Rev. A \textbf{13}, 1411–1415 (1976).
\vsp

[2] N. Böwering, T. Lischke, B. Schmidtke, N. Müller, T. Khalil, U. Heinzmann, Phys. Rev. Lett. \textbf{86}, 1187–1190 (2001).
\vsp

[3] C. Lux, M. Wollenhaupt, T. Bolze, Q. Liang, J. Köhler, C. Sarpe, T. Baumert, Angew. Chem. Int. Ed., \textbf{51}, 5001–5005 (2012).
}

\vspace{\baselineskip}