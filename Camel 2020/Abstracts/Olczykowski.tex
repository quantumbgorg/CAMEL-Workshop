\title{SPIN STATE DYNAMICS IN A BICHROMATIC MICROWAVE FIELD}
% Spin State Dynamics in a Bichromatic Microwave Field

\underline{P. Olczykowski} \index{Olczykowski P.}
%Piotr Olczykowski

{\normalsize{\vspace{-4mm}
ul. Studencka 6/10A,
31-116 Krak\'ow,
Poland



\email piotr.olczykowski@uj.edu.pl}}

Driving an open spin system by two strong, nearly degenerate fields enables addressing populations of individual spin states and observation of complex resonance structures in experiments with nitrogen vacancy in diamond. These resonances consist of wide and broad components which show nontrivial dependence on microwave field intensity: while the width of one of them undergoes a strong power broadening, the other one exhibits a peculiar field-induced stabilization. The observations suggest existence of dark and bright states in analogy with the familiar coherent population trapping (CPT) effect.

In my talk, I will present a theoretical description of the phenomena in terms of a nonunitary evolution operator on the vector space spanned by superpositions of populations rather than wave functions as in a standard CPT. The bright and dark states are identified as eigenvectors of the nonunitary evolution operator and thus their corresponding eigenvalues are interpreted as their lifetimes. These states are respectively strongly and weakly coupled with environment and with each other. The significance of the work lies in the experimental measurement and theoretical interpretation of the dark and the bright state lifetimes as the widths of the broad and wide components of the resonance, respectively. I will present the renormalization procedure of the main equation in terms of Shirley-Floquet representation. This approach can be easily applied for numerical simulations relevant for strong microwave fields.


\vspace{\baselineskip}