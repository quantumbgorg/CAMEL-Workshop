\title{EFFICIENT AND ROBUST SIGNAL SENSING BY SEQUENCES OF ADIABATIC CHIRPED PULSES}
% Efficient and robust signal sensing by sequences of adiabatic chirped pulses

\underline{G. Genov} \index{Genov G.}
%Genko Genov

{\normalsize{\vspace{-4mm}
Institut for Quantum Optics, Ulm University, Albert-Einstein-Allee 11, Ulm 89081, Germany



\email genko.genov@uni-ulm.de}}

We propose a scheme for sensing of an oscillating field in systems with large inhomogeneous broadening and driving field inhomogeneity by applying sequences of phased, adiabatic, chirped pulses. The latter act as a double filter for dynamical decoupling, where the adiabatic changes of the mixing angle during the pulses rectify the signal and partially remove frequency noise. The sudden changes between the pulses act as instantaneous $\pi$ pulses in the adiabatic basis for additional noise suppression. We also use the phases of the pulses to correct for other errors, e.g., due to non-adiabatic couplings.

Our technique improves significantly the coherence time in comparison to standard XY8 dynamical decoupling in realistic simulations in NV centers with large inhomogeneous broadening and is suitable for experimental implementations with substantial driving field inhomogeneity, thus allowing for improved sensing in a wide range of experimental applications.

Beyond the theoretical proposal, we also present proof-of-principle experimental results for quantum sensing of an oscillating field in NV centers in diamond, demonstrating superior performance compared to standard techniques.


\vspace{\baselineskip}