\title{TRAPPED RYDBERG IONS FOR QUANTUM COMPUTING AND SIMULATION}
% Trapped Rydberg ions for quantum computing and simulation

\underline{A. Mokhberi} \index{Mokhberi A.}
%Arezoo Mokhberi

{\normalsize{\vspace{-4mm}
Staudingerweg 7, 55128 Mainz, Germany



\email arezoo.mokhberi@uni-mainz.de}}

Excitation of trapped ions to their Rydberg states offers a unique opportunity for combining advantages of precisely controllable trapped-ion qubits with long-range, tunable Rydberg interactions [1,2]. As an exciting application, we proposed a method for fast entangling operations using Rydberg trapped ions that are shuttled in a Paul trap [3]. The state-dependent kick is resulted from impulsive electric pulses [4] acting on ions in Rydberg states with huge polarizability [5]. This gives rise to a geometric phase gate which is controlled using experimental parameters [3], and it is orders of magnitude faster as compared to typical trapped-ion quantum gates. More recent studies open doors to engineering the coupling between electronic and vibrational degrees of freedom in Rydberg atoms held in optical tweezers [6]. This can be used as a versatile tool for designing many-body interactions in arrays of cold Rydberg atoms in optical tweezers. We also discuss our new experimental setup in Mainz that aims for coherent manipulation of Rydberg states of 40Ca$^+$ ions. In this experiment, high power ultra-violet lasers are used to drive ionic Rydberg transitions via a two-step excitation scheme, and we present the results for Rydberg spectroscopy of S and D series.

{\normalsize
[1] Feldker et al., Phys. Rev. Lett. 115, 173001(2015).
\vsp

[2] Higgins et al., Phys. Rev. Lett. 119, 220501 (2017).
\vsp

[3] Vogel et al., Phys. Rev. Lett. 123, 153603 (2019).
\vsp

[4] Walther et al., Phys. Rev. Lett. 109, 080501 (2012).
\vsp

[5] Higgins et al., Phys. Rev. Lett. 123, 153602 (2019).
\vsp

[6] Gambetta et al., Phys. Rev. Lett. 124, 043402 (2020).
}

\vspace{\baselineskip}