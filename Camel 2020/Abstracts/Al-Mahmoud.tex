\title{QUANTUM-INSPIRED BROADBAND POLARIZATION ROTATORS}
% Quantum-inspired broadband polarization rotators

\underline{M. Al-Mahmoud} \index{Al-Mahmoud M.}
%Mouhamad Al-Mahmoud

{\normalsize{\vspace{-4mm}
Theoretical Physics Division, Department of Physics, Sofia University,
James Bourchier 5 Blvd, 1164 Sofia, Bulgaria



\email mouhamadmahmoud1@gmail.com}}

An optical polarization rotator is an element that rotates the polarization of a linearly polarized input wave by a fixed angle that is independent of the input polarization direction. In order to make a broadband rotator, we theoretically proposed and experimentally verified a simple rotator’s scheme. It is composed of only three ($N=3$) waveplates: a full-wave plate retarder (i.e. phase shift between polarization components $\phi=-2\pi$) is sandwiched between two half-wave plates retarders ($\phi=\pi$)). Such a simple composite polarization rotator is shown to be very robust against variations of the individual wave plates retardations. This leads to a broadband operation with full agreement between theoretical expectations and experimental results [1]. The above scheme can be easily tuned since the rotation angle can be changed by rotating only one of the half-wave plates. Besides the three wave-plates rotator, we will also discuss alternative composite-pulses-inspired broadband schemes composed of an even numbers of half-wave plates retarders ($N$ = 2, 4, 6, 8 or 10) [2].

{\normalsize
[1] Al-Mahmoud, M., Coda, V., Rangelov, A., & Montemezzani, G. (2020). Physical Review Applied, 13(1), 014048.
\vsp

[2] Stoyanova, E., Al-Mahmoud, M., Hristova, H., Rangelov, A., Dimova, E., & Vitanov, N. V. (2019). Journal of Optics, 21(10), 105403.
}

\vspace{\baselineskip}