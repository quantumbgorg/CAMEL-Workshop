\title{DEVELOPMENT AND CHARACTERIZATION OF AN APPARATUS FOR A MULTI-ION FREQUENCY STANDARD}
% Development and characterization of an apparatus for a multi-ion frequency standard

\underline{C. Wunderlich} \index{Wunderlich C.}
%Christof Wunderlich

{\normalsize{\vspace{-4mm}
University of Siegen, Department of Physics, 57068 Siegen, Germany



\email christof.wunderlich@uni-siegen.de}}

The opticlock consortium (www.opticlock.de) develops a compact transportable optical clock for non-specialist users with a projected uncertainty of order $10^{-16}$. This clock, based on the 2S$_{1/2}$ - 2D$_{3/2}$ resonance with wavelength near 436 nm in a single $^{171}$Yb$^+$ ion, could be further improved using a frequency standard based on multiple ions. For this purpose, a segmented four layer ion trap for confining a linear Coulomb crystal of $^{171}$Yb$^+$ ions [1] and a compact vacuum interface, allowing for excellent optical access, is used. Here, we will focus on the design aspects and construction process of the new setup and give details regarding optical [2], electrical and vacuum aspects and present the experimental status of the complete linear trap setup.

{\normalsize
[1] M. Brinkmann, A.Didier, T. Mehlst\"aubler, Physikalisch-Technische Bundesanstalt,
Bundesallee 100, 38116 Braunschweig, Germany.

[2] S. Brakhane, D. Meschede, Institut f\"ur Angewandte Physik der Universit\"at Bonn, Wegelerstr. 8, 53115 Bonn, Germany.
}

\vspace{\baselineskip}