\title{DEVELOPING PRACTICAL QUANTUM COMPUTERS WITH TRAPPED IONS}
% Developing practical quantum computers with trapped ions

\underline{W. Hensinger} \index{Hensinger W.}
%Winfried Hensinger

{\normalsize{\vspace{-4mm}
Sussex Centre for Quantum Technologies, Department of Physics and Astronomy, University of Sussex, Brighton BN1 9QH, United Kingdom




\email w.k.hensinger@sussex.ac.uk}}

Trapped ions are arguably the most mature technology capable of constructing practical large scale quantum computers. We are now moving away from fundamental physics studies towards tackling the required engineering tasks in order build such machines. By inventing a new method where voltages applied to a quantum computer microchip are used to implement entanglement operations, we have managed to remove one of the biggest barriers traditionally faced to build a large-scale quantum computer using trapped ions, namely having to precisely align billions of lasers to execute quantum gate operations [1]. We developed a practical blueprint to construct such a machine [2].

I will present on update on the development and construction of practical prototype machines based on the approach above.

The cost of enabling connectivity in Noisy-Intermediate-Scale-Quantum devices is an important factor in determining computational power. We have created a qubit routing algorithm [3] which enables efficient global connectivity in a particular trapped ion quantum computing architecture. The routing algorithm was characterized by comparison against both a strict lower bound, and a positional swap based routing algorithm. We propose an error model which can be used to estimate the achievable circuit depth and quantum volume of the device as a function of experimental parameters. We use a new metric based on quantum volume, but with native two qubit gates, to assess the cost of connectivity relative to the upper bound of free, all to all connectivity. The metric was also used to assess a square grid superconducting device. We compare these two architectures and find that for the shuttling parameters used, the trapped ion design has a substantially lower cost associated with connectivity.

{\normalsize
[1]        Trapped-ion quantum logic with global radiation fields, S. Weidt, J. Randall, S. C. Webster, K. Lake, A. E. Webb, I. Cohen, T. Navickas, B. Lekitsch, A. Retzker, and W. K. Hensinger, Phys. Rev. Lett. {\bf 117}, 220501 (2016).
\vsp

[2]     Blueprint for a microwave trapped ion quantum computer, B. Lekitsch, S. Weidt, A.G. Fowler, K. Mølmer, S.J. Devitt, Ch. Wunderlich, and W.K. Hensinger, Science Advances 3, e1601540 (2017).
\vsp

[3]     Efficient Qubit Routing for a Globally Connected Trapped Ion Quantum Computer, Mark Webber, Steven Herbert, Sebastian Weidt, Winfried Hensinger, arXiv:2002.12782 [quant-ph] (2020).
}

\vspace{\baselineskip}