\title{LIFTING THE LIMITS OF CAVITY-ENHANCED SINGLE PHOTON EMISSION}
% Lifting the limits of cavity-enhanced single photon emission

\underline{A. Kuhn} \index{Kuhn A.}
%Axel Kuhn

{\normalsize{\vspace{-4mm}
University of Oxford,
Dept. of Physics,
Parks Road, Oxford,
OX1 3PU, United Kingdom




\email axel.kuhn@physics.ox.ac.uk}}

Purcell-enhanced single-photon emission into a cavity [1] is at the heart of many schemes for interfacing quantum states of light and matter in quantum networks [2]. We show that the intra-cavity coupling of orthogonal polarisation modes in a birefringent cavity [3] allows for the emitter and photon to be decoupled prior to emission from the cavity mode, enabling photon extraction efficiencies that exceed the limits of Purcell enhancement [4], which were hitherto considered being fundamental. Tailored cavity birefringence is seen to mitigate the tradeoff between stronger emitter-cavity coupling and efficient photon extraction, providing significant advantages over single-mode cavities. We generalise this approach and show that engineered coupling between states of the emitter can equivalently hide the emitter from the photon, ultimately allowing the extraction efficiency to approach its fundamental upper limit. The principles proposed in this work can be applied in multiple ways to any emitter-cavity system, paving the way to surpassing the traditional limitations with technologies that exist today.

{\normalsize
[1] A. Kuhn, chapter 1 in Engineering the Atom-Photon Interaction, A. Predojevic and M. W. Mitchell (edts), pages 3–37, Springer (2015).
\vsp

[2] H. J. Kimble, Nature \textbf{453}, 1023 (2008).
\vsp

[3] T. D. Barrett, O. Barter, D. Stuart, B. Yuen, and A. Kuhn, Phys. Rev. Lett. \textbf{122}, 083602 (2019).
\vsp

[4] T. D. Barrett, T. Doherty, and A. Kuhn: Pushing Purcell-enhancement beyond its limits, submitted, arXiv:1903.08628 (2019).
}

\vspace{\baselineskip}