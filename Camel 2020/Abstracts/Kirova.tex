\title{ATOM LOCALIZATION USING RYDBERG-EIT/ CPT WITH SPATIALLY DEPENDENT FIELDS}
% Atom localization using Rydberg-EIT/ CPT with spatially dependent fields

\underline{T. Kirova} \index{Kirova T.}
%Teodora Kirova

{\normalsize{\vspace{-4mm}
Jelgavas iela 3,
Riga,LV-1004, Latvia



\email teo@lu.lv}}

Atom localization has been of constant interest in quantum mechanics, however modern quantum optics tools have made realization of such experiments possible.   Different methods exist for subwavelength atom localization, while in this work we investigate the possibility to attain localization using interacting Rydberg atoms.
We consider a scheme of two Rydberg atoms interacting via van der Waals $vdW$ interaction, where each atom is in a three-level ladder EIT configuration.
The ground and middle states are coupled by a traveling wave probe field, while the middle and Rydberg states are connected via a standing-wave coupling field.
We derive an analytical expression for the steady state Rydberg level population, as well as the form of the parameter $S$, which describes the energy shift to the Rydberg state induced by the $vdW$ interaction with other excited atoms, which are usually situated beyond the blockade radius.

Our numerical calculations show that when the coupling field detuning equals $S$ at the nodes of the standing-wave, one can obtain perfect atom localization with Rydberg level population equal to 1.

Localization is possible also when the probe field intensity is increased. However, it is accompanied by spectral line widening and a loss of localization sharpness. Similar numerical results are obtained with a coupling field standing-wave in two directions, leading to lattice-like structure of localized atoms in $XY$ plane, e. g. $2D$ localization. In order to confirm atom localization in x-space, we also study localization in p-space, according to Heisenberg uncertainty principle. Finally, we investigate the possibilities to attain atom localization using a coupling field which carries Optical Angular Momentum.


\vspace{\baselineskip}