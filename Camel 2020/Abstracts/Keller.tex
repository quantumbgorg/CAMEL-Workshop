\title{COUPLING IONS TO OPTICAL CAVITIES}
% Coupling Ions to Optical Cavities

\underline{M. Keller} \index{Keller M.}
%Matthias Keller

{\normalsize{\vspace{-4mm}
University of Sussex,
Department of Physics and Astronomy,
Pevensey 2,
Brighton BN1 9QH,
UK



\email m.k.keller@sussex.ac.uk}}

The complementary benefits of trapped ions and photons as carriers of quantum information make it appealing to combine them in a joint system. Ions provide low decoherence rates, long storage times and high readout efficiency, while photons are ideal candidates for the transmission of quantum states over long distances. To interface the quantum states of ions and photons efficiently, we use calcium ions coupled to an optical high-finesse cavity via a Raman transition.

We employ two approaches to combine trapped ions and optical cavities. In one system, calcium ions are trapped in a linear ion trap with the optical cavity collinear to the trap axis. In this weakly coupled system we explore novel schemes to generate photons with unparalleled control and high indistinguishability. We employ cavity assisted Raman transitions to generate single photons. By carefully choosing the cavity assisted Raman transition, the adverse effect of the atomic decay on the distinguishability of the emitted photons can be mitigated. We demonstrated an improvement of the HOM visibility (integrated over the entire photon pulse) from 51\%, for a conventional cavity assisted Raman transition, to 72\%. The predicted visibility is >90\%. We attribute the reduction to noise in our system.

In a second system we couple a single ion to an optical fibre cavity with small mode volume. In this system, a fibre cavity is integrated into the electrodes of an endcap style ion trap. We have recently demonstrated the first strong coupling of a single ion to an optical cavity and observed vacuum induced transparency in this system. By employing cavity cooling, we have been able to improve the ion’s localisation in the cavity and increase the coherent ion-photon coupling.


\vspace{\baselineskip}
