\title{NOVEL APPLICATIONS OF COMPOSITE PULSES}
% Novel applications of composite pulses

\underline{B. Torosov} \index{Torosov B.}
%Boyan Torosov

{\normalsize{\vspace{-4mm}
72 Tsarigradsko shose



\email torosov@gmail.com}}

We demonstrate two different applications of the composite-pulse technique. First, we introduce a method for detection of chiral molecules using sequences of three pulses driving a closed-loop three-state quantum system. The left-and right-handed enantiomers have identical optical properties (transition frequencies and transition dipole moments) with the only difference being the sign of one of the couplings. We identify twelve different sequences of resonant pulses for which chiral resolution with perfect contrast occurs. In all of them the first and third pulses are $\pi/2$-pulses and the middle pulse is a $\pi$-pulse. In addition, one of the three pulses must have a phase shift of $\pi/2$ with respect to the other two. The simplicity of the proposed chiral resolution technique allows for straightforward extensions to more efficient and more robust implementations by replacing the single $\pi/2$ and $\pi$-pulses by composite pulses. We present specific examples of chiral resolution by composite pulses which compensate errors in the pulse areas and the detuning of the driving fields. As a second application, we present a systematic and general approach to producing quantum logic gates in Raman qubits. The method is based on the theory of composite pulses, which is adapted to be used in the current context. The technique uses the same composite phases to produce X gates, Hadamard gates, and rotation gates, while only the Rabi frequencies need to be different. For the creation of a phase shift gate, a slightly different approach is used, which is still general and arbitrarily accurate. All composite phases are given by an analytical formula, which makes the method scalable.

\vspace{\baselineskip}