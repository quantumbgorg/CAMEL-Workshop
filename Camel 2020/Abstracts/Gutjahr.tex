\title{FROM GOOD IDEAS TO WORKING PROTOTYPES, OR HOW TO GET A START-UP COMPANY OFF THE GROUND}
% From good ideas to working prototypes, or how to get a start-up company off the ground

\underline{J. Gutjahr} \index{Gutjahr J.}
%Joscha Gutjahr

{\normalsize{\vspace{-4mm}
2nd floor
The Wolfram center,
Long Hanborough,
OX29 8FD,
Oxfordshire,
Great Britain



\email joscha@verivin.com}}

VeriVin deploys Raman spectroscopy for through-the-barrier analysis of liquids and started to actively develop a portable dedicated spectrometer in 2018. This is of particular importance for the wine industry, where we test for authenticity and faults. Other substances of interest are spirits, honey and a large variety of further liquids of either nutritional or technical relevance.

In this talk, I will present how to commercialize a good scientific idea, illustrating the pathway from an initial lab-based development to a small successful startup company. We now have a first working product after less than two years of development. The list of milestones to get there range from elaborating the initial idea and IP registration over optical bench testing, assembly of OEM components, to the design of a first prototype and the implementation of various software elements, such as database infrastructure and user interface platforms. Other more economic aspects I will be discussing relate to funding and grants for start-up companies, cost reduction, the hiring of key staff, outsourcing of development, product design relating to client needs and marketability.

\vspace{\baselineskip}