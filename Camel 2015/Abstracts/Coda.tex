\title{ANALOG TO QUANTUM DYNAMICS WITH LIGHT COUPLING IN PHOTOINDUCED WAVEGUIDES}

\underline{V. Coda} \index{Coda V}

{\normalsize{\vspace{-4mm}
LMOPS,
Locaux de CentraleSup\'{e}lec,
2 rue E Belin,
57070 METZ,
France

\email coda5@univ-lorraine.fr}}

Thanks to the formal analogy between the Schr\"{o}dinger equation governing the dynamics of a quantum system and the
paraxial wave equation of light that propagates in a guiding structure, analogies can provide a useful laboratory tool to
easily visualize the temporal evolution of a quantum system. This can be also exploited to bring new concepts from
quantum physics to the optical network community.  For instance, adiabatic transfer of population can correspond in
coupled waveguides to a robust broadband transfer of energy in waveguides.

In this work the realization of integrated structures uses photorefractive materials (where an internal space-charge electric
field arises under illumination and creates a modification of the refractive index via the electro-optic effect). It takes
advantage of a unique experimental versatile platform that uses a biased photorefractive crystal and lateral illumination for
a rapid realization of any reconfigurable 3-D waveguide pattern with engineered local properties. This technique allows to
create refractive index structures with various reconfigurable profiles. Moreover, unlike in the case of conventional
fabrication techniques, a longitudinal variation of the propagation constants or a defect in a periodic waveguide array can
be easily implemented. Therefore the photo-induced technique is ideal for a fast, easy and low cost demonstration of new
concepts, ideas and components based on analogies with quantum dynamics and classical optics.

\vspace{\baselineskip}
