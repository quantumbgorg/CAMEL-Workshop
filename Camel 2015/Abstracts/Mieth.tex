\title{EXPERIMENTAL IMPLEMENTATION OF COHERENCE POPULATION MAPPING IN A DOPED SOLID}

\underline{S. Mieth} \index{Mieth S}

{\normalsize{\vspace{-4mm}
\darmstadt

\email Simon.Mieth@physik.tu-darmstadt.de}}

We present the first experimental implementation of a novel coherence population mapping (CPM) scheme. The idea of our CPM protocol is to map the amplitude and phase of an arbitrary atomic coherence onto a population distribution in a three state system. Thus, the information stored in the atomic coherence is no more temporally limited by the typically fast decoherence time, but prolonged up to the population relaxation time. The protocol does not require complex dynamical decoupling sequences or static magnetic fields, but uses a rather short write (and read) sequence only. After the storage time, we retrieve the atomic coherence with a defined phase, and at a third of the initial amplitude. In comparison to an earlier approach, the novel composite solution of CPM requires only two coherent coupling fields and control of their phase.
We show experimental results on storage and retrieval of a radio frequency coherence, mapped onto populations in the long lived hyperfine ground states of a rare earth ion doped solid. In addition, we compare the protocol to the well-known stimulated photon echo (SPE). In contrast to the latter, CPM permits efficient storage of any coherence at arbitrary phase, with storage times reaching the regime of one minute. We note, that the technique is a quite general approach towards long-term data storage, i.e. it is not restricted to our specific implementation in the inhomogeneous manifold of a doped solid.

\vspace{\baselineskip}
