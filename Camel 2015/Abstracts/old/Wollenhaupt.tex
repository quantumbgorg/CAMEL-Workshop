\title{RECENT DEVELOPMENTS IN MULTIPHOTON PHOTOELECTRON CIRCULAR DICHROISM (PECD)}

\underline{M. Wollenhaupt} \index{Wollenhaupt M}

{\normalsize{\vspace{-4mm}
Carl von Ossietzky Universit\"{a}t Oldenburg, Institut f\"{u}r Physik,\\
Carl-von-Ossietzky-Strasse 9-11,
D-26129 Oldenburg,
Germany

\email matthias.wollenhaupt@uni-oldenburg.de}}

PECD describes the asymmetry in the photoelectron angular distribution (PAD) after ionization of randomly oriented chiral molecules in the gas phase with circularly polarized light. PECD was observed in one photon ionization using synchrotron radiation. Recently, we have measured PECD by femtosecond REMPI of camphor and fenchone molecules [1]. In our experiments strong contributions of higher order Legendre polynomials were observed. To apply PECD as a sensitive analytical tool, quantitative measures to evaluate the experimental PECD data are required. For one photon ionization, parameters to characterize the asymmetry of the PAD based on the forward/ backward asymmetries have been developed [2]. Although this method can be extended to the multiphoton case, we show that measures based on the forward/backward asymmetry are generally not sufficient to quantify the multiphoton PECD. We suggest a more general measure based on the decomposition of the PAD into their gerade and ungerade part. In addition, a measure to evaluate images from noncylinder symmetrical PAD is introduced. These measures are evaluated on experimental multiphoton PECD data from camphor molecules.


{\normalsize
[1]  C. Lux et al., Angew Chem Int Ed \textbf{51}, 5001 (2012).
\vsp

[2] L. Nahon et al., J Chem Phys \textbf{125}, 114309 (2006).
}

\vspace{\baselineskip}
