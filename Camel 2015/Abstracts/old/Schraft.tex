\title{UNIVERSAL COMPOSITE PULSES FOR REPHASING OF ATOMIC COHERENCES IN A DOPED SOLID}

\underline{D. Schraft} \index{Schraft D}

{\normalsize{\vspace{-4mm}}
Institute of Applied Physics, Technical University of Darmstadt, Hochschulstrasse 6, 64289 Darmstadt, Germany

\email daniel.schraft@physik.tu-darmstadt.de}

Composite pulses (CP) have been used for decades in nuclear magnetic resonance, and since recently, also in quantum information processing as a powerful tool to drive excitation processes via robust pathways. The general idea of CP is to improve the performance of an excitation process driven by a single pulse, by applying a sequence of pulses with appropriately chosen phases. Usually these CP compensate fluctuations in a single experimental parameter only. Here we introduce universal CP [1] for robust system inversion, compensating variations in any experimental parameters (i.e. pulse amplitude, pulse duration, static detuning, etc.), and also operate independent of the pulse shape.
We demonstrate the performance of universal CP by inversion of atomic coherences in a rare earth ion-doped solid (Pr:YSO). Such doped solids are an attractive medium to implement solid-state quantum memories. The media exhibit long decoherence times and small homogenous optical line width, while maintaining the advantages of solids, i.e. large density and scalability. Typically, these memories rely on atomic coherences, driven between inhomogeneously broadened levels. Hence, robust rephasing protocols are required to cope with dephasing. Our experimental data confirm improved robustness of universal CP, in comparison with standard $\pi$-pulses, with regard to variations in pulse area, static detuning, and different pulse shapes.

{\normalsize
[1] G. T. Genov, D. Schraft, T. Halfmann, and N. V. Vitanov, arXiv:1403.1201.
}

\vspace{\baselineskip}
