\title{SOME FUNDAMENTAL ASPECTS AND APPLICATIONS OF NON-MARKOVIAN QUANTUM DYNAMICS}

\underline{J. Piilo}\index{Piilo J}

{\normalsize{\vspace{-4mm}
Department of Physics $\&$ Astronomy,
University of Turku,
FI-20014 Turun Yliopisto,
Finland

\email jyrki.piilo@utu.fi}}

Non-Markovian quantum dynamics and memory effects have recently attracted quite a large amount of attention both from
fundamental and applicative perspective. We begin by recalling the concept of nonlocal memory effects [1] and show how this can
be used to perform efficient teleportation with mixed states [2] and to protect a polarization entangled photon pair against
decoherence when distributed in optical fibers [3]. From a more fundamental point of view,  we discuss the universality of
quantum memory effects [4] and conclude by presenting some schemes for discrete non-Markovian quantum walks implemented
with photons.

{\normalsize

[1] E.-M. Laine et al., Phys. Rev. Lett. \textbf{108}, 210402 (2012).
\vsp

[2] E.-M. Laine, H.-P. Breuer, and J. Piilo, Scientific Reports \textbf{4}, 4620 (2014).
\vsp

[3] G.-Y. Xiang et al., arXiv:1401.5091.
\vsp

[4] B.-H. Liu et al., arXiv:1403.4261.
}

\vspace{\baselineskip}
