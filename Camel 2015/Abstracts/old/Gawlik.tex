\title{COHERENCE EFFECTS IN COLD RUBIDIUM ATOMS}

\underline{W. Gawlik} \index{Gawlik W}

{\normalsize{\vspace{-4mm}
Institute of Physics, Jagiellonian University,
Reymonta 4, 30-059 Krakow, Poland

\email gawlik@uj.edu.pl}}

We report on our experiments on nonlinear magneto-optical effects in laser-cooled, near-degenerate rubidium samples. Atoms are laser-cooled and subsequently transferred to the crossed-beam optical dipole trap (ODT), formed by focusing of 1070-nm laser beams. The tight confinement of atoms enables magnetic field probing with a high spatial resolution of a few tens of micrometers.
Interaction of atoms with a near-resonant, linearly polarized light leads to an effective creation of long-lived ground-state Zeeman coherences, which is observed through the nonlinear Faraday effect or free induction decay (FID) signals of the Larmor precession. Our experiments show that high rotation angles of a few degrees and coherence lifetimes of a few milliseconds can be achieved with cold atoms released from the magneto-optical trap (MOT). Moreover, we are able to optically detect the weak rf (kHz regime) magnetic fields.
Application of these effects to precision magnetometry and its potential limits are presented. By employing the amplitude-modulation of the laser beam, for instance, we are able to measure high (geophysical range) magnetic fields.
The presented work describes the experiments with a far off-resonant ODT, which enables much longer observation times. Our goal is to study coherence effects in the temperatures down to the limit of quantum degeneracy, i.e., a Bose-Einstein condensate and with the sensitivity close to the photonic-shot noise limit.

\vspace{\baselineskip}
