\title{SELECTIVE EXCITATION OF ELECTRONIC STATES IN $K_2$}

\underline{H. Braun} \index{Braun H}

{\normalsize{\vspace{-4mm}
Universit\"at Kassel,
Experimentalphysik III,
Heinrich-Plett-Str. 40,
34132 Kassel,
Germany

\email braun@physik.uni-kassel.de}}

Recently we reported on the selective excitation of $K_2$ using spectrally phase-shaped femtosecond laser pulses [1]. A detailed analysis of quantum dynamic simulations reveals the mechanism at play [2]. By bespoke tailoring of the electric field the coupled electron-nuclear dynamics in molecules are steered. In this way we can manipulate the delicate interplay between the driving laser field and the light induced dynamics in the molecule and efficiently populate different target channels not accessible in the perturbative excitation regime. In addition first experimental evidence shows that femtosecond laser pulses, that are shaped with spectral phases consisting of second and third order polynomial modulation [3], offer a high degree of control over the populations in selected electronic states of the potassium dimer.

{\normalsize
[1] T. Bayer et al., Phys. Rev. Lett. \textbf{110}, 123003 (2013).
\vsp

[2] H. Braun et al., J. Phys. B., in print (2014).
\vsp

[3] J. Schneider et al., Phys. Chem. Chem. Phys. \textbf{13}, 8733 (2011).
}

\vspace{\baselineskip}
