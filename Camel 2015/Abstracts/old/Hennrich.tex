\title{EXPERIMENTAL QUANTUM COMPUTATIONS ON A TOPOLOGICALLY ENCODED QUBIT}

\underline{M. Hennrich} \index{Hennrich M}

{\normalsize{\vspace{-4mm}
University of Innsbruck,
Experimental Physics,
Technikerstr. 25,
6020 Innsbruck,
Austria

\email markus.hennrich@uibk.ac.at}}

The construction of a quantum computer remains a fundamental scientific and technological challenge, in particular due to unavoidable noise. Quantum states and operations can be protected from errors using protocols for fault-tolerant quantum computing (FTQC). Here we present a step towards this by implementing a quantum error correcting code, encoding one qubit in entangled states distributed over 7 trapped-ion qubits. We demonstrate the capability of the code to detect one bit flip, phase flip or a combined error of both, regardless on which of the qubits they occur. Furthermore, we apply combinations of the entire set of logical single-qubit Clifford gates on the encoded qubit to explore its computational capabilities. The implemented 7-qubit code is the first realization of a complete Calderbank-Shor-Steane (CSS) code and constitutes a central building block for FTQC schemes based on concatenated elementary quantum codes. It also represents the smallest fully functional instance of the color code, opening a route towards topological FTQC.

\vspace{\baselineskip}
