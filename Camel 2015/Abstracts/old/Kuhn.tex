\title{CODING OF QUBITS, QUTRITS AND QUQUADS IN CAVITY-QED PHOTONS}

\underline{A. Kuhn}\index{Kuhn A}

{\normalsize{\vspace{-4mm}
University of Oxford, Clarendon Laboratory, Parks Road, OX1 3PU Oxford, United Kingdom

\email axel.kuhn@physics.ox.ac.uk}}

Photons acting as flying information carriers are key to many modern applications of quantum technology like, e.g., linear optics
quantum computing (LOQC) and long-distance quantum communication. The arbitrary control of the photonic states in space and
time is crucial to the success of these schemes. To this end, we have been developing new methods for the interfacing of atoms
and photons in a cavity-qed, which allow for full quantum control of photon emission and absorption. In particular, we are using a
vacuum-stimulated Raman process to control the spatio-temporal properties of single photons to an unprecedented degree,
which allows us to obtain photons of arbitrary shape. We have been demonstrating the singleness and indistinguishability of these
photons in sophisticated photon-photon correlation experiments, and we also achieved a single-photon production efficiency of
85\% at a repetition rate of 1 MHz. Over and above that, we have been using this scheme to encode arbitrary qubits, qutrits and
even ququads within the spatio-temporal mode profile of individual photons. The fidelity of this technique has been verified in
time resolved quantum-homodyne measurements to be better than 96\%. Such a close-to-perfect control of photonic
wavefunctions might well change the way we think about quantum logic today. For instance, when using qutrits or ququads,
powerful ternary or quaternary quantum logic concepts can be realised without the need for additional resources.

\vspace{\baselineskip}
