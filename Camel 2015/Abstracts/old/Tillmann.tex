\title{BOSONSAMPLING WITH CONTROLLABLE DISTINGUISHABILITY}

\underline{M. Tillmann} \index{Tillmann M}

{\normalsize{\vspace{-4mm}
Faculty of Physics, University of Vienna,
Quantum Information Science and Quantum Computation,
Boltzmanngasse 5, A-1090 Vienna, Austria

\email Max.Tillmann@univie.ac.at}}

Photonic BosonSampling computers have inspired significant interest because they have the potential to solve computationally hard problems efficiently using only a few-dozen indistinguishable photons. However, their ability to outperform classical computers is currently limited by their tight error tolerance. Whereas practical BosonSampling computation depends on the quality of the apparatus it is the photons distinguishability which is the fundamentally limiting source of error. Here we develop a method describing the transition probabilities of photons with arbitrary distinguishability through any linear-optical network. We test this experimentally by tuning the temporal delay of the input-photons. Our approach provides tighter estimates for the underlying BosonSampling distribution by relating the output to the transition matrix immanants, enabling the main source of errors to be quantified. This is essential for experimentally realizable implementations. Our method may enable generalized BosonSampling computation through the use of immanants and not just permanents. 


%{\normalsize
%[1] H. Takahashi et al. , New J. Phys. \textbf{15}, 053011 (2013).
%}

\vspace{\baselineskip}
