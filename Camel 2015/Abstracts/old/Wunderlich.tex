\title{ADDRESSING A QUANTUM BYTE AND ROBUST DYNAMICAL DECOUPLING FOR CONDITIONAL QUANTUM GATES}

\underline{Ch. Wunderlich} \index{Wunderlich Ch}

{\normalsize{\vspace{-4mm}
Universtit\"{a}t Siegen,
Department Physik,
57068 Siegen

\email christof.wunderlich@uni-siegen.de}}

We report on recent experimental achievements using trapped $^{171}$Yb$^+$ ions that interact via Magnetic Gradient Induced Coupling (MAGIC).

The addressing of a particular qubit within a quantum register is a key prerequisite for scalable
quantum computing. We demonstrate addressing of individual qubits within a quantum byte
(eight qubits) and measure a crosstalk associated with the application of single-qubit gates on the
order of $10^{-5}$, breaching an important threshold for fault-tolerant quantum computing [1].

Dynamical decoupling (DD) sequences are applied to protect CNOT gates against decoherence. The sequences  employed here are robust against imperfections of DD pulses that
otherwise may destroy quantum information or interfere with gate dynamics.
A CNOT gate is implemented, despite the gate time being more
than one order of magnitude longer than the intrinsic coherence time of the
system [2].

{\normalsize
[1] Ch. Piltz, Th. Sriarunothai, A. Varon, and Ch. Wunderlich, arXiv:1403.8043.
\vsp

[2] Ch. Piltz, B. Scharfenberger, A. Khromova, A. F. Varon, and Ch. Wunderlich, Phys. Rev. Lett. {\bf 110}, 200501 (2013).
}

\vspace{\baselineskip}
