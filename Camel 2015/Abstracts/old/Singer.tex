\title{DETERMINISTIC ION SOURCE REALIZING NANOMETER RESOLUTION AND THERMODYNAMICS WITH TRAPPED IONS}

\underline{K. Singer} \index{Singer K}

{\normalsize{\vspace{-4mm}}
Staudingerweg 20, Germany

\email kilian.singer@uni-mainz.de}

Novel ion trap geometries for deterministic high resolution ion
implantation are presented which are obtained by highly efficient field
 calculation methods [1]. I will present our recent progress with a
segmented ion trap with mK laser cooled ions which serves as a high
resolution deterministic single ion source. It can operate with a huge
range of sympathetically cooled ion species, isotopes or ionic
molecules. We have deterministically extracted a predetermined number of
ions on demand [2], moved the ions inside the trap [3] without exciting
any motional quanta and reached extraction accuracies of less than 10 nm [4].
These results a first step in the realization of an atomic nano
assembler, a novel device capable of placing an exactly defined number
of atoms or molecules into solid state substrates with sub nano meter
precision in depth and lateral position. The project is motivated by the
quest for novel tailored solid state quantum materials generated by
deterministic high resolution ion implantation. Ions can also be moved
by heat realizing a heat engine scaled down to a single ion [5]. We
propose driving the trapped ion in an Otto cycle, oscillating in a
specially designed linear Paul trap and coupled to engineered laser
reservoirs. We present detailed Monte Carlo simulations and the
calculation of the efficiency of such single ion heat engine, exceeding
the standard Carnot limit when employing a squeezed thermal reservoir
[6]. Finally, I will discuss how ions can be moved in the radial
direction performing the experimental investigation of the Kibble-Zurek
mechanism, where control parameters are tailored such that a structural
phase transition from a linear to a zigzag configuration of the crystal
is crossed. Trapped ions serve here as a clean model system to
investigate universal laws of defect formation when such transition is
crossed fast and causally separated regions form. The amount of defects
is predicted by the Kibble-Zurek mechanism. We have experimentally
determined the universal scaling exponent for defect formation and
confirm the scaling law for the inhomogeneous Kibble-Zurek effect
accurately at the percent level [7].

{\normalsize
[1] K. Singer, U. G. Poschinger, M. Murphy, P. A. Ivanov, F. Ziesel, T.
Calarco, F. Schmidt- Kaler, Rev. Mod. Phys. \textbf{82}, 2609 (2010).
\vsp

[2] W. Schnitzler, N. M. Linke, R. Fickler, J. Meijer, F. Schmidt-Kaler,
and K. Singer, Phys. Rev. Lett. \textbf{102}, 070501 (2009).
\vsp

[3] A. Walther, F. Ziesel, T. Ruster, S. T. Dawkins, K. Ott, M.
Hettrich, K. Singer, F. Schmidt-Kaler, U. G. Poschinger, Phys. Rev.
Lett. \textbf{109}, 080501 (2012).
\vsp

[4] G. Jacob, K. Groot-Berning, S. Wolf, S. Ulm, L.
  Couturier, U. G. Poschinger, F. Schmidt-Kaler, K. Singer, arxiv 1405.6480 (2014).
\vsp

[5] O. Abah, J. RoГџnagel, G. Jacob, S. Deffner, F. Schmidt-Kaler, K.
Singer, E. Lutz, Phys. Rev. Lett. \textbf{109}, 203006 (2012).
\vsp

[6] J. RoГџnagel et al., Phys. Rev. Lett. \textbf{112}, 030602 (2014).
\vsp

[7] S. Ulm, J. RoГџnagel, G. Jacob, C. DegГјnther, S. T. Dawkins, U. G.
Poschinger, R. Nigmatullin, A. Retzker, M. B. Plenio, F. Schmidt-Kaler,
K. Singer, Nature Communications \textbf{4}, 2290 (2013).
}

\vspace{\baselineskip}
