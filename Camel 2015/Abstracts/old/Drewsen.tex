\title{ROTATIONAL COOLING OF COULOMB-CRYSTALLIZED MOLECULAR IONS BY A HELIUM BUFFER GAS}

\underline{M. Drewsen} \index{Drewsen M}

{\normalsize{\vspace{-4mm}
Department of Physics and Astronomy,
Aarhus University,
Ny Munkegade 120,
DK-8000 Aarhus C,
Denmark

\email drewsen@phys.au.dk}}

In this talk, I will discuss recent experimental results on helium buffer-gas cooling of the rotational degrees of freedom of MgH+ molecular ions, which are trapped and sympathetically crystallized in a linear radio-frequency quadrupole trap [1]. With helium collision rates of only ~10 s$^{-1}$, i.e. four to five orders of magnitude lower than in usual buffer gas cooling settings, we have cooled a single molecular ion to an unprecedented measured low rotational temperature of 7.5 K. In addition, by only varying the shape and/or the number of atomic and molecular ions in larger Coulomb crystals, we have tuned the effective rotational temperature from ~7 K up to ~60 K by changing the micromotion energy. The very low helium collision rate may potentially even allow for sympathetic sideband cooling of single molecular ions, and eventually make quantum-logic spectroscopy of buffer gas cooled molecular ions feasible. Furthermore, application of the presented cooling scheme to complex molecular ions should have the potential of single or few-state manipulations of individual molecules of biochemical interest. This latter perspective can hopefully be exploited to disentangle various processes happening in complex molecules, like light harvesting complexes.

{\normalsize
[1]	A. K. Hansen, O. O. Versolato, S. B. Kristensen, A. Gingell, M. Schwarz,
    A. Windberger, J. Ullrich, J. R. Crespo Lіpez-Urrutia and M. Drewsen, Nature vol. \textbf{508}, 76 (2014).
}


\vspace{\baselineskip}
