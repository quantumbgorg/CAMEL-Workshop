\title{NONCLASSICAL STATE PREPARATIONS BY RESERVOIR ENGINEERING IN AN ION TRAP}

\underline{H. Lo}\index{Lo H}

{\normalsize{\vspace{-4mm}
Trapped Ion Quantum Information Group,
Institute for Quantum Electronics,
ETH Zurich,
Otto-Stern-Weg 1, HPF E23,
8093 Zurich

\email hylo@phys.ethz.ch}}

Coupling a quantum system to an engineered reservoir provides new opportunities for quantum state preparations, and for studying
the dynamics of open quantum systems. In an ion trap, we can form engineered reservoirs because we precisely control interactions
between the ion itself and the electromagnetic vacuum. We have recently demonstrated reservoir engineering of the motional states
of a single calcium ion, realizing nonclassical states of the motion such as squeezed and coherent states as the steady-state of the
dynamics. The spin-motion coupling utilizes a multichromatic laser field, producing jump operators in an engineered motional basis,
which for us are the displaced and squeezed Fock states. Taking advantage of this engineered coupling, we have demonstrated novel
methods for state diagnosis and control working directly in the engineered eigenstate basis. The steady-state dissipative motion may
also be used as an environment for further quantum systems, or for investigating many-body dissipative dynamics. I will describe
how we plan to use two-species ion chains to investigate open systems using these methods, and will describe experimental
progress towards this goal.

\vspace{\baselineskip}
