\title{QUANTUM STORAGE OF HERALDED SINGLE PHOTONS IN A PR BASED SOLID-STATE MEMORY}

\underline{M. Mazzera} \index{Mazzera M}

{\normalsize{\vspace{-4mm}
ICFO-The Institute of Photonic Sciences,
Mediterranean Technology Park,
Av. Carl Friedrich Gauss, num. 3,
08860 Castelldefels (Barcelona), Spain

\email margherita.mazzera@icfo.es}}

Crystalline insulators doped with rare earth ions are promising systems for the realization of quantum memories, core elements in the architecture of quantum repeaters.
Despite the extraordinary properties of Pr:YSO,  including high efficiency [1] and long storage times [2], the storage of quantum light has never been achieved in this material. In fact, the hyperfine separation of the excited state establishes a tight bound to the bandwidth of the single photons to be stored ($<$ 4 MHz).
We report on the reversible mapping of heralded single photons to long lived collective optical excitation in Pr:YSO using the atomic frequency comb protocol [3]. The ultra-narrow band photons resonant with the memory are created by cavity-enhanced spontaneous parametric down-conversion and heralded by the detection of photons at the telecom wavelength [4]. We demonstrate quantum correlation between the retrieved and the heralding photons up to storage times of 4.5 us, more than 20 times longer than previous realizations with solid state systems [5]. Pr possesses an energy level scheme which allows the transfer to the ground state. However, the closely spaced hyperfine levels of the ground state (10.2 MHz) make the filtering of the technical excess noise introduced by the strong transferring pulses challenging. Adopting a narrow filtering strategy based on hole burning in a second Pr:YSO crystal, the spin-wave storage of weak coherent pulses, with an average number of photons per pulse lower than one, has been demonstrated for the first time.

{\normalsize
[1] M. P. Hedges, et al., Nature \textbf{465} 1052 (2010).
\vsp

[2] G. Heinze, et al., Phys. Rev. A \textbf{111} 033601 (2013).
\vsp

[3] M. Afzelius, et al., Phys. Rev. A \textbf{79} 052329 (2009).
\vsp

[4] J. Fekete, et al., Phys. Rev. Lett. \textbf{110} 220502 (2013).
\vsp

[5] D. Riel\"{o}nder, et al., Phys. Rev. Lett. \textbf{112} 040504 (2014).
}

\vspace{\baselineskip}
