\title{INTERFACING SINGLE IONS AND PHOTONS VIA CAVITY QED}

\underline{H. Takahashi} \index{Takahashi H}

{\normalsize{\vspace{-4mm}
University of Sussex, Pevensey 2 Flamer, Brighton, UK BN1 9QH

\email email:ht74@sussex.ac.uk}}

The complementary benefits of trapped ions and photons as carriers of quantum information make it appealing to interface them at the single quantum level. Single ions provide long storage times, high fidelity coherent control and efficient state readout, whereas single photons travel over a long distance with small decoherence.  The need for coherently interfacing single ions and photons naturally leads to cavity QED with single ions, where an optical cavity with small mode volume is used to achieve quantum mechanical coupling between two systems.

At University of Sussex we are currently working on two distinct ion-cavity QED experiments. In one of them, a cavity collinear to the axis of a linear Paul trap is employed where a moderate ion-photon coupling is expected. The simultaneous couplings of multiple ions to the same cavity mode can be exploited, for example, for probabilistic generation of entanglement. A stronger coupling can be achieved by using a miniature fibre cavity. We have developed a novel endcap-type ion trap which tightly integrates a high finesse fibre cavity inside the electrodes [1]. This design significantly reduces possible disturbances to the trap caused by the fibres’ dielectric surfaces and as a result allows us to bring the fibres as close as 150 $\mu$m to the ion. We have fabricated fibre cavities which are stable over a wide range of cavity length up to a few hundred $\mu$m, so that they can be used in this endcap trap.  With strong ion-photon coupling, a deterministic transfer of quantum states between ions and photons becomes possible. We present the current status and future prospects of the two experiments.


{\normalsize
[1] H. Takahashi et al. , New J. Phys. \textbf{15}, 053011 (2013).
}

\vspace{\baselineskip}
