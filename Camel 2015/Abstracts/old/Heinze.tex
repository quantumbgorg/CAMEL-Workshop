\title{REPHASING OF SINGLE SPIN EXCITATIONS FOR TEMPORALLY MULTIPLEXED QUANTUM MEMORIES}

\underline{G. Heinze} \index{Heinze G}

{\normalsize{\vspace{-4mm}
ICFO -- The Institute of Photonic Sciences,
Mediterranean Technology Park,
Av. Carl Friedrich Gauss, 3
08860 Castelldefels (Barcelona), Spain

\email georg.heinze@icfo.es}}

Future long-distance quantum communication networks critically rely on realistic quantum memories (QM) for light. Such devices allow coherent and reversible transfer of quantum information between flying qubits (typ. encoded in photons) and long-lived matter qubits (typ. encoded in atomic states).

An important benchmark for realistic QMs is the capability of temporal multiplexing, i.e. storing several qubits at the same time and selectively read them out afterwards. Although cold atomic ensembles provide excellent QMs, temporal multiplexed storage of single photons has not been achieved in these systems yet. We now demonstrate a significant step towards this goal.

In our experiment, we apply laser cooled Rubidium atoms, to form a QM based on the DLCZ-scheme. An incoming photon creates a single collective spin excitation via a Raman transition. The scattered photon is then used to indicate the charged QM. After the storage time, the QM can be read out by the inverse Raman process. To obtain the temporal multiplexing capability, we combine the DLCZ-memory with an externally controlled inhomogeneous broadening of the spin transition. By inverting the sign of the broadening, spin excitations which were stored at differing times will rephase at corresponding times after the switching point. This enables creation, storage and selective readout of single collective excitations in different temporal modes within the same atomic ensemble.

%{\normalsize
%[1] G. Heinze, C. Hubrich, and T. Halfmann, Phys. Rev. Lett. \textbf{111}, 033601 (2013).
%\vsp
%}

\vspace{\baselineskip}
