\title{FEEDBACK-ENHANCED SPATIAL ORGANIZATION OF BOSE-EINSTEIN CONDENSATE IN AN OPTICAL LATTICE}

\underline{D. Ivanov}
\index{Ivanov D}

{\normalsize{
\vspace{-4mm} Faculty of physics, St-Petersburg State University, Ulianovskaya 3,
Petrodvorets, St. Petersburg, 198504, Russia

%\vspace{-4mm} $^{2}$ \unisofia

\email ivanov-den@yandex.ru}}

We consider a Bose-Einstein condensate placed in an optical lattice potential with the lattice depth
controlled by a feedback loop. The control signal for the loop is the backward Bragg scattering of a
weak probe light. The wavelength of the probe light is equal to the double period of the lattice.
Under this condition the maximum Bragg scattering is obtained if the condensate wavefunction
represents narrow peaks localized near the center of each lattice site. The feedback loop is
designed to increase the lattice potential as the Bragg scattering increases, making the
periodically localized condensate wavefunction more favorable. We assume that the probe and the
scattered light are counterpropagating modes of a ring cavity, since the scattered signal is
expected to be small and should be collected. The time response of the system due to the cavity is
analyzed.
We perform quantum analysis of the system using the formalism of the positive P-representation. We show that there is no additional quantum noise introduced into the system via the
feedback loop, which is considered as an incoherent (containing measurement) electronic feedback.
Solving numerically the stochastic differential equations derived from the positive P-representation
Fokker-Planck equation we determine the effect of the quantum correlations between the scattered
light and the condensate.  The quantum statistical properties of the light and the condensate are
found as well. 

\vspace{\baselineskip}
