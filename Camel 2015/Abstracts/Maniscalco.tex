\title{THERMODYNAMIC MEANING AND POWER OF NON-MARKOVIANITY}

\underline{S. Maniscalco} \index{Maniscalco S}

{\normalsize{\vspace{-4mm}
Turun yliopisto, 20014 Turku, Finland

\email smanis@utu.fi}}

We establish a connection between non-Markovian memory effects and thermodynamical quantities such as work. We show how
memory effects can be interpreted as revivals of work that can be extracted from a quantum system. We prove that non-Markovianity
may allow an increase in the extractable work even when the entropy of the system is increasing. Our results have important
implications both in quantum thermodynamics and in quantum information theory. In the former context they pave the way to the
understanding of concepts like work in a non-Markovian open system scenario. In the latter context they lead to interesting
consequences for quantum state merging protocols in presence of noise.

%{\normalsize
%[1] P. A. Ivanov, N. I. Karchev, N. V. Vitanov and D. G. Angelakis,  arXiv:1405.6071.
%}

\vspace{\baselineskip}
