\title{TOWARDS RELIABLE QUANTUM COMPUTING}

\underline{D. Suter} \index{Suter D}

{\normalsize{\vspace{-4mm}
Fakult\"{a}t Physik,
TU Dortmund,
44227 Dortmund,
Germany

\email Dieter.Suter@tu-dortmund.de}}

Quantum information is a very valuable, but also very fragile resource.
Accordingly, protecting it against environmental noise remains one of the biggest challenges
on the road to scalable quantum computing. Another challenge is the effect of experimental
imperfections, which also degrade the fidelity of the quantum information.
A number of techniques have been designed to fight these problems and extend the lifetime of quantum coherence.
Protection schemes can be passive, by storing information in those parts of Hilbert space that are least affected
by environmental noise, or active, by applying control operations that average the environmental perturbations to
zero. The control operations have to be robust, such that the resulting operations are insensitive to the unavoidable
experimental imperfections of the control fields.
The talk will highlight some of these approaches and illustrate them with examples from different qubit systems.

\vspace{\baselineskip}
