\title{SINGLE-SHOT SHAPED PULSES FOR ROBUST REPHASING OF ATOMIC COHERENCES}

\underline{D. Schraft} \index{Schraft D}

{\normalsize{\vspace{-4mm}
\darmstadt

\email daniel.schraft@physik.tu-darmstadt.de}}

Applications of quantum control, e.g. for quantum information processing, require protocols of high fidelity, while being at the same time robust with respect to fluctuations in experimental parameters and system inhomogeneities. To tackle these requirements, in the last decades, a number of different approaches (i.e. composite pulses, rapid (composite) adiabatic passage, etc.) have been proposed. Here we present a powerful, new technique, i.e. single shot shaped pulses (SSSP) [1]. The general idea of SSSP is to design smooth time-dependent phase and amplitude variations in a single pulse, to drive a two-level system on specific excitation pathways, which are robust with regard to errors. We demonstrate the performance of SSSP by inversion of atomic coherences in a rare earth ion-doped solid. Such doped solids are attractive media to implement solid-state quantum memories. They exhibit long decoherence times and small homogenous linewidth, while maintaining the advantages of solids, i.e. large density and scalability. Typically, these memories rely on atomic coherences, driven between inhomogeneously broadened levels. Hence, robust rephasing techniques are required to cope with dephasing. Our experimental data and simulations confirm the pronounced robustness of SSSP with regard to variations in pulse area and detuning, making them a versatile tool to control coherent quantum dynamics.

{\normalsize
[1] D. Daems, A. Ruschhaupt, D. Sugny, and S. Gu\'erin, Phys. Rev. Lett. \textbf{111}, 050404 (2013).
}

\vspace{\baselineskip}
