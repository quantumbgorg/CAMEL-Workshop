\title{MULTIDIMENSIONAL HIGH HARMONIC SPECTROSCOPY}

\underline{B. Bruner} \index{Bruner B}

{\normalsize{\vspace{-4mm}
Department of Physics of Complex Systems,
Weismann Building Rm. 250,
Weizmann Institute of Science,
Rehovot 76100, Israel

\email barry.bruner@weizmann.ac.il}}

Multidimensional spectroscopy has become an indispensable tool in ultrafast science over the past couple of decades.  By spreading the measured spectra into multiple dimensions, one can uncover signatures of intra and intermolecular couplings and lineshape information that are hidden in one dimensional absorption or Raman spectra.  More recently, high harmonic generation (HHG) has opened up a new frontier in ultrafast science where attosecond time resolution and Angstrom spatial resolution are accessible in a single measurement.   Incorporating concepts of ultrafast 2D spectroscopy into the field of HHG spectroscopy will be essential for revealing the underlying complex dynamics behind attosecond scale phenomena.  These new measurement schemes integrate perturbative nonlinear optics with strong-field physics, and are capable of measuring tunnel ionization dynamics with high precision.  By applying these schemes to molecules, we can begin composing the features of a conventional ultrafast 2D spectrum by resolving multiple ionization channels and reconstructing their relative signal contributions. 

%{\normalsize
%[1] P. A. Ivanov, N. I. Karchev, N. V. Vitanov and D. G. Angelakis,  arXiv:1405.6071.
%}

\vspace{\baselineskip}
