\title{SHORTCUTS TO ADIABATICITY: STABLE POPULATION TRANSFER IN QUANTUM SYSTEMS}

\underline{A. Kiely} \index{Kiely A}

{\normalsize{\vspace{-4mm}
Department of Physics,
University College Cork,
Cork,
Ireland

\email kiely.anthony1@gmail.com}}

Fast and stable population transfer is a vital ingredient for upcoming quantum technologies
and quantum information tasks. The area of shortcuts to adiabaticity attempts to
tackle this problem by providing methods of population transfer which are faster than
adiabatic passage (avoiding problems with short decoherence times) but
still stable. I will outline two systems where these methods
have been applied. In the first part I will describe how population transfer in two- and three-level
systems can be made stable against unwanted transitions to nearby levels by using a combination of
invariant based inverse-engineering and perturbation theory [1]. In the second part, I will show how
the motional state of a charged particle in a Penning trap can be changed in a fast and stable way by
only changing the magnetic field strength [2].

{\normalsize
[1] A. Kiely and A. Ruschhaupt, J. Phys. B: At. Mol. Opt. Phys. \textbf{47}, 115501 (2014).
\vsp

[2] A. Kiely, J. P. L. McGuinness, J. G Muga and A. Ruschhaupt, J. Phys. B: At. Mol. Opt. Phys. \textbf{48}, 075503 (2015).
}

\vspace{\baselineskip}
