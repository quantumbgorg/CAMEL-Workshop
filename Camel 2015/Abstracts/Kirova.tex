\title{CONSEQUENCES OF DRESSED STATES CREATION IN ATOMIC NONDEGENERATE HYPERFINE LEVELS}

\underline{T. Kirova} \index{Kirova T}

{\normalsize{\vspace{-4mm}
Institute of Ayomic Physics and Spectroscopy, University of Latvia, Skunu iela 4, Riga, LV-1586, Latvia

\email teo@lu.lv}}

We conduct theoretical studies of the formation of laser induced adiabatic states in a multilevel
two-state quantum system with hyperfine (HF) splitting which is essential in observing laser-matter
interaction under experimental conditions. Two theoretical models were independently developed: the
first -based on solving the Schr\"{o}dinger's equation, while the second one- on the Optical Bloch
Equations using the Split Propagation Technique. Both models included the interaction of the laser
fields with all HF- and Zeeman sublevels in the system and produced similar results, which gave us
confidence in the validity of our prediction. Our theoretical treaty revealed a number of nontrivial
results in the manifestation of HF adiabatic states depending on the coupling laser intensity:
(i) At intermediate coupling field Rabi frequencies the HF operator is comparable to the coupling
one, which results in noticeable mixing between different orthogonal adiabatic wave functions, in
particular, between individual sets of bright and dark states. The mixing populates dressed states
which are decoupled from the excitation scheme, e.g. ``gray'' states, and gives rise to a number of
extra AT multiplets.
(ii) At large coupling field strengths, the HF atomic operator is treated perturbatively; the number
of bright and dark states and their explicit representations is found from the classical Morris-
Shore transformation. Multiple dark states are created from the ``grey'' ones, while some ``bright'' AT
components gradually die with the increase of coupling laser power.
Based on analysis of our results, an interesting observation was made: the excitation process to the
final sate is going via a number of partial two-photon pathways with varying intermediate HF levels,
which leads to destructive interference between the respective probability amplitudes and hence to
creation an artificial selection rule $\Delta F \equiv 0$ for the two-photon transition.


\vspace{\baselineskip}
