\title{SLOW, STORED AND STATIONARY LIGHT IN A ONE-DIMENSIONAL ULTRACOLD ATOMIC ENSEMBLE}

\underline{F. Blatt} \index{Blatt F}

{\normalsize{\vspace{-4mm}
\darmstadt

\email frank.blatt@physik.tu-darmstadt.de}}

Engineering strong light-matter interactions at the single-photon level is a key requirement for quantum information processing, or quantum nonlinear optics (QNLO). They can be created, e.g., by implementing electromagnetically induced transparency (EIT) with a Kerr-type nonlinearity in atomic ensembles of optical depth (OD) above 1000. A promising method towards tightly confined matter and light is based on loading atomic ensembles into hollow-core photonic crystal fibers (HCPCFs).
Using HCPCFs filled with a room-temperature gas already led to the demonstration of a broadband quantum memory, EIT, and photon switching. But implementation of narrowband coherent effects as required, e.g., for QNLO is here not possible. Filling laser-cooled atoms from a magneto-optical trap into the fiber is a more promising strategy towards this goal. EIT-based slow light and photon switching was already demonstrated in such a system for an OD$\approx 30$.
Having recently demonstrated the preparation of an ultracold medium of OD$=1000$ inside a HCPCF, we here report on the first demonstration of narrowband EIT, slow light, and light storage at the few-photon level in a highly opaque HCPCF-based medium. Furthermore, we present preliminary results on the generation of stationary light pulses (SLPs), i.e., stopped light pulses with non-vanishing photonic component. This represents a major step towards the goal of implementing QNLO with Kerr nonlinearities in atomic ensembles.

\vspace{\baselineskip}
