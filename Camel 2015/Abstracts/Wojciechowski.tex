\title{MAGNETOMETRY USING COLD ATOMIC ENSAMBLES}

\underline{A. Wojciechowski} \index{Wojciechowski A}

{\normalsize{\vspace{-4mm}
Institute of Physics,
Jagiellonian University,
St. Lojasiewicza 11,
30-348 Krakow, Poland

\email a.wojciechowski@uj.edu.pl}}

We report on our experiments on nonlinear magneto-optical effects in laser-cooled, near-degenerate rubidium samples. After
the loading and cooling, cloud of atoms is released from a Magneto-Optical Trap (MOT). Interaction of atoms with a near-
resonant, linearly polarized light leads to an effective creation of long-lived ground-state Zeeman coherences, which is
observed through the nonlinear Faraday effect or free induction decay signals of the Larmor precession. Our experiments
showed that high rotation angles of a few degrees and coherence lifetimes of a few milliseconds are achieved in mG fields
with cold atoms released from MOT in a simple magnetic shielding. Alternatively, with dc field compensation and without the
shield, we are able to detect changes in the local magnetic field inside our vacuum system after switching the trap off. The
observed signals reveal the contributions from induced eddy currents, as well as, external oscillating fields.
In the next-generation setup, we keep atoms in a far off-resonant optical dipole trap (ODT) inside an improved magnetic
shield. The ODT provides long relaxation time and large on-axis optical depth, which result in the improved sensitivity to the
magnetic field. Moreover, the tight confinement of atoms enables magnetic field probing with a spatial resolution of a few tens
of micrometers.

\vspace{\baselineskip}
