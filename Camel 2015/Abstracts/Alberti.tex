\title{QUANTUM WALKS WITH NEUTRAL ATOMS}

\underline{A. Alberti} \index{Alberti A}

{\normalsize{\vspace{-4mm}
Institut f\"ur Angewandte Physik,
Wegelerstr. 8,
53115 Bonn, Germany

\email alberti@iap.uni-bonn.de}}

The quantum walk is a prime example of quantum transport: a spinor particle is delocalised
over a very large Hilbert space through discrete steps in space and time. We implement this
transport using ultracold atoms moving in a deep optical lattice.

By creating artificial electric fields, we observe textbook transport phenomena like spin
orbit coupling, Bloch oscillations, or Landau-Zener tunnelling in a single experiment. We
unravel the unique character of electric quantum walks by studying very different transport
regimes, which depend on the commensurability of the electric field [1]. We experimentally
observe ballistic delocalization for rational fields and dynamical localization for irrational
ones [2]. Physical insight into the ``quantumness'' of the walk is obtained by an analysis of
decoherence phenomena [3] and application of ideal negative measurements, i.e. interaction-free
measurements [4].

The controlled interaction of exactly two quantum walkers remains a daunting but highly
attractive experimental challenge.

{\normalsize
[1] C. Cedzich, T. Ryb\'{a}r, A. H. Werner, A. Alberti, M. Genske and R. F. Werner, Phys. Rev. Lett. \textbf{111}, 160601 (2013).
\vsp

[2] M. Genske, W. Alt, A. Steffen, A. H. Werner, R. F. Werner, D. Meschede, A. Alberti, Phys. Rev. Lett. \textbf{110}, 190601 (2013).
\vsp

[3] A. Alberti, W. Alt, R. Werner, and D. Meschede, New J. Phys. \textbf{16}, 123052 (2014).
\vsp

[4] C. Robens, W. Alt, D. Meschede, C. Emary, and A. Alberti, Phys. Rev. X \textbf{5}, 011003 (2015).
}

\vspace{\baselineskip}
