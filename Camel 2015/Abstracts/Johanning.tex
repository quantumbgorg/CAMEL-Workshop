\title{3D POSITION SENSING OF IONS AND LONG EQUIDISTANT ION STRINGS}

\underline{M. Johanning} \index{Johanning M}

{\normalsize{\vspace{-4mm}
Universtit\"{a}t Siegen,
Department Physik,
57068 Siegen

\email johanning@physik.uni-siegen.de}}

We implemented 3d position sensing by using a tomographic focus scanning
technique and demonstrate a precision substantially below diffraction
limit. We used position sensing to determine and null electric dc fields
for sensitive and fast micromotion compensation. Another application is
the quantitative verification of vertical shuttling in a 5 wire surface
trap which is achieved by varying the ac component of the middle wire.
Finally we demonstrate the sensitivity of the technique by measuring
forces in the yocto newton range arising from light pressure experienced
by a single atom.

In the second part we discuss the creation of long equidistant linear
strings in linear ion traps which are interesting as a quantum register
for quantum information and simulation applications, as well as for
metrology. We discuss the required axial potential and how it can be
realized in segmented traps, the normal mode spectrum of the crystal and
motional eigenstates, and the effective spin-spin interaction when the
string is exposed to an inhomogeneous magnetic field. We investigate the
radial confinement and normal modes and the transition to an almost
homogeneous zigzag crystal when the radial confinement is reduced below
the critical value.

%{\normalsize
%[1] Ch. Piltz, Th. Sriarunothai, A. Varon, and Ch. Wunderlich, arXiv:1403.8043.
%\vsp

%[2] Ch. Piltz, B. Scharfenberger, A. Khromova, A. F. Varon, and Ch. Wunderlich, Phys. Rev. Lett. {\bf 110}, 200501 (2013).
%}

\vspace{\baselineskip}
