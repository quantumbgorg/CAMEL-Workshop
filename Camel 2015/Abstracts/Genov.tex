\title{PHASE-INSENSITIVE MAPPING OF COHERENCES ONTO LONG-LIVED POPULATIONS}

\underline{G. Genov} \index{Genov G}

{\normalsize{\vspace{-4mm}
\darmstadt

\email genko.genov@physik.tu-darmstadt.de}}

Storage of optical information in quantum systems typically relies on atomic coherences, i.e.
coherent superpositions of states. During storage, the atomic coherences are subject to
decoherence, which destroys the encoded information. Various protocols have been implemented to
reduce the effects that lead to decoherence, e.g. dynamical decoupling. However, these
protocols are usually rather complicated, which leads to difficulties in their implementation.

We theoretically develop a novel phase-insensitive coherence population mapping (CPM) protocol
for storage of coherences well beyond the decoherence time of a system up to the population
relaxation time. The information about the coherence is written in the populations of a three-
state system by a composite sequence of phase shifted pulses requiring only two coupling fields
(or a single field with two different polarizations), which allows for a wide variety of
experimental implementations. After the storage, the full information about the stored
coherence is distributed equally among all three coherences of the system. The retrieval
efficiency is insensitive to the phase of the stored signal unlike other storage protocols,
e.g., the well-known stimulated-photon echo.

\vspace{\baselineskip}
