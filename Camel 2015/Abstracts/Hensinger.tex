\title{QUANTUM INFORMATION PROCESSING WITH LONG-WAVELENGTH RADIATION}

\underline{W. Hensinger} \index{Hensinger W}

{\normalsize{\vspace{-4mm}
Ion Quantum Technology Group, Department of Physics and Astronomy,
University of Sussex, Falmer, Brighton, East Sussex, BN1 9QH, United Kingdom

\email w.k.hensinger@sussex.ac.uk}}

To this point, entanglement operations on ion qubits have predominantly been performed using
lasers. When scaling to larger qubit numbers this will become more challenging due to the
required engineering that might be required. Using long wavelength radiation along with static
magnetic field gradients provides an architecture capable of significantly simplifying the
construction of a large scale quantum computer. Instead of aligning pairs of Raman laser beams
into designated entanglement zones, the use of a single microwave horn outside the vacuum
system is sufficient. Such gate operations are vulnerable to decoherence due to fluctuating
magnetic fields, however the use of microwave-dressed states protects against this source of
noise; with radio-frequency fields being used for qubit manipulation.

I will report the realisation of spin-motion entanglement using long-wavelength radiation. We
have also developed a new method to efficiently prepare dressed state qubits and qutrits,
thereby significantly reducing the experimental complexity of gate operations with dressed-
state qubits. While ground state cooling is not required for quantum gates, it may increase the
entanglement fidelity of two-qubit gates. I will report the demonstration of ground state
cooling  using long wavelength radiation providing an additional tool for microwave quantum
logic with trapped ions. I will also report the demonstration a long-wavelength two-ion quantum
gate using dressed states.

I will present a vision to scale this scheme to build a large scale quantum computer and
present results concerning the development of ion microchips that can be used for this purpose.

\vspace{\baselineskip}
