\title{RYDBERG EXCITATION OF TRAPPED STRONTIUM IONS}

\underline{M. Hennrich} \index{Hennrich M}

{\normalsize{\vspace{-4mm}
Department of Physics,
Stockholm University,
106 91, Stockholm,
Sweden

\email markus.hennrich@uibk.ac.at}}

Trapped Rydberg ions are a novel approach for quantum information processing [1,2]. It joins the sophisticated quantum control of trapped ions with the strong dipolar interaction between Rydberg atoms. For trapped ions this method promises to speed up entangling interactions [3] and to enable such operations in larger ion crystals [4].
In this presentation, we will report on the first realization of trapped strontium Rydberg ions. A single ion was confined in a linear Paul trap and excited to the 26S Rydberg state using a two-photon excitation with 243nm and 308nm laser light. The transitions we observed are narrow (a few MHz linewidth) and the excitation can be performed repeatedly which indicates that the Rydberg ions are stable in the trap. The tunability of the 304-309nm laser enables us to excite our strontium ions to even higher Rydberg levels. Such highly excited levels are required to achieve a strong interaction between neighbouring Rydberg ions in the trap as will be required for quantum gates using the Rydberg interaction.

{\normalsize
[1] M. M\"{u}ller, L. Liang, I. Lesanovsky, P. Zoller, New J. Phys. \textbf{10}, 093009 (2000).
\vsp

[2] F. Schmidt-Kaler, et al., New J. Phys. \textbf{13}, 075014 (2011).
\vsp

[3] W. Li, I. Lesanovsky, Appl. Phys. B \textbf{114}, 37-44 (2014).
\vsp

[4] W. Li, et al., Phys. Rev. A \textbf{87}, 052304 (2013).
}

\vspace{\baselineskip}
