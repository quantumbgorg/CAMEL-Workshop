\title{SPECTROSCOPY OF IONS LOCALIZED IN A PERIODIC OPTICAL POTENTIAL}

\underline{T. Laupretre} \index{Laupretre T}

{\normalsize{\vspace{-4mm}
Institute of Physics and Astronomy,
University of Aarhus,
Ny Munkegade 120,
8000 Aarhus C,
Denmark

\email thomas.laupretre@phys.au.dk}}

We report on the spectroscopy of an ion pinned by the standing-wave field of an optical resonator. In the experiments, a 40Ca$^{+}$ ion, held in a linear Paul trap, is placed in the periodic potential of an optical field close to resonance with the D3/2 - P1/2  transition at 866nm. Pinning of a single ion and multidimensional few-ion Coulomb crystals in such an intracavity optical lattice has been previously demonstrated using scattering from the optical lattice itself as a measure of the ions’ average potential energy. However, such a method assumed a model for the distribution for the ion’s energy in the standing wave field while the lattice depth was inferred indirectly through a measurement of the field intensity at the cavity output. Here, we measure the ion fluorescence spectrum by a near-resonant probe field applied orthogonally to the lattice. The obtained spectrum depends on the position-dependent perturbation of the energy levels by the lattice, and thus contains information about the lattice depth and the ion’s energy distribution inside the periodic potential.
The interplay between Coulomb-interacting particles and optical potentials provides an interesting setting for the simulation of various many-body physics models, such as the Frenkel-Kontorova model of friction. A better understanding of the dynamics of ions in optical potentials is also valuable for ion Coulomb crystal-based cavity QED experiments.

%{\normalsize
%[1] P. A. Ivanov, N. I. Karchev, N. V. Vitanov and D. G. Angelakis,  arXiv:1405.6071.
%}

\vspace{\baselineskip}
