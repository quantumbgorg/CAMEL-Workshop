\title{TRANSPORT QUANTUM LOGIC GATES}

\underline{L. de Clercq} \index{de Clercq L}

{\normalsize{\vspace{-4mm}
Otto-Stern-Weg 1, HPF E12, Switzerland

\email ludwigd@phys.ethz.ch}}

Quantum computation technologies are at the point where scaling up becomes a crucial endeavor.
In our ion-trap quantum computating system we are working towards scaling using the Quantum
``CCD'' architecture [1,2]. We have demonstrated single-qubit logic gates by transporting ions
through static laser beams, allowing the control complexity to be transferred from bulky
acousto-optic devices to the time-dependence of the voltages applied to the trap electrodes
[3]. We explore the scalability of this scheme by simultaneously performing independent gates
on two ions in different zones of our multi-zone ion trap using a retro-reflected laser beam.
By manipulating the transport velocities of the ions we have demonstrated control over
independent single qubit rotations within this scheme. Using laser beam arrangements in which
the first-order Doppler shift plays an important role, we show that we can extract the velocity
trajectory of an ion interacting with a laser beam over time. I will also briefly discuss new
results on fast transport of ions using bang-bang control, allowing the exploration of ion-
laser interactions at the 10 kilo-quanta level of vibrational motion.

{\normalsize
[1] D. Kielpinski et al., Nature \textbf{417}, 709-711 (2002).
\vsp

[2] D. Wineland et. al., J. Res. Natl. Inst. Stand. Technol. \textbf{103}, 259 (1999).
\vsp

[3] D. Leibfried et al., Phys. Rev. A \textbf{76}, 032324 (2007).
}


\vspace{\baselineskip}
