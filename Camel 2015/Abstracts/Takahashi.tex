\title{ION TRAP CAVITY QED EXPERIMENTS AT SUSSEX}

\underline{H. Takahashi} \index{Takahashi H}

{\normalsize{\vspace{-4mm}
Pevensey 2, Falmer,
Brighton, East Sussex BN1 9QH,
United Kingdom 

\email ht74@sussex.ac.uk}}

We are currently working on three distinct ion-cavity QED experiments. In one of them, a cavity collinear to the axis of a linear Paul trap is employed where a moderate ion-photon coupling is expected. The simultaneous couplings of multiple ions to the same cavity mode can be exploited, for example, for probabilistic generation of entanglement. A stronger coupling can be achieved by using a miniature fibre cavity. We have developed a novel endcap-type ion trap which tightly integrates a high finesse fibre cavity inside the electrodes [1]. With strong ion-photon coupling, a deterministic transfer of quantum states between ions and photons becomes possible. Finally, the third trap combines the benefits of the former two by employing a miniature linear trap with a fibre cavity collinear to the trap axis. This configuration allows us to strongly couple single ions in a linear string simultaneously to the cavity mode. This system can be used for ion-photon interface with
  collectively enhanced coupling and cavity cooling of molecular ions.
We will present an overview of these three on-going experiments and their future prospects.

{\normalsize
[1] H. Takahashi et al., New. J. Phys. \textbf{15}, 053011 (2013).
}

\vspace{\baselineskip}
