\title{THIRD HARMONIC MICROSCOPY: NEW INVESTIGATIONS AND APPLICATIONS}

\underline{T. Halfmann} \index{Halfmann T}

{\normalsize{\vspace{-4mm}
\darmstadt

\email thomas.halfmann@physik.tu-darmstadt.de}}

Nonlinear microscopy, driven by ultra-fast laser pulses, is already an established tool with large scientific and commercial relevance. The broad field found numerous applications in physics, chemistry, engineering, biology, and medicine. Examples are imaging using signals from two-photon laser-induced fluorescence (2PLIF), coherent anti-Stokes Raman scattering (CARS), or second (SHG) and third harmonic generation (THG). In contrast to linear microscopy, harmonic generation enables imaging also of otherwise fully transparent media - without the need for markers or staining.

The talk reports on studies of third harmonic emission, driven by tightly focused ultra-short laser pulses at interfaces in solid or liquid samples. We show how the THG intensity profile varies considerably (and quite surprisingly) with the interface orientation [1,2]. This has important consequences, e.g. for imaging of samples with arbitrary surface structure, multi-focus microscopy, or single-shot determination of surface orientations. Moreover, we demonstrate new applications of THM to image commercially relevant micro-fluidic devices (“labs-on-a-chip”). THM enables high-resolution imaging of such usually fully transparent devices and dynamics of liquids therein [3].

{\normalsize
[1] U. Petzold, C. Wenski, A. Romanenko, and T. Halfmann, JOSA B \textbf{30}, 1725 (2013).
\vsp

[2] U. Petzold, A. B\"{u}chel, and T. Halfmann, Opt. Expr. \textbf{20}, 3654 (2012).
\vsp

[3] U. Petzold, A. B\"{u}chel, S. Hardt, and T. Halfmann, Exp. Fluids \textbf{53}, 777 (2012).
}

\vspace{\baselineskip}
