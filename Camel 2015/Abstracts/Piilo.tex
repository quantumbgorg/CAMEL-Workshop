\title{EFFICIENT SUPERDENSE CODING IN THE PRESENCE OF NON-MARKOVIAN NOISE}

\underline{J. Piilo} \index{Piilo J}

{\normalsize{\vspace{-4mm}
Turun yliopisto, 20014 Turku, Finland

\email jyrki.piilo@utu.fi}}

Many quantum information tasks rely on entanglement, which is used as a resource, for example, to enable efficient and
secure communication. Typically, noise, accompanied by loss of entanglement, reduces the efficiency of quantum protocols.
We demonstrate experimentally a superdense coding scheme with noise, where the decrease of entanglement in Alice\\\'s
encoding state does not reduce the efficiency of the information transmission. Having almost fully dephased classical two-
photon polarization state at the time of encoding, we reach values of mutual information close to 1.52 (1.89) with 3-state  (4-state) encoding. This high efficiency relies both on non-Markovian features, that Bob exploits just before his Bell-state
measurement, and on very high visibility (99.6\%) of the Hong-Ou-Mandel interference within the experimental set-up.Our
proof-of-principle results pave the way for exploiting non-Markovianity to improve the efficiency and security of quantum
information processing tasks.

{\normalsize
[1] B.-H. Liu, X.-M. Hu, Y.-F. Huang, C.-F Li, G.-C. Guo, A. Karlsson, E.-M. Laine, S. Maniscalco, C. Macchiavello and J. Piilo,
arXiv:1504.07572.
}

\vspace{\baselineskip}
