\title{GENERATION OF SINGLE PHOTONS WITH HIGHLY TUNABLE WAVEFORM FROM A QUANTUM MEMORY}

\underline{G. Heinze} \index{Heinze G}

{\normalsize{\vspace{-4mm}
ICFO -- The Institute of Photonic Sciences,
Mediterranean Technology Park,
Av. Carl Friedrich Gauss, 3
08860 Castelldefels (Barcelona), Spain

\email georg.heinze@icfo.es}}

A vast range of experiments in quantum information science and technology rely on single photons as carrier of information. Single photon sources are thus key components, which have been continuously improved over the past years. An important parameter of such a source is the temporal shape or the spectral characteristics of the emitted photons [1-4]. The generation of ultra-long single photons is for example an essential requirement for precise interactions with media of spectrally sharp structure like trapped atoms, ions or doped solids, which have been proposed as quantum memories for light.
We report on a photon source with highly tunable photon shape based on a cold ensemble of Rubidium atoms. We follow the DLCZ protocol [5] to implement an emissive-type of quantum memory, which can be operated as a photon pair source with controllable delay. By varying the waveform of the driving read-pulse, we find that the emitted photon follows the driving shape with a high degree of precision. We generate photons with temporal durations varying over three orders of magnitude up to 10 $\mu$s without a significant change of the read-out efficiency. We prove the non-classicality of the emitted photons by measuring their antibunching, showing near single photon behavior at low excitation probabilities. We also show that the photons are emitted in a pure state by measuring unconditional autocorrelation functions. Finally, to demonstrate the usability of the source for realistic applications, we create ultra-long single photons with a rising exponential shape which is preferable for efficient photon storage.

{\normalsize
[1] M. D. Eisaman, L. Childress, A. Andr\'e et al., Phys. Rev. Lett. \textbf{93}, 233602 (2004).
\vsp

[2] M. Almendros, J. Huwer, N. Piro et al., Phys. Rev. Lett. \textbf{103}, 213601 (2009).
\vsp

[3] P. B. R. Nisbet-Jones et al., New J. Phys. \textbf{13}, 103036 (2011).
\vsp

[4] L. Zhao, X. Guo, C. Liu, Y. Sun, M. M. T. Loy, and S. Du, Optica \textbf{1}, 84 (2014).
\vsp

[5] L. M. Duan, M. D. Lukin, J. I. Cirac, and P. Zoller, Nature \textbf{414}, 413 (2001).
}

\vspace{\baselineskip}
