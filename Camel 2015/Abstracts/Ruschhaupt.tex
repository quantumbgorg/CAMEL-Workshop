\title{SHORTCUTS TO ADIABATICITY: OPTIMISATION OF A NOISY ION TRANSPORT}

\underline{A. Ruschhaupt} \index{Ruschhaupt A}

{\normalsize{\vspace{-4mm}
Department of Physics,
University College Cork,
Cork,
Ireland

\email aruschhaupt@ucc.ie}}

Shortcuts to Adiabaticity are mostly-analytical derived schemes for the fast and robust control of quantum systems. We apply these tools to design a trap trajectory for shuttling a single ion in a harmonic trap. We theoretically investigate the motional excitation of the ion caused by spring-constant noise and position noise fluctuations of the harmonic trap during the trap shuttling process. We derive trap trajectories with low sensitivity on these noises for several noise spectra. This is done by combining invariant-based inverse engineering, perturbation theory, and optimal control. The effect of spring-constant drifts of the trap is also analysed, even in the case of two ions of different mass. We again design transport protocols to suppress or mitigate the final excitation energy.

{\normalsize
[1] E. Torrontegui, S. Ibanez, S. Martinez-Garaot, M. Modugno, A. del Campo, D. Guery-Odelin, A. Ruschhaupt, X. Chen, and J. G. Muga, Adv. At. Mol. Opt. Phys. \textbf{62}, 117 (2013).
\vsp

[2] X-J Lu, J. G. Muga, X. Chen, U. G. Poschinger, F. Schmidt-Kaler, and A. Ruschhaupt, Phys. Rev. A \textbf{89}, 063414 (2014);
X-J Lu, M. Palmero, A. Ruschhaupt, X. Chen and J. G. Muga, Phys. Scr. (2015), in print.
}

\vspace{\baselineskip}
