\title{MICROFABRICATED ION TRAPS FOR MICROWAVE QUANTUM CONTROL AND METROLOGY}
%Microfabricated ion traps for microwave quantum control and metrology

\underline{A. Bautista-Salvador} \index{Bautista-Salvador A}

{\normalsize{\vspace{-4mm}
Physikalisch-Technische Bundesanstalt,
Bundesallee 100,
D-38116 Braunschweig

\email amado.bautista@ptb.de}}

In quantum technologies based on trapped atomic ions, internal and motional degrees of freedom are routinely manipulated using laser light. Scaling laser-based qubit operations to a large number of ions, however, remains a challenging task. In the first part of this talk, we describe experiments carried out in a surface-electrode trap with integrated microwave conductors. The optimized trap features the necessary set of conductors to induce single and multi-qubit quantum operations while keeping a low residual field and high magnetic field gradient. Beryllium ions are loaded into the trap via photoionization and a ``single-push-button'' loading scheme using a pulsed laser. At 22.3 $\mathrm{mT}$, where the $\left|F=2,m_F=1\right>$ and $\left|F=1,m_F=1\right>$ states form a first-order field-independent qubit for long coherence times, we demonstrate initialization and control of the qubit. We develop a model of 2D microwave near-fields parametrized by 5 parameters characterizing the strength of the zero and first order terms in a field expansion, their spatial orientation and the field polarization. We perform a spatial characterization of the microwave near-field with a single ion as a local field probe and find good agreement with full-wave numerical simulations within our 2D near-field model. Using the spatial variation of the microwave near-field, we demonstrate motional sideband transitions and motional-sideband ground state cooling on our field-independent qubit. This is the essential prerequisite towards entangling multi-qubit quantum logic gates. We show first steps towards the realization of a multi-layer trap structure with integrated microwave control for scalability.

In the second part, we describe a related project in which we are developing laser-based, quantum logic inspired cooling and detection techniques for precision experiments with single (anti-)protons. The ultimate application is a test of CPT invariance by comparing the proton and antiproton magnetic moment within the BASE collaboration. We present first steps towards the realization of a microfabricated Penning trap stack allowing the realization of a double-well potential for coupling a single (anti-)proton to a single $^9$Be$^+$ ion. The stack is made of silicon wafers structured using deep reactive ion etching (DRIE). In this experiment, the large magnetic field of the Penning trap ($\approx 5$ T) leads to a large ground state energy splitting of order 140 GHz. We present a stimulated-Raman pulsed laser system allowing us to bridge this frequency splitting using pairs of comb teeth of the mode-locked laser. We show first results on stimulated-Raman transitions in $^9$Be$ +$ transitions using this novel laser source, implemented at low magnetic field for demonstration purposes.

%{\normalsize
%[1] B. Rousseaux, D. Dzsotjan, G. Colas des Francs, H. R. Jauslin, C. Couteau, and S. Gu\'{e}rin, Phys. Rev. B \textbf{93}, 045422 (2016).
%\vsp
%
%[2] F. Lindenfelser et al., Rev. Sci. Instrum. \textbf{86}, 033107 (2015).
%\vsp
%
%[3] R. J. Hendricks et al., Phys. Rev. A \textbf{77}, 021401(R) (2008).
%}



\vspace{\baselineskip} 