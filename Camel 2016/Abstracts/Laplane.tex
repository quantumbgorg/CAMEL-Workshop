\title{GENERATION OF PHOTON PAIRS WITH CONTROLLABLE DELAY USING A CRYSTAL}
%Generation of photon pairs with controllable delay using a crystal

\underline{C. Laplane} \index{Laplane C}

{\normalsize{\vspace{-4mm}
Universit\'{e} de Gen\`{e}ve,
GAP-Optique,
Chemin de Pinchat 22,
CH-1211 Gen\`{e}ve 4

\email cyril.laplane@unige.ch}}

One of the main challenges in quantum information science is the distribution of entanglement over long distances [1]. For quantum communication or distributed quantum computing, it requires the implementation of quantum memories. Such devices could serve as a light-matter interface between computing nodes and communication channel [2]. Moreover it is an essential building block of quantum repeaters [3].

Ensemble-based quantum memories couple strongly to light due to their intrinsic collective enhancement effect. Rare-earth ions doped crystals are promising candidates for quantum memories [4] thanks to their long coherence times at cryogenic temperature. They present atomic-like properties while being free of motional decoherence.
It is still a challenge to interface efficiently entangled photon pair source and ensemble based quantum memories. One solution proposed by Duan, Lukin, Cirac and Zoller [5] is to use the atomic ensemble as the source of photon pairs.

We have chosen to implement a modified version of this approach, the AFC-DLCZ [6]. Using a Europium doped Y$_2$SiO$_5$ crystal we are able to generate photon pairs with a controllable delay up to a few milliseconds thanks to spin-wave storage and coherent spin control [7]. Furthermore, the multimode capacity of the AFC scheme allows us to temporally multiplex the generation of photon pairs. Such a device would constitute a solid-state quantum network node capable of multiplexing and long-duration storage.

{\normalsize
[1] N. Gisin et al., Nat. Photon. \textbf{1}, 165–71 (2007).
\vsp

[2] H. J. Kimble, Nature \textbf{453}, 1023–30 (2008).
\vsp

[3] N. Sangouard et al., Rev. Mod. Phys. \textbf{83}, 33 (2011).
\vsp

[4] W. Tittel et al., Laser Photon. Rev. \textbf{4}, 244 (2010).
\vsp

[5] L. M. Duan et al., Nature \textbf{414}, 413 (2001).
\vsp

[6] P. Sekatski et al., Phys. Rev. A \textbf{83}, 053840 (2011).
\vsp

[7] P. Jobez et al., Phys. Rev. Lett. \textbf{114}, 230502 (2015).
}



\vspace{\baselineskip} 