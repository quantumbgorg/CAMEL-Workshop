\title{TOWARDS NOVEL QUANTUM MECHANICS APPLICATION: ENHANCED SENSORS AND SECURE MONEY}
%Towards novel quantum mechanics application: enhanced sensors and secure money

\underline{J. Bernardoff} \index{Bernardoff J}

{\normalsize{\vspace{-4mm}
Dagobertstr. 4,
55116 Mainz, Germany

%\vspace{-4mm} $^{2}$ \unisofia

\email josselin.bernardoff@gmail.com}}

Control of quantum information in different systems has achieved enough progress to be now used in
technology to improve on classical limits (on sensitivity, security, etc.). Following this idea, one
system that seems within reach is the quantum memory and its applications, for which prepared
quantum bits just need keep their integrity for a time significant at the human scale, say hours,
without being manipulated.

Color centers in diamond -- more specifically Nitrogen-Vacancy centers -- interfaced to nearby nuclear
spins have exceptional properties and coherence time (T2*) of seconds has been reached at room
temperature, with a measured potential lifetime over a day. In that regime, quantum money or
signature schemes become possible. Interest for this research spans from the fundamental aspects of
quantum cryptography that aims at designing maximally secure quantum money protocols to the
possibility of building a compact practical device.

With very similar tools, sensors operating in the quantum regime can also be developed: a huge
field of sensing with high sensitivity, resolution or compactness has emerged that will be a
parallel research direction in the group.

\vspace{\baselineskip}
