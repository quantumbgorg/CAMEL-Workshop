\title{PRECISION SPECTROSCOPY IN ION COULOMB CRYSTALS}
%Precision spectroscopy in Ion Coulomb Crystals

\underline{T. Mehlst\"{a}ubler} \index{Mehlst\"{a}ubler T}

{\normalsize{\vspace{-4mm}
QUEST-Institute at PTB,
Bundesallee 100,
38116 Braunschweig

\email tanja.mehlstaeubler@ptb.de}}

In order to exploit their full potential and to resolve frequencies with a fractional frequency instability of 10$^{-18}$ and below, optical ion clocks need to integrate over many days to weeks. For the characterisation of systematic shifts of the clock, as well as for applications, such as relativistic geodesy, the long averaging time scales put up fundamental limits. Scaling up the number of ions for optical clock spectroscopy is a natural way to significantly reduce integration times and to relax the requirements on clock laser stability, but was hindered so far by the poor control of the dynamics of coupled many body systems, on-axis micromotion and systematic shifts due to interacting ions.

In our experiment we implement linear chains of Yb$^+$ and In$^+$ ions for a first evaluation of optical clock operation. For optimal control of the ion motion, scalable high-precision ion traps are engineered in the clean room facilities of PTB. To reduce systematic shifts due to blackbody radiation, the trap is made of AlN ceramics chips. The new ion trap with minimized axial micromotion allows us to trap and cool large ion Coulomb crystals and study many-body physics with trapped ions. The realization of topological defects in 2D crystals opens up a new research field of non-equilibrium dynamics and nonlinear physics in ion Coulomb crystals and using them as quantum simulators of well-controlled solid-state systems.

%{\normalsize
%[1] J. P. Home et al., Science \textbf{325}, 1227 (2009).
%\vsp
%
%[2] F. Lindenfelser et al., Rev. Sci. Instrum. \textbf{86}, 033107 (2015).
%\vsp
%
%[3] R. J. Hendricks et al., Phys. Rev. A \textbf{77}, 021401(R) (2008).
%}



\vspace{\baselineskip} 