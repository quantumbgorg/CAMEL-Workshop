\title{HIGH OPTICAL ACCESS ION TRAPS USING PCF TECHNOLOGY AND ALL IR LASER COOLING OF CALCIUM IONS}
%High optical access ion traps using PCF technology and all IR laser cooling of calcium ions

\underline{F. Lindenfelser} \index{Lindenfelser F}

{\normalsize{\vspace{-4mm}
ETH Z\"{u}rich IQE TIQI,
Otto-Stern-Weg 1, HPF E14,
8093 Z\"{u}rich, Switzerland

\email friederl@phys.ethz.ch}}

Trapped ions can be used as high fidelity qubits and using micro-scale electrode-structures many ions can be combined and potentially used as a quantum processor or simulator [1]. One challenge is to achieve high optical access to single trapping sites. To address this we are developing ion traps based on photonic crystal fibre (PCF) technology. This fabrication method holds the promise of simultaneous optical and electrical alignment since the trapping structures themselves could be used to guide light. I will describe our first PCF-cane based surface-electrode ion trap, which does not guide light but offers high optical access [2]. We have used this to perform all IR-laser cooling and control of calcium ions by using a tightly focused laser beam to strongly drive the ion's dipole forbidden 4S to 3D transition [3].
This minimizes the possibility for stray charges built up on dielectrics of the trapping structure due to the presence of UV laser light, and is of interest in the further miniaturization of ion traps. We demonstrate cooling throughout a range of conditions from the weak-binding (Doppler) to the strong-binding (sideband) limits. I will also briefly discuss our research on scaling up the trapped-ion quantum processor using fast transport of ions and by means of trapping in optical lattices.

{\normalsize
[1] J. P. Home et al., Science \textbf{325}, 1227 (2009).
\vsp

[2] F. Lindenfelser et al., Rev. Sci. Instrum. \textbf{86}, 033107 (2015).
\vsp

[3] R. J. Hendricks et al., Phys. Rev. A \textbf{77}, 021401(R) (2008).
}



\vspace{\baselineskip} 