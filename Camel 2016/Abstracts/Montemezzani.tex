\title{ADIABATIC PASSAGE OF LIGHT IN COUPLED WAVEGUIDES WITH LONGITUDINALLY MODULATED DETUNING}
%Adiabatic passage of light in coupled waveguides with longitudinally modulated detuning

\underline{G. Montemezzani} \index{Montemezzani G}

{\normalsize{\vspace{-4mm}
Laboratoire Mat\'eriaux Optiques, Photonique et Syst\`emes (LMOPS),
Universit\'e de Lorraine et Sup\'elec 2, rue E. Belin
57070 METZ (France)

\email germano.montemezzani@univ-lorraine.fr}}

The evanescent coupling of light waves between waveguides can be described in the framework of the coupled wave theory, that present a formal analogy with the Schr\"{o}din-ger equation describing the quantum population dynamics among coupled quantum levels in the rotating wave approximation. While these analogies are inspiring an increasing interest for fundamental studies, they also lead to several potential applications in integrated optics. Adiabatic passage of light can be achieved for instance in configurations inspired by the Stimulated Raman Adiabatic Passage (STIRAP) process by means of a longitudinal modulation of the coupling constants between identical waveguides which is controlled by their relative distances. If the waveguides differ from each other, the relative detuning of their longitudinal propagation constants provide an additional control parameter being analogous to the detuning of the transition between quantum levels. We will focus mainly on the description of waveguide systems presenting a longitudinal modulation of the detuning as well as the coupling constant being analogous to the Rapid Adiabatic Passage (RAP) or the two-state STIRAP process. While the RAP-like process leads to an adiabatic passage of light equivalent to a broadband directional coupler, the two-state STIRAP leads to achromatic beam splitting. Experiments are performed using reconfigurable waveguides induced by a properly shaped lateral control illumination of a ferroelectric SrxBa1-xNb2O6 exhibiting the photorefractive effect.

%{\normalsize
%[1] Suchowski et al., Phys. Rev. A. \textbf{78}, 063821 (2008).
%}

\vspace{\baselineskip} 