\title{ENHANCEMENT OF HIGH HARMONIC GENERATION USING TWO-COLOUR LASER FIELDS}
%Enhancement of high harmonic generation using two-colour laser fields

\underline{B. Bruner} \index{Bruner B}

{\normalsize{\vspace{-4mm}
Department of Physics of Complex Systems,
Weizmann Building Rm. 250,
Weizmann Institute of Science,
Rehovot 76100, Israel

\email barry.bruner@weizmann.ac.il}}

Advancements in high harmonic generation (HHG) have led to the development of tabletop XUV and soft x-ray light sources for attosecond science.  However, the low conversion efficiency from the strong driving laser fields to short wavelength HHG light poses a significant practical limitation for the use of these sources in experimental applications.  We show that a two-colour driving field produces a considerable enhancement of the ionization rate compared to that a single-colour field, leading to huge increases in the HHG efficiency.  We use a tunable mid-IR (1300-1600 nm) source as a driving field and a weaker 800 nm beam as an assisting field.  By systematically varying the field parameters we can observe increases in HHG efficiency of over two orders of magnitude.  All experiments were performed using gas jets with very short interaction lengths ($<$ 1 mm), suggesting that it is the ionization enhancement, and not extended phase matching effects, that play the dominant role in the process. The effect is not specific to a particular laser wavelength of the driving field or any specific atomic or molecular system.  The robustness of this approach makes great strides toward improving the simplicity and practicality of high flux HHG sources.

%{\normalsize
%[1] Suchowski et al., Phys. Rev. A. \textbf{78}, 063821 (2008).
%}

\vspace{\baselineskip} 