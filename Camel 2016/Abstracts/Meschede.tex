\title{FROM CONTROLLED QUANTUM PROCESSES TO QUANTUM TECHNOLOGY}
%From controlled quantum processes to quantum technology

\underline{D. Meschede} \index{Meschede D}

{\normalsize{\vspace{-4mm}
Institut f\"{u}r Angewandte Physik,
Wegeler Str. 8,
53115 Bonn,
Germany

\email meschede@uni-bonn.de}}

With quantum optical experiments –- interacting isolated atoms and photons -- the feasibility of controlled quantum processes has been demonstrated typically for systems with a small number of degrees of freedom. I will present examples (two atoms radiating jointly, a Leggett-Garg experiment disproving the role of classical trajectories, simulation of electric fields) which illustrate the role of important and fundamental physical concepts such as the quantum superposition principle.
An interesting question is whether we can put forward our experimental findings to obtain significance for technological developments. I will discuss aspects of this question.

%{\normalsize
%[1] Suchowski et al., Phys. Rev. A. \textbf{78}, 063821 (2008).
%}

\vspace{\baselineskip} 