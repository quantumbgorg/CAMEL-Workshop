\title{LASER-COOLED IONS IN MULTIPOLE TRAPS}
%Laser-cooled ions in multipole traps

\underline{M. Knoop} \index{Knoop M}

{\normalsize{\vspace{-4mm}
CNRS/Aix-Marseille University

\email martina.knoop@univ-amu.fr}}

Transporting charged particles between different traps has become an important feature in high-precision spectroscopy experiments to take advantage of different experimental environments. In our double linear radio-frequency trap, we have implemented a fast protocol allowing to shuttle large ion clouds very efficiently between traps, in times shorter than a millisecond. Moreover, our shuttling protocol is a  one-way process, allowing to add ions to an existing cloud without loss of the already trapped sample. This feature makes  accumulation possible, resulting in the creation of large ion clouds. Experimental results show, that ion clouds of large size are reached with laser-cooling, however, the described mechanism does not rely on any cooling process.

Our experimental set-up is composed of 3 linear traps, one of them with a multipole configuration. Numerical simulations show that the density distribution of a large cloud should be hollow in the centre, and that these features become steeper for colder ions. I will present images of laser-cooled ions in the octupole trap, which show that defects play a major role.

%{\normalsize
%[1] J. Struck, C. \ddot{O}lschläger, M. Weinberg, P. Hauke, J. Simonet, A. Eckardt, M. Lewenstein, K.
%Sengstock, and P. Windpassinger, Phys. Rev. Lett. \textbf{108}, 225304 (2012).
%\vsp
%
%[2] K. Jim\acute{e}nez-Garcia, L. J. LeBlanc, R. A. Williams, M. C. Beeler, A. R. Perry, and I. B.
%Spielman, Phys. Rev. Lett. \textbf{108}, 225303 (2012).
%\vsp
%
%[3] J. Polo, J. Mompart and V. Ahufinger, Phys. Rev. A \textbf{93}, 033613 (2016).
%}



\vspace{\baselineskip} 