\title{CONSTRUCTING A LONG-WAVELENGTH RADIATION TRAPPED-ION QUANTUM COMPUTER}
%Constructing a long-wavelength radiation trapped-ion quantum computer

\underline{W. Hensinger} \index{Hensinger W}

{\normalsize{\vspace{-4mm}
Ion Quantum Technology Group,
Department of Physics and Astronomy,
University of Sussex,
Falmer, Brighton, East Sussex, BN1 9QH,
United Kingdom

\email w.k.hensinger@sussex.ac.uk}}

Trapped ions are a promising tool for building a large-scale quantum computer. The number of required radiation fields (such as lasers) for the realisation of quantum gates in any proposed ion-based architecture scales with the number of ions inside the quantum computer, posing a major challenge when imagining a device with millions of qubits. Here I present a fundamentally different approach, where this scaling entirely vanishes. The method is based on individually controlled voltages applied to each logic gate location to facilitate the actual gate operation analogous to a traditional transistor architecture within a classical computer processor. Instead of aligning numerous pairs of Raman laser beams into designated entanglement zones, the use of a single microwave source outside the vacuum system is sufficient. We have demonstrated the key principle of this approach by implementing a two-qubit quantum gate based on long-wavelength radiation where we generate a maximally entangled two-qubit state with fidelity 0.985(12). Finally, I will discuss the engineering blueprint for a large-scale microwave trapped-ion quantum computer and the construction of a demonstrator device at the University of Sussex.

%{\normalsize
%[1] J. Struck, C. \ddot{O}lschläger, M. Weinberg, P. Hauke, J. Simonet, A. Eckardt, M. Lewenstein, K.
%Sengstock, and P. Windpassinger, Phys. Rev. Lett. \textbf{108}, 225304 (2012).
%\vsp
%
%[2] K. Jim\acute{e}nez-Garcia, L. J. LeBlanc, R. A. Williams, M. C. Beeler, A. R. Perry, and I. B.
%Spielman, Phys. Rev. Lett. \textbf{108}, 225303 (2012).
%\vsp
%
%[3] J. Polo, J. Mompart and V. Ahufinger, Phys. Rev. A \textbf{93}, 033613 (2016).
%}



\vspace{\baselineskip} 