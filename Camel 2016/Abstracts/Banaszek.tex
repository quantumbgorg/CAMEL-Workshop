\title{QUANTUM METROLOGY GETS REAL}
%Quantum metrology gets real

\underline{K. Banaszek} \index{Banaszek K}

{\normalsize{\vspace{-4mm}
Faculty of Physics, University of Warsaw, Pasteura 5, Poland

\email Konrad.Banaszek@fuw.edu.pl}}

Quantum physics holds the promise of enhanced performance in metrology and sensing by exploiting non-classical
phenomena such as multiparticle interference. Specific designs for quantum-enhanced schemes
need to take into account noise and imperfections present in real-life implementations. This talk will
review selected recent results in realistic quantum metrology, starting from interferometric phase
estimation with common impairments, such as photon loss, and ending with general scaling laws implied
by the geometry of quantum channels. In many situations, although qualitatively improved asymptotic
scaling of ideal noise-free protocols is lost, quantum physics can usefully offer performance beyond
the standard shot noise limit. As a concrete example, a comparison of the fundamental quantum
interferometry bound with the recently achieved sensitivity of the squeezed-light-enhanced GEO600
gravitational wave detector indicates its nearly optimal operation given the present amount of optical
loss. Finally, the potential of mode-engineering techniques exploiting multiple degrees of freedom to
alleviate deleterious effects induced, e.g. by residual distinguishability in multiphoton interference
is highlighted.

%{\normalsize
%[1] J. P. Home et al., Science \textbf{325}, 1227 (2009).
%\vsp
%
%[2] F. Lindenfelser et al., Rev. Sci. Instrum. \textbf{86}, 033107 (2015).
%\vsp
%
%[3] R. J. Hendricks et al., Phys. Rev. A \textbf{77}, 021401(R) (2008).
%}



\vspace{\baselineskip} 