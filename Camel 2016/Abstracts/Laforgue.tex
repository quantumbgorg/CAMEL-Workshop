\title{ROBUST NOT-GATE BY SINGLE SHOT SHAPED PULSES}
%Robust not-gate by single shot shaped pulses

\underline{X. Laforgue} \index{Laforgue X}

{\normalsize{\vspace{-4mm}
\dijon
\vspace{-4mm}

\darmstadt

\email xavier.laforgue@physik.tu-darmstadt.de}}

Coherent control of quantum states is a crucial requirement for applications in quantum information technology, e.g., to implement logical gates. This typically requires manipulation of population distributions in a quantum system, e.g., to invert the populations of two states. Standard short $\pi$-pulse interactions are a conventional choice here. However, they feature low robustness with regard to fluctuations of experimental parameters, which hampers applications. On the other hand, adiabatic passage processes provide robustness, but require long interaction times or high intensities.

In order to combine the advantages of both techniques, a number of different proposals have been made, e.g., composite pulses, parallel adiabatic passage, and shortcuts to adiabaticity. The latter also includes the concept of single-shot shaped pulses (SSSP)[1]. These usually aim at achieving high population inversion efficiencies. They provide an analytic approach in comparison to purely numerical techniques from optimal control.

In this contribution, we present a novel SSSP protocol to implement a fast and robust quantum NOT-gate. We combine the advantages of analytical and numerical techniques by hybridizing the analytic derivation of SSSP with state-of-the art numerical optimization. This serves to derive the temporal evolution of the control field. The applicability, efficiency and robustness of the pulses, derived by this hybrid control technique, are experimentally demonstrated for rephasing of atomic coherences in a $\mathrm{Pr}^{3+}:\mathrm{Y}_2\mathrm{SiO}_5$ crystal.

{\normalsize
[1] D. Daems, A. Ruschhaupt, D. Sugny, and S. Gu\'erin, Phys. Rev. Lett. \textbf{111}, 050404 (2013).
}

\vspace{\baselineskip} 