\title{DIPOLAR QUANTUM GASES, MODULATED INTERACTIONS, AND IMPURITY TRANSPORT}
%Dipolar quantum gases, modulated interactions, and impurity transport

\underline{H. N\"{a}gerl} \index{N\"{a}gerl H}

{\normalsize{\vspace{-4mm}
Institut f\"{u}r Experimentalphysik,
Universit\"{a}t Innsbruck,
Technikerstrasse 25/4,
A-6020 Innsbruck

\email christoph.naegerl@uibk.ac.at}}

Ultracold dipolar systems are of high interest for quantum many-body physics and quantum simulation. Our goal is to prepare dipolar RbCs molecules [1] in an optical lattice at unity filling. Atomic Rb and Cs samples are mixed in the lattice to efficiently form Rb-Cs atom pairs as precursors to ground-state molecules. We go through the superfluid-to-Mott-insulator phase transition twice, first for Cs, then for Rb on top of Cs to form Rb-Cs atom pairs. We estimate the filling fraction to be above 30% in the center of our trap [2].
I will then turn to experiments in the context of driven quantum many-body systems. We demonstrate Floquet engineering of a Hubbard model featuring occupation-dependent tunneling via periodically modulated interactions [3] and provide evidence for Bloch-type oscillations for an impurity traveling through a fermionized Bose gas that apparently acts as a lattice [4].


{\normalsize
[1] T. Takekoshi et al., Phys. Rev. Lett. \textbf{113}, 205301 (2014).
\vsp

[2] L. Reichs\"{o}llner et al., in preparation.
\vsp

[3] F. Meinert et al., arXiv:1602.02657 (2016).
\vsp

[4] F. Meinert et al., in preparation.
}



\vspace{\baselineskip} 