\title{QUANTUM OPTICS AND CONTROL WITH SURFACE PLASMON POLARITONS AT THE NANOSCALE}
%Quantum optics and control with surface plasmon polaritons at the nanoscale

\underline{S. Gu\'{e}rin} \index{Gu\'{e}rin S}

{\normalsize{\vspace{-4mm}
\dijon

\email sguerin@u-bourgogne.fr}}

Controlling quantum emitters (atoms, molecules, quantum dots, etc.), light and its interactions is a key issue for implementing all-optical
devices and information processing at the quantum level. This generally necessitates a strong coupling of emitters to photonic
modes, as achieved by a high Q-cavity.

A nanoscale plasmonic platform has been envisioned to transpose strong coupling of quantum optics to plasmonics which takes
advantage of the strong mode confinement of surface plasmon polaritons (SPP). Recent progress in quantum plasmonics showed the
possibility of quantum emitters to reach the strong coupling regime to SPP fields. However, its application appears notoriously
limited in practical situations due to the intrinsic presence of numerous and lossy modes, which complicates the description and the
interpretation of the interaction, and introduces strong decoherence in the system. These drawbacks are known to limit severely the
practical use of quantum plasmons for coherent manipulation of quantum emitters at the nanoscale.

In this presentation, we show how to solve these two issues [1]. First, we derive an effective model from the fully quantized
description allowing one to interpret the complete plasmonic coupling as a multimode lossy cQED interaction. The potential of our
model is demonstrated in a second step: We show that we can engineer a specific coherent manipulation, where externally
manipulating a pair of emitters coupled by SPPs, we make full use of the strong coupling while circumventing plasmonic losses via
the use of unpopulated plasmonic modes. This is achieved by adapting the technique of stimulated Raman adiabatic passage
(STIRAP), known to lead to an efficient and robust transfer of population between two metastable states commonly coupled to a lossy
excited state, via a dark state immune to loss, thanks to adiabaticity usually reached for relatively modest durations and energies of
the external control pulses. Here the STIRAP technique and its variant, the fractional STIRAP, are applied between the multi-emitter
states driven by external fields: They allow the robust control of a complete or partial population transfer between two emitters,
leading in particular to their high-fidelity entanglement.

{\normalsize
[1] B. Rousseaux, D. Dzsotjan, G. Colas des Francs, H. R. Jauslin, C. Couteau, and S. Gu\'{e}rin, Phys. Rev. B \textbf{93}, 045422 (2016).
%\vsp
%
%[2] F. Lindenfelser et al., Rev. Sci. Instrum. \textbf{86}, 033107 (2015).
%\vsp
%
%[3] R. J. Hendricks et al., Phys. Rev. A \textbf{77}, 021401(R) (2008).
}



\vspace{\baselineskip} 