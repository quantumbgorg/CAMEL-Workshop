\title{STOPPED LIGHT AT HIGH STORAGE EFFICIENCY IN A Pr$^{3+}$:Y$_2$SiO$_5$ CRYSTAL}
%Stopped light at high storage efficiency in a Pr$^{3+}$:Y$_2$SiO$_5$ crystal

\underline{M. Hain} \index{Hain M}

{\normalsize{\vspace{-4mm}
\darmstadt

\email marcel.hain@physik.tu-darmstadt.de}}

Storage of photonic qubits typically requires atomic media (i.e. quantum memories) of large optical depth -- no matter which light storage protocol is applied. The larger the optical depth, the larger the storage and retrieval efficiency. Solid media, such as rare-earth ion-doped crystals, usually have a rather low optical depth leading to rather low efficiencies. Nevertheless, they are easy to handle, integrate and scale and permit very long storage times. As an example, in our recent work on light storage by electromagnetically induced transparency (EIT) in a Pr$^{3+}$:Y$_2$SiO$_5$ crystal we obtained ultra-long storage times up to one minute, but at efficiencies in the regime of 1\% \mbox{only [1].}

We now implemented a ring-type multipass setup to increase the efficiency of ``stopped light'' in the solid-state memory. The application of control techniques (e.g. dynamical decoupling, ZEFOZ) to prolong the storage time remains possible in the multipass geometry. Using our setup we obtained an optical depth up to $\sim 96$ and storage efficiencies for classical light pulses up to 76\% [2]. This exhibits the largest value reported so far for solid state systems and protocols related to EIT.

The talk discusses the technical implementation of the multipass configuration and optimization procedures for efficient light storage by EIT.

{\normalsize
[1] G. Heinze, C. Hubrich, and T. Halfmann, Phys. Rev. Lett. \textbf{111}, 033601 (2013).
\vsp

[2] D. Schraft, M. Hain, N. Lorenz, and T. Halfmann, Phys. Rev. Lett. \textbf{116}, 073602 (2016).
}

\vspace{\baselineskip} 