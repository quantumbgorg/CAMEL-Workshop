\title{SINGLE-ION THERMAL MACHINES}
%Single-ion thermal machines

\underline{D. Crivelli} \index{Crivelli D}

{\normalsize{\vspace{-4mm}
Universit{\"a}t Kassel,
FB10 -- Institut f{\"u}r Physik,
Heinrich-Plett-Stra{\ss}e 40,
34132 Kassel

\email dawid.crivelli@gmail.com}}

Thermodynamic machines can be reduced to the ultimate atomic limit [1], using a single ion as a working agent. The confinement in a linear Paul trap with tapered geometry allows for coupling axial and radial modes of oscillation.
The heat-engine is driven thermally by coupling it alternately to hot and cold reservoirs, using the output power of the engine to drive a harmonic oscillation [2].
From direct measurements of the ion dynamics, the thermodynamic cycles for various temperature differences of the reservoirs can be determined [3] and the efficiency compared with analytical estimates.
The mechanism can be inverted to act as a heat-pump, transferring heat along the axial direction from the cold to the hot reservoir, under external electric field driving.

{\normalsize
[1] J. Ro{\ss}nagel, S. T. Dawkins, K. N. Tolazzi, O. Abah, E. Lutz, F. Schmidt-Kaler, and K. Singer, Science \textbf{352}, Issue 6283, pp. 325-329.
\vsp

[2] O. Abah, J. Ro{\ss}nagel, G. Jacob, S. Deffner, F. Schmidt-Kaler, K. Singer, and E. Lutz, Phys. Rev. Lett. \textbf{109}, 203006 (2012).
\vsp

[3] J. Ro{\ss}nagel et al., New J. Phys. \textbf{17}, 045004 (2015).
}



\vspace{\baselineskip} 