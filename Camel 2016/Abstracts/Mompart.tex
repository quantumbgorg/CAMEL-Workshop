\title{GEOMETRICALLY INDUCED COMPLEX TUNNELINGS FOR ULTRACOLD ATOMS CARRYING OAM}
%Geometrically induced complex tunnelings for ultracold atoms carrying OAM

\underline{J. Mompart} \index{Mompart J}

{\normalsize{\vspace{-4mm}
Universitat Aut\`{o}noma de Barcelona, E-08193, Bellaterra, Spain

\email jordi.mompart@uab.cat}}

Controllable complex tunneling amplitudes for ultracold atoms have been induced in one
dimensional (1D) optical lattices either by a suitable forcing of the optical lattice [1] or by
a combination of radio frequency and optical Raman coupling fields [2]. We show [3] that
complex tunneling amplitudes appear naturally in the dynamics of orbital angular momentum
states for a single ultracold atom trapped in two dimensional (2D) systems of sided coupled
cylindrically symmetric identical traps.

We consider two 2D in-line ring potentials and three 2D rings in a triangular configuration.
The full dynamics Hilbert space consists of a set of decoupled manifolds spanned by ring states
with identical vibrational and orbital angular momentum quantum numbers. Recalling basic
geometric symmetries of the system, we demonstrate that the tunneling amplitudes between
different ring states, named cross-couplings, with (without) variation of the winding number,
are complex (real). Moreover, we show that a complex self-coupling between states with opposite
winding number within a ring arises due to the breaking of cylindrical symmetry induced by the
presence of additional rings and that these complex couplings can be controlled geometrically.
Although for two in-line rings, the complex cross-coupling contribution is shown to give a non-physically
relevant phase, we demonstrate that, in a triangular ring configuration, it leads to
the possibility of engineering spatial dark states, which allows manipulating the transport of
angular momentum states via quantum interference.

{\normalsize
[1] J. Struck, C. \"{O}lschläger, M. Weinberg, P. Hauke, J. Simonet, A. Eckardt, M. Lewenstein, K.
Sengstock, and P. Windpassinger, Phys. Rev. Lett. \textbf{108}, 225304 (2012).
\vsp

[2] K. Jim\'{e}nez-Garcia, L. J. LeBlanc, R. A. Williams, M. C. Beeler, A. R. Perry, and I. B.
Spielman, Phys. Rev. Lett. \textbf{108}, 225303 (2012).
\vsp

[3] J. Polo, J. Mompart and V. Ahufinger, Phys. Rev. A \textbf{93}, 033613 (2016).
}



\vspace{\baselineskip} 