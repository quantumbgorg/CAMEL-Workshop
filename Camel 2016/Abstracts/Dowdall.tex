\title{QUANTUM CONTROL OF MANY FERMION SYSTEMS UTILIZING PAULI-EXCLUSION PRINCIPLE}
%Quantum control of many fermion systems utilizing Pauli-exclusion principle

\underline{T. Dowdall} \index{Dowdall T}

{\normalsize{\vspace{-4mm}
Department of Physics,
University College Cork,
Cork, Ireland

\email t.dowdall@umail.ucc.ie}}

Quick manipulation of particle states is required for advances in quantum information processing.
Typically this is done by using adiabatic schemes to alter the Hamiltonian of a system, adiabatic processes have the advantage of keeping particles in the instantaneous eigenstate of the Hamiltonian.  They are however susceptible to noise and are generally quite slow. In this study we propose a new method for controlling many particle systems that utilizes the Pauli-exclusion principle. This new method will allow quick and robust control of such systems, for example will we demonstrate how this new technique enables high fidelity transport or expansion of a variety of traps. We analyse numerous schemes for these purposes.

%{\normalsize
%[1] Suchowski et al., Phys. Rev. A. \textbf{78}, 063821 (2008).
%}

\vspace{\baselineskip} 