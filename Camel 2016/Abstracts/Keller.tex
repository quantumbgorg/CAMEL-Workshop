\title{INTERFACING IONS AND PHOTONS}
%Interfacing ions and photons

\underline{M. Keller} \index{Keller M}

{\normalsize{\vspace{-4mm}
University of Sussex,
Pevensey 2,
Brighton, BN1 9QH

\email m.k.keller@sussex.ac.uk}}

The complementary benefits of trapped ions and photons as carriers of quantum information make
it appealing to combine them in a joint system. Ions provide low decoherence rates, long
storage times and high readout efficiency, while photons travel over long distances. To
interface the quantum states of ions and photons efficiently, we use calcium ions coupled to an
optical high-finesse cavity via a Raman transition.

For strong ion-cavity coupling, deterministic transfer of quantum states between ions and
photons is possible. Each basis state of the ion is linked with one polarization mode of the
cavity. Through a partially transparent cavity mirror, a freely propagating photon is generated
which can be used to distribute quantum information, for example to entangle distant ions in a
network of multiple quantum nodes. Ion-cavity systems can also be employed to create
entanglement between ions trapped in the same cavity. For moderate coupling, quantum
entanglement may be generated probabilistically. Ions coupled to two orthogonally polarized
cavity modes are projected to an entangled state upon detection of photons emitted from the
cavity with different polarization.

The realization of these schemes requires the development of novel techniques to combine ion
traps with miniature optical cavities, as the strength of the ion-photon coupling increases
with shrinking cavity mode volume. We are presently testing two different setups, optimized for
the respective interaction regimes mentioned above.

%{\normalsize
%[1] Suchowski et al., Phys. Rev. A. \textbf{78}, 063821 (2008).
%}

\vspace{\baselineskip} 