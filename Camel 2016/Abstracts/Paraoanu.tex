\title{EXPERIMENTAL REALIZATION OF STIMULATED RAMAN ADIABATIC PASSAGE IN A TRANSMON}
%Experimental realization of stimulated Raman adiabatic passage in a transmon

\underline{S. Paraoanu} \index{Paraoanu S}

{\normalsize{\vspace{-4mm}
Department of Applied Physics,
Aalto University,
PO Box 15100,
00076 Aalto,
Finland

\email thomas.walther@physik.tu-darmstadt.de}}

The adiabatic manipulation of quantum states is a powerful technique from quantum optics and atomic physics. Previous work on the Autler-Townes effect in superconducting phase qutrits [1], on Stueckelberg interference [2], and on the effect of motional averaging in transmons [3] has added evidence that superconducting circuits truly behave as controllable artificial atoms.
Here we benchmark the stimulated Raman adiabatic passage process for circuit quantum
electrodynamics, by using the first three levels of a transmon qubit [4]. To realize this coherent transfer, we use two adiabatic Gaussian-shaped control microwave pulses coupled
to the first and the second transition. In this ladder configuration, we measure a population transfer efficiency above 80\% between the ground state and the second excited state.  The advantage of this technique is robustness against errors in the timing of the control pulses. By doing quantum tomography at successive moments during the Raman pulses, we investigate the transfer of the
population in time-domain. We also show that this protocol can be reversed by applying
a third adiabatic pulse. Furthermore, we study the effect of applying the adiabatic
Raman sequence to a superposition between the ground and the first excited state,
and we present experimental results for the case of a quasi-degenerate intermediate
level. The result is one step towards the realization of holonomic quantum computing and
quantum simulators with superconducting circuits [5].

{\normalsize
[1] M. A. Sillanp\"{a}\"{a} et al., Phys. Rev. Lett. \textbf{103}, 193601 (2009);
J. Li et al., Phys. Rev. B \textbf{84}, 104527 (2011);
J. Li et al., Sci. Rep. \textbf{2}, 645 (2012).
\vsp

[2] M. P. Silveri et al., New J. Phys. \textbf{17}, 043058 (2015).
\vsp

[3] J. Li, Nat. Commun. \textbf{4}, 1420 (2013).
\vsp

[4] K. S. Kumar et al., Nat. Commun. \textbf{7}, 10628 (2016).
\vsp

[5] G. S. Paraoanu, J. Low. Temp. Phys. \textbf{175}, 633-654 (2014).
}



\vspace{\baselineskip} 