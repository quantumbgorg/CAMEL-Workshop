\title{MEASURING ION HEATING IN A SURFACE TRAP WITH VARIABLE ION-ELECTRODE SEPARATION}
%Measuring ion heating in a surface trap with variable ion-electrode separation

\underline{I. Boldin} \index{Boldin I}

{\normalsize{\vspace{-4mm}
\siegen

\email boldin@physik.uni-siegen.de}}

Planar electrode ion traps are considered to have great potential for quantum information science as: a) they allow for complex electrode structures that might be needed for future quantum processors; b) ions can be trapped tens of micrometers above the surface, therefore strong static and oscillating field gradients can be imposed on ions and utilized in the realization of microwave qubit processing.

Reducing the ion-surface separation is advantageous for increasing EM field gradients for qubit manipulation, but the noise from the electrode surface also increases. This noise heats up motional degrees of freedom of an ion. At a height of tens of microns this effect becomes a major concern in quantum information processing experiments. The heating rate is orders of magnitude greater then expected from Johnson noise on the electrode surface. This effect is known as ``anomalous heating'' as it is not fully understood.
The heating rate has been measured in numerous experiments, however to our knowledge there are no direct measurements of its dependence on ion-electrode separation in a planar electrode configuration. Our planar electrode ion trap allows for variation of ion height above the surface in the range of 45-155 $\mu m$ by applying an additional RF field to the central electrode of the trap. Here we present the results of measuring the heating rates at variable ion-surface separation.

%{\normalsize
%[1] Suchowski et al., Phys. Rev. A. \textbf{78}, 063821 (2008).
%}

\vspace{\baselineskip} 