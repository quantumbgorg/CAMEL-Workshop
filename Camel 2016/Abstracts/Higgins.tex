\title{OBSERVATION OF EIT AND TRAP EFFECTS WITH A SINGLE RYDBERG ION}
%Observation of EIT and trap effects with a single Rydberg ion

\underline{G. Higgins} \index{Higgins G}

{\normalsize{\vspace{-4mm}
Stockholm University,
AlbaNova University Centre,
Roslagstullsbacken 21,
SE -- 106 91 Stockholm

\email gerard.higgins@fysik.su.se}}

Trapped Rydberg ions are a novel approach to quantum information processing, which joins the
advanced quantum control of trapped ions with the strong dipolar interactions between Rydberg atoms
[1-2]. It promises to speed up entangling interactions and to enable such operations in larger ion
crystals.

In the experiments presented here a single strontium ion was trapped in the centre of a linear Paul
trap and excited to Rydberg S- and D-states by a two-photon transition via an intermediate state.
For quantum information applications the Rydberg excitation needs to be controlled coherently.
Using the second excitation laser as a pump and the first excitation laser as a probe we observed
EIT-the first observed coherent phenomenon with a Rydberg ion. The strong electric fields used for
trapping Rydberg ions give rise to phenomena which are not usually observed in neutral Rydberg atom
experiments, including the quadrupole shift and Floquet sidebands, which were observed directly in
the Rydberg excitation spectra.

In the future in this project we plan to excite the ions to even higher Rydberg states and to apply
microwave fields to induce large oscillating dipole moments. In this way, ions should be able to
interact at a distance as will be required for quantum gates via the Rydberg interaction.

{\normalsize
[1] M. M\"{u}ller, et al., New J. Phys. \textbf{10}, 093009 (2008).
\vsp

[2] T. Feldker, et al., Phys. Rev. Lett. \textbf{115}, 173001 (2015).
}



\vspace{\baselineskip} 